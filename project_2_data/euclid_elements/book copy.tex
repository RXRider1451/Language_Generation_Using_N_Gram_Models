
\documentclass[oneside]{book}
\usepackage[reqno]{amsmath}		
\usepackage{amssymb}			
\usepackage[english]{babel}		
\usepackage[utf8]{inputenc}		
\usepackage{graphicx}                   
\usepackage{makeidx}			
\usepackage{wrapfig}			
\usepackage{needspace}			
\newcounter{wrapwidth}
\makeindex


\def\changemargin#1#2{\list{}{\rightmargin#2\leftmargin#1}\item[]}
\let\endchangemargin=\endlist


\newcommand{\ibksp}{\hspace{-1.5em}}


\newcommand\pgram{
	\setlength{\unitlength}{1pt}
	\begin{picture}(11,8)(0,0)
	\linethickness{0.5pt}
	\put(1,0){\line(1,0){9}}
	\put(1,0){\line(0,1){5}}
	\put(1,5){\line(1,0){9}}
	\put(10,0){\line(0,1){5}}
	\end{picture}}


\newcommand\mybook[2]{
	\newpage\begin{center}{\LARGE #1.} \bigskip\\{\large \uppercase{#2}}\\ \bigskip
        \rule[1ex]{3cm}{.2pt}\\ \bigskip {\large DEFINITIONS.}\end{center}
	\noindent
	\addtocontents{toc}{\bigskip\protect\centering {\large #1.}\medskip}
	\addcontentsline{toc}{section}{\protect \ibksp\textsf{#2,}}
	\addcontentsline{toc}{section}{Definitions,}
}


\newcommand\myprop[2]{
\bigskip\Needspace*{4\baselineskip}\begin{center}\textsc{#1}\\\medskip\emph{#2}\par\end{center}
}


\newcommand\mypropl[2]{
\bigskip\Needspace*{4\baselineskip}\begin{center}\textsc{#1}\end{center}
\hspace{\parindent}\emph{#2}\par\medskip
}


\newcommand\mypropflow[5]{
\bigskip\Needspace*{4\baselineskip}\begin{center}\textsc{#1}\end{center}
\setcounter{wrapwidth}{#3}
\begin{wrapfigure}[#4]{r}{\value{wrapwidth}pt}
\begin{center}
\vspace{-0.3in}

\end{center}
\end{wrapfigure}\par
\indent\emph{\protect{#2}}\par\medskip
}


\newcommand\mypropflot[5]{
\vspace{0.5\baselineskip}
\bigskip\Needspace*{4\baselineskip}
\setcounter{wrapwidth}{#3}
\begin{wrapfigure}[#4]{r}{\value{wrapwidth}pt}
\begin{center}
\vspace{-0.3in}

\end{center}
\end{wrapfigure}
\hspace{1.5\parindent}\textsc{#1}\par\medskip
\indent\emph{#2}\par\medskip
}


\newcommand\exhead[1]{
\Needspace*{5\baselineskip}\begin{center}
\textsf{#1}
\end{center}
}


\newcommand\exheadin[1]{
\par\medskip\Needspace*{4\baselineskip}
\hspace{1.5\parindent}\textsf{#1}\par\medskip
}




\newcommand\imgflow[3]{
\setcounter{wrapwidth}{#1}

\begin{wrapfigure}[#2]{r}{\value{wrapwidth}pt}
\begin{center}
\vspace{-0.3in}

\end{center}
\end{wrapfigure}
}


\newcommand\imgcent[2]{
\begin{center}

\end{center}
}

\begin{document}
\pagestyle{empty}

\thispagestyle{empty}
\small
\begin{verbatim}
Project Gutenberg's First Six Books of the Elements of Euclid,
by John Casey

This eBook is for the use of anyone anywhere in the United States and
most other parts of the world at no cost and with almost no restrictions
whatsoever. You may copy it, give it away or re-use it under the terms
of the Project Gutenberg License included with this eBook or online at
www.gutenberg.org. If you are not located in the United States, you
will have to check the laws of the country where you are located before
using this eBook.

Title: The First Six Books of the Elements of Euclid

Subtitle: And Propositions I.-XXI. of Book XI., and an
          Appendix on the Cylinder, Sphere, Cone, etc.,

Author: John Casey

Author: Euclid

Release Date: April 14, 2007 [eBook #21076]
[Most recently updated: July 18, 2022]

Language: English

Character set encoding: UTF-8

Produced by Joshua Hutchinson, Keith Edkins and the
Online Distributed Proofreading Team at https://www.pgdp.net
(This file was produced from images from the Cornell University
Library: Historical Mathematics Monographs collection.)
Revised by Richard Tonsing.

*** START OF THE PROJECT GUTENBERG EBOOK ELEMENTS OF EUCLID ***
\end{verbatim}
\normalsize

\frontmatter
\pagestyle {empty}




\begin{center}
\texttt{Production Note}
\end{center}

\noindent
\texttt{Cornell University Library produced this volume to replace the \\
irreparably deteriorated original. It was scanned using Xerox \\
software and equipment at 600 dots per inch resolution and com- \\
pressed prior to storage using CCITT Group 4 compression. The \\
digital data were used to create Cornell's replacement volume \\
on paper that meets the ANSI Standard Z39.48-1984. The produc- \\
tion of this volume was supported in part by the Commission on \\
Preservation and Access and the Xerox Corporation. Digital \\
file copyright by Cornell University Library 1991.}

\bigskip\bigskip
\scriptsize \noindent \textsc{Transcriber's Note:} \emph{The Index has been regenerated
to fit the pagination of this edition. Despite the author's stated hope that ``few misprints
have escaped detection'' there were several, which have here been corrected and
noted at the end of the text.} \normalsize





\vfill
\begin{center}
\framebox{\begin{minipage}[c]{4in}
\begin{center}
{\Large CORNELL UNIVERSITY LIBRARY}

\medskip
\rule{1in}{1pt}
\medskip

{\large\sf THE EVAN WILHELM EVANS \\
MATHEMATICAL SEMINARY LIBRARY}

\medskip
\rule{1in}{1pt}
\medskip

{\small THE GIFT OF} \\
LUCIEN AUGUSTUS WAIT

\end{center}
\medskip
{\small \dotfill}
\end{minipage}
}
\end{center}
\vfill
\newpage






\vfill
\begin{center}
{\Large THE ELEMENTS OF EUCLID.}
\end{center}
\vfill
\newpage



\vfill

\begin{center}
{\large \textsf{NOW READY}}

\medskip
\rule{0.5in}{1pt}
\medskip

\textbf{Price 3s.}

\textbf{A TREATISE ON ELEMENTARY TRIGONOMETRY,}\\
\emph{With Numerous Examples and Questions for Examination.}


\medskip
\rule{1in}{1pt}
\medskip

{\small\textbf{Third Edition, Revised and Enlarged, Price 3s.\ 6d., Cloth.}} \\
A SEQUEL TO THE FIRST SIX BOOKS OF THE \\
ELEMENTS OF EUCLID, \\
\emph{Containing an Easy Introduction to Modern Geometry:} \\
\textbf{With numerous Examples.}

\medskip
\rule{1in}{1pt}
\medskip

{\small\textbf{Third Edition, Price 4s.\ 6d.; or in two parts, each 2s.\ 6d.}} \\
THE ELEMENTS OF EUCLID, BOOKS I.---VI\@., AND \\
PROPOSITIONS I.---XXI\@., OF BOOK XI\@.; \\
\emph{Together with an Appendix on the Cylinder, Sphere, \\
Cone, \&c.: with} \\
\textbf{Copious Annotations \& numerous Exercises.}

\medskip
\rule{1in}{1pt}
\medskip

\textbf{Price 6s.} \\
A KEY TO THE EXERCISES IN THE FIRST SIX \\
BOOKS OF CASEY'S ELEMENTS OF EUCLID.

\medskip
\rule{1in}{1pt}
\medskip

\textbf{Price 7s.\ 6d.}

A TREATISE ON THE ANALYTICAL GEOMETRY OF \\
THE POINT, LINE, CIRCLE, \& CONIC SECTIONS, \\
\emph{Containing an Account of its most recent Extensions,} \\
\textbf{With numerous Examples.}

\medskip
\rule{1in}{1pt}
\medskip

DUBLIN: HODGES, FIGGIS, \& CO. \\
LONDON: LONGMANS \& CO.
\end{center}
\vfill



\newpage
\begin{center}

{\LARGE THE FIRST SIX BOOKS}\bigskip

{\small OF THE}\bigskip

{\Huge ELEMENTS OF EUCLID,}\bigskip

{\small AND}\bigskip

{\Large PROPOSITIONS I.-XXI\@. OF BOOK XI\@.,}\bigskip

{\small AND AN}\bigskip

{\Large \textit{APPENDIX ON THE CYLINDER, SPHERE,\\
CONE, E{\normalsize TC.}},}\bigskip

{\small WITH}\bigskip

COPIOUS ANNOTATIONS AND NUMEROUS EXERCISES.\bigskip

{\small BY}\medskip

{\LARGE J\,O\,H\,N\, \,C\,A\,S\,E\,Y,\, \,LL\@.\,D.,\, \,F.\,R.\,S.,}\medskip

{\small FELLOW OF THE ROYAL UNIVERSITY OF IRELAND;\\
MEMBER OF COUNCIL, ROYAL IRISH ACADEMY;\\
MEMBER OF THE MATHEMATICAL SOCIETIES OF LONDON AND FRANCE;\\
AND PROFESSOR OF THE HIGHER MATHEMATICS AND OF\\
MATHEMATICAL PHYSICS IN THE CATHOLIC UNIVERSITY OF IRELAND.}\bigskip


\imgcent{144}{f000}

\bigskip

\textit{THIRD EDITION, REVISED AND ENLARGED.}\smallskip

DUBLIN: HODGES, FIGGIS, \& CO., GRAFTON-ST.

LONDON: LONGMANS, GREEN, \& CO.

1885.

\end{center}



\newpage

\begin{center}
DUBLIN\medskip

PRINTED AT THE UNIVERSITY PRESS,\medskip

BY PONSONBY AND WELDRICK
\end{center}



\newpage
\frontmatter
\pagestyle{plain}


\begin{center}
{\LARGE PREFACE.}

\bigskip

\rule[1ex]{1.5cm}{.2pt}

\bigskip

\end{center}

\noindent This edition of the Elements of Euclid, undertaken
at the request of the principals of some of
the leading Colleges and Schools of Ireland, is
intended to supply a want much felt by teachers
at the present day---the production of a work
which, while giving the unrivalled original in
all its integrity, would also contain the modern
conceptions and developments of the portion of
Geometry over which the Elements extend. A
cursory examination of the work will show that
the Editor has gone much further in this latter
direction than any of his predecessors, for it will
be found to contain, not only more actual matter
than is given in any of theirs with which he is
acquainted, but also much of a special character,
which is not given, so far as he is aware, in any
former work on the subject. The great extension

of geometrical methods in recent times has made
such a work a necessity for the student, to enable
him not only to read with advantage, but even to
understand those mathematical writings of modern
times which require an accurate knowledge
of Elementary Geometry, and to which it is in
reality the best introduction.

In compiling his work the Editor has received
invaluable assistance from the late Rev.\ Professor
Townsend\index{Townsend}, \textsc{s.f.t.c.d.} The book was rewritten
and considerably altered in accordance with his
suggestions, and to that distinguished Geometer
it is largely indebted for whatever merit it possesses.

The Questions for Examination in the early part
of the First Book are intended as specimens,
which the teacher ought to follow through the
entire work. Every person who has had experience
in tuition knows well the importance
of such examinations in teaching Elementary
Geometry.

The Exercises, of which there are over eight
hundred, have been all selected with great care.
Those in the body of each Book are intended as
applications of Euclid's Propositions. They are

for the most part of an elementary character, and
may be regarded as common property, nearly
every one of them having appeared already in
previous collections. The Exercises at the end
of each Book are more advanced; several are
due to the late Professor Townsend, some are
original, and a large number have been taken
from two important French works---\textsc{Catalan's}\index{Catalan}
\textit{Théorèmes et Problèmes de Géométrie Elémentaire},
and the \textit{Traité de Géométrie}, by \textsc{Rouché}\index{Rouché} and
\textsc{De Comberousse}\index{Comberousse}.

The second edition has been thoroughly revised
and greatly enlarged. The new matter includes
several alternative proofs, important examination
questions on each of the books, an explanation of
the ratio of incommensurable quantities, the first
twenty-one propositions of Book XI\@., and an
Appendix on the properties of the Prism, Pyramids,
Cylinder, Sphere, and Cone.

The present Edition has been very carefully
read throughout, and it is hoped that few misprints
have escaped detection.

The Editor is glad to find from the rapid sale of
former editions (each 3000 copies) of his Book, and
its general adoption in schools, that it is likely to

accomplish the double object with which it was
written, viz.\ to supply students with a Manual
that will impart a thorough knowledge of the
immortal work of the great Greek Geometer, and
introduce them, at the same time, to some of the
most important conceptions and developments of
the Geometry of the present day.
\bigskip

\begin{flushright}
JOHN CASEY. \mbox\quad
\end{flushright}
\bigskip

\noindent
\begin{tabular}{@{}c@{}}
\textsc{86, South Circular-road, Dublin.}\\
\textit{November}, 1885.
\end{tabular}


\tableofcontents






























































































\newpage
\mainmatter
\pagestyle{plain}
\addcontentsline{toc}{section}{\ibksp\textsc{Introduction,}}
\begin{center}
{\LARGE THE ELEMENTS OF EUCLID.}

\bigskip\rule[1ex]{3cm}{.2pt}\bigskip

{\Large INTRODUCTION.}
\end{center}


\noindent Geometry is the Science of figured Space. Figured
Space is of one, two, or three dimensions, according
as it consists of lines, surfaces, or solids. The
boundaries of solids are surfaces; of surfaces, lines;
and of lines, points. Thus it is the province of
Geometry to investigate the properties of solids,
of surfaces, and of the figures described on surfaces.
The simplest of all surfaces is the plane,
and that department of Geometry which is occupied
with the lines and curves drawn on a plane
is called \textit{Plane Geometry}; that which demonstrates
the properties of solids, of curved surfaces, and the
figures described on curved surfaces, is \textit{Geometry}

\textit{of Three Dimensions}. The simplest lines that can
be drawn on a plane are the right line and circle,
and the study of the properties of the point, the
right line, and the circle, is the introduction to
Geometry, of which it forms an extensive and important
department. This is the part of Geometry
on which the oldest Mathematical Book in existence,
namely, Euclid's \textit{Elements}, is written, and
is the subject of the present volume. The \textit{conic
sections} and \textit{other curves} that can be described on
a plane form special branches, and complete the
divisions of this, the most comprehensive of all the
Sciences. The student will find in Chasles'\index{Chasles} \textit{Aperçu
Historique} a valuable history of the origin and
the development of the methods of Geometry.

\vspace{0.2in}
\begin{center}
\rule[1ex]{5cm}{.2pt}
\end{center}
\vspace{0.2in}

In the following work, when figures are not
drawn, the student should construct them from the
given directions. The Propositions of Euclid will
be printed in larger type, and will be referred to
by Roman numerals enclosed in brackets. Thus
[III\@. \textsc{xxxii.}] will denote the 32nd Proposition
of the 3rd Book. The number of the Book will
be given only when different from that under

which the reference occurs. The general and the
particular enunciation of every Proposition will be
given in one. By omitting the letters enclosed
in parentheses we have the general enunciation,
and by reading them, the particular. The annotations
will be printed in smaller type.     The
following symbols will be used in them:---

\begin{tabular}{l c c}
Circle & will be denoted by & $\bigodot$ \\
Triangle & ,, & $\triangle$ \\
Parallelogram & ,, & $\pgram$ \\
Parallel lines & ,, & $\parallel$ \\
Perpendicular & ,, & $\perp$
\end{tabular}

In addition to these we shall employ the usual
symbols $+$, $-$, \&c.\ of Algebra, and also the sign
of congruence, namely $\equiv$. This symbol has been
introduced by the illustrious Gauss.\index{Gauss}






\mybook{BOOK I}{Theory of Angles, Triangles, Parallel Lines, and parallelograms.}

\begin{center}
\textsc{The Point.}\index{Points}
\end{center}

\textsc{i.} A point is that which has position but not dimensions.

\begin{footnotesize}
A geometrical magnitude which has three dimensions, that is,
length, breadth, and thickness, is a solid; that which has two
dimensions, such as length and breadth, is a surface; and that
which has but one dimension is a line. But a point is neither a
solid, nor a surface, nor a line; hence it has no dimensions---that
is, it has neither length, breadth, nor thickness.
\par\end{footnotesize}

\begin{center}
\textsc{The Line.}\index{Line}
\end{center}

\textsc{ii}. A line is length without breadth.

\begin{footnotesize}
A line is space of one dimension. If it had any breadth, no
matter how small, it would be space of two dimensions; and if in
addition it had any thickness it would be space of three dimensions;
hence a line has neither breadth nor thickness.
\par\end{footnotesize}

\textsc{iii.} The intersections of lines and their extremities
are points.




\imgflow{108}{2}{f262}

\textsc{iv.} A line which lies evenly between its extreme
points is called a straight or right
line, such as $AB$.

\begin{footnotesize}
If a point move without changing its direction it will describe
a right line. The direction in which a point moves in called its
``\textit{sense}.'' If the moving point continually changes its direction
it will describe a curve; hence it follows that only one right line
can be drawn between two points. The following Illustration is
due to Professor Henrici\index{Henrici}:---``If we suspend a weight by a string,
the string becomes stretched, and we say it is straight, by which
we mean to express that it has assumed a peculiar definite shape.
If we mentally abstract from this string all thickness, we obtain
the notion of the simplest of all lines, which we call a straight
line.''
\par\end{footnotesize}

\begin{center}
\textsc{The Plane.}
\end{center}

\textsc{v.} A surface\index{Surface} is that which has length and breadth.

\begin{footnotesize}
A surface is space of two dimensions. It has no thickness, for if
it had any, however small, it would be space of three dimensions.
\par\end{footnotesize}

\textsc{vi.}  When a surface is such that the right line joining
any two arbitrary points in it lies wholly in the
surface, it is called a \textit{plane}.\index{Plane}

\begin{footnotesize}
A plane is perfectly flat and even, like the surface of still
water, or of a smooth floor.---\textsc{Newcomb}.\index{Newcomb}
\par\end{footnotesize}

\newpage
\begin{center}
\textsc{Figures.}\index{Figures}
\end{center}

\textsc{vii.} Any combination of points, of lines, or of points
and lines in a plane, is called a \textit{plane} figure. If a figure
be formed of points only it is called a \textit{stigmatic} figure;
and if of right lines only, a \textit{rectilineal} figure.\index{Figures!rectilineal}

\textsc{viii.} Points which lie on the same right line are
called \textit{collinear} points.\index{Points!collinear} A figure formed of collinear
points is called a \textit{row} of points.\index{Points!row of}

\begin{center}
\textsc{The Angle.}\index{Angle}
\end{center}

\textsc{ix.} The inclination of two right lines extending out
from one point in different directions is called a \textit{rectilineal}
angle.



\textsc{x.} The two lines are called the \textit{legs},\index{Angle!legs of} and the point
the \textit{vertex}\index{Angle!vertex of}\index{Vertex!of angle} of the angle.


\imgflow{113}{11}{f001}
\begin{footnotesize}
A right line drawn from the vertex and turning about it in the
plane of the angle, from the position of coincidence with one leg
to that of coincidence with the other, is said to turn through the
angle, and the angle is the greater as the quantity of turning is
the greater. Again, since the line may turn from one position to
the other in either of two ways, two angles are formed by two
lines drawn from a point.

Thus if $AB$, $AC$ be the legs, a line
may turn from the position $AB$ to the
position $AC$ in the two ways indicated
by the arrows. The smaller of the angles
thus formed is to be understood as
the angle contained by the lines. The
larger, called a \textit{re-entrant} angle\index{Angle!re-entrant}, seldom
occurs in the ``Elements.''
\par\end{footnotesize}

\textsc{xi.} \textit{Designation of Angles.}---A particular angle in a
figure is denoted by three letters, as $BAC$, of which the
middle one, $A$, is at the vertex, and the other two along
the legs. The angle is then read $BAC$.


\imgcent{300}{f002}

\textsc{xii.} The angle formed by joining two or more angles
together is called their \textit{sum}. Thus the sum of the two
angles $ABC$, $PQR$ is the angle $AB'R$,
\imgflow{129}{7}{f003}

formed by applying
the side $QP$ to the side $BC$, so that the vertex $Q$ shall
fall on the vertex $B$, and the
side $QR$ on the opposite side
of $BC$ from $BA$.

\textsc{xiii.} When the sum of two
angles $BAC$, $CAD$ is such that
the legs $BA$, $AD$ form one right
line, they are called \textit{supplements} of each other.

\begin{footnotesize}
Hence, when one line stands on another, the two angles which
it makes on the same side of that on which it stands are supplements
of each other.
\par\end{footnotesize}




\imgflow{121}{7}{f004}
\textsc{xiv}. When one line stands on another, and makes
the adjacent angles\index{Angle!adjacent} at both
sides of itself equal, each of
the angles is called a \textit{right}
angle\index{Angle!right}, and the line which
stands on the other is called
a \textit{perpendicular}\index{Lines!perpendicular} to it.


\begin{footnotesize}
Hence a right angle is equal to its supplement.
\par\end{footnotesize}

\textsc{xv}. An \textit{acute} angle\index{Angle!acute} is one which is less than a right
angle, as $A$.


\imgcent{266}{f005}

\textsc{xvi}. An \textit{obtuse} angle is one which is greater than a
right angle, as $BAC$.

\begin{footnotesize}
The supplement\index{Angle!supplement of} of an acute angle is obtuse, and conversely, the
supplement of an obtuse
angle is acute.
\par\end{footnotesize}

\imgflow{100}{7}{f006}
\textsc{xvii}. When the sum of two angles
is a right angle, each is called the
\textit{complement}\index{Angle!complement of} of the other. Thus, if
the angle $BAC$ be right, the angles
$BAD$, $DAC$ are complements of each
other.

\begin{center}
\textsc{Concurrent Lines.}\index{Lines!Concurrent}
\end{center}

\textsc{xviii}. Three or more right lines passing through the
same point are called \textit{concurrent} lines.

\textsc{xix}. A system of more than three concurrent lines is
called a \textit{pencil} of lines\index{Lines!pencil of}. Each line of a pencil is called
a \textit{ray}\index{Ray}, and the common point through which the rays
pass is called the \textit{vertex}\index{Vertex!of pencil}.



\begin{center}
\textsc{The Triangle.}\index{Triangle}
\end{center}

\textsc{xx}. A \textit{triangle} is a figure formed by three right
lines joined end to end. The three lines are called
its \textit{sides}.

\textsc{xxi}. A triangle whose three sides are unequal is said
to be \textit{scalene}\index{Triangle!scalene}, as $A$; a triangle having two sides equal,
to be \textit{isosceles}\index{Triangle!isosceles}, as $B$; and and having all its sides equal,
to be \textit{equilateral}\index{Triangle!equilateral}, as $C$.


\imgcent{233}{f007}

\textsc{xxii}. A \textit{right-angled} triangle\index{Triangle!right-angled} is one that has one of
its angles a right angle, as $D$. The side which subtends
the right angle is called the \textit{hypotenuse}\index{Hypotenuse}.

\textsc{xxiii}. An \textit{obtuse-angled} triangle\index{Triangle!obtuse-angled} is one that has one
of its angles obtuse\index{Angle!obtuse}, as $E$.

\textsc{xxiv}. An \textit{acute-angled} triangle is one that has its three
angles acute, as $F$.


\imgcent{233}{f008}

\textsc{xxv}. An \textit{exterior} angle\index{Angle!exterior} of a triangle is one that is
formed by any side and the continuation of another
side.

\begin{footnotesize}
Hence a triangle has six exterior angles; and also each exterior
angle is the supplement of the adjacent interior angle.
\par\end{footnotesize}


\begin{center}
\textsc{The Polygon.}\index{Polygon}
\end{center}

\textsc{xxvi}. A \textit{rectilineal} figure bounded by more than three
right lines is usually called a polygon.

\textsc{xxvii}. A polygon is said to be \textit{convex} when it has no
re-entrant angle.


\textsc{xxviii}. A polygon of four sides is called a \textit{quadrilateral}.\index{Quadrilateral}

\textsc{xxix}. A quadrilateral whose four sides are equal is
called a \textit{lozenge}.\index{Lozenge}

\textsc{xxx}. A lozenge which has a right angle is called a
\textit{square}.\index{Square}

\textsc{xxxi}. A polygon which has five sides is called a
\textit{pentagon}\index{Pentagon}; one which has six sides, a \textit{hexagon}, and so on.


\begin{center}
\textsc{The Circle.}
\end{center}


\imgflow{95}{7}{f009}
\textsc{xxxii}. A \textit{circle}\index{Circle} is a plane figure formed by a curved
line called the \textit{circumference}\index{Circle!circumference}, and is
such that all right lines drawn
from a certain point within the
figure to the circumference are
equal to one another. This point
is called the \textit{centre}\index{Centre}.

\textsc{xxxiii}. A \textit{radius}\index{Circle!radius of} of a circle is
any right line drawn from the centre to the circumference,
such as $CD$.

\textsc{xxxiv}. A \textit{diameter}\index{Circle!diameter of} of a circle is a right line drawn
through the centre and terminated both ways by the
circumference, such as $AB$.

\begin{footnotesize}
From the definition of a circle it follows at once that the path
of a movable point in a plane which remains at a constant distance
from a fixed point is a circle; also that any point $P$ in the plane is
inside, outside, or on the circumference of a circle according as
its distance from the centre is \textit{less} than, \textit{greater} than, or \textit{equal} to,
the radius.
\par\end{footnotesize}

\begin{center}
\textsc{Postulates.}\index{Postulate}
\end{center}

Let it be granted that---

\textsc{i}. A right line may be drawn from any one point to
any other point.

\begin{footnotesize}
When we consider a straight line contained between two fixed
points which are its ends, such a portion is called a \textit{finite straight
line}.
\par\end{footnotesize}

\textsc{ii}. A terminated right line may be produced to any
length in a right line.



\begin{footnotesize}
Every right line may extend without limit in either direction
or in both. It is in these cases called an \textit{indefinite} line. By this
postulate a finite right line may be supposed to be produced,
whenever we please, into an indefinite right line.
\par\end{footnotesize}

\textsc{iii}. A circle may be described from any centre, and
with any distance from that centre as radius.



\imgflow{100}{3}{f010}
\begin{footnotesize}
If there be two points $A$ and $B$, and if with any instruments,
such as a ruler and pen, we draw a line
from $A$ to $B$, this will evidently have some
irregularities, and also some breadth and
thickness. Hence it will not be a geometrical line no matter how
nearly it may approach to one. This is the reason that Euclid
postulates the drawing of a right line from one point to another.
For if it could be accurately done there would be no need for his
asking us to let it be granted. Similar observations apply to the
other postulates. It is also worthy of remark that Euclid never
takes for granted the doing of anything for which a geometrical
construction, founded on other problems or on the foregoing postulates,
can be given.
\par\end{footnotesize}


\begin{center}
\textsc{Axioms.}\index{Axioms}
\end{center}

\textsc{i}. Things which are equal to the same, or to equals,
are equal to each other.

\begin{footnotesize}
Thus, if there be three things, and if the first, and the second,
be each equal to the third, we infer by this axiom that the first is
equal to the second. This axiom relates to all kinds of magnitude.
The same is true of Axioms \textsc{ii}., \textsc{iii}., \textsc{iv}., \textsc{v}., \textsc{vi}., \textsc{vii}., \textsc{ix}.;
but \textsc{viii}., \textsc{x}., \textsc{xi}., \textsc{xii}., are strictly geometrical.
\par\end{footnotesize}

\textsc{ii}. If equals be added to equals the sums will be
equal.

\textsc{iii}. If equals be taken from equals the remainders
will be equal.

\textsc{iv}. If equals be added to unequals the sums will be
unequal.

\textsc{v}. If equals be taken from unequals the remainders
will be unequal.

\textsc{vi}. The doubles of equal magnitudes are equal.

\textsc{vii}. The halves of equal magnitudes are equal.

\textsc{viii}. Magnitudes that can be made to coincide are
equal.


\begin{footnotesize}
The placing of one geometrical magnitude on another, such as a
line on a line, a triangle on a triangle, or a circle on a circle, \&c.,
is called \emph{superposition}. The superposition employed in Geometry
is only \emph{mental}, that is, we conceive one magnitude placed on the
other; and then, if we can prove that they coincide, we infer, by
the present axiom, that they are equal. Superposition involves the
following principle, of which, without explicitly stating it, Euclid
makes frequent use:---``Any figure may be transferred from one
position to another without change of form or size.''
\par\end{footnotesize}

\textsc{ix}. The whole is greater than its part.

\begin{footnotesize}
This axiom is included in the following, which is a fuller statement:---
\par\end{footnotesize}

\textsc{ix$'$}. The whole is equal to the sum of all its parts.

\textsc{x}. Two right lines cannot enclose a space.

\begin{footnotesize}
This is equivalent to the statement, ``If two right lines have two
points common to both, they coincide in direction,'' that is, they
form but one line, and this holds true even when one of the points
is at infinity.
\par\end{footnotesize}

\textsc{xi}. All right angles are equal to one another.

\begin{footnotesize}
This can be proved as follows:---Let there be two right lines
$AB$, $CD$, and two perpendiculars to them, namely, $EF$, $GH$, then
if $AB$, $CD$ be made to coincide by superposition, so that the point
$E$ will coincide with $G$; then since a right angle is equal to its
supplement, the line $EF$ must coincide with $GH$. Hence the
angle $AEF$ is equal to $CGH$.
\end{footnotesize}

\textsc{xii}. If two right lines $(AB,\ CD)$ meet a third line
$(AC)$, so as to make the sum
of the two interior angles
$(BAC,\ ACD)$ on the same
side less than two right angles,
these lines being produced
shall meet at some
\emph{finite distance}.


\imgcent{132}{f011}

This axiom is the converse of Prop.~\textsc{xvii}., Book~I.



\begin{center}
\textsc{Explanation of Terms.}
\end{center}

\emph{Axioms}.---``Elements of human reason,'' according to
\textsc{Dugald Stewart}, are certain general propositions, the
truths of which are self-evident, and which are so fundamental,
that they cannot be inferred from any propositions
which are more elementary; in other words,
they are incapable of demonstration. ``That two sides
of a triangle are greater than the third'' is, perhaps,
self-evident; but it is not an axiom, inasmuch as it can
be inferred by demonstration from other propositions;
but we can give no proof of the proposition that
``things which are equal to the same are equal to one
another,'' and, being self-evident, it is an axiom.

\emph{Propositions}\index{Proposition} which are not axioms are properties of
figures obtained by processes of reasoning. They are
divided into theorems and problems.

A \emph{Theorem}\index{Theorem} is the formal statement of a property that
may be demonstrated from known propositions. These
propositions may themselves be theorems or axioms. A
theorem consists of two parts, the \emph{hypothesis}\index{Hypothesis}, or that
which is assumed, and the \emph{conclusion}\index{Conclusion}, or that which is
asserted to follow therefrom. Thus, in the typical
theorem,
\[
  \text{\ If } X \text{\ is } Y,\text{\ then }Z \text{\ is } W,
\tag{\textsc{i.}}
\]
the hypothesis is that $X$ is $Y$, and the conclusion is
that $Z$ is $W$.

\emph{Converse Theorems}\index{Theorem!converse of}\index{Proposition!converse of}.---Two theorems are said to be
converse, each of the other, when the hypothesis of
either is the conclusion of the other. Thus the converse
of the theorem (\textsc{i}.) is---
\[
  \text{\ If } Z \text{\ is } W, \text{\ then } X \text{\ is } Y.
\tag{\textsc{ii}.}
\]

From the two theorems (\textsc{i.}) and (\textsc{ii.}) we may infer

two others, called their \emph{contrapositives}\index{Contrapositive}\index{Proposition!contrapositive of}. Thus the contrapositive
\begin{align*}
&\text{of (\textsc{i}.) is, If $Z$ is not $W$, then $X$ is not $Y$;} \tag{\textsc{iii}.}
\\
&\text{of (\textsc{ii}.) is, If $X$ is not $Y$, then $Z$ is not $W$.} \tag{\textsc{iv}.}
\end{align*}

The theorem (\textsc{iv}.) is called the \emph{obverse}\index{Propositions!obverse of} of (\textsc{i}.), and
(\textsc{iii}.) the obverse of (\textsc{ii}.).

A \emph{Problem}\index{Problem} is a proposition in which something is
proposed to be done, such as a line to be drawn, or a
figure to be constructed, under some given conditions.

The \emph{Solution}\index{Solution} of a problem is the method of construction
which accomplishes the required end.

The \emph{Demonstration}\index{Demonstration} is the proof, in the case of a
theorem, that the conclusion follows from the hypothesis;
and in the case of a problem, that the construction
accomplishes the object proposed.

The \emph{Enunciation}\index{Enunciation} of a problem consists of two parts,
namely, the \emph{data}, or things supposed to be given, and
the \emph{quaesita}, or things required to be done.

\emph{Postulates} are the elements of geometrical construction,
and occupy the same relation with respect to
problems as axioms do to theorems.

A \emph{Corollary}\index{Corollary} is an inference or deduction from a proposition.

A \emph{Lemma}\index{Lemma} is an auxiliary proposition required in the
demonstration of a principal proposition.

A \emph{Secant}\index{Secant} or \emph{Transversal}\index{Lines!transversal} is a line which cuts a system
of lines, a circle, or any other geometrical figure.

\emph{Congruent}\index{Congruent} figures are those that can be made to
coincide by superposition. They agree in shape and
size, but differ in position. Hence it follows, by
Axiom~\textsc{viii}., that corresponding parts or portions of
congruent figures are congruent, and that congruent
figures are equal in every respect.

\emph{Rule of Identity}\index{Identity, rule of}.---Under this name the following
principle will be sometimes referred to:---``If there is
but one $X$ and one $Y$, then, from the fact that $X$ is $Y$,
it necessarily follows that $Y$ is $X$.''---\textsc{Syllabus}.



\addcontentsline{toc}{section}{Propositions \textsc{i.--xlviii.,}}
\myprop{PROP\@.~I.---Problem.}{On a given finite right line $(AB)$ to construct an equilateral triangle.}


\imgflow{164}{10}{f012}
\textbf{Sol.}---With $A$ as centre, and $AB$ as radius, describe
the circle $BCD$ (Post.~\textsc{iii.}). With $B$ as centre, and $BA$
as radius, describe the
circle $ACE$, cutting the
former circle in $C$. Join
$CA$, $CB$ (Post.~\textsc{i.}). Then
$ABC$ \emph{is the equilateral
triangle required.}

\textbf{Dem.}---Because $A$ is
the centre of the circle
$BCD$, $AC$ is equal to
$AB$ (Def.~\textsc{xxxii.}). Again, because $B$ is the centre of
the circle $ACE$, $BC$ is equal to $BA$. Hence we have
proved.
\begin{flalign*}
           &&  AC &= AB,  &&\\
\text{and }&&  BC &= AB.  &&\phantom{and }
\end{flalign*}

But things which are equal to the same are equal to
one another (Axiom~\textsc{i.}); therefore $AC$ is equal to $BC$;
therefore the three lines $AB$, $BC$, $CA$ are equal to one
another. Hence the triangle $ABC$ is equilateral (Def.~\textsc{xxi.});
and it is described on the given line $AB$, \emph{which
was required to be done.}


\exhead{Questions for Examination.}

\begin{footnotesize}
1.~What is the \emph{datum} in this proposition?

2.~What is the \emph{quaesitum}?

3.~What is a finite right line?

4.~What is the opposite of finite?

5.~In what part of the construction is the third postulate
quoted? and for what purpose? Where is the first postulate
quoted?

6.~Where is the first axiom quoted?

7.~What use is made of the definition of a circle? What is a
circle?

8.~What is an equilateral triangle?
\par\end{footnotesize}


\exhead{Exercises.}

\begin{footnotesize}
The following exercises are to be solved when the pupil has
mastered the First Book:---

1.~If the lines $AF$, $BF$ be joined, the figure $ACBF$ is a lozenge.

2.~If $AB$ be produced to $D$ and $E$, the triangles $CDF$ and
$CEF$ are equilateral.

3.~If $CA$, $CB$ be produced to meet the circles again in $G$ and $H$,
the points $G$, $F$, $H$ are collinear, and the triangle $GCH$ is equilateral.

4.~If $CF$ be joined, $CF^{2} = 3AB^{2}$.

5.~Describe a circle in the space $ACB$, bounded by the line $AB$
and the two circles.
\par\end{footnotesize}


\myprop{PROP\@.~II\@.---Problem.}{From a given point $(A)$ to draw a right line equal to
a given finite right line $(BC)$.}


\imgflow{150}{13}{f013}
\textbf{Sol.}---Join $AB$ (Post.~\textsc{i.}); on $AB$ describe the equilateral
triangle $ABD$
[\textsc{i.}]. With $B$ as centre,
and $BC$ as radius, describe
the circle $ECH$
(Post~\textsc{iii.}). Produce
$DB$ to meet the circle
$ECH$ in $E$ (Post.~\textsc{ii.}).
With $D$ as centre, and
$DE$ as radius, describe
the circle $EFG$ (Post.~\textsc{iii.}).
Produce $DA$ to
meet this circle in $F$.
$AF$ \emph{is equal to} $BC$.

\textbf{Dem.}---Because $D$ is
the centre of the circle $EFG$, $DF$ is equal to $DE$ (Def.\

\textsc{xxxii.}). And because $DAB$ is an equilateral triangle,
$DA$ is equal to $DB$ (Def.~\textsc{xxi.}). Hence we have
\begin{flalign*}
&&               & DF = DE,  &&\\
& \text{and } &  & DA = DB;  &&\phantom{and }
\end{flalign*}
and taking the latter from the former, the remainder
$AF$ is equal to the remainder $BE$ (Axiom~\textsc{iii.}). Again,
because $B$ is the centre of the circle $ECH$, $BC$ is equal
to $BE$; and we have proved that $AF$ is equal to $BE$;
and things which are equal to the same thing are equal
to one another (Axiom~\textsc{i.}). Hence $AF$ is equal to $BC$.
\emph{Therefore from the given point $A$ the line $AF$ has been
drawn equal to $BC$}.

\begin{footnotesize}
It is usual with commentators on Euclid to say that he allows
the use of the \emph{rule} and \emph{compass}. Were such the case this Proposition
would have been unnecessary. The fact is, Euclid's object
was to teach Theoretical and not Practical Geometry, and the only
things he postulates are the drawing of right lines and the describing
of circles. If he allowed the mechanical use of the rule and
compass he could give methods of solving many problems that
go beyond the limits of the ``geometry of the point, line, and
circle.''---\textit{See} Notes D, F at the end of this work.
\par\end{footnotesize}

\exhead{Exercises.}

\begin{footnotesize}
1.~Solve the problem when the point $A$ is in the line $BC$ itself.

2.~Inflect from a given point $A$ to a given line $BC$ a line equal
to a given line. State the number of solutions.
\par\end{footnotesize}



\myprop{PROP\@. III\@.---Problem.}{From the greater $(AB)$ of two given right lines to cut off
a part equal to $(C)$ the less.}


\imgflow{133}{10}{f014}
\textbf{Sol.}---From $A$, one of the extremities of $AB$, draw
the right line $AD$ equal to $C$
[\textsc{ii.}]; and with $A$ as centre,
and $AD$ as radius, describe
the circle $EDF$ (Post.~\textsc{iii.})
cutting $AB$ in $E$. $AE$ shall
be equal to $C$.

\textbf{Dem.}---Because $A$ is the
centre of the circle $EDF$, $AE$
is equal to $AD$ (Def.~\textsc{xxxii.}),
and $C$ is equal to $AD$ (const.); and things which are
equal to the same are equal to one another (Axiom~\textsc{i.});
therefore $AE$ is equal to $C$. \emph{Wherefore from $AB$, the
greater of the two given lines, a part, $AE$, has been out off
equal to $C$, the less}.


\exhead{Questions for Examination.}

\begin{footnotesize}
1.~What previous problem is employed in the solution of this?

2.~What postulate?

3.~What axiom in the demonstration?

4.~Show how to produce the less of two given lines until the
whole produced line becomes equal to the greater.
\par\end{footnotesize}


\mypropl{PROP\@.~IV\@.---Theorem.}{If two triangles $(BAC,\ EDF)$ have two sides $(BA,\ AC)$
of one equal respectively to two sides $(ED,\ DF)$ of the
other, and have also the angles $(A,\ D)$ included by those
sides equal, the triangles shall be equal in every respect---that
is, their bases or third sides $(BC,\ EF)$ shall be equal,
and the angles $(B,\ C)$ at the base of one shall be respectively
equal to the angles $(E,\ F)$ at the base of the other;

namely, those shall be equal to which the equal sides are
opposite.}


\imgflow{163}{8}{f015}

\textbf{Dem.}---Let us conceive the triangle $BAC$ to be applied
to $EDF$, so that the
point $A$ shall coincide with
$D$, and the line $AB$ with
$DE$, and that the point $C$
shall be on the same side
of $DE$ as $F$; then because
$AB$ is equal to $DE$, the
point $B$ shall coincide
with $E$. Again, because the angle $BAC$ is equal to
the angle $EDF$, the line $AC$ shall coincide with $DF$;
and since $AC$ is equal to $DF$ (hyp.), the point $C$ shall
coincide with $F$; and we have proved that the point
$B$ coincides with $E$. Hence two points of the line $BC$
coincide with two points of the line $EF$; and since two
right lines cannot enclose a space, $BC$ must coincide
with $EF$. Hence the triangles agree in every respect;
\textit{therefore $BC$ is equal to $EF$, the angle $B$ is equal to the
angle $E$, the angle $C$ to the angle $F$, and the triangle $BAC$
to the triangle $EDF$.}


\exhead{Questions for Examination.}

\begin{footnotesize}
1.~How many parts in the hypothesis of this Proposition? \textit{Ans.}\
Three. Name them.

2.~How many in the conclusion? Name them.

3.~What technical term is applied to figures which agree in
everything but position? \textit{Ans.}\ They are said to be congruent.

4.~What is meant by superposition?

5.~What axiom is made use of in superposition?

6.~How many parts in a triangle? \textit{Ans.}\ Six; namely, three
sides and three angles.

7.~When it is required to prove that two triangles are congruent,
how many parts of one must be given equal to corresponding
parts of the other? \textit{Ans.}\ In general, any three except
the three angles. This will be established in Props.~\textsc{viii.}\ and
\textsc{xxvi.}, taken along with \textsc{iv.}

8.~What property of two lines having two common points is
quoted in this Proposition? They must coincide.
\par\end{footnotesize}


\exhead{Exercises.}

\begin{footnotesize}
1.~The line that bisects the vertical angle of an isosceles triangle
bisects the base perpendicularly.

2.~If two adjacent sides of a quadrilateral be equal, and the
diagonal bisects the angle between them, their other sides are
equal.

3.~If two lines be at right angles, and if each bisect the other,
then any point in either is equally distant from the extremities of
the other.

4.~If equilateral triangles be described on the sides of any
triangle, the distances between the vertices of the original triangle
and the opposite vertices of the equilateral triangles are equal.
(This Proposition should be proved after the student has read
Prop.~\textsc{xxxii.})
\par\end{footnotesize}


\mypropl{PROP\@.~V.---Theorem.}{The angles $(ABC,\ ACB)$ at the base $(BC)$ of an isosceles
triangle are equal to one another, and if the equal sides
$(AB,\ AC)$ be produced, the external angles $(DEC,\ ECB)$
below the base shall be equal.}


\imgflow{128}{12}{f016}

\textbf{Dem.}---In $BD$ take any point $F$, and from $AE$, the
greater, cut off $AG$ equal to
$AF$ [\textsc{iii}]. Join $BG$, $CF$
(Post.~\textsc{i.}). Because $AF$ is
equal to $AG$ (const.), and $AC$
is equal to $AB$ (hyp.), the
two triangles $FAC$, $GAB$
have the sides $FA$, $AC$ in
one respectively equal to the
sides $GA$, $AB$ in the other;
and the included angle $A$ is
common to both triangles.
Hence [\textsc{iv.}] the base $FC$ is
equal to $GB$, the angle $AFC$ is equal to $AGB$, and the
angle $ACF$ is equal to the angle $ABG$.

Again, because $AF$ is equal to $AG$ (const.), and $AB$
to $AC$ (hyp.), the remainder, $BF$, is equal to $CG$ (Axiom
\textsc{iii}); and we have proved that $FC$ is equal to $GB$, and

the angle $BFC$ equal to the angle $CGB$. Hence the
two triangles $BFC$, $CGB$ have the two sides $BF$, $FC$ in
one equal to the two sides $CG$, $GB$ in the other; and the
angle $BFC$ contained by the two sides of one equal to the
angle $CGB$ contained by the two sides of the other.
Therefore [\textsc{iv.}] these triangles have the angle $FBC$
equal to the angle $GCB$, \emph{and these are the angles below
the base}. Also the angle $FCB$ equal to $GBC$; but the
whole angle $FCA$ has been proved equal to the whole
angle $GBA$. Hence the remaining angle $ACB$ is equal
to the remaining angle $ABC$, \emph{and these are the angles
at the base}.

\begin{footnotesize}

\imgflow{145}{9}{f017}

\smallskip
\textbf{Observation.}---The great difficulty which beginners find in this
Proposition is due to the fact
that the two triangles $ACF$,
$ABG$ overlap each other.
The teacher should make
these triangles separate, as
in the annexed diagram, and
point out the corresponding
parts thus:---
\begin{align*}
  AF & = AG,  \\
  AC & = AB;  \\
\text{angle } FAC &= \text{angle } GAB.
\end{align*}

\begin{flalign*}
& \text{\indent Hence [\textsc{iv.}],} &
  \text{angle } ACF &= \text{angle } ABG.  &&\\
& \text{and } &
  \text{angle } AFC &= \text{angle } AGB.  &&
\end{flalign*}

The student should also be shown how to apply one of the
triangles to the other, so as to bring them into coincidence. Similar
Illustrations may be given of the triangles $BFC$, $CGB$.


\imgflow{110}{10}{f018}

The following is a very easy proof of this Proposition. Conceive
the $\triangle~ACB$ to be turned, without
alteration, round the line $AC$,
until it falls on the other side. Let
$ACD$ be its new position; then the
angle $ADC$ of the displaced triangle is
evidently equal to the angle $ABC$, with
which it originally coincided. Again,
the two $\triangle s~BAC$, $CAD$ have the sides
$BA$, $AC$ of one respectively equal to
the sides $AC$, $AD$ of the other, and
the included angles equal; therefore [\textsc{iv.}] the angle $ACB$ opposite

to the side $AB$ is equal to the angle $ADC$ opposite to the
side $AC$; but the angle $ADC$ is equal to $ABC$; therefore $ACB$
is equal to $ABC$.
\par\end{footnotesize}

\emph{Cor.}---Every equilateral triangle is equiangular.

\textsc{Def.}---\emph{A line in any figure, such as $AC$ in the preceding
diagram, which is such that, by folding the plane of the
figure round it, one part of the diagram will coincide with
the other, is called an} \textsc{axis of symmetry}\index{Axis!of symmetry} \emph{of the figure.}

\exhead{Exercises.}

\begin{footnotesize}
1.~Prove that the angles at the base are equal without producing
the sides. Also by producing the sides through the vertex.

2.~Prove that the line joining the point $A$ to the intersection of
the lines $CF$ and $BG$ is an axis of symmetry of the figure.

3.~If two isosceles triangles be on the same base, and be either
at the same or at opposite sides of it, the line joining their vertices
is an axis of symmetry of the figure formed by them.

4.~Show how to prove this Proposition by assuming as an
axiom that every angle has a bisector.

5.~Each diagonal of a lozenge is an axis of symmetry of the
lozenge.

6.~If three points be taken on the sides of an equilateral triangle,
namely, one on each side, at equal distances from the
angles, the lines joining them form a new equilateral triangle.
\par\end{footnotesize}


\myprop{PROP\@.~VI\@.---Theorem.}{If two angles $(B,\ C)$ of a triangle be equal, the sides
$(AC,\ AB)$ opposite to them are also equal.}


\imgflow{79}{8}{f019}

\textbf{Dem.}---If $AB$, $AC$ are not equal, one must be greater
than the other. Suppose $AB$ is the
greater, and that the part $BD$ is equal
to $AC$. Join $CD$ (Post.~\textsc{i.}). Then the
two triangles $DBC$, $ACB$ have $BD$ equal
to $AC$, and $BC$ common to both. Therefore
the two sides $DB$, $BC$ in one are
equal to the two sides $AC$, $CB$ in the
other; and the angle $DBC$ in one is
equal to the angle $ACB$ in the other (hyp). Therefore

[\textsc{iv.}] the triangle $DBC$ is equal to the triangle
$ACB$---the less to the greater, which is absurd; \emph{hence
$AC$, $AB$ are not unequal, that is, they are equal}.


\exhead{Questions for Examination.}

\begin{footnotesize}
1.~What is the hypothesis in this Proposition?

2.~What Proposition is this the converse of?

3.~What is the obverse of this Proposition?

4.~What is the obverse of Prop.~\textsc{v.}?

5.~What is meant by an indirect proof?

6.~How does Euclid generally prove converse Propositions?

7.~What false assumption is made in the demonstration?

8.~What does this assumption lead to?
\par\end{footnotesize}


\mypropl{PROP\@.~VII---Theorem.}{If two triangles $(ACB,\ ADB)$ on the same base $(AB)$
and on the same side of it have one pair of conterminous
sides $(AC,\ AD)$ equal to one another, the other pair of
conterminous sides $(BC,\ BD)$ must be unequal.}


\imgflow{80}{8}{f020}

\textbf{Dem.}---1. Let the vertex of each triangle be without
the other. Join $CD$. Then because
$AD$ is equal to $AC$ (hyp.), the triangle
$ACD$ is isosceles; therefore [\textsc{v.}] the
angle $ACD$ is equal to the angle $ADC$;
but $ADC$ is greater than $BDC$ (Axiom
\textsc{ix.}); therefore $ACD$ is greater than
$BDC$: much, more is $BCD$ greater
than $BDC$. Now if the side $BD$ were
equal to $BC$, the angle $BCD$ would be
equal to $BDC$ [\textsc{v.}]; but it has been proved to be
greater. \emph{Hence $BD$ is not equal to $BC$}.


\imgflow{160}{10}{f021}

2.~Let the vertex of one triangle $ADB$ fall within the
other triangle $ACB$. Produce the sides $AC$, $AD$ to $E$ and $F$.

Then because $AC$ is equal to $AD$ (hyp.), the triangle $ACD$
is isosceles, and [\textsc{v.}] the
external angles $ECD$,
$FDC$ at the other side of
the base $CD$ are equal;
but $ECD$ is greater than
$BCD$ (Axiom~\textsc{ix.}). Therefore
$FDC$ is greater than
$BCD$: much more is
$BDC$ greater than $BCD$;
but if $BC$ were equal
to $BD$, the angle $BDC$ would be equal to $BCD$ [\textsc{v.}];
\textit{therefore $BC$ cannot be equal to $BD$}.

3.~If the vertex $D$ of the second triangle fall on the
line $BC$, it is evident that $BC$ and $BD$ are unequal.


\exhead{Questions for Examination.}

\begin{footnotesize}
1.~What use is made of Prop.~\textsc{vii.}? \textit{Ans.}\ As a lemma to
Prop.~\textsc{viii.}

2.~In the demonstration of Prop.~\textsc{vii.}\ the contrapositive of
Prop.~\textsc{v.}\ occurs; show where.

3.~Show that two circles can intersect each other only in one
point on the same side of the line joining their centres, and hence
that two circles cannot have more than two points of intersection.
\par\end{footnotesize}



\mypropflow{PROP\@.~VIII\@.---Theorem.}{If two triangles $(ABC,\ DEF)$ have two sides $(AB,\ AC)$
of one respectively equal to two sides $(DE,\ DF)$ of the
other, and have also the base $(BC)$ of one equal to the base
$(EF)$ of the other; then the two triangles shall be equal,
and the angles of one shall be respectively equal to the
angles of the other---namely, those shall be equal to which
the equal sides are opposite.}{160}{8}{f022}

\textbf{Dem.}---Let the triangle $ABC$ be applied to $DEF$,
so that the point $B$ will coincide with $E$, and the
line $BC$ with the line $EF$; then because $BC$ is equal

to $EF$, the point $C$ shall coincide with $F$. Then if the
vertex $A$ fall on the same side of $EF$ as the vertex $D$,
the point $A$ must coincide
with $D$; for if not,
let it take a different
position $G$; then we have
$EG$ equal to $BA$, and $BA$
is equal to $ED$ (hyp.).
Hence (Axiom~\textsc{i.}) $EG$ is
equal to $ED$: in like
manner, $FG$ is equal to $FD$, and this is impossible [\textsc{vii.}].
Hence the point $A$ must coincide with $D$, and the
triangle $ABC$ agrees in every respect with the triangle
$DEF$; \emph{and therefore the three angles of one are respectively
equal to the three angles of the other---namely, $A$ to
$D$, $B$ to $E$, and $C$ to $F$, and the two triangles are equal.}

This Proposition is the converse of \textsc{iv.}, and is the
second case of the congruence of triangles in the
Elements.


\imgcent{177}{f023}

\begin{footnotesize}
\emph{Philo's Proof.}\index{Philo, proofs by}---Let the equal bases be applied as in the foregoing
proof, but let the vertices be on the opposite sides; then let
$BGC$ be the position which $EDF$ takes. Join $AG$. Then because
$BG = BA$, the angle $BAG = BGA$. In like manner the angle $CAG
= CGA$. Hence the whole angle $BAC = BGC$; but $BGC = EDF$
therefore $BAC = EDF$.
\par\end{footnotesize}



\myprop{PROP\@. IX\@.---Problem.}{To bisect a given rectilineal angle $(BAC)$.}


\imgflow{120}{13}{f024}

\textbf{Sol.}---In $AB$ take any point $D$, and cut off [\textsc{iii.}] $AE$
equal to $AD$. Join $DE$ (Post.~\textsc{i.}),
and upon it, on the side remote
from $A$, describe the equilateral
triangle $DEF$ [\textsc{i.}] Join
$AF$. \emph{$AF$ bisects the given angle
$BAC$.}

\textbf{Dem.}---The triangles $DAF$,
$EAF$ have the side $AD$ equal
to $AE$ (const.) and $AF$ common;
therefore the two sides $DA$, $AF$
are respectively equal to $EA$,
$AF$, and the base $DF$ is equal to
the base $EF$, because they are the sides of an equilateral
triangle (Def.~\textsc{xxi.}). Therefore [\textsc{viii.}] the
angle $DAF$ is equal to the angle $EAF$; \emph{hence the angle
$BAC$ is bisected by the line $AF$}.

\emph{Cor.}---The line $AF$ is an axis of symmetry of the
figure.

\exhead{Questions for Examination.}

\begin{footnotesize}
1.~Why does Euclid describe the equilateral triangle on the
side remote from $A$?

2.~In what case would the construction fail, if the equilateral
triangle were described on the other side of $DE$?
\par\end{footnotesize}

\exhead{Exercises.}

\begin{footnotesize}
1.~Prove this Proposition without using Prop.~\textsc{viii.}

2.~Prove that $AF$ is perpendicular to $DE$.

3.~Prove that any point in $AF$ is equally distant from the
points $D$ and $E$.

4.~Prove that any point in $AF$ is equally distant from the lines
$AB$, $AC$.
\par\end{footnotesize}




\myprop{PROP\@.~X.---Problem.}{To bisect a given finite right line $(AB)$.}


\imgflow{100}{9}{f025}

\textbf{Sol.}---Upon $AB$ describe an equilateral triangle $ACB$
[\textsc{i.}]. 
Bisect the angle\index{Angle!bisection of} $ACB$ by the
line $CD$ [\textsc{ix.}], meeting $AB$ in $D$, \emph{then
$AB$ is bisected in $D$}.

\textbf{Dem.}---The two triangles $ACD$,
$BCD$, have the side $AC$ equal to $BC$,
being the sides of an equilateral
triangle, and $CD$ common. Therefore
the two sides $AC$, $CD$ in one
are equal to the two sides $BC$, $CD$ in the other; and the
angle $ACD$ is equal to the angle $BCD$ (const.). Therefore
the base $AD$ is equal to the base $DB$ [\textsc{iv.}]. \emph{Hence
$AB$ is bisected in $D$}.

\exhead{Exercises.}

\begin{footnotesize}
1.~Show how to bisect a finite right line by describing two
circles.

2.~Every point equally distant from the points $A$, $B$ is in the
line $CD$.
\par\end{footnotesize}



\mypropflow{PROP\@.~XI\@.---Problem.}{From a given point $(C)$ in a given right line $(AB)$ to draw
a right line perpendicular to the given line.}{190}{8}{f026}

\textbf{Sol.}---In $AC$ take any point $D$, and make $CE$ equal
to $CD$ [\textsc{iii.}]. Upon
$DE$ describe an
equilateral triangle
$DFE$ [\textsc{i.}]. Join $CF$.
\emph{Then $CF$ shall be at
right angles to $AB$.}

\textbf{Dem.}---The two
triangles $DCF$, $ECF$
have $CD$ equal to $CE$ (const.) and $CF$ common; therefore
the two sides $CD$, $CF$ in one are respectively equal

to the two sides $CE$, $CF$ in the other, and the base $DF$
is equal to the base $EF$, being the sides of an equilateral
triangle (Def.~\textsc{xxi}.); therefore [\textsc{viii}.] the angle
$DCF$ is equal to the angle $ECF$, and they are adjacent 
angles. Therefore (Def.~\textsc{xiii}.) each of them is a right
angle, \emph{and $CF$ is perpendicular to $AB$ at the point $C$}.

\exhead{Exercises.}

\begin{footnotesize}
1.~The diagonals of a lozenge bisect each other perpendicularly.\index{Diagonal!middle points of}

2.~Prove Prop.~\textsc{xi.} without using Prop.~\textsc{viii}.

3.~Erect a line at right angles to a given line at one of its extremities
without producing the line.

4.~Find a point in a given line that shall be equally distant
from two given points.

5.~Find a point in a given line such that, if it be joined to two
given points on opposite sides of the line, the angle formed by the
joining lines shall be bisected by the given line.

6.~Find a point that shall be equidistant from three given
points.
\par\end{footnotesize}


\myprop{PROP\@. XII\@.---Problem.}{To draw a perpendicular to a given indefinite right line
$(AB)$ from a given point $(C)$ without it.}


\imgflow{180}{8}{f027}

\textbf{Sol.}---Take any point $D$ on the other side of $AB$, and
describe (Post.~\textsc{iii}.)
a circle, with $C$ as
centre, and $CD$ as
radius, meeting $AB$
in the points $F$ and $G$.
Bisect $FG$ in $H$ [\textsc{x}.].
Join $CH$ (Post.~\textsc{i}.).
$CH$ \textit{shall be at right
angles to $AB$.}

\textbf{Dem.}---Join $CF$, $CG$. Then the two triangles $FHC$,
$GHC$ have $FH$ equal to $GH$ (const.), and $HC$ common;

and the base $CF$ equal to the base $CG$, being radii of
the circle $FDG$ (Def.~\textsc{xxxii}.). Therefore the angle $CHF$
is equal to the angle $CHG$ [\textsc{viii}.], and, being adjacent
angles, they are right angles (Def.~\textsc{xiii}.). \emph{Therefore
$CH$ is perpendicular to $AB$}.


\exhead{Exercises.}

\begin{footnotesize}
1.~Prove that the circle cannot meet $AB$ in more than two
points.

2.~If one angle of a triangle be equal to the sum of the other two,
the triangle can be divided into the sum of two isosceles triangles,
and the base is equal to twice the line from its middle point to the
opposite angle.
\par\end{footnotesize}


\mypropl{PROP\@. XIII\@.---Theorem.}{The adjacent angles $(ABC,\ ABD)$ which one right line
$(AB)$ standing on another $(CD)$ makes with it are either
both right angles, or their sum is equal to two right
angles.}


\imgcent{230}{f028}

\textbf{Dem.}---If $AB$ is perpendicular to $CD$, as in fig.~1,
the angles $ABC$, $ABD$ are right angles. If not, draw
$BE$ perpendicular to $CD$ [\textsc{xi}.]. Now the angle $CBA$ is
equal to the sum of the two angles $CBE$, $EBA$ (Def.~\textsc{xi}.).
Hence, adding the angle $ABD$, the sum of the angles

$CBA$, $ABD$ is equal to the sum of the three angles $CBE$,
$EBA$, $ABD$. In like manner, the sum of the angles
$CBE$, $EBD$ is equal to the sum of the three angles
$CBE$, $EBA$, $ABD$. And things which are equal to the
same are equal to one another. Therefore the sum of
the angles $CBA$, $ABD$ is equal to the sum of the angles
$CBE$, $EBD$; but $CBE$, $EBD$ are right angles; \emph{therefore
the sum of the angles $CBA$, $ABD$ is two right angles.}

\smallskip
\begin{footnotesize}
\emph{Or thus:} Denote the angle $EBA$ by $\theta$; then evidently
\\[1ex]
\begin{tabular}{@{}l@{\hspace{3em}}r@{\ }l}
the angle  &   $CBA$ & $=$ right angle${} + \theta$; \\[1ex]
the angle  &   $ABD$ & $=$ right angle${} - \theta$; \\[1ex]
therefore  & $CBA + ABD$ & $=$ two right angles.
\end{tabular}
\end{footnotesize}

\smallskip
\emph{Cor.}~1.---The sum of two supplemental angles is two
right angles.

\emph{Cor.}~2.---Two right lines cannot have a common
segment.

\emph{Cor.}~3.---The bisector of any angle bisects the corresponding
re-entrant angle.

\emph{Cor.}~4.---The bisectors of two supplemental angles
are at right angles to each other.

\emph{Cor.}~5.---The angle $EBA$ is half the difference of the
angles $CBA$, $ABD$.



\mypropflow{PROP\@. XIV\@.--Theorem.}{If at a point $(B)$ in a right line $(BA)$ two other right
lines $(CB,\ BD)$ on opposite sides make the adjacent angles
$(CBA,\ ABD)$ together equal to two right angles, these
two right lines form one continuous line.}{140}{9}{f029}

\textbf{Dem.}---If $BD$ be not the continuation of $CB$, let $BE$
be its continuation. Now, since $CBE$ is a right line,
and $BA$ stands on it, the sum of the angles $CBA$, $ABE$

is two right angles (\textsc{xiii}.); and the sum of the angles
$CBA$, $ABD$ is two right angles
(hyp.); therefore the sum of
the angles $CBA$, $ABE$ is equal
to the sum of the angles
$CBA$, $ABD$. Reject the angle
$CBA$, which is common, and
we have the angle $ABE$ equal
to the angle $ABD$---that is,
a part equal to the whole---which
is absurd. \textit{Hence $BD$ must be in the same right
line with $CB$.}


\myprop{PROP\@.~XV\@.---Theorem.}{If two right lines $(AB,\ CD)$ intersect one another,
the opposite angles are equal $(CEA = DEB$, and $BEC = AED)$.}


\imgflow{120}{8}{f030}

\textbf{Dem.}---Because the line $AE$ stands on $CD$, the sum
of the angles $CEA$, $AED$ is two
right angles [\textsc{xiii}.]; and because
the line $CE$ stands on $AB$,
the sum of the angles $BEC$,
$CEA$ is two right angles; therefore
the sum of the angles $CEA$,
$AED$ is equal to the sum of the
angles $BEC$, $CEA$. Reject the
angle $CEA$, which is common, and we have \textit{the angle
$AED$ equal to $BEC$}. In like manner, \textit{the angle $CEA$
is equal to $DEB$}.

The foregoing proof may be briefly given, by saying
that opposite angles are equal because they have a
common supplement.

\exhead{Questions for Examination on Props. XIII\@., XIV\@., XV\@.}

\begin{footnotesize}
1.~What problem is required in Euclid's proof of Prop.~\textsc{xiii}.?

2.~What theorem? \textit{Ans.}\ No theorem, only the axioms.

3.~If two lines intersect, how many pairs of supplemental
angles do they make?



4.~What relation does Prop.~\textsc{xiv}.~bear to Prop.~\textsc{xiii}.?

5.~What three lines in Prop.~\textsc{xiv}.\ are concurrent?

6.~What caution is required in the enunciation of Prop.~\textsc{xiv}.?

7.~State the converse of Prop.~\textsc{xv}. Prove it.

8.~What is the subject of Props.~\textsc{xiii., xiv., xv.}? \textit{Ans.}\ Angles
at a point.
\par\end{footnotesize}



\mypropflot{PROP\@.~XVI\@.---Theorem.}{If any side $(BC)$ of a triangle $(ABC)$ be produced, the
exterior angle $(ACD)$ is greater than either of the interior\index{Angle!interior}
non-adjacent angles.}{110}{11}{f031}

\textbf{Dem.}---Bisect $AC$ in $E$ [\textsc{x}.]. Join $BE$ (Post.~\textsc{i}.).
Produce it, and from the produced
part cut off $EF$ equal to $BE$ [\textsc{iii}].
Join $CF$. Now because $EC$ is equal
to $EA$ (const.), and $EF$ is equal
to $EB$, the triangles $CEF$, $AEB$
have the sides $CE$, $EF$ in one
equal to the sides $AE$, $EB$ in the
other; and the angle $CEF$ equal
to $AEB$ [\textsc{xv}.]. Therefore [\textsc{iv}.]
the angle $ECF$ is equal to $EAB$;
but the angle $ACD$ is greater
than $ECF$; therefore the angle
$ACD$ is greater than $EAB$.

In like manner it may be shown, if the side $AC$ be
produced, that the exterior angle $BCG$ is greater than
the angle $ABC$; but $BCG$ is equal to $ACD$ [\textsc{xv}.]. Hence
$ACD$ is greater than $ABC$. \emph{Therefore $ACD$ is greater
than either of the interior non-adjacent angles $A$ or $B$ of
the triangle $ABC$.}

\textit{Cor.~1.}---The sum of the three interior angles of the
triangle $BCF$ is equal to the sum of the three interior
angles of the triangle $ABC$.

\textit{Cor.~2.}---The area of $BCF$ is equal to the area of
$ABC$.



\textit{Cor.~3.}---The lines $BA$ and $CF$, if produced, cannot
meet at any finite distance. For, if they met at any
finite point $X$, the triangle $CAX$ would have an exterior
angle $BAC$ equal to the interior angle $ACX$.


\myprop{PROP\@.~XVII\@.---Theorem.}{Any two angles $(B,\ C)$ of a triangle $(ABC)$ are together
less than two right angles.}


\imgflow{105}{7}{f032}

\textbf{Dem.}---Produce $BC$ to $D$; then the exterior angle
$ACD$ is greater than $ABC$ [\textsc{xvi.}]:
to each add the angle $ACB$, and
we have the sum of the angles
$ACD$, $ACB$ greater than the sum
of the angles $ABC$, $ACB$; but the
sum of the angles $ACD$, $ACB$ is
two right angles [\textsc{xiii.}]. \textit{Therefore
the sum of the angles $ABC$,
$ACB$ is less than two right angles.}

In like manner we may show that the sum of the
angles $A$, $B$, or of the angles $A$, $C$, is less than two
right angles.

\textit{Cor.~1.}---Every triangle must have at least two
acute angles.

\textit{Cor.~2.}---If two angles of a triangle be unequal, the
lesser must be acute.

\exhead{Exercise.}

\begin{footnotesize}
Prove Prop.~\textsc{xvii.}\ without producing a side.
\par\end{footnotesize}



\mypropl{PROP\@.~XVIII\@.---Theorem.}{If in any triangle $(ABC)$ one side $(AC)$ be greater than
another $(AB)$, the angle opposite to the greater side is
greater than the angle opposite to the less.}


\imgflow{133}{7}{f033}

\textbf{Dem.}---From $AC$ cut off $AD$ equal to $AB$ [\textsc{iii}].
Join $BD$ (Post.~\textsc{i.}). Now
since $AB$ is equal to $AD$, the
triangle $ABD$ is isosceles;
therefore [\textsc{v.}] the angle $ADB$
is equal to $ABD$; but the angle
$ADB$ is greater than the angle
$ACB$ [\textsc{xvi.}]; therefore $ABD$
is greater than $ACB$.  \textit{Much more is the angle $ABC$
greater than the angle $ACB$.}


\imgflow{133}{6}{f034}

\textit{Or thus:} From $A$ as centre,
with the lesser side $AB$ as radius,
describe the circle $BED$,
cutting $BC$ in $E$. Join $AE$.
Now since $AB$ is equal to $AE$,
the angle $AEB$ is equal to
$ABE$; but $AEB$ is greater than
$ACB$ (\textsc{xvi}.); \textit{therefore $ABE$ is greater than $ACB$.}

\exhead{Exercises.}

\begin{footnotesize}
1.~If in the second method the circle cut the line $CB$ produced
through $B$, prove the Proposition.

2.~This Proposition may be proved by producing the less side.

3.~If two of the opposite sides of a quadrilateral be respectively
the greatest and least, the angles adjacent to the least are greater
than their opposite angles.

4.~In any triangle, the perpendicular from the vertex opposite
the side which is not less than either of the remaining sides falls
within the triangle.
\par\end{footnotesize}



\mypropl{PROP\@.~XIX\@.---Theorem.}{If one angle $(B)$ of a triangle $(ABC)$ be greater than
another angle $(C)$, the side $(AC)$ which it opposite to the
greater angle is greater than the side $(AB)$ which is opposite
to the less.}


\imgflow{115}{8}{f035}

\textbf{Dem.}---If $AC$ be not greater than $AB$, it must be
either equal to it or less than it.
Let us examine each case:---

1.~If $AC$ were equal to $AB$,
the triangle $ACB$ would be
isosceles, and then the angle $B$
would be equal to $C$ [\textsc{v}.]; but it
is not by hypothesis; therefore
$AB$ is not equal to $AC$.

2.~If $AC$ were less than $AB$, the angle $B$ would be
less than the angle $C$ [\textsc{xviii}.]; but it is not by hypothesis;
therefore $AC$ is not less than $AB$; and since $AC$ is
neither equal to $AB$ nor less than it, \emph{it must be greater}.

\exhead{Exercises.}

\begin{footnotesize}
1.~Prove this Proposition by a direct demonstration.

2.~A line from the vertex of an isosceles triangle to any point
in the base is less than either of the equal sides, but greater if the
point be in the base produced.

3.~Three equal lines could not be drawn from the same point
to the same line.

4.~The perpendicular is the least line which can be drawn from
a given point to a given line; and of all others that may be
drawn to it, that which is nearest to the perpendicular is less
than any one more remote.

5.~If in the fig., Prop.~\textsc{xvi}., $AB$ be the greatest side of the
$\triangle~ABC$, $BF$ is the greatest side of the $\triangle~FBC$, and the angle $BFC$
is less than half the angle $ABC$.

6.~If $ABC$ be a $\triangle$ having $AB$ not greater than $AC$, a line $AG$,
drawn from $A$ to any point $G$ in $BC$, is less than $AC$. For the
angle $ACB$ [\textsc{xviii}.] is not greater than $ABC$; but $AGC$ [\textsc{xvi}.] is
greater than $ABC$; therefore $AGC$ is greater than $ACG$. \emph{Hence
$AC$ is greater than $AG$}.
\par\end{footnotesize}



\myprop{PROP\@. XX\@.---Theorem.}{The sum of any two sides $(BA,\ AC)$ of a triangle $(ABC)$
is greater than the third.}


\imgflow{150}{10}{f036}

\textbf{Dem.}---Produce $BA$ to $D$ (Post.~\textsc{ii}.), and make $AD$
equal to $AC$ [\textsc{iii}.]. Join
$CD$. Then because $AD$ is
equal to $AC$, the angle
$ACD$ is equal to $ADC$ (\textsc{v}.);
therefore the angle $BCD$
is greater than the angle
$BDC$; hence the side $BD$
opposite to the greater angle
is greater than $BC$ opposite
to the less [\textsc{xix}.]. Again,
since $AC$ is equal to $AD$,
adding $BA$ to both, we have the sum of the sides $BA$,
$AC$ equal to $BD$. \emph{Therefore the sum of $BA$, $AC$ is
greater than $BC$.}

\begin{footnotesize}
\emph{Or thus}: Bisect the angle $BAC$ by $AE$ [\textsc{ix}.] Then the angle
$BEA$ is greater than $EAC$; but $EAC = EAB$ (const.); therefore
the angle $BEA$ is greater than $EAB$. Hence $AB$ is greater than
$BE$ [\textsc{xix}.]. In like manner $AC$ is greater than $EC$. \emph{Therefore
the sum of $BA$, $AC$ is greater than $BC$.}
\par\end{footnotesize}

\exhead{Exercises.}

\begin{footnotesize}
1.~In any triangle, the difference between any two sides is less
than the third.

2.~If any point within a triangle be joined to its angular points,
the sum of the joining lines is greater than its semiperimeter.

3.~If through the extremities of the base of a triangle, whose
sides are unequal, lines be drawn to any point in the bisector of
the vertical angle, their difference is less than the difference of
the sides.

4.~If the lines be drawn to any point in the bisector of the
external vertical angle, their sum is greater than the sum of the
sides.


5.~Any side of any polygon is less than the sum of the remaining
sides.

6.~The perimeter of any triangle is greater than that of any
inscribed triangle, and less than that of any circumscribed triangle.

7.~The perimeter of any polygon is greater than that of any
inscribed, and less than that of any circumscribed, polygon of the
same number of sides.

8.~The perimeter of a quadrilateral is greater than the sum of
its diagonals.

\textbf{Def.}---\emph{A line drawn from any angle of a triangle to the middle
point of the opposite side is called a median\index{Median} of the triangle.}

9.~The sum of the three medians of a triangle is less than its
perimeter.

10.~The sum of the diagonals of a quadrilateral is less than the
sum of the lines which can be drawn to its angular points from
any point except the intersection of the diagonals.
\par\end{footnotesize}


\mypropl{PROP\@. XXI\@.---Theorem.}{If two lines $(BD,\ CD)$ be drawn to a point $(D)$ within
a triangle from the extremities of its base $(BC)$, their sum
is less than the sum of the remaining sides $(BA,\ CA)$, but
they contain a greater angle.}


\imgflow{125}{8}{f037}

\textbf{Dem.}---1. Produce $BD$ (Post.~\textsc{ii}.) to meet $AC$ in $E$.
Then, in the triangle $BAE$, the
sum of the sides $BA$, $AE$ is
greater than the side $BE$ [\textsc{xx}.]:
to each add $EC$, and we have
the sum of $BA$, $AC$ greater
than the sum of $BE$, $EC$. Again,
the sum of the sides $DE$, $EC$ of
the triangle $DEC$ is greater
than $DC$: to each add $BD$, and
we get the sum of $BE$, $EC$ greater than the sum of $BD$,
$DC$; but it has been proved that the sum of $BA$, $AC$ is
greater than the sum of $BE$, $EC$. \emph{Therefore much
more is the sum of $BA$, $AC$ greater than the sum of
$BD$, $DC$.}



2.~The external angle $BDC$ of the triangle $DEC$ is
greater than the internal angle $BEC$ [\textsc{xvi}.], and the angle
$BEC$, for a like reason, is greater than $BAC$. \emph{Therefore
much more is $BDC$ greater than $BAC$.}

Part 2 may be proved without producing either of
the sides $BD$, $DC$. Thus: join $AD$ and produce it to
meet $BC$ in $F$; then the angle $BDF$ is greater than
the angle $BAF$ [\textsc{xvi}.], and $FDC$ is greater than $FAC$.
\emph{Therefore the whole angle $BDC$ is greater than $BAC$.}


\imgflow{138}{8}{f038}

\exheadin{Exercises.}

\begin{footnotesize}
1.~The sum of the lines drawn from any point within a triangle
to its angular points is less than
the perimeter. (Compare Ex.~2,
last Prop.)

2.~If a convex polygonal line
$ABCD$ lie within a convex polygonal
line $AMND$ terminating in
the same extremities, the length of
the former is less than that of the
latter.

\par\end{footnotesize}


\mypropl{PROP\@.~XXII\@.---Problem.}{To construct a triangle whose three sides shall be respectively
equal to three given lines $(A,\ B,\ C)$, the sum of every
two of which is greater than the third.}

\textbf{Sol.}---Take any right line $DE$, terminated at $D$, but
unlimited towards $E$, and cut off [\textsc{iii}.] $DF$ equal to $A$,
$FG$ equal to $B$, and $GH$ equal to $C$. With $F$ as centre,
and $FD$ as radius, describe the circle $KDL$ (Post.~\textsc{iii}.);
and with $G$ as centre, and $GH$ as radius, describe the
circle $KHL$, intersecting the former circle in $K$. Join
$KF$, $KG$. $KFG$ \emph{is the triangle required.}


\imgcent{290}{f039}

\textbf{Dem.}---Since $F$ is the centre of the circle $KDL$, $FK$
is equal to $FD$; but $FD$ is equal to $A$ (const.); therefore

(Axiom \textsc{i}.) $FK$ is equal to $A$. In like manner $GK$
is equal to $C$, and $FG$ is equal to $B$ (const.) \textit{Hence
the three sides of the triangle $KFG$ are respectively equal
to the three lines $A$, $B$, $C$}.

\exhead{Questions for Examination.}

\begin{footnotesize}
1.~What is the reason for stating in the enunciation that the
sum of every two of the given lines must be greater than the
third?

2.~Prove that when that condition is fulfilled the two circles
must intersect.

3.~Under what conditions would the circles not intersect?

4.~If the sum of two of the lines were equal to the third, would
the circles meet? Prove that they would not intersect.
\par\end{footnotesize}


\myprop{PROP\@.~XXIII\@.---Problem.}{At a given point $(A)$ in a given right line $(AB)$ to make
an angle equal to a given rectilineal angle $(DEF)$.}


\imgcent{270}{f040}

\textbf{Sol.}---In the sides $ED$, $EF$ of the given angle take
any arbitrary points $D$ and $F$. Join $DF$, and construct
[\textsc{xxii}.] the triangle $BAC$, whose sides, taken in order,
shall be equal to those of $DEF$---namely, $AB$ equal to

$ED$, $AC$ equal to $EF$, and $CB$ equal to $FD$; then the
angle $BAC$ will [\textsc{viii}.] be equal to $DEF$. \textit{Hence it is
the required angle.}


\exhead{Exercises.}

\begin{footnotesize}
1.~Construct a triangle, being given two sides and the angle
between them.

2.~Construct a triangle, being given two angles and the side
between them.

3.~Construct a triangle, being given two sides and the angle
opposite to one of them.

4.~Construct a triangle, being given the base, one of the angles
at the base, and the sum or difference of the sides.

5.~Given two points, one of which is in a given line, it is required
to find another point in the given line, such that the sum
or difference of its distances from the former points may be given.
Show that two such points may be found in each case.
\par\end{footnotesize}


\mypropl{PROP\@.~XXIV\@.---Theorem.}{If two triangles $(ABC,\ DEF)$ have two sides $(AB,\ AC)$
of one respectively equal to two sides $(DE,\ DF)$ of the
other, but the contained angle $(BAC)$ of one greater than
the contained angle $(EDF)$ of the other, the base of that
which has the greater angle is greater than the base of the
other.}


\imgflow{180}{12}{f041}

\textbf{Dem.}---Of the two sides $AB$, $AC$, let $AB$ be the one
which is not the greater, and with it make the angle
$BAG$ equal to $EDF$ [\textsc{xxiii}.]. Then because $AB$ is not
greater than $AC$, $AG$ is less than $AC$ [\textsc{xix}., Exer.~6].

Produce $AG$ to $H$, and make $AH$ equal to $DF$ or $AC$
[\textsc{iii.}]. Join $BH$, $CH$.

In the triangles
$BAH$, $EDF$, we have
$AB$ equal to $DE$
(hyp.), $AH$ equal to
$DF$ (const.), and the
angle $BAH$ equal
to the angle $EDF$
(const.); therefore the
base [\textsc{iv.}] $BH$ is equal
to $EF$. Again, because
$AH$ is equal to $AC$ (const.), the triangle $ACH$ is isosceles;
therefore the angle $ACH$ is equal to $AHC$ [\textsc{v.}];
but $ACH$ is greater than $BCH$; therefore $AHC$ is
greater than $BCH$: much more is the angle $BHC$
greater than $BCH$, and the greater angle is subtended
by the greater side [\textsc{xix.}]. Therefore $BC$ is greater
than $BH$; but $BH$ has been proved to be equal to $EF$;
\textit{therefore $BC$ is greater than $EF$.}\par\medskip

\begin{footnotesize}
The concluding part of this Proposition may be proved without
joining $CH$, thus:---
\\
\begin{tabular*}{\textwidth}{@{} l@{\extracolsep\fill} l r}
& $BG + GH > BH$ [\textsc{xx.}], &\phantom{therefore}  \\[.5ex]
& $AG + GC > AC$ [\textsc{xx.}]; \\[.5ex]
therefore & $BC + AH > BH + AC$; \\[.5ex]
but & $AH = AC$ (const.); \\[.5ex]
\textit{therefore} & \multicolumn{1}{c}{$BC \textit{ is } > BH$.}
\end{tabular*}\par\medskip

\textit{Or thus:} Bisect the angle $CAH$ by $AO$. Join $OH$. Now in
the $\triangle$s $CAO$, $HAO$ we have the sides $CA$, $AO$ in one equal to
the sides $AH$, $AO$ in the other, and the contained angles equal;
therefore the base $OC$ is equal to the base $OH$ [\textsc{iv.}]: to each add
$BO$, and we have $BC$ equal to the sum of $BO$, $OH$; but the sum
of $BO$, $OH$ is greater than $BH$ [\textsc{xx.}]. \textit{Therefore $BC$ is greater
than $BH$, that is, greater than $EF$.}
\par\end{footnotesize}

\exhead{Exercises.}

\begin{footnotesize}
1.~Prove this Proposition by making the angle $ABH$ to the
left of $AB$.

2.~Prove that the angle $BCA$ is greater than $EFD$.
\par\end{footnotesize}



\mypropl{PROP\@.~XXV\@.---Theorem.}{If two triangles $(ABC,\ DEF)$ have two sides $(AB,\ AC)$
of one respectively equal to two sides $(DE,\ DF)$ of the
other, but the base $(BC)$ of one greater than the base
$(EF)$ of the other, the angle $(A)$ contained by the sides of
that which has the greater base is greater them the angle
$(D)$ contained by the sides of the other.}


\imgflow{190}{10}{f042}

\textbf{Dem.}---If the angle $A$ be not greater than $D$, it must
be either equal to
it or less than it.
We shall examine
each case:---

1.~If $A$ were
equal to $D$, the triangles
$ABC$, $DEF$
would have the two
sides $AB$, $AC$ of one
respectively equal
to the two sides $DE$, $DF$ of the other, and the angle $A$
contained by the two sides of one equal to the angle $D$
contained by the two sides of the other. Hence [\textsc{iv}.]
$BC$ would be equal to $EF$; but $BC$ is, by hypothesis,
greater than $EF$; hence the angle $A$ is not equal to
the angle $D$.

2.~If $A$ were less than $D$, then $D$ would be greater
than $A$, and the triangles $DEF$, $ABC$ would have the two
sides $DE$, $DF$ of one respectively equal to the two sides
$AB$, $AC$ of the other, and the angle $D$ contained by the
two sides of one greater than the angle $A$ contained by
the two sides of the other. Hence [\textsc{xxiv}.] $EF$ would be
greater than $BC$; but $EF$ (hyp.) is not greater than $BC$.
Therefore $A$ is not less than $D$, and we have proved
that it is not equal to it; \emph{therefore it must be greater}.


\imgflow{190}{10}{f043}

\emph{Or thus, directly}: Construct the triangle $ACG$, whose
three sides $AG$, $GC$, $CA$ shall be respectively equal to
the three sides $DE$, $EF$, $FD$ of the triangle $DEF$ [\textsc{xxii}.].

Join $BG$. Then because $BC$ is greater than $EF$, $BC$ is
greater than $CG$. Hence [\textsc{xviii}.] the angle $BGC$ is
greater than $GBC$;
and make (\textsc{xxiii}.)
the angle $BGH$ equal
to $GBH$, and join
$AH$. Then [\textsc{vi}.] $BH$
is equal to $GH$.
Therefore the triangles
$ABH$, $AGH$
have the sides $AB$,
$AH$ of one equal to
the sides $AG$, $AH$ of the other, and the base $BH$ equal to
$GH$. Therefore [\textsc{viii}.] the angle $BAH$ is equal to $GAH$.
\emph{Hence the angle $BAC$ is greater than $CAG$, and therefore
greater than $EDF$.}

\exhead{Exercise.}

\begin{footnotesize}
Demonstrate this Proposition directly by cutting off from $BC$
a part equal to $EF$.
\par\end{footnotesize}


\mypropl{PROP\@.~XXVI\@.---Theorem.}{If two triangles $(ABC,\ DEF)$ have two angles $(B,\ C)$
of one equal respectively to two angles $(E,\ F)$ of the other,
and a side of one equal to a side similarly placed with
respect to the equal angles of the other, the triangles are
equal in every respect.}

\textbf{Dem.}---This Proposition breaks up into two according
as the sides given to be equal are the sides adjacent
to the equal angles, namely $BC$ and $EF$, or those
opposite equal angles.


\imgflow{140}{9}{f044}

1.~Let the equal sides be $BC$ and $EF$; then if $DE$ be
not equal to $AB$, suppose $GE$ to be equal to it. Join
$GF$; then the triangles $ABC$, $GEF$ have the sides $AB$,
$BC$ of one respectively equal to the sides $GE$, $EF$ of
the other, and the angle $ABC$ equal to the angle $GEF$

(hyp.); therefore [\textsc{iv}.] the angle $ACB$ is equal to the
angle $GFE$; but the angle
$ACB$ is (hyp.) equal to
$DFE$; hence $GFE$ is equal
to $DFE$---a part equal to
the whole, which is absurd;
therefore $AB$ and $DE$ are
not unequal, that is, they
are equal.  Consequently
the triangles $ABC$, $DEF$
have the sides $AB$, $BC$ of one respectively equal to the
sides $DE$, $EF$ of the other; and the contained angles
$ABC$ and $DEF$ equal; \emph{therefore} [\textsc{iv}.] \emph{$AC$ is equal to
$DF$, and the angle $BAC$ is equal to the angle $EDF$.}


\imgflow{160}{9}{f045}

2.~Let the sides given to be equal be $AB$ and $DE$; 
it is required to prove that
$BC$ is equal to $EF$, and
$AC$ to $DF$. If $BC$ be
not equal to $EF$, suppose
$BG$ to be equal to it.
Join $AG$. Then the triangles
$ABG$, $DEF$ have
the two sides $AB$, $BG$
of one respectively equal
to the two sides $DE$, $EF$ of the other, and the angle
$ABG$ equal to the angle $DEF$; therefore [\textsc{iv}.] the
angle $AGB$ is equal to $DFE$; but the angle $ACB$ is
equal to $DFE$ (hyp.). Hence (Axiom~\textsc{i}.) the angle
$AGB$ is equal to $ACB$, that is, the exterior angle of
the triangle $ACG$ is equal to the interior and non-adjacent
angle, which [\textsc{xvi}.] is impossible. \emph{Hence $BC$
must be equal to $EF$, and the same as in $1$, $AC$ is equal
to $DF$, and the angle $BAC$ is equal to the angle $EDF$.}\par\medskip

\begin{footnotesize}
This Proposition, together with \textsc{iv.}\ and \textsc{viii}., includes all the
cases of the congruence of two triangles. Part~I. may be proved
immediately by superposition. For it is evident if $ABC$ be applied
to $DEF$, so that the point $B$ shall coincide with $E$, and the line
$BC$ with $EF$, since $BC$ is equal to $EF$, the point $C$ shall coincide
with $F$; and since the angles $B$, $C$ are respectively equal to the
angles $E$, $F$, the lines $BA$, $CA$ shall coincide with $ED$ and $FD$.
\emph{Hence the triangles are congruent.}
\par\end{footnotesize}\bigskip



\textsc{Def.}---\textit{If every point on a geometrical figure satisfies an
assigned condition, that figure is called the locus\index{Locus} of the
point satisfying the condition.} Thus, for example, a
circle is the locus of a point whose distance from the
centre is equal to its radius.


\exhead{Exercises.}

\begin{footnotesize}
1.~The extremities of the base of an isosceles triangle are equally
distant from any point in the perpendicular from the vertical angle
on the base.

2.~If the line which bisects the vertical angle of a triangle also
bisects the base, the triangle is isosceles.

3.~The locus of a point which is equally distant from two fixed
lines is the pair of lines which bisect the angles made by the
fixed lines.

4.~In a given right line find a point such that the perpendiculars
from it on two given lines may be equal. State also the number
of solutions.

5.~If two right-angled triangles have equal hypotenuses, and
an acute angle of one equal to an acute angle of the other, they
are congruent.

6.~If two right-angled triangles have equal hypotenuses, and
a side of one equal to a side of the other, they are congruent.

7.~The bisectors of the three internal angles of a triangle are
concurrent.

8.~The bisectors of two external angles and the bisector of the
third internal angle are concurrent.

9.~Through a given point draw a right line, such that perpendiculars
on it from two given points on opposite sides may be
equal to each other.

10.~Through a given point draw a right line intersecting two
given lines, and forming an isosceles triangle with them.
\par\end{footnotesize}

\begin{center}
\textsc{Parallel Lines.}\index{Lines!parallel}
\end{center}

\textsc{Def.~i.}---\textit{If two right lines in the same plane be such
that, when produced indefinitely, they do not meet at any
finite distance, they are said to be} \textsc{parallel}.

\textsc{Def.~ii.}---A \emph{parallelogram}\index{parallelogram} is a quadrilateral, both
pairs of whose opposite sides are parallel.



\textsc{Def.~iii}.---The right line joining either pair of opposite
angles of a quadrilateral is called a \emph{diagonal}\index{Diagonal}.

\textsc{Def.~iv}.---If both pairs of opposite sides of a quadrilateral
be produced to meet, the right line joining their
points of intersection is called its \emph{third diagonal}\index{Diagonal!third of a quadrilateral}.

\textsc{Def.~v}.---A quadrilateral which has one pair of opposite
sides parallel is called a \emph{trapezium}\index{Trapezium}.


\imgflow{110}{8}{f046}

\textsc{Def.~vi}.---If from the extremities of one right line
perpendiculars be drawn to another, the intercept between
their feet is called the \emph{projection}\index{Projection}\index{Lines!projection of} of the first line
on the second.


\textsc{Def.~vii}\label{def8}.---When a right line
intersects two other right lines
in two distinct points it makes
with them eight angles, which
have received special names in
relation to one another. Thus,
in the figure---1, 2; 7, 8 are
called \emph{exterior} angles; 3, 4; 5, 6, \emph{interior}\index{Angle!interior} angles.
Again, 4; 6; 3, 5 are called \emph{alternate}\index{Angle!alternate} angles; lastly,
1, 5; 2, 6; 3, 8; 4, 7 are called \emph{corresponding} angles.

\myprop{PROP\@.~XXVII\@.---Theorem.}{If a right line $(EF)$ intersecting two right lines $(AB,\
CD)$ makes the alternate angles $(AEF,\ EFD)$ equal to
each other, these lines are parallel.}


\imgflow{160}{7}{f047}

\textbf{Dem.}---If $AB$ and $CD$ are not parallel they must meet,
if produced, at some finite
distance: if possible let
them meet in $G$; then
the figure $EGF$ is a triangle,
and the angle $AEF$
is an exterior angle, and
$EFD$ a non-adjacent interior
angle. Hence [\textsc{xvi.}] $AEF$ is greater than $EFD$;
but it is also equal to it (hyp.), that is, both equal and
greater, which is absurd. \emph{Hence $AB$ and $CD$ are
parallel.}\par\medskip



\textit{Or thus:} Bisect $EF$ in $O$; turn the whole figure
round $O$ as a centre, so that $EF$ shall fall on itself;
then because $OE = OF$, the point $E$ shall fall on $F$;
and because the angle $AEF$ is equal to the angle $EFD$,
the line $EA$ will occupy the place of $FD$, and the line
$FD$ the place of $EA$; therefore the lines $AB$, $CD$ interchange
places, and the figure is symmetrical with respect
to the point $O$. Hence, if $AB$, $CD$ meet on one
side of $O$, they must also meet on the other side; but
two right lines cannot enclose a space (Axiom~\textsc{x}.);
therefore they do not meet at either side. \textit{Hence they
are parallel.}


\mypropl{PROP\@.~XXVIII\@.---Theorem.}{If a right line $(EF)$ intersecting two right lines $(AB,\ CD)$
makes the exterior angle $(EGB)$ equal to its corresponding
interior angle $(GHD)$, or makes two interior
angles $(BGH,\ GHD)$ on the same side equal to two right
angles, the two right lines are parallel.}


\imgflow{133}{8}{f048}

\textbf{Dem.}---1. Since the lines $AB$, $EF$ intersect, the
angle $AGH$ is equal to $EGB$
[\textsc{xv}.]; but $EGB$ is equal to
$GHD$ (hyp.); therefore $AGH$
is equal to $GHD$, and they
are alternate angles. \textit{Hence}
[\textsc{xxvii}.] \textit{$AB$ is parallel to
$CD$.}

2.~Since $AGH$ and $BGH$
are adjacent angles, their sum is equal to two right
angles [\textsc{xiii}.]; but the sum of $BGH$ and $GHD$ is two
right angles (hyp.); therefore rejecting the angle $BGH$
we have $AGH$ equal $GHD$, and they are alternate
angles; \textit{therefore $AB$ is parallel to $CD$} [\textsc{xxvii}.].




\mypropl{PROP\@.~XXIX\@.---Theorem.}{If a right line $(EF)$ intersect two parallel right lines
$(AB,\ CD)$, it makes---$1$.~the alternate angles $(AGH, GHD)$
equal to one another; $2$.~the exterior angle $(EGB)$ equal
to the corresponding interior angle $(GHD)$; $3$.~the two
interior angles $(BGH,\ GHD)$ on the same side equal to
two right angles.}


\imgflow{135}{8}{f049}

\textbf{Dem.}---If the angle $AGH$ be not equal to $GHD$, one
must be greater than the
other. Let $AGH$ be the
greater; to each add $BGH$,
and we have the sum of the
angles $AGH$, $BGH$ greater
than the sum of the angles
$BGH$, $GHD$; but the sum of
$AGH$, $BGH$ is two right
angles; therefore the sum of $BGH$, $GHD$ is less than
two right angles, and therefore (Axiom \textsc{xii.}) the lines
$AB$, $CD$, if produced, will meet at some finite distance:
but since they are parallel (hyp.) they cannot meet at
any finite distance. \textit{Hence the angle $AGH$ is not unequal
to $GHD$---that is, it is equal to it.}

2.~Since the angle $EGB$ is equal to $AGH$ [\textsc{xv.}], and
$GHD$ is equal to $AGH$ (1), $EGB$ \emph{is equal to} $GHD$
(Axiom \textsc{i}.).

3.~Since $AGH$ is equal to $GHD$ (1), add $HGB$ to each,
and we have the sum of the angles $AGH$, $HGB$ equal
to the sum of the angles $GHD$, $HGB$; but the sum of the
angles $AGH$, $HGB$ [\textsc{xiii.}] is two right angles; \textit{therefore
the sum of the angles $BGH$, $GHD$ is two right angles.}

\exhead{Exercises.}

\begin{footnotesize}
1.~Demonstrate both parts of Prop.\ \textsc{xxviii.}\ without using Prop.\
\textsc{xxvii}.

2.~The parts of all perpendiculars to two parallel lines intercepted
between them are equal.

3.~If $ACD$, $BCD$ be adjacent angles, any parallel to $AB$ will
meet the bisectors of these angles in points equally distant from
where it meets $CD$.



4.~If through the middle point $O$ of any right line terminated
by two parallel right lines any other secant be drawn, the intercept
on this line made by the parallels is bisected in $O$.

5.~Two right lines passing through a point equidistant from
two parallels intercept equal portions on the parallels.

6.~The perimeter of the parallelogram, formed by drawing
parallels to two sides of an equilateral triangle from any point in
the third side, is equal to twice the side.

7.~If the opposite sides of a hexagon be equal and parallel, its
diagonals are concurrent.

8.~If two intersecting right lines be respectively parallel to two
others, the angle between the former is equal to the angle between
the latter. For if $AB$, $AC$ be respectively parallel to $DE$,
$DF$, and if $AC$, $DE$ meet in $G$, the angles $A$, $D$ are each equal to
$G$ [\textsc{xxix}.].
\par\end{footnotesize}


\myprop{PROP\@.~XXX\@.---Theorem.}{If two right lines $(AB,\ CD)$ be parallel to the same right
line $(EF)$, they are parallel to one another.}


\imgflow{125}{7}{f050}

\textbf{Dem.}---Draw any secant
$GHK$. Then since $AB$ and
$EF$ are parallel, the angle $AGH$
is equal to $GHF$ [\textsc{xxix}.]. In like
manner the angle $GHF$ is equal
to $HKD$ [\textsc{xxix}.]. Therefore the
angle $AGK$ is equal to the
angle $GKD$ (Axiom~\textsc{i}.). \textit{Hence }
[\textsc{xxvii}.] \textit{$AB$ is parallel to $CD$.}


\myprop{PROP\@.~XXXI\@.---Problem.}{Through a given point $(C)$ to draw a right line parallel
to a given right line.}


\imgflow{125}{5}{f051}

\textbf{Sol.}---Take any point $D$ in
$AB$. Join $CD$ (Post.~\textsc{i}.), and
make the angle $DCE$ equal to
the angle $ADC$ [\textsc{xxiii}.]. \textit{The
line $CE$ is parallel to $AB$}
[\textsc{xxvii}.].


\exhead{Exercises.}

\begin{footnotesize}
1.~Given the altitude of a triangle and the base angles, construct
it.

2.~From a given point draw to a given line a line making with
it an angle equal to a given angle. Show that there will be two
solutions.

3.~Prove the following construction for trisecting a given line
$AB$:---On $AB$ describe an equilateral $\triangle\ ABC$. Bisect the angles
$A$, $B$ by the lines $AD$, $BD$, meeting in $D$; through $D$ draw
parallels to $AC$, $BC$, meeting $AB$ in $E$, $F$: $E$, $F$ are the points of
trisection of $AB$.

4.~Inscribe a square in a given equilateral triangle, having its
base on a given side of the triangle.

5.~Draw a line parallel to the base of a triangle so that it may
be---1.~equal to the intercept it makes on one of the sides from
the extremity of the base; 2.~equal to the sum of the two intercepts
on the sides from the extremities of the base; 3.~equal to
their difference. Show that there are two solutions in each case.

6.~Through two given points in two parallel lines draw two
lines forming a lozenge with the given parallels.

7.~Between two lines given in position place a line of given
length which shall be parallel to a given line. Show that there
are two solutions.
\par\end{footnotesize}


\mypropflow{PROP\@.~XXXII\@.---Theorem.}{If any side $(AB)$ of a triangle $(ABC)$ be produced $($to
$D)$, the external angle $(CBD)$ is equal to the sum of the
two internal\index{Angle!interior} non-adjacent angles $(A,\ C)$, and the sum of
the three internal angles is equal to two right angles.}{140}{9}{f052}

\textbf{Dem.}---Draw $BE$ parallel to $AC$ [\textsc{xxxi}.]. Now since
$BC$ intersects the parallels $BE$,
$AC$, the alternate angles $EBC$,
$ACB$ are equal [\textsc{xxix}.]. Again,
since $AB$ intersects the parallels
$BE$, $AC$, the angle $EBD$ is
equal to $BAC$ [\textsc{xxix}.]; hence
the whole angle $CBD$ is equal
to the sum of the two angles
$ACB$, $BAC$: to each of these add the angle $ABC$ and

we have the sum of $CBD$, $ABC$ equal to the sum of the
three angles $ACB$, $BAC$, $ABC$: but the sum of $CBD$,
$ABC$ is two right angles [\textsc{xiii}.]; \emph{hence the sum of the
three angles $ACB$, $BAC$, $ABC$ is two right angles}.

\emph{Cor.}~1.---If a right-angled triangle be isosceles, each
base angle is half a right angle.

\emph{Cor.}~2.---If two triangles have two angles in one
respectively equal to two angles in the other, their
remaining angles are equal.

\emph{Cor.}~3.---Since a quadrilateral can be divided into
two triangles, the sum of its angles is equal to four
right angles.

\emph{Cor.}~4.---If a figure of $n$ sides be divided into triangles
by drawing diagonals from any one of its angles
there will be $(n-2)$ triangles; hence the sum of its
angles is equal $2(n-2)$ right angles.

\emph{Cor.}~5.---If all the sides of any convex polygon be
produced, the sum of the external angles is equal to
four right angles.

\emph{Cor.}~6.---Each angle of an equilateral triangle is
two-thirds of a right angle.

\emph{Cor.}~7.---If one angle of a triangle be equal to the
sum of the other two, it is a right angle.

\emph{Cor.}~8.---Every right-angled triangle can be divided
into two isosceles triangles by a line drawn from the
right angle to the hypotenuse.


\exhead{Exercises.}

\begin{footnotesize}
1.~Trisect a right angle.

2.~Any angle of a triangle is obtuse, right, or acute, according
as the opposite side is greater than, equal to, or less than, twice
the \emph{median} drawn from that angle.

3.~If the sides of a polygon of $n$ sides be produced, the sum of
the angles between each alternate pair is equal to $2(n-4)$ right
angles.

4.~If the line which bisects the external vertical angle be
parallel to the base, the triangle is isosceles.

5.~If two right-angled $\triangle$s $ABC$, $ABD$ be on the same hypotenuse
$AB$, and the vertices $C$ and $D$ be joined, the pair of angles
subtended by any side of the quadrilateral thus formed are equal.

6.~The three perpendiculars of a triangle are concurrent.



7.~The bisectors of two adjacent angles of a parallelogram are
at right angles.

8.~The bisectors of the external angles of a quadrilateral form a
circumscribed quadrilateral, the sum of whose opposite angles is
equal to two right angles.

9.~If the three sides of one triangle be respectively perpendicular
to those of another triangle, the triangles are equiangular.

10.~Construct a right-angled triangle, being given the hypotenuse
and the sum or difference of the sides.

11.~The angles made with the base of an isosceles triangle by
perpendiculars from its extremities on the equal sides are each
equal to half the vertical angle.

12.~The angle included between the internal bisector of one
base angle of a triangle and the external bisector of the other base
angle is equal to half the vertical angle.

13.~In the construction of Prop.\ \textsc{xviii.}\ prove that the angle
$DBC$ is equal to half the difference of the base angles.

14.~If $A$, $B$, $C$ denote the angles of a $\triangle$, prove that $\frac{1}{2} (A + B)$,
$\frac{1}{2} (B + C)$, $\frac{1}{2} (C + A)$ will be the angles of a $\triangle$ formed by any
side and the bisectors of the external angles between that side and
the other sides produced.
\par\end{footnotesize}


\myprop{PROP\@.~XXXIII\@.---Theorem.}{The right lines $(AC,\ BD)$ which join the adjacent extremities
of two equal and parallel right lines $(AB,\ CD)$ are equal and parallel.}


\imgflow{133}{9}{f053}

\textbf{Dem.}---Join $BC$. Now since $AB$ is parallel to $CD$,
and $BC$ intersects them, the
angle $ABC$ is equal to the
alternate angle $DCB$ [\textsc{xxix.}].
Again, since $AB$ is equal to
$CD$, and $BC$ common, the triangles
$ABC$, $DCB$ have the
sides $AB$, $BC$ in one respectively
equal to the sides $DC$,
$CB$ in the other, and the angles $ABC$, $DCB$ contained
by those sides equal; therefore [\textsc{iv.}] the base $AC$ is
equal to the base $BD$, and the angle $ACB$ is equal to
the angle $CBD$; but these are alternate angles; hence
[\textsc{xxvii.}] $AC$ is parallel to $BD$, and it has been proved
equal to it. \emph{Therefore $AC$ is both equal and parallel to $BD$.}



\exhead{Exercises.}

\begin{footnotesize}
1.~If two right lines $AB$, $BC$ be respectively equal and parallel
to two other right lines $DE$, $EF$, the right line $AC$ joining the
extremities of the former pair is equal to the right line $DF$ joining
the extremities of the latter.

2.~Right lines that are equal and parallel have equal projections
on any other right line; and conversely, parallel right lines that
have equal projections on another right line are equal.

3.~Equal right lines that have equal projections on another
right line are parallel.

4.~The right lines which join transversely the extremities of
two equal and parallel right lines bisect each other.
\par\end{footnotesize}


\mypropl{PROP\@.~XXXIV\@.---Theorem.}{The opposite sides $(AB,\ CD;\ AC,\ BD)$ and the opposite
angles $(A,\ D;$ $B,\ C)$ of a parallelogram are equal to
one another, and either diagonal bisects the parallelogram.}


\imgflow{133}{8}{f054}

\textbf{Dem.}---Join $BC$. Since $AB$ is parallel to $CD$, and
$BC$ intersects them, the angle
$ABC$ is equal to the angle
$BCD$ [\textsc{xxix.}]. Again, since
$BC$ intersects the parallels $AC$,
$BD$, the angle $ACB$ is equal to
the angle $CBD$; hence the triangles
$ABC$, $DCB$ have the two
angles $ABC$, $ACB$ in one respectively
equal to the two angles $BCD$, $CBD$ in the other,
and the side $BC$ common. \emph{Therefore} [\textsc{xxvi.}] \emph{$AB$ is equal
to $CD$, and $AC$ to $BD$; the angle $BAC$ to the angle $BDC$,
and the triangle $ABC$ to the triangle $BDC$.}

Again, because the angle $ACB$ is equal to $CBD$, and
$DCB$ equal to $ABC$, \emph{the whole angle $ACD$ is equal to the
whole angle $ABD$.}

\textit{Cor.}~1.---If one angle of a parallelogram be a right
angle, all its angles are right angles.

\textit{Cor.}~2.---If two adjacent sides of a parallelogram be
equal, it is a lozenge.



\textit{Cor.}~3.---If both pairs of opposite sides of a quadrilateral
be equal, it is a parallelogram.

\textit{Cor.}~4.---If both pairs of opposite angles of a quadrilateral
be equal, it is a parallelogram.

\textit{Cor.}~5.---If the diagonals of a quadrilateral bisect
each other, it is a parallelogram.

\textit{Cor.}~6.---If both diagonals of a quadrilateral bisect
the quadrilateral, it is a parallelogram.

\textit{Cor.}~7.---If the adjacent sides of a parallelogram be
equal, its diagonals bisect its angles.

\textit{Cor.}~8.---If the adjacent sides of a parallelogram be
equal, its diagonals intersect at right angles.

\textit{Cor.}~9.---In a right-angled parallelogram the diagonals
are equal.

\textit{Cor.}~10.---If the diagonals of a parallelogram be perpendicular
to each other, it is a lozenge.

\textit{Cor.}~11.---If a diagonal of a parallelogram bisect the
angles whose vertices it joins, the parallelogram is a
lozenge.

\exhead{Exercises.}

\begin{footnotesize}
1.~The diagonals of a parallelogram bisect each other.

2.~If the diagonals of a parallelogram be equal, all its angles
are right angles.

3.~Divide a right line into any number of equal parts.

4.~The right lines joining the adjacent extremities of two
unequal parallel right lines will meet, if produced, on the side
of the shorter parallel.

5.~If two opposite sides of a quadrilateral be parallel but not
equal, and the other pair equal but not parallel, its opposite angles
are supplemental.

6.~Construct a triangle, being given the middle points of its
three sides.

7.~The area of a quadrilateral is equal to the area of a triangle,
having two sides equal to its diagonals, and the contained angle
equal to that between the diagonals.
\par\end{footnotesize}



\myprop{PROP\@.~XXXV\@.---Theorem.}{Parallelograms on the same base $(BC)$ and between the
same parallels are equal.}


\imgflow{175}{8}{f055}

\textbf{Dem.}---1. Let the sides $AD$, $DF$ of the parallelograms
$AC$, $BF$ opposite to the
common base $BC$ terminate
in the same
point $D$, then [\textsc{xxxiv}.]
each parallelogram is
double of the triangle
$BCD$. \emph{Hence they are
equal to one another.}

2.~Let the sides $AD$, $EF$ (figures ($\alpha$), ($\beta$)) opposite
to $BC$ not terminate in the same point.


\imgcent{338}{f056}

Then because $ABCD$ is a parallelogram, $AD$ is equal
to $BC$ [\textsc{xxxiv}.]; and since $BCEF$ is a parallelogram, $EF$
is equal to $BC$; therefore (see fig.~($\alpha$)) take away $ED$,
and in fig.~($\beta$) add $ED$, and we have in each case $AE$
equal to $DF$, and $BA$ is equal to $CD$ [\textsc{xxxiv}.]. Hence
the triangles $BAE$, $CDF$ have the two sides $BA$, $AE$ in
one respectively equal to the two sides $CD$, $DF$ in the
other, and the angle $BAE$ [\textsc{xxix}.] equal to the angle
$CDF$; hence [\textsc{iv}.] the triangle $BAE$ is equal to the triangle
$CDF$; and taking each of these triangles in succession
from the quadrilateral $BAFC$, \emph{there will remain
the parallelogram $BCFE$ equal to the parallelogram $BCDA$.}

\emph{Or thus:} The triangles $ABE$, $DCF$ have [\textsc{xxxiv}.]
the sides $AB$, $BE$ in one respectively equal to the sides

$DC$, $CF$ in the other, and the angle $ABE$ equal to the
angle $DCF$ [\textsc{xxix}., Ex.~8]. Hence the triangle $ABE$ is
equal to the triangle $DCF$; and, taking each away from
the quadrilateral $BAFC$, \emph{there will remain the parallelogram
$BCFE$ equal to the parallelogram $BCDA$.}\par\medskip

\begin{footnotesize}
\textsf{Observation}.---By the second method of proof the subdivision
of the demonstration into cases is avoided. It is easy to see that
either of the two parallelograms $ABCD$, $EBCF$ can be divided
into parts and rearranged so as to make it congruent with the
other. This Proposition affords the first instance in the Elements
in which equality which is not congruence occurs. This equality
is expressed algebraically by the symbol $=$, while congruence
is denoted by $\equiv$, called also the symbol of identity. Figures that
are congruent are said to be \emph{identically equal}\index{Identically equal}.
\par\end{footnotesize}

\myprop{PROP\@.~XXXVI\@.---Theorem.}{Parallelograms $(BD,\ FH)$ on equal bases $(BC,\ FG)$ and
between the same parallels are equal.}


\imgflow{190}{8}{f057}

\textbf{Dem.}---Join $BE$, $CH$. Now since $FH$ is a parallelogram,
$FG$ is equal
to $EH$ [\textsc{xxxiv}.]; but
$BC$ is equal to $FG$
(hyp.); therefore
$BC$ is equal to $EH$
(Axiom~\textsc{i}.). Hence
$BE$, $CH$, which join
their adjacent extremities,
are equal and parallel; therefore $BH$ is a
parallelogram. Again, since the parallelograms $BD$,
$BH$ are on the same base $BC$, and between the same
parallels $BC$, $AH$, they are equal [\textsc{xxxv}.]. In like
manner, since the parallelograms $HB$, $HF$ are on the
same base $EH$, and between the same parallels $EH$,
$BG$, they are equal. Hence $BD$ and $FH$ are each equal
to $BH$. \emph{Therefore} (Axiom~\textsc{i}.) \emph{$BD$ is equal to $FH$.}

\begin{footnotesize}
\textsf{Exercise}.---Prove this Proposition without joining $BE$, $CH$.
\par\end{footnotesize}



\myprop{PROP\@.~XXXVII\@.---Theorem.}{Triangles $(ABC,\ DBC)$ on the same base $(BC)$ and between
the same parallels $(AD,\ BC)$ are equal.}


\imgflow{190}{8}{f058}

\textbf{Dem.}---Produce $AD$ both ways. Draw $BE$ parallel
to $AC$, and $CF$ parallel
to $BD$ [\textsc{xxxi}.]
Then the figures
$AEBC$, $DBCF$ are
parallelograms; and
since they are on
the same base $BC$,
and between the
same parallels $BC$, $EF$ they are equal [\textsc{xxxv}.]. Again,
the triangle $ABC$ is half the parallelogram $AEBC$
[\textsc{xxxiv}.], because the diagonal $AB$ bisects it. In like

manner the triangle $DBC$ is half the parallelogram
$DBCF$, because the diagonal DC bisects it, and halves
of equal things are equal (Axiom \textsc{vii}.). \emph{Therefore the
triangle $ABC$ is equal to the triangle $DBC$.}

\exhead{Exercises.}

\begin{footnotesize}
1.~If two equal triangles be on the same base, but on opposite
sides, the right line joining their vertices is bisected by the base.

2.~Construct a triangle equal in area to a given quadrilateral
figure.

3.~Construct a triangle equal in area to a given rectilineal
figure.

4.~Construct a lozenge equal to a given parallelogram, and
having a given side of the parallelogram for base.

5.~Given the base and the area of a triangle, find the locus of
the vertex.

6.~If through a point $O$, in the production of the diagonal $AC$
of a parallelogram $ABCD$, any right line be drawn cutting the
sides $AB$, $BC$ in the points $E$, $F$, and $ED$, $FD$ be joined, the
triangle $EFD$ is less than half the parallelogram.
\par\end{footnotesize}



\myprop{PROP\@.~XXXVIII\@.---Theorem.}{Two triangles on equal bases and between the same
parallels are equal.}

\textbf{Dem.}---By a construction similar to the last, we see
that the triangles are the halves of parallelograms, on
equal bases, and between the same parallels. Hence
they are the halves of equal parallelograms [\textsc{xxxvi}.].
\emph{Therefore they are equal to one another.}

\exhead{Exercises.}

\begin{footnotesize}
1.~Every median of a triangle bisects the triangle.

2.~If two triangles have two sides of one respectively equal to
two sides of the other, and the contained angles supplemental,
their areas are equal.

3.~If the base of a triangle be divided into any number of equal
parts, right lines drawn from the vertex to the points of division
will divide the whole triangle into as many equal parts.

4.~Right lines from any point in the diagonal of a parallelogram
to the angular points through which the diagonal does not
pass, and the diagonal, divide the parallelogram into four triangles
which are equal, two by two.

5.~If one diagonal of a quadrilateral bisects the other, it also
bisects the quadrilateral, and conversely.

6.~If two $\triangle$s $ABC$, $ABD$ be on the same base $AB$, and between
the same parallels, and if a parallel to $AB$ meet the sides $AC$, $BC$
in the point $E$, $F$; and the sides $AD$, $BD$ in the point $G$, $H$; then
$EF = GH$.

7.~If instead of triangles on the same base we have triangles on
equal bases and between the same parallels, the intercepts made by
the sides of the triangles on any parallel to the bases are equal.

8.~If the middle points of any two sides of a triangle be joined,
the triangle so formed with the two half sides is one-fourth of the
whole.

9.~The triangle whose vertices are the middle points of two
sides, and any point in the base of another triangle, is one-fourth
of that triangle.

10.~Bisect a given triangle by a right line drawn from a given
point in one of the sides.

11.~Trisect a given triangle by three right lines drawn from a
given point within it.



12.~Prove that any right line through the intersection of the
diagonals of a parallelogram bisects the parallelogram\label{parallellogram}.

13.~The triangle formed by joining the middle point of one of
the non-parallel sides of a trapezium to the extremities of the
opposite side is equal to half the trapezium.
\par\end{footnotesize}


\myprop{PROP\@.~XXXIX\@.---Theorem.}{Equal triangles $(BAC,\ BDC)$ on the same base $(BC)$ and
on the same side of it are between the same parallels.}


\imgflow{128}{8}{f059}

\textbf{Dem.}---Join $AD$. Then if $AD$ be not parallel to $BC$,
let $AE$ be parallel to it, and
let it cut $BD$ in $E$. Join $EC$.
Now since the triangles $BEC$,
$BAC$ are on the same base $BC$,
and between the same parallels
$BC$, $AE$, they are equal [\textsc{xxxvii}.];
but the triangle $BAC$ is equal
to the triangle $BDC$ (hyp.).
Therefore (Axiom~\textsc{i}.) the triangle $BEC$ is equal to
the triangle $BDC$---that is, a part equal to the whole
which is absurd. \emph{Hence $AD$ must be parallel to $BC$.}


\mypropl{PROP\@.~XL\@.---Theorem.}{Equal triangles $(ABC,\ DEF)$ on equal bases $(BC,\ EF)$
which form parts of the same right line, and on the same
side of the line, are between the same parallels.}


\imgflow{170}{7}{f060}

\textbf{Dem.}---Join $AD$. If $AD$ be not parallel to $BF$, let
$AG$ be parallel to it.
Join $GF$. Now since
the triangles $GEF$ and
$ABC$ are on equal bases
$BC$, $EF$, and between
the same parallels $BF$,
$AG$, they are equal
[\textsc{xxxviii}.]; but the triangle
$DEF$ is equal to the triangle $ABC$ (hyp.). Hence

$GEF$ is equal to $DEF$ (Axiom~\textsc{i}.)---that is, a part equal
to the whole, which is absurd. \emph{Therefore $AD$ must be
parallel to $BF$.}\par\medskip

\textsc{Def}.---\emph{The altitude of a triangle is the perpendicular
from the vertex on the base.}


\exhead{Exercises.}

\begin{footnotesize}
1.~Triangles and parallelograms of equal bases and altitudes
are respectively equal.

2.~The right line joining the middle points of two sides of a
triangle is parallel to the third; for the medians from the extremities
of the base to these points will each bisect the original
triangle. Hence the two triangles whose base is the third side
and whose vertices are the points of bisection are equal.

3.~The parallel to any side of a triangle through the middle
point of another bisects the third.

4.~The lines of connexion of the middle points of the sides of
a triangle divide it into four congruent triangles.

5.~The line of connexion of the middle points of two sides of
a triangle is equal to half the third side.

6.~The middle points of the four sides of a convex quadrilateral,
taken in order, are the angular points of a parallelogram whose
area is equal to half the area of the quadrilateral.

7.~The sum of the two parallel sides of a trapezium is double
the line joining the middle points of the two remaining sides.

8.~The parallelogram formed by the line of connexion of the
middle points of two sides of a triangle, and any pair of parallels
drawn through the same points to meet the third side, is equal to
half the triangle.

9.~The right line joining the middle points of opposite sides
of a quadrilateral, and the right line joining the middle points
of its diagonals, are concurrent.
\par\end{footnotesize}



\mypropl{PROP\@.~XLI\@.---Theorem.}{If a parallelogram $(ABCD)$ and a triangle $(EBC)$ be
on the same base $(BC)$ and between the same parallels, the
parallelogram is double of the triangle.}


\imgflow{128}{8}{f061}

\textbf{Dem.}---Join $AC$. The parallelogram $ABCD$ is double
of the triangle $ABC$ [\textsc{xxxiv}.];
but the triangle $ABC$ is equal
to the triangle $EBC$ [\textsc{xxxvii}.].
\emph{Therefore the parallelogram
$ABCD$ is double of the triangle
$EBC$.}

\textit{Cor.}~1.---If a triangle and
a parallelogram have equal altitudes, and if the base
of the triangle be double of the base of the parallelogram,
the areas are equal.

\textit{Cor.}~2.---The sum of the triangles whose bases are
two opposite sides of a parallelogram, and which have
any point between these sides as a common vertex, is
equal to half the parallelogram.


\myprop{PROP\@.~XLII\@.---Problem.}{To construct a parallelogram equal to a given triangle
$(ABC)$, and having an angle equal to a given angle $(D)$.}


\imgcent{224}{f062}

\textbf{Sol.}---Bisect $AB$ in $E$. Join $EC$. Make the angle
$BEF$ [\textsc{xxiii}.] equal to $D$. Draw $CG$ parallel to $AB$
[\textsc{xxxi}.], and $BG$ parallel to $EF$. $EG$ is \emph{a parallelogram
fulfilling the required conditions.}



\textbf{Dem.}---Because $AE$ is equal to $EB$ (const.), the triangle
$AEC$ is equal to the triangle $EBC$ [\textsc{xxxviii}.],
therefore the triangle $ABC$ is double of the triangle
$EBC$; but the parallelogram $EG$ is also double of the
triangle $EBC$ [\textsc{xli}.], because they are on the same base
$EB$, and between the same parallels $EB$ and $CG$.
Therefore the parallelogram $EG$ is equal to the triangle
$ABC$, and it has (const.) the angle $BEF$ equal to $D$.
\emph{Hence $EG$ is a parallelogram fulfilling the required conditions.}



\mypropflow{PROP\@.~XLIII\@.---Theorem.}
{The parallels $(EF,\ GH)$ through any point $(K)$ in one
of the diagonals $(AC)$ of a parallelogram divide it into
four parallelograms, of which the two $(BK,\ KD)$ through
which the diagonal does not pass, and which are called the
\textsc{complements}\index{parallelogram!complements of} of the other two, are equal.}{140}{8}{f063}

\textbf{Dem.}---Because the diagonal bisects the parallelograms
$AC$, $AK$, $KC$ we have
[\textsc{xxxiv}.] the triangle $ADC$
equal to the triangle $ABC$,
the triangle $AHK$ equal to
$AEK$, and the triangle $KFC$
equal to the triangle $KGC$.
Hence, subtracting the sums
of the two last equalities from
the first, we get \emph{the parallelogram $DK$ equal to the
parallelogram $KB$.}

\textit{Cor.}~1.---If through a point $K$ within a parallelogram
$ABCD$ lines drawn parallel to the sides make
the parallelograms $DK$, $KB$ equal, $K$ is a point in the
diagonal $AC$.

\textit{Cor.}~2.---The parallelogram $BH$ is equal to $AF$, and
$BF$ to $HC$.

\begin{footnotesize}
\textit{Cor.}~2. supplies an easy demonstration of a fundamental Proposition
in Statics.
\par\end{footnotesize}



\exhead{Exercises.}


\imgflow{170}{13}{f064}

\begin{footnotesize}
1.~\emph{If $EF$, $GH$ be parallels to the adjacent sides of a parallelogram
$ABCD$, the diagonals
$EH$, $GF$ of two of
the four $\pgram$s into which
they divide it and one of
the diagonals of $ABCD$
are concurrent.}

\textbf{Dem.}---Let $EH$, $GF$
meet in $M$; through $M$
draw $MP$, $MJ$ parallel to
$AB$, $BC$. Produce $AD$,
$GH$, $BC$ to meet $MP$,
and $AB$, $EF$, $DC$ to meet
$MJ$. Now the complement $OF = FJ$: to each add the $\pgram~FL$,
and we get the figure $OFL = \pgram~CJ$. Again, the complement
$PH=HK$ [\textsc{xliii}.]: to each add the $\pgram~OC$, and we get the $\pgram~PC
=\text{\ figure } OFL$. Hence the $\pgram~PC = CJ$. 
Therefore they are
about the same diagonal [\textsc{xliii}., \textit{Cor}.~1]. \emph{Hence $AC$ produced
will pass through $M$.}

2.~\emph{The middle points of the three diagonals $AC$, $BD$, $EF$ of
a quadrilateral $ABCD$ are
collinear.}


\imgcent{158}{f065}

\textbf{Dem.}---Complete the
$\pgram~AEBG$. Draw $DH$, $CI$
parallel to $AG$, $BG$. Join
$IH$, and produce; then $AB$,
$CD$, $IH$ are concurrent
(Ex.~1); therefore $IH$ will
pass through $F$. Join $EI$,
$EH$. Now [\textsc{xi}., Ex.~2, 3]
the middle points of $EI$,
$EH$, $EF$ are collinear, but
[\textsc{xxxiv}., Ex.~1] the middle
points of $EI$, $EH$ are the
middle points of $AC$, $BD$.
\emph{Hence the middle points of $AC$, $BD$, $EF$ are collinear.}
\par\end{footnotesize}



\mypropl{PROP\@.~XLIV\@.---Problem.}{To a given, right line $(AB)$ to apply a parallelogram
which shall be equal to a given triangle $(C)$, and have one
of its angles equal to a given angle $(D)$.}


\imgcent{240}{f066}

\textbf{Sol.}---Construct the parallelogram $BEFG$ [\textsc{xlii}.] equal
to the given triangle $C$, and having the angle $B$ equal
to the given angle $D$, and so that its side $BE$ shall be in
the same right line with $AB$. Through $A$ draw $AH$
parallel to $BG$ [\textsc{xxxi}.], and produce $FG$ to meet it in $H$.
Join $HB$. Then because $HA$ and $FE$ are parallels, and
$HF$ intersects them, the sum of the angles $AHF$, $HFE$
is two right angles [\textsc{xxix}.]; therefore the sum of the
angles $BHF$, $HFE$ is less than two right angles; and
therefore (Axiom~\textsc{xii}.) the lines $HB$, $FE$, if produced,
will meet as at $K$. Through $K$ draw $KL$ parallel to
$AB$ [\textsc{xxxi}.], and produce $HA$ and $GB$ to meet it in the
points $L$ and $M$. \emph{Then $AM$ is a parallelogram fulfilling
the required conditions.}

\textbf{Dem.}---The parallelogram $AM$ is equal to $GE$
[\textsc{xliii}.]; but $GE$ is equal to the triangle $C$ (const.);
therefore $AM$ is equal to the triangle $C$. Again, the
angle $ABM$ is equal to $EBG$ [\textsc{xv}.], and $EBG$ is equal
to $D$ (const.); therefore the angle $ABM$ is equal to $D$;
and $AM$ is constructed on the given line; \emph{therefore it is
the parallelogram required.}




\myprop{PROP\@.~XLV\@.---Problem.}{To construct a parallelogram equal to a given rectilineal
figure $(ABCD)$, and having an angle equal to a given rectilineal angle $(X)$.}


\imgcent{230}{f067}

\textbf{Sol.}---Join $BD$. Construct a parallelogram $EG$
[\textsc{xlii.}] equal to the triangle $ABD$, and having the
angle $E$ equal to the given angle $X$; and to the right
line $GH$ apply the parallelogram $HI$ equal to the triangle
$BCD$, and having the angle $GHK$ equal to $X$
[\textsc{xliv.}], and so on for additional triangles if there be
any. \textit{Then $EI$ is a parallelogram fulfilling the required
conditions.}

\textbf{Dem.}---Because the angles $GHK$, $FEH$ are each
equal to $X$ (const.), they are equal to one another: to
each add the angle $GHE$, and we have the sum of the
angles $GHK$, $GHE$ equal to the sum of the angles
$FEH$, $GHE$; but since $HG$ is parallel to $EF$, and $EH$
intersects them, the sum of $FEH$, $GHE$ is two right
angles [\textsc{xxix.}]. Hence the sum of $GHK$, $GHE$ is two
right angles; therefore $EH$, $HK$ are in the same right
line [\textsc{xiv.}].

Again, because $GH$ intersects the parallels $FG$, $EK$,
the alternate angles $FGH$, $GHK$ are equal [\textsc{xxix.}]: to
each add the angle $HGI$, and we have the sum of the
angles $FGH$, $HGI$ equal to the sum of the angles $GHK$,
$HGI$; but since $GI$ is parallel to $HK$, and $GH$ intersects
them, the sum of the angles $GHK$, $HGI$ is equal
to two right angles [\textsc{xxix.}]. Hence the sum of the

angles $FGH$, $HGI$ is two right angles; therefore $FG$
and $GI$ are in the same right line [\textsc{xiv}.].

Again, because $EG$ and $HI$ are parallelograms, $EF$
and $KI$ are each parallel to $GH$; hence [\textsc{xxx}.] $EF$ is
parallel to $KI$, and the opposite sides $EK$ and $FI$ are
parallel; therefore $EI$ is a parallelogram; and because
the parallelogram $EG$ (const.) is equal to the triangle
$ABD$, and $HI$ to the triangle $BCD$, the whole parallelogram
$EI$ is equal to the rectilineal figure $ABCD$, and
it has the angle $E$ equal to the given angle $X$. \emph{Hence
$EI$ is a parallelogram fulfilling the required conditions.}\par\medskip

\begin{footnotesize}
It would simplify Problems \textsc{xliv}., \textsc{xlv}., if they were stated as
the constructing of rectangles, and in this special form they
would be better understood by the student, since rectangles are
the simplest areas to which others are referred.
\par\end{footnotesize}


\exhead{Exercises.}

\begin{footnotesize}
1.~Construct a rectangle equal to the sum of two or any
number of rectilineal figures.

2.~Construct a rectangle equal to the difference of two given
figures.
\par\end{footnotesize}


\myprop{PROP\@.~XLVI\@.---Problem.}{On a given right line $(AB)$ to describe a square.}


\imgflow{90}{9}{f255}

\textbf{Sol.}---Erect $AD$ at right angles to $AB$ [\textsc{xi}.], and
make it equal to $AB$ [\textsc{iii}.]. Through
$D$ draw $DC$ parallel to $AB$ [\textsc{xxxi}.],
and through $B$ draw $BC$ parallel to
$AD$; \emph{then $AC$ is the square required.}

\textbf{Dem.}---Because $AC$ is a parallelogram,
$AB$ is equal to $CD$ [\textsc{xxxiv}.];
but $AB$ is equal to $AD$ (const.); therefore
$AD$ is equal to $CD$, and $AD$ is
equal to $BC$ [\textsc{xxxiv}.]. Hence the four sides are equal;
therefore $AC$ is a lozenge, and the angle $A$ is a right
angle. \emph{Therefore $AC$ is a square} (Def.~\textsc{xxx}.).




\exhead{Exercises.}

\begin{footnotesize}
1.~The squares on equal lines are equal; and, conversely, the
sides of equal squares are equal.

2.~The parallelograms about the diagonal of a square are
squares.

3.~If on the four sides of a square, or on the sides produced,
points be taken equidistant from the four angles, they will be the
angular points of another square, and similarly for a regular
pentagon, hexagon, \&c.

4.~Divide a given square into five equal parts; namely, four
right-angled triangles, and a square.
\par\end{footnotesize}


\myprop{PROP\@.~XLVII\@.---Theorem.}{In a right-angled triangle $(ABC)$ the square on the
hypotenuse $(AB)$ is equal to the sum of the squares on the other two sides $(AC,\ BC)$.}


\imgflow{175}{15}{f068}

\textbf{Dem.}---On the sides $AB$, $BC$, $CA$ describe squares
[\textsc{xlvi}.]. Draw $CL$ parallel
to $AG$. Join $CG$,
$BK$. Then because the
angle $ACB$ is right
(hyp.), and $ACH$ is
right, being the angle
of a square, the sum of
the angles $ACB$, $ACH$
is two right angles;
therefore $BC$, $CH$ are
in the same right line
[\textsc{xiv}.]. In like manner
$AC$, $CD$ are in the
same right line. Again,
because $BAG$ is the
angle of a square it is a right angle: in like manner
$CAK$ is a right angle. Hence $BAG$ is equal to $CAK$:
to each add $BAC$, and we get the angle $CAG$ equal to

$KAB$. Again, since $BG$ and $CK$ are squares, $BA$ is
equal to $AG$, and $CA$ to $AK$. Hence the two triangles
$CAG$, $KAB$ have the sides $CA$, $AG$ in one respectively
equal to the sides $KA$, $AB$ in the other, and the contained
angles $CAG$, $KAB$ also equal. Therefore [\textsc{iv}.]
the triangles are equal; but the parallelogram $AL$ is
double of the triangle $CAG$ [\textsc{xli}.], because they are on
the same base $AG$, and between the same parallels $AG$
and $CL$. In like manner the parallelogram
$AH$ is double
of the triangle $KAB$, because they are on the same base
$AK$, and between the same parallels $AK$ and $BH$; and
since doubles of equal things are equal (Axiom \textsc{vi}.),
the parallelogram $AL$ is equal to $AH$. In like manner
it can be proved that the parallelogram $BL$ is equal to
$BD$. \emph{Hence the whole square $AF$ is equal to the sum of
the two squares $AH$ and $BD$.}\par\newpage 


\imgflow{115}{15}{f069}

\begin{footnotesize}

\emph{Or thus:} Let all the squares be made in reversed directions.
Join $CG$, $BK$, and through $C$ draw $OL$
parallel to $AG$. Now, taking the $\angle BAC$
from the right $\angle$s $BAG$, $CAK$, the remaining
$\angle$s $CAG$, $BAK$ are equal.
Hence the $\triangle$s $CAG$, $BAK$ have the side
$CA = AK$, and $AG = AB$, and the
$\angle CAG= BAK$; therefore [\textsc{iv}.] they are
equal; and since [\textsc{xli}.] the $\pgram$s $AL$, $AH$
are respectively the doubles of these
triangles, they are equal. In like manner
the $\pgram$s  $BL$, $BD$ are equal; hence
the whole square $AF$ is equal to the
sum of the two squares $AH$, $BD$.

This proof is shorter than the usual
one, since it is not necessary to prove
that $AC$, $CD$ are in one right line. In a similar way the
Proposition may be proved by taking any of the eight figures
formed by turning the squares in all possible directions. Another
simplification of the proof would be got by considering that the
point $A$ is such that one of the $\triangle$s $CAG$, $BAK$ can be turned
round it in its own plane until it coincides with the other; and
hence that they are congruent.
\par\end{footnotesize}




\exhead{Exercises.}

\begin{footnotesize}
1.~The square on $AC$ is equal to the rectangle $AB\,.\,AO$, and
the square on $BC = AB\,.\,BO$.

2.~The square on $CO = AO\,.\,OB$.

3.~$AC^2 - BC^2 = AO^2 - BO^2$.

4.~Find a line whose square shall be equal to the sum of two
given squares.

5.~Given the base of a triangle and the difference of the
squares of its sides, the locus of its vertex is a right line perpendicular
to the base.

6.~The transverse lines $BK$, $CG$ are perpendicular to each
other.

7.~If $EG$ be joined, its square is equal to $AC^2 + 4BC^2$.

8.~The square described on the sum of the sides of a right-angled
triangle exceeds the square on the hypotenuse by four
times the area of the triangle (\emph{see} fig., \textsc{xlvi}., Ex.~3).  More
generally, if the vertical angle of a triangle be equal to the angle
of a regular polygon of $n$ sides, then the regular polygon of
$n$ sides, described on a line equal to the sum of its sides, exceeds
the area of the regular polygon of $n$ sides described on the base
by $n$ times the area of the triangle.

9.~If $AC$ and $BK$ intersect in $P$, and through $P$ a line be drawn
parallel to $BC$, meeting $AB$ in $Q$; then $CP$ is equal to $PQ$.

10.~Each of the triangles $AGK$ and $BEF$, formed by joining
adjacent corners of the squares, is equal to the right-angled triangle
$ABC$.

11.~Find a line whose square shall be equal to the difference
of the squares on two lines.

12.~The square on the difference of the sides $AC$, $CB$ is less
than the square on the hypotenuse by four times the area of the
triangle.

13.~If $AE$ be joined, the lines $AE$, $BK$, $CL$, are concurrent.

14.~In an equilateral triangle, three times the square on any
side is equal to four times the square on the perpendicular to it
from the opposite vertex.

15.~On $BE$, a part of the side $BC$ of a square $ABCD$, is
described the square $BEFG$, having its side $BG$ in the continuation
of $AB$; it is required to divide the figure $AGFECD$ into
three parts which will form a square.

16.~Four times the sum of the squares on the medians which
bisect the sides of a right-angled triangle is equal to five times
the square on the hypotenuse.


17.~If perpendiculars be let fall on the sides of a polygon from
any point, dividing each side into two segments, the sum of the
squares on one set of alternate segments is equal to the sum of the
squares on the remaining set.

18.~The sum of the squares on lines drawn from any point to
one pair of opposite angles of a rectangle is equal to the sum of
the squares on the lines from the same point to the remaining
pair.

19.~Divide the hypotenuse of a right-angled triangle into two
parts, such that the difference between their squares shall be
equal to the square on one of the sides.

20.~From the extremities of the base of a triangle perpendiculars
are let fall on the opposite sides; prove that the sum of the
rectangles contained by the sides and their lower segments is
equal to the square on the base.
\par\end{footnotesize}


\mypropl{PROP\@.~XLVIII\@.---Theorem.}{If the square on one side $(AB)$ of a triangle be equal to
the sum of the squares on the remaining sides $(AC,\ CB)$,
the angle $(C)$ opposite to that side is a right angle.}


\imgflow{107}{9}{f070}

\textbf{Dem.}---Erect $CD$ at right angles to $CB$ [\textsc{xi}.], and
make $CD$ equal to $CA$ [\textsc{iii}.]. Join
$BD$. Then because $AC$ is equal
to $CD$, the square on $AC$ is equal
to the square on $CD$: to each add
the square on $CB$, and we have the
sum of the squares on $AC$, $CB$
equal to the sum of the squares
on $CD$, $CB$; but the sum of the
squares on $AC$, $CB$ is equal to the
square on $AB$ (hyp.), and the sum of the squares on
$CD$, $CB$ is equal to the square on $BD$ [\textsc{xlvii}.]. Therefore
the square on $AB$ is equal to the square on $BD$.
Hence $AB$ is equal to $BD$ [\textsc{xlvi}., Ex.~1]. Again, because
$AC$ is equal to $CD$ (const.), and $CB$ common to
the two triangles $ACB$, $DCB$, and the base $AB$ equal
to the base $DB$, the angle $ACB$ is equal to the angle
$DCB$; but the angle $DCB$ is a right angle (const.).
\emph{Hence the angle $ACB$ is a right angle.}\par\medskip




\imgflow{125}{7}{f071}

\begin{footnotesize}
The foregoing proof forms an exception to Euclid's demonstrations
of converse propositions, for it
is direct. The following is an indirect
proof:---If $CB$ be not at right
angles to $AC$, let $CD$ be perpendicular
to it. Make $CD = CB$. Join $AD$.
Then, as before, it can be proved
that $AD$ is equal to $AB$, and $CD$ is
equal to $CB$ (const.). This is contrary
to Prop.~\textsc{vii}. \emph{Hence the angle $ACB$ is a right angle.}
\par\end{footnotesize}

\addcontentsline{toc}{section}{Questions for Examination,}
\exhead{Questions for Examination on Book I.}

\begin{footnotesize}
1.~What is Geometry?

2.~What is geometric magnitude? \emph{Ans}.~That which has extension
in space.

3.~Name the primary concepts of geometry. \emph{Ans}.~Points,
lines, surfaces, and solids.

4.~How may lines be divided? \emph{Ans}.~Into straight and curved.

5.~How is a straight line generated? \emph{Ans}.~By the motion of
a point which has the same direction throughout.

6.~How is a curved line generated? \emph{Ans}.~By the motion of a
point which continually changes its direction.

7.~How may surfaces be divided? \emph{Ans}.~Into planes and
curved surfaces.

8.~How may a plane surface be generated. \emph{Ans}.~By the
motion of a right line which crosses another right line, and
moves along it without changing its direction.

9.~Why has a point no dimensions?

10.~Why has a line neither breadth nor thickness?

11.~How many dimensions has a surface?

12.~What is Plane Geometry?

13.~What portion of plane geometry forms the subject of the
``First Six Books of Euclid's Elements''? \emph{Ans}.~The geometry
of the \emph{point}, \emph{line}, and \emph{circle}.

14.~What is the subject-matter of Book I.?

15.~How many conditions are necessary to fix the position of a
point in a plane? \emph{Ans}.~Two; for it must be the intersection of
two lines, straight or curved.

16.~Give examples taken from Book I.


17.~In order to construct a line, how many conditions must be
given? \textit{Ans}.\ Two; as, for instance, two points through which
it must pass; or one point through which it must pass and a line
to which it must be parallel or perpendicular, \&c.

18.~What problems on the drawing of lines occur in Book I.?
\textit{Ans.}\ \textsc{ii., ix., xi., xii., xxiii., xxxi.,} in each of which, except
Problem 2, there are two conditions. The direction in
Problem 2 is indeterminate.

19.~How many conditions are required in order to describe a
circle? \textit{Ans.}\ Three; as, for instance, the position of the centre
(which depends on two conditions) and the length of the radius
(compare Post.\ \textsc{iii.}).

20.~How is a proposition proved indirectly? \textit{Ans.}\ By proving
that its contradictory is false.

21.~What is meant by the obverse of a proposition?

22.~What propositions in Book I. are the obverse respectively
of Propositions \textsc{iv., v., vi., xxvii.}?

23.~What proposition is an instance of the \textit{rule of identity}?

24.~What are congruent figures?

25.~What other name is applied to them? \textit{Ans.}\ They are
said to be identically equal.

26.~Mention all the instances of equality which are not
congruence that occur in Book I.

27.~What is the difference between the symbols denoting
congruence and identity?

28.~Classify the properties of triangles and parallelograms
proved in Book I.

29.~What proposition is the converse of Prop.\ \textsc{xxvi.}, Part I.?

30.~Define \textit{adjacent, exterior, interior, alternate} angles
respectively.

31.~What is meant by the projection of one line on another?

32.~What are meant by the medians of a triangle?

33.~What is meant by the third diagonal of a quadrilateral?

34.~Mention some propositions in Book I. which are particular
cases of more general ones that follow.

35.~What is the sum of all the exterior angles of any
rectilineal figure equal to?

36.~How many conditions must be given in order to construct
a triangle? \textit{Ans.}\ Three; such as the three sides, or two sides
and an angle, \&c.
\par\end{footnotesize}


\addcontentsline{toc}{section}{Exercises,}
\exhead{Exercises on Book I.}

\begin{footnotesize}
1.~Any triangle is equal to the fourth part of that which is
formed by drawing through each vertex a line parallel to its
opposite side.

2.~The three perpendiculars of the first triangle in question 1
are the perpendiculars at the middle points of the sides of the
second triangle.

3.~Through a given point draw a line so that the portion
intercepted by the legs of a given angle may be bisected in the point.

4.~The three medians of a triangle are concurrent.

5.~The medians of a triangle divide each other in the ratio of
$2:1$.

6.~Construct a triangle, being given two sides and the median
of the third side.

7.~In every triangle the sum of the medians is less than the
perimeter, and greater than three-fourths of the perimeter.

8.~Construct a triangle, being given a side and the two
medians of the remaining sides.

9.~Construct a triangle, being given the three medians.

10.~The angle included between the perpendicular from the
vertical angle of a triangle on the base, and the bisector of the
vertical angle, is equal to half the difference of the base angles.

11.~Find in two parallels two points which shall be equidistant
from a given point, and whose line of connexion shall be parallel
to a given line.

12.~Construct a parallelogram, being given two diagonals and
a side.

13.~The smallest median of a triangle corresponds to the
greatest side.

14.~Find in two parallels two points subtending a right angle
at a given point and equally distant from it.

15.~The sum of the distances of any point in the base of an
isosceles triangle from the equal sides is equal to the distance of
either extremity of the base from the opposite side.

16.~The three perpendiculars at the middle points of the sides
of a triangle are concurrent. Hence prove that perpendiculars
from the vertices on the opposite sides are concurrent [\textit{see}
Ex.~2].

17.~Inscribe a lozenge in a triangle having for an angle one
angle of the triangle.


18.~Inscribe a square in a triangle having its base on a side of
the triangle.

19.~Find the locus of a point, the sum or the difference of
whose distance from two fixed lines is equal to a given length.

20.~The sum of the perpendiculars from any point in the
interior of an equilateral triangle is equal to the perpendicular from
any vertex on the opposite side.

21.~The distance of the foot of the perpendicular from either
extremity of the base of a triangle on the bisector of the vertical
angle, from the middle point of the base, is equal to half the
difference of the sides.

22.~In the same case, if the bisector of the external vertical
angle be taken, the distance will be equal to half the sum of the
sides.

23.~Find a point in one of the sides of a triangle such that the
sum of the intercepts made by the other sides, on parallels drawn
from the same point to these sides, may be equal to a given length.

24.~If two angles have their legs respectively parallel, their
bisectors are either parallel or perpendicular.

25.~If lines be drawn from the extremities of the base of a
triangle to the feet of perpendiculars let fall from the same points
on either bisector of the vertical angle, these lines meet on the
other bisector of the vertical angle.

26.~The perpendiculars of a triangle are the bisectors of the
angles of the triangle whose vertices are the feet of these
perpendiculars.

27.~Inscribe in a given triangle a parallelogram whose diagonals
shall intersect in a given point.

28.~Construct a quadrilateral, the four sides being given in
magnitude, and the middle points of two opposite sides being
given in position.

29.~The bases of two or more triangles having a common
vertex are given, both in magnitude and position, and the sum of the
areas is given; prove that the locus of the vertex is a right line.

30.~If the sum of the perpendiculars let fall from a given
point on the sides of a given rectilineal figure be given, the locus
of the point is a right line.

31.~$ABC$ is an isosceles triangle whose equal sides are $AB$, $AC$;
$B'C'$ is any secant cutting the equal sides in $B'$, $C'$, so that
$AB' + AC' = AB + AC$: prove that $B'C'$ is greater than $BC$.

32.~$A$, $B$ are two given points, and $P$ is a point in a given line
$L$; prove that the difference of $AP$ and $PB$ is a maximum when
$L$ bisects the angle $APB$; and that their sum is a minimum if it
bisects the supplement.


33.~Bisect a quadrilateral by a right line drawn from one of
its angular points.

34.~$AD$ and $BC$ are two parallel lines cut obliquely by $AB$, and
perpendicularly by $AC$; and between these lines we draw $BED$,
cutting $AC$ in $E$, such that $ED = 2AB$; prove that the angle $DBC$
is one-third of $ABC$.

35.~If $O$ be the point of concurrence of the bisectors of the
angles of the triangle $ABC$, and if $AO$ produced meet $BC$ in $D$,
and from $O$, $OE$ be drawn perpendicular to $BC$; prove that the
angle $BOD$ is equal to the angle $COE$.

36.~If the exterior angles of a triangle be bisected, the three
external triangles formed on the sides of the original triangle are
equiangular.

37.~The angle made by the bisectors of two consecutive angles
of a convex quadrilateral is equal to half the sum of the
remaining angles; and the angle made by the bisectors of two opposite
angles is equal to half the difference of the two other angles.

38.~If in the construction of the figure, Proposition \textsc{xlvii.},
$EF$, $KG$ be joined,
\[
EF^2 + KG^2 = 5AB^2.
\]

39.~Given the middle points of the sides of a convex polygon
of an odd number of sides, construct the polygon.

40.~Trisect a quadrilateral by lines drawn from one of its
angles.

41.~Given the base of a triangle in magnitude and position and
the sum of the sides; prove that the perpendicular at either extremity
of the base to the adjacent side, and the external bisector
of the vertical angle, meet on a given line perpendicular to the
base.

42.~The bisectors of the angles of a convex quadrilateral form
a quadrilateral whose opposite angles are supplemental. If the
first quadrilateral be a parallelogram, the second is a rectangle;
if the first be a rectangle, the second is a square.

43.~The middle points of the sides $AB$, $BC$, $CA$ of a triangle
are respectively $D$, $E$, $F$; $DG$ is drawn parallel to $BF$ to meet
$EF$; prove that the sides of the triangle $DCG$ are respectively
equal to the three medians of the triangle $ABC$.

44.~Find the path of a billiard ball started from a given point
which, after being reflected from the four sides of the table, will
pass through another given point.


45.~If two lines bisecting two angles of a triangle and terminated
by the opposite sides be equal, the triangle is isosceles.

46.~State and prove the Proposition corresponding to Exercise
41, when the base and difference of the sides are given.

47.~If a square be inscribed in a triangle, the rectangle under
its side and the sum of the base and altitude is equal to twice the
area of the triangle.

48.~If $AB$, $AC$ be equal sides of an isosceles triangle, and if
$BD$ be a perpendicular on $AC$; prove that $BC^2 = 2AC\,.\,CD$.

49.~The sum of the equilateral triangles described on the legs
of a right-angled triangle is equal to the equilateral triangle described
on the hypotenuse.

50.~Given the base of a triangle, the difference of the base
angles, and the sum or difference of the sides; construct it.

51.~Given the base of a triangle, the median that bisects the
base, and the area; construct it.

52.~If the diagonals $AC$, $BD$ of a quadrilateral $ABCD$ intersect
in $E$, and be bisected in the points $F$, $G$, then
\[
4 \; \triangle \; EFG = (AEB + ECD)-(AED + EBC).
\]

53.~If squares be described on the sides of any triangle, the
lines of connexion of the adjacent corners are respectively---(1) the
doubles of the medians of the triangle; (2) perpendicular to
them.
\par\end{footnotesize}




\newpage\begin{center}{\LARGE BOOK II\@.} \bigskip\\{\large \uppercase{Theory of Rectangles}}\\ \bigskip
\addtocontents{toc}{\bigskip\protect\centering {\large BOOK II\@.}\medskip}
\addcontentsline{toc}{section}{\protect \ibksp\textsf{Theory of Rectangles,}}
\end{center}

\noindent Every Proposition in the Second Book has either a
square or a rectangle\index{Rectangle} in its enunciation. Before commencing
it the student should read the following preliminary
explanations: by their assistance it will be
seen that this Book, which is usually considered difficult,
will be rendered not only easy, but almost intuitively
evident.\par\medskip

\begin{footnotesize}
1.~As the linear unit is that by which we express all
linear measures, so the square unit is that to which all
superficial measures are referred. Again, as there are
different linear units in use, such as in this country,
inches, feet, yards, miles, \&c., and in France, metres,
and their multiples or sub-multiples, so different square
units are employed.

2.~A square unit is the square described on a line
whose length is the linear unit. Thus a square inch is
the square described on a line whose length is an inch;
a square foot is the square described on a line whose
length is a foot, \&c.

3.~If we take a linear foot, describe a square on it,
divide two adjacent sides each into twelve equal parts,
and draw parallels to the sides, we evidently divide the
square foot into square inches; and as there will manifestly
be 12 rectangular parallelograms, each containing
12 square inches, the square foot contains 144
square inches.


In the same manner it can be shown that a square
yard contains 9 square feet; and so in general the
square described on any line contains $n^2$ times the
square described on the $n^{th}$ part of the line. Thus, as
a simple case, the square on a line is four times the
square on its half. On account of this property the
second power of a quantity is called its square; and,
conversely, the square on a line $AB$ is expressed symbolically
by $AB^2$.

4.~If a rectangular parallelogram be such that two
adjacent sides contain respectively $m$ and $n$ linear units,
by dividing one side into $m$ and the other into $n$ equal
parts, and drawing parallels to the sides, the whole area
is evidently divided into $mn$ square units. Hence the
area of the parallelogram is found by multiplying its
length by its breadth, and this explains why we say
(\emph{see} Def.~\textsc{iv}.) a rectangle is contained by any two adjacent
sides; for if we multiply the length of one by
the length of the other we have the area. Thus, if
$AB$, $AD$ be two adjacent sides of a rectangle, the rectangle
is expressed by $AB\,.\,AD$.
\par\end{footnotesize}

\bigskip
\begin{center}
\textsc{Definitions.}
\end{center}
\addcontentsline{toc}{section}{Definitions,}


\imgflow{120}{2}{f072}

\textsc{i}. If a point $C$ be taken in a line $AB$, the parts $AC$,
$CB$ are called \emph{segments}\index{Lines!segment of}, and $C$ a
\emph{point of division}.


\imgflow{140}{2}{f073}

\textsc{ii}. If C be taken in the line $AB$ produced, $AC$, $CB$ are
still called the segments of
the line $AB$; but $C$ is called
a point of \emph{external} division.

\textsc{iii}. A parallelogram whose angles are right angles is
called a \emph{rectangle}.


\imgflow{135}{8}{f074}

\textsc{iv}. A rectangle is said to
be contained by any two adjacent
sides. Thus the rectangle
$ABCD$ is said to be contained
by $AB$, $AD$, or by $AB$, $BC$,
\&c.


\textsc{v}. The rectangle contained by two separate lines
such as $AB$ and
$CD$ is the parallelogram
formed by erecting a perpendicular to $AB$, at
$A$, equal to $CD$, and drawing parallels: the area of the
rectangle will be $AB\,.\,CD$.


\imgcent{208}{f075}


\imgflow{140}{8}{f076}

\textsc{vi}. In any parallelogram the figure which is composed
of either of the parallelograms
about a diagonal
and the two complements
[\emph{see} I., \textsc{xliii}.] is called a
gnomon. Thus, if we take
away either of the parallelograms
$AO$, $OC$ from the
parallelogram $AC$, the remainder
is called a \emph{gnomon}.\index{Gnomon}


\addcontentsline{toc}{section}{Propositions \textsc{i.--xiv.,}}
\mypropl{PROP\@.~I.---Theorem.}{If there be two lines $(A,\ BC)$, one of which is divided
into any number of parts $(BD,\ DE,\ EC)$, the rectangle
contained by the two lines $(A,\ BC)$, is equal to the sum of
the rectangles contained by the undivided line $(A)$ and the
several parts of the divided line.}


\imgflow{185}{8}{f077}

\textbf{Dem.}---Erect $BF$ at right angles to $BC$ [I., \textsc{xi}.] and
make it equal to $A$. Complete the parallelogram $BK$
(Def.~\textsc{v}.). Through
$D$, $E$ draw $DG$, $EH$
parallel to $BF$. Because
the angles at
$B$, $D$, $E$ are right
angles, each of the
quadrilaterals $BG$,
$DH$, $EK$ is a rectangle.
Again, since $A$ is equal to $BF$ (const.), the
rectangle contained by $A$ and $BC$ is the rectangle contained
by $BF$ and $BC$ (Def.~\textsc{v}.); but $BK$ is the rectangle
contained by $BF$ and $BC$. Hence the rectangle

contained by $A$ and $BC$ is $BK$. In like manner the
rectangle contained by $A$ and $BD$ is $BG$. Again, since
$A$ is equal to $BF$ (const.), and $BF$ is equal to $DG$
[I.~\textsc{xxxiv}.], $A$ is equal to $DG$. Hence the rectangle
contained by $A$ and $DE$ is the figure $DH$ (Def.~\textsc{v}.). In
like manner the rectangle contained by $A$ and $EC$ is
the figure $EK$. Hence we have the following identities:---
\begin{center}
\begin{tabular}{c@{\ }c@{\ }r@{\ }c@{\ }c}
Rectangle & contained & by $A$ & and & $BD \equiv BG$.  \\
    ,,    &     ,,    &    $A$ & ,,  & $DE \equiv DH$.  \\
    ,,    &     ,,    &    $A$ & ,,  & $EC \equiv EK$.  \\
    ,,    &     ,,    &    $A$ & ,,  & $BC \equiv BK$.
\end{tabular}
\end{center}

But $BK$ is equal to the sum of $BG$, $DH$, $EK$
(I., Axiom~\textsc{ix}.). \emph{Therefore the rectangle contained by $A$
and $BC$ is equal to the sum of the rectangles contained by
$A$ and $BD$, $A$ and $DE$, $A$ and $EC$.}\par\medskip

\begin{footnotesize}
If we denote the lines $BD$, $DE$, $EC$ by $a$, $b$, $c$, the Proposition
asserts that the rectangle contained by $A$, and $a+b+c$ is equal to
the sum of the rectangles contained by $A$ and $a$, $A$ and $b$, $A$ and $c$,
or, as it may be written, $A (a+b+c) = Aa + Ab + Ac$. This corresponds
to the distributive law in multiplication, and shows that
rectangles in Geometry, and products in Arithmetic and Algebra,
are subject to the same rules.\par\medskip

\textbf{Illustration.}---Suppose $A$ to be 6 inches; $BD$, 5 inches;
$DE$, 4 inches; $EC$, 3 inches; then $BC$ will be 12 inches; and
the rectangles will have the following values:---
\begin{center}
\begin{tabular}{c@{\ }c@{\ }c}
Rectangle & $A\,.\,BC = 6\times 12 = 72$ & square inches.  \\
    ,,    & $A\,.\,BD = 6\times \phantom{1}5 = 30$ &  ,,   \\
    ,,    & $A\,.\,DE = 6\times \phantom{1}4 = 24$ &  ,,   \\
    ,,    & $A\,.\,EC = 6\times \phantom{1}3 = 18$ &  ,,
\end{tabular}
\end{center}

Now the sum of the three last rectangles, viz. 30, 24, 18, is 72.
Hence the rectangle $A\,.\,BC = A\,.\,BD + A\,.\,DE + A\,.\,EC$.
\par\end{footnotesize}\medskip

The Second Book is occupied with the relations between
the segments of a line divided in various ways.
All these can be proved in the most simple manner
by Algebraic Multiplication. We recommend the student

to make himself acquainted with the proofs by
this method as well as with those of Euclid. He will
thus better understand the meaning of each Proposition.\par\medskip

\emph{Cor.~\textrm{1}.---The rectangle contained by a line and the
difference of two others is equal to the difference of the
rectangles contained by the line and each of the others.}

\emph{Cor.~\textrm{2}.---The area\index{Area} of a triangle is equal to half the
rectangle contained by its base and perpendicular.}\par\medskip


\imgflow{120}{7}{f078}

\textbf{Dem.}---From the vertex $C$ let fall the perpendicular
$CD$. Draw $EF$ parallel to $AB$,
and $AE$, $BF$ each parallel to
$CD$. Then $AF$ is the rectangle
contained by $AB$ and $BF$;
but $BF$ is equal to $CD$. Hence
$AF = AB\,.\,CD$; but [I.~\textsc{xli}.]
the triangle $ABC$ is $=$ half the
parallelogram $AF$. \emph{Therefore
the triangle $ABC$ is $=\tfrac{1}{2}AB\,.\,CD$.}

\mypropl{PROP\@.~II\@.---Theorem.}{If a line $(AB)$ be divided into any two parts $($at $C)$,
the square on the whole line is equal to the sum of the
rectangles contained by the whole and each of the segments
$(AC,\ CB)$.}


\imgflow{100}{10}{f079}

\textbf{Dem.}---On $AB$ describe the square $ABDF$ [I.~\textsc{xlvi}.],
and through $C$ draw $CE$ parallel to
$AF$ [I.~\textsc{xxxi}.]. Now, since $AB$ is
equal to $AF$, the rectangle contained
by $AB$ and $AC$ is equal to
the rectangle contained by $AF$ and
$AC$; but $AE$ is the rectangle contained
by $AF$ and $AC$. \emph{Hence the
rectangle contained by $AB$ and $AC$
is equal to $AE$. In like manner
the rectangle contained by $AB$ and $CB$ is equal to the

figure $CD$. Therefore the sum of the two rectangles
$AB\,.\,AC$, $AB\,.\,CB$ is equal to the square on $AB$}.
\begin{footnotesize}
\begin{flalign*}
&\text{\indent\emph{Or thus: }}& AB &= AC + CB,  &&\\
&\text{and }& AB &= AB.  &&\\
&\text{Hence, multiplying, we get }& AB^2 &= AB\,.\,AC + AB\,.\,CB.
&\phantom{Hence,\ multiplying,\ we\ get\ }&
\end{flalign*}

This Proposition is the particular case of \textsc{i}.\ when the divided
and undivided lines are equal, hence it does not require a separate
Demonstration.
\par\end{footnotesize}

\mypropl{PROP\@.~III\@.---Theorem.}{If a line $(AB)$ be divided into two segments $($at $C)$, the
rectangle contained by the whole line and either segment
$(CB)$ is equal to the square on that segment together with
the rectangle contained by the segments.}


\imgflow{130}{6}{f080}

\textbf{Dem.}---On $BC$ describe the square $BCDE$ [I.~\textsc{xlvi}.].
Through $A$ draw $AF$ parallel
to $CD$: produce $ED$ to meet
$AF$ in $F$. Now since $CB$ is
equal to $CD$, the rectangle
contained by $AC$, $CB$ is equal
to the rectangle contained by
$AC$, $CD$; but the rectangle contained by $AC$, $CD$ is the
figure $AD$. Hence the rectangle $AC\,.\,CB$ is equal to
the figure $AD$, and the square on $CB$ is the figure $CE$.
Hence the rectangle $AC\,.\,CB$, together with the square
on $CB$, is equal to the figure $AE$.\par\medskip


Again, since $CB$ is equal to $BE$, the rectangle $AB\,.\,CB$
is equal to the rectangle $AB\,.\,BE$; but the rectangle
$AB\,.\,BE$ is equal to the figure $AE$. Hence the rectangle
$AB\,.\,CB$ is equal to the figure $AE$. And since
things which are equal to the same are equal to one
another, \emph{the rectangle $AC\,.\,CB$, together with the square
on $CB$, is equal to the rectangle $AB\,.\,CB$}.
\begin{footnotesize}
\begin{flalign*}
&\text{\indent\emph{Or thus:}}&  AB &= AC + CB,  &&\\
&&                  CB &= CB.  &&\\
&\text{Hence }& AB\,.\,CB &= AC\,.\,CB + CB^2. &\phantom{\indent Or\ thus:\ }
\end{flalign*}

Prop.~\textsc{iii}.\ is the particular case of Prop.~\textsc{i}., when the undivided
line is equal to a segment of the divided line.
\par\end{footnotesize}


\mypropl{PROP\@.~IV\@.---Theorem.}{If a line $(AB)$ be divided into any two parts $($at $C)$,
the square on the whole line is equal to the sum of the
squares on the parts $(AC,\ CB)$, together with twice their
rectangle.}


\imgflow{100}{9}{f081}

\textbf{Dem.}---On $AB$ describe a square $ABDE$. Join $EB$;
through $C$ draw $CF$ parallel to $AE$,
intersecting $BE$ in $G$; and through
$G$ draw $HI$ parallel to $AB$.

Now since $AE$ is equal to $AB$,
the angle $ABE$ is equal to $AEB$
[I.~\textsc{v}.]; but since $BE$ intersects the
parallels $AE$, $CF$, the angle $AEB$ is
equal to $CGB$ [I.~\textsc{xxix}.]. Hence
the angle $CBG$ is equal to $CGB$,
and therefore [I.~\textsc{vi}.] $CG$ is equal to $CB$; but $CG$ is
equal to $BI$ and $CB$ to $GI$. Hence the figure $CBIG$
is a lozenge, and the angle $CBI$ is right. Hence
(I., Def.~\textsc{xxx}.) it is a square. In like manner the
figure $EFGH$ is a square.

Again, since $CB$ is equal to $CG$, the rectangle $AC\,.\,CB$
is equal to the rectangle $AC\,.\,CG$; but $AC\,.\,CG$ is the
figure $AG$ (Def.~\textsc{iv}.). Therefore the rectangle $AC\,.\,CB$
is equal to the figure $AG$. Now the figures $AG$, $GD$
are equal [I.~\textsc{xliii}.], being the complements about the
diagonal of the parallelogram $AD$. Hence the parallelograms
$AG$, $GD$ are together equal to twice the
rectangle $AC\,.\,CB$. Again, the figure $HF$ is the square
on $HG$, and $HG$ is equal to $AC$. Therefore $HF$ is
equal to the square on $AC$, and $CI$ is the square on
$CB$; but the whole figure $AD$, which is the square
on $AB$, is the sum of the four figures $HF$, $CI$, $AG$,
$GD$. \emph{Therefore the square on $AB$ is equal to the sum
of the squares on $AC$, $CB$, and twice the rectangle
$AC\,.\,CB$.}\par\medskip



\imgflow{105}{10}{f082}

\begin{footnotesize}
\textit{Or thus:} On $AB$ describe the square $ABDE$, and cut off $AH$,
$EG$, $DF$ each equal to $CB$. Join $CF$, $FG$,
$GH$, $HC$. Now the four $\triangle$s $ACH$, $CBF$,
$FDG$, $GEH$ are evidently equal; therefore
their sum is equal to four times the
$\triangle ACH$; but the $\triangle ACH$ is half the
rectangle $AC\,.\,AH$ (\textsc{i}.\ \textit{Cor.}\ 2)\label{1cor2}---that is, equal
to half the rectangle $AC\,.\,CB$. Therefore
the sum of the four triangles is equal to
$2AC\,.\,CB$.

Again, the figure $CFGH$ is a square
[I.\ \textsc{xlvi}., \textit{Cor.}\ 3], and equal to $AC^2+ AH^2$
[I.\ \textsc{xlvii}.]---that is, equal to $AC^2+CB^2$. \textit{Hence the whole figure
$ABDE=AC^2+CB^2+2AC\,.\,CB$.}\par\medskip

\begin{flalign*}
&\text{\indent\emph{Or thus:}}&  AB &= AC + CB.&   \\
&\text{Squaring, we get }&  AB^2 = &AC^2 + 2AC\,.\,CB + CB^2.&  \phantom{\text{Squaring, we get }}
\end{flalign*}
\end{footnotesize}\par\medskip

\textit{Cor.}\ 1.---\textit{The parallelograms about the diagonal of a
square are squares.}


\imgflow{120}{7}{f083}

\textit{Cor.}\ 2.---\textit{The square on a line is equal to four times the
square on its half.}

For let $AB = 2AC$, then $AB^2= 4AC^2$.

This \textit{Cor.}\ may be proved by
the First Book thus: Erect $CD$
at right angles to $AB$, and make
$CD = AC$ or $CB$. Join $AD$, $DB$.

\begin{flalign*}
&\text{\indent Then}&  AD^2 = AC^2 &+ CD^2=2 AC^2&  \\
&\text{\indent In like manner,}&  DB^2 &= 2 CB^2;&  \phantom{\text{In like manner,}} \\
&\text{therefore}&  AD^2 + DB^2 &= 2AC^2 + 2CB^2 = 4AC^2.&
\end{flalign*}

But since the angle $ADB$ is right, $AD^2 + DB^2 = AB^2$;
\begin{flalign*}
&\text{therefore}&  AB^2 &= 4AC^2.& \phantom{\text{foretherefore}}
\end{flalign*}

\textit{Cor.}\ 3.---\textit{If a line be divided into any number of parts,
the square on the whole is equal to the sum of the squares
on all the parts, together with twice the sum of the rectangles
contained by the several distinct pairs of parts.}



\exhead{Exercises.}

\begin{footnotesize}
1.~Prove Proposition \textsc{iv}.\ by using Propositions \textsc{ii}.\ and \textsc{iii}.

2.~If from the vertical angle of a right-angled triangle a perpendicular
be let fall on the hypotenuse, its square is equal to the
rectangle contained by the segments of the hypotenuse.

3.~From the hypotenuse of a right-angled triangle portions are
cut off equal to the adjacent sides; prove that the square on the
middle segment is equal to twice the rectangle contained by the
extreme segments.

4.~In any right-angled triangle the square on the sum of the
hypotenuse and perpendicular, from the right angle on the
hypotenuse, exceeds the square on the sum of the sides by the square
on the perpendicular.

5.~The square on the perimeter of a right-angled triangle is
equal to twice the rectangle contained by the sum of the
hypotenuse and one side, and the sum of the hypotenuse and the other
side.
\par\end{footnotesize}

\mypropl{PROP\@.~V.---Theorem.}{If a line $(AB)$ be divided into two equal parts $($at $C)$,
and also into two unequal parts $($at $D)$, the rectangle
$(AD\,.\,DB)$ contained by the unequal parts, together with
the square on the part $(CD)$ between the points of section,
is equal to the square on half the line.}


\imgflow{150}{7}{f084}

\textbf{Dem.}---On $CB$ describe the square $CBEF$ [I.~\textsc{xlvi}.].
Join $BF$. Through $D$ draw
$DG$ parallel to $CF$, meeting
$BF$ in $H$. Through $H$
draw $KM$ parallel to $AB$,
and through $A$ draw $AK$
parallel to $CL$ [I.~\textsc{xxxi}.].

The parallelogram $CM$
is equal to $DE$ [I.~\textsc{xliii}., \textit{Cor.}~2];\label{cor3} 
but $AL$ is equal to
$CM$ [I.~\textsc{xxxvi}.], because they are on equal bases $AC$,

$CB$, and between the same parallels; therefore $AL$ is
equal to $DE$: to each add $CH$, and we get the parallelogram
$AH$ equal to the gnomon $CMG$; but $AH$ is
equal to the rectangle $AD\,.\,DH$, and therefore equal
to the rectangle $AD\,.\,DB$, since $DH$ is equal to $DB$
[\textsc{iv}., \emph{Cor.}~1]; therefore the rectangle $AD\,.\,DB$ is
equal to the gnomon $CMG$, and the square on $CD$ is
equal to the figure $LG$. \textit{Hence the rectangle $AD\,.\,DB$,
together with the square on $CD$, is equal to the whole
figure $CBEF$---that is, to the square on $CB$.}

\begin{flalign*}
&\indent\textit{Or thus:} & AD &=          AC+CD =   BC+CD;  &&\\
&                         & DB &= \phantom{AC+CD =}\: BC-CD;  &&\\
&\text{therefore }&      AD\,.\,BD &= (BC + CD) (BC-CD) = BC^2-CD^2.  &&\\
&\text{Hence     }&\multispan{2}{$\hfill
  AD\,.\,BD + CD^2 = BC^2.  \phantom{= BC^2-CD^2.}\hfill$}
\end{flalign*}

\textit{Cor}.~1.---The rectangle $AD\,.\,DB$ is the rectangle
contained by the sum of the lines $AC$, $CD$ and their
difference; and we have proved it equal to the difference
between the square on $AC$ and the square on $CD$.
\textit{Hence the difference of the squares on two lines is equal
to the rectangle contained by their sum and their difference.}

\textit{Cor}.~2.---The perimeter of the rectangle $AH$ is equal
to $2AB$, and is therefore independent of the position of
the point $D$ on the line $AB$; and the area of the same
rectangle is less than the square on half the line by the
square on the segment between $D$ and the middle
point of the line; therefore, when $D$ is the middle
point, the rectangle will have the maximum area.
\textit{Hence, of all rectangles having the same perimeter, the
square has the greatest area.}



\exhead{Exercises.}

\begin{footnotesize}
1.~Divide a given line so that the rectangle contained by its
parts may have a maximum area.

2.~Divide a given line so that the rectangle contained by its
segments may be equal to a given square, not exceeding the
square on half the given line.

3.~The rectangle contained by the sum and the difference of
two sides of a triangle is equal to the rectangle contained by the
base and the difference of the segments of the base, made by the
perpendicular from the vertex.

4.~The difference of the sides of a triangle is less than the
difference of the segments of the base, made by the perpendicular
from the vertex.

5.~The difference between the square on one of the equal sides
of an isosceles triangle, and the square on any line drawn from
the vertex to a point in the base, is equal to the rectangle contained
by the segments of the base.

6.~The square on either side of a right-angled triangle is equal
to the rectangle contained by the sum and the difference of the
hypotenuse and the other side.
\par\end{footnotesize}

\mypropl{PROP\@. VI\@.---Theorem.}{If a line $(AB)$ be bisected $($at $C)$, and divided externally
in any point $(D)$, the rectangle $(AD\,.\,BD)$ contained
by the segments made by the external point, together with
the square on half the line, is equal to the square on the
segment between the middle point and the point of external
division.}


\imgflow{150}{9}{f085}

\textbf{Dem.}---On $CD$ describe the square $CDFE$ [I.~\textsc{xlvi}.],
and join $DE$; through $B$
draw $BHG$ parallel to $CE$
[I.~\textsc{xxxi}.], meeting $DE$ in
$H$; through $H$ draw $KLM$
parallel to $AD$; and through
$A$ draw $AK$ parallel to $CL$.
Then because $AC$ is equal to
$CB$, the rectangle $AL$ is equal
to $CH$ [I.~\textsc{xxxvi}.]; but the complements $CH$, $HF$ are

equal [I.~\textsc{xliii}.]; therefore $AL$ is equal to $HF$. To
each of these equals add $CM$ and $LG$, and we get $AM$
and $LG$ equal to the square $CDFE$; but $AM$ is equal to
the rectangle $AD\,.\,DM$, and therefore equal to the
rectangle $AD\,.\,DB$, since $DB$ is equal to $DM$; also $LG$
is equal to the square on $CB$, and $CDFE$ is the square
on $CD$. \textit{Hence the rectangle $AD\,.\,DB$, together with the
square on $CB$, is equal to the square on $CD$.} 

\Needspace*{4\baselineskip}\textit{Or thus:---}\par\medskip


\imgflow{140}{9}{f086}

\textbf{Dem.}---On $CB$ describe the square $CBEF$ [I. \textsc{xlvi}.].\label{Ixlvi}
Join $BF$. Through $D$ draw
$DG$ parallel to $CF$, meeting
$FB$ produced in $H$. Through
$H$ draw $KM$ parallel to $AB$.
Through $A$ draw $AK$ parallel 
to $CL$ [I.~\textsc{xxxi}.].

The parallelogram $CM$ is
equal to $DE$ [I.~\textsc{xliii}.]; but
$AL$ is equal to $CM$ [I.~\textsc{xxxvi}.],
because they are on equal bases $AC$, $CB$, and between
the same parallels; therefore $AL$ is equal to $DE$. To
each add $CH$, and we get the parallelogram $AH$ equal
to the gnomon $CMG$; but $AH$ is equal to the rectangle
$AD\,.\,DH$, and therefore equal to the rectangle $AD\,.\,DB$,
since $DH$ is equal to $DB$ [\textsc{iv.}, \textit{Cor.}~1]; therefore the
rectangle $AD\,.\,DB$ is equal to the gnomon $CMG$, and
the square on $CB$ is the figure $CE$. \textit{Therefore the rectangle
$AD\,.\,DB$, together with the square on $CB$, is equal
to the whole figure $LHGF$---that is, equal to the square
on $LH$ or to the square on $CD$.}

\begin{flalign*}
&\indent\textit{Or thus:}&
   AD ={}&          AC + CD =  CD + CB;  &&\phantom{Or\ thus}
\\
&& BD ={}& \phantom{AC + CD =\ } CD - CB.  &&\\
&\text{Hence }&
  AD\,.\,DB ={}& (CD+CB) (CD-CB) = CD^2-CB^2;  &&\\
&\text{therefore }& AD & \,.\,DB + CB^2 = CD^2.  &&
\end{flalign*}


\exhead{Exercises.}

\begin{footnotesize}
1.~Show that Proposition~\textsc{vi}.\ is reduced to Proposition~\textsc{v}.\ by
producing the line in the opposite direction.

2.~Divide a given line externally, so that the rectangle contained
by its segments may be equal to the square on a given
line.

3.~Given the difference of two lines and the rectangle contained
by them; find the lines.

4.~The rectangle contained by any two lines is equal to the
square on half the sum, minus the square on half the difference.

5.~Given the sum or the difference of two lines and the difference
of their squares; find the lines.

6.~If from the vertex $C$ of an isosceles triangle a line $CD$ be
drawn to any point in the base produced, prove that $CD^2-CB^2
= AD\,.\,DB$.

7.~Give a common enunciation which will include Propositions
\textsc{v}.\ and \textsc{vi}.
\par\end{footnotesize}

\mypropl{PROP\@.~VII\@.---Theorem.}{If a right line $(AB)$ be divided into any two parts
$($at $C)$, the sum of the squares on the whole line $(AB)$
and either segment $(CB)$ is equal to twice the rectangle
$(2AB\,.\,CB)$ contained by the whole line and that segment,
together with the square on the other segment.}


\imgflow{95}{8}{f087}

\textbf{Dem.}---On $AB$ describe the square $ABDE$. Join $BE$.
Through $C$ draw $CG$ parallel to $AE$,
intersecting $BE$ in $F$. Through $F$
draw $HK$ parallel to $AB$.

Now the square $AD$ is equal to the
three figures $AK$, $FD$, and $GH$: to
each add the square $CK$, and we
have the sum of the squares $AD$, $CK$
equal to the sum of the three figures
$AK$, $CD$, $GH$; but $CD$ is equal to $AK$; therefore the
sum of the squares $AD$, $CK$ is equal to twice the
figure $AK$, together with the figure $GH$. Now $AK$
is the rectangle $AB\,.\,BK$; but $BK$ is equal to $BC$;
therefore $AK$ is equal to the rectangle $AB\,.\,BC$, and

$AD$ is the square on $AB$; $CK$ the square on $CB$; and $GH$
is the square on $HF$, and therefore equal to the square
on $AC$. \textit{Hence the sum of the squares on $AB$ and $BC$
is equal to twice the rectangle $AB\,.\,BC$, together with the
square on $AC$.}\par\medskip


\imgflow{120}{12}{f088}

\begin{footnotesize}
\textit{Or thus:} On $AC$ describe the square $ACDE$. Produce the sides
$CD$, $DE$, $EA$, and make each produced
part equal to $CB$. Join $BF$, $FG$, $GH$,
$HB$. Then the figure $BFGH$ is a square
[I.~\textsc{xlvi}., Ex.~3], and it is equal to the

square on $AC$, together with the four
equal triangles $HAB$, $BCF$, $FDG$, $GEH$.
Now [I.~\textsc{xlvii}.], the figure $BFGH$ is
equal to the sum of the squares on $AB$,
$AH$---that is, equal to the sum of the
squares on $AB$, $BC$; and the sum of the
four triangles is equal to twice the rectangle
$AB\,.\,BC$, for each triangle is equal
to half the rectangle $AB\,.\,BC$. \textit{Hence the sum of the squares on
$AB$, $BC$ is equal to twice the rectangle $AB\,.\,BC$, together with the
square on $AC$.}
\begin{flalign*}
&\indent\textit{Or thus: }& AC = AB-BC; &
&&\phantom{\indent Or\ thus:\ }\\
&\text{therefore }& AC^2 = AB^2 - 2AB\,.\,BC &+ BC^2;  &&\\
&\text{therefore }& AC^2 + 2AB\,.\,BC = AB^2 &+ BC^2.  &&
\end{flalign*}

Comparison of \textsc{iv}.\ and \textsc{vii}.

By \textsc{iv}., square on sum $=$ sum of squares $+$ twice rectangle.

By \textsc{vii}., square on difference = sum of squares-twice
rectangle.
\par\end{footnotesize}

\textit{Cors.}\ from \textsc{iv}.\ and \textsc{vii}.

1.~Square on the sum, the sum of the squares, and
the square on the difference of any two lines, are in
arithmetical progression.

2.~Square on the sum $+$ square on the difference of
any two lines $=$ twice the sum of the squares on the
lines (Props.~\textsc{ix}.\ and \textsc{x}.).

3.~The square on the sum $-$ the square on the difference
of any two lines $=$ four times the rectangle under
lines (Prop.~\textsc{viii}.).



\mypropl{PROP\@.~VIII\@.--Theorem.}{If a line $(AB)$ be divided into two parts $($at $C)$, the
square on the sum of the whole line $(AB)$ and either segment
$(BC)$ is equal to four times the rectangle contained
by the whole line $(AB)$ and that segment, together with
the square on the other segment $(AC)$.}


\imgflow{110}{8}{f089}

\textbf{Dem.}---Produce $AB$ to $D$. Make $BD$ equal to $BC$.
On $AD$ describe the square $AEFD$
[I.~\textsc{~xlvi}.]. Join $DE$. Through
$C$, $B$ draw $CH$, $BL$ parallel to $AE$
[I.~\textsc{~xxxi}.], and through $K$, $I$ draw
$MN$, $PO$ parallel to $AD$.

Since $CO$ is the square on $CD$,
and $CK$ the square on $CB$, and
$CB$ is the half of $CD$, $CO$ is equal
to four times $CK$ [\textsc{iv.,} \textit{Cor.}~1].
Again, since $CG$, $GI$ are the sides of equal squares,
they are equal [I.~\textsc{xlvi.}, \textit{Cor.}~1]. Hence the parallelogram
$AG$ is equal to $MI$ [I.~\textsc{~xxxvi}.]. In like manner
$IL$ is equal to $JF$; but $MI$ is equal to $IL$ [I.~\textsc{~xliii}.].
Therefore the four figures $AG$, $MI$, $IL$, $JF$ are all
equal; hence their sum is equal to four times $AG$; and
the square $CO$ has been proved to be equal to four times
$CK$. Hence the gnomon $AOH$ is equal to four times
the rectangle $AK$---that is, equal to four times the
rectangle $AB\,.\,BC$, since $BC$ is equal to $BK$.

Again, the figure $PH$ is the square on $PI$, and
therefore equal to the square on $AC$. Hence the
whole figure $AF$, that is, \textit{the square on $AD$, is equal
to four times the rectangle $AB\,.\,BC$, together with the
square on $AC$.}\par\medskip


\imgflow{105}{9}{f090}

\begin{footnotesize}
\textit{Or thus:} Produce $BA$ to $D$, and make $AD = BC$. On $DB$
describe the square $DBEF$. Cut off $BG$, $EI$, $FL$ each equal to
$BC$. Through $A$ and $I$ draw lines parallel to $DF$, and through $G$
and $L$, lines parallel to $AB$.


Now it is evident that the four rectangles.
$AG$, $GI$, $IL$, $LA$ are all equal; but $AG$
is the rectangle $AB\,.\,BG$ or $AB\,.\,BC$.
Therefore the sum of the four rectangles
is equal to $4AB\,.\,BC$. Again, the figure
$NP$ is evidently equal to the square on
$AC$. Hence the whole figure, which is
the square on $BD$, or \emph{the square on the
sum of $AB$ and $BC$, is equal to $4AB\,.\,BC
+ AC^2$}.
\begin{flalign*}
&\indent\text{\emph{Or thus}: }&
      AB + BC &= AC +2BC;  &&\phantom{\indent Or\ thus: }\\
&\text{therefore }&
  (AB + BC)^2 &= AC^2 + 4AC\,.\,CB + 4BC^2  &&\\
&&            &= AC^2 + 4 (AC + CB)\,.\,CB  &&\\
&&            &= AC^2 + 4AB\,.\,BC.  &&
\end{flalign*}
\par\end{footnotesize}

Direct sequence from \textsc{v}.\ or \textsc{vi}.

Since by \textsc{v}.\ or \textsc{vi}.\ the rectangle contained by any two
lines is $=$ the square on half their sum $-$ the square on
half their difference; therefore four times the rectangle
contained by any two lines $=$ the square on their sum
$-$ the square on their difference.

Direct sequence of \textsc{viii}.\ from \textsc{iv}.\ and \textsc{vii}.

By \textsc{iv.}, the square on the sum $=$ the sum of the squares
$+$ twice the rectangle.

By \textsc{vii.}, the square on the difference $=$ the sum of the
squares $-$ twice the rectangle. Therefore, by subtraction,
the square on the sum $-$ the square on the
difference $=$ four times the rectangle.


\exhead{Exercises.}

\begin{footnotesize}
1.~In the figure [I.~\textsc{xlvii}.] if $EF$, $GK$ be joined, prove
$EF^2-CO^2 = (AB + BO)^2$.

2.~Prove \hfill $GK^2 - EF^2 = 3AB (AO - BO)$. \hfill \phantom{\indent 2. Prove }

3.\footnote
  {Ex.~3 occurs in the solution of the problem of the inscription
  of a regular polygon of seventeen sides in a circle. \emph{See} note C.}
Given the difference of two lines $= R$, and their rectangle
$= 4R^2$; find the lines.
\par\end{footnotesize}


\mypropl{PROP\@.~IX\@.---Theorem.}{If a line $(AB)$ be bisected $($at $C)$ and divided into two
unequal parts $($at $D)$, the sum of the squares on the unequal
parts $(AD,\ DB)$ is double the sum of the squares on half
the line $(AC)$, and on the segment $(CD)$ between the points
of section.}


\imgflow{145}{9}{f091}

\textbf{Dem}.---Erect $CE$ at right angles to $AB$, and make it
equal to $AC$ or $CB$. Join
$AE$, $EB$. Draw $DF$ parallel
to $CE$, and $FG$ parallel to
$CD$. Join $AF$.

Because $AC$ is equal to $CE$,
and the angle $ACE$ is right,
the angle $CEA$ is half a right
angle. In like manner the
angles $CEB$, $CBE$ are half right angles; therefore the
whole angle $AEF$ is right. Again, because $GF$ is parallel
to $CB$, and $CE$ intersects them, the angle $EGF$ is
equal to $ECB$; but $ECB$ is right (const.); therefore
$EGF$ is right; and $GEF$ has been proved to be half a
right angle; therefore the angle $GFE$ is half a right
angle [I.\ \textsc{xxxii}.]. Therefore [I.\ \textsc{vi}.] $GE$ is equal to
$GF$. In like manner $FD$ is equal to $DB$.

Again, since $AC$ is equal to $CE$, $AC^2$ is equal to
$CE^2$; but $AE^2$ is equal to $AC^2 + CE^2$ [I.\ \textsc{xlvii}.].
Therefore $AE^2$ is equal to $2AC^2$. In like manner $EF^2$
is equal to $2GF^2$ or $2CD^2$. Therefore $AE^2 + EF^2$ is
equal to $2AC^2 + 2CD^2$; but $AE^2 + EF^2$ is equal to $AF^2$
[I.\ \textsc{xlvii}.]. Therefore $AF^2$ is equal to $2AC^2 + 2CD^2$.

Again, since $DF$ is equal to $DB$, $DF^2$ is equal to $DB^2$:
to each add $AD^2$, and we get $AD^2 + DF^2$ equal to $AD^2
+ DB^2$; but $AD^2 + DF^2$ is equal to $AF^2$; therefore $AF^2$
is equal to $AD^2 + DB^2$; and we have proved $AF^2$ equal
to $2AC^2 + 2CD^2$. \textit{Therefore $AD^2 + DB^2$ is equal to $2AC^2
+ 2CD^2$.}\par\medskip

\begin{footnotesize}
\begin{flalign*}
&\textit{\indent Or thus:} &&AD = AC +CD;\ DB = AC-CD.& \\
&\text{\indent Square and add, and we get} &&AD^2 + DB^2 = 2AC^2 + 2CD^2.&\phantom{\text{\indent Square and ad}}
\end{flalign*}
\par\end{footnotesize}




\exhead{Exercises.}

\begin{footnotesize}
1.~The sum of the squares on the segments of a line of given
length is a minimum when it is bisected.

2.~Divide a given line internally, so that the sum of the squares
on the parts may be equal to a given square, and state the limitation
to its possibility.

3.~If a line $AB$ be bisected in $C$ and divided unequally in $D$,
\[
  AD^2 + DB^2 =  2AD\,.\,DB + 4CD^2.
\]

4.~Twice the square on the line joining any point in the hypotenuse
of a right-angled isosceles triangle to the vertex is equal to
the sum of the squares on the segments of the hypotenuse.

5.~If a line be divided into any number of parts, the continued
product of all the parts is a maximum, and the sum of their squares
is a minimum when all the parts are equal.
\par\end{footnotesize}


\mypropl{PROP\@.~X.---Theorem.}{If a line $(AB)$ be bisected $($at $C)$ and divided externally
$($at $D)$, the sum of the squares on the segments $(AD,\ DB)$
made by the external point is equal to twice the square on
half the line, and twice the square on the segment between
the points of section.}


\imgflow{184}{11}{f092}

\textbf{Dem.}---Erect $CE$ at right angles to $AB$, and make
it equal to $AC$ or $CB$.
Join $AE$, $EB$. Draw
$DF$ parallel to $CE$, and
produce $EB$. Now
since $DF$ is parallel to
$EC$, the angle $BDF$ is
$=$ to $BCE$ [I.\ \textsc{xxix}.],
and [I.\ \textsc{xv}.] the angle
$DBF$ is $=$ to $EBC$;
but the sum of the
angles $BCE$, $EBC$ is
less than two right angles [I.\ \textsc{xvii}.]; therefore the sum

of the angles $BDF$, $DBF$ is less than two right angles,
and therefore [I., Axiom \textsc{xii}.] the lines $EB$, $DF$, if produced,
will meet. Let them meet in $F$. Through $F$
draw $FG$ parallel to $AB$, and produce $EC$ to meet it in
$G$. Join $AF$.

Because $AC$ is equal to $CE$, and the angle $ACE$ is right,
the angle $CEA$ is half a right angle. In like manner
the angles $CEB$, $CBE$ are half right angles; therefore
the whole angle $AEF$ is right. Again, because $GF$
is parallel to $CB$, and $GE$ intersects them, the angle $EGF$
is equal to $ECB$ [I.\ \textsc{xxix}.]; but $ECB$ is right (const.);
therefore $EGF$ is right, and $GEF$ has been proved to
be half a right angle; therefore [I.\ \textsc{xxxii}.] $GFE$ is
half a right angle, and therefore [I.\ \textsc{vi}.] $GE$ is equal
to $GF$. In like manner $FD$ is equal to $DB$.

Again, since $AC$ is equal to $CE$, $AC^2$ is equal to $CE^2$;
but $AE^2$ is equal to $AC^2$ + $CE^2$ [I.\ \textsc{xlvii}.]; therefore
$AE^2$ is equal to $2AC^2$. In like manner $EF^2$ is equal to
$2GF^2$ or $2CD^2$; therefore $AE^2 + EF^2$ is equal to $2AC^2
+ 2CD^2$; but $AE^2 + EF^2$ is equal to $AF^2$ [I.\ \textsc{xlvii}.].
Therefore $AF^2$ is equal to $2AC^2 + 2CD^2$.

Again, since $DF$ is equal to $DB$, $DF^2$ is equal to
$DB^2$: to each add $AD^2$, and we get $AD^2 + DF^2$ equal to
$AD^2 + DB^2$; but $AD^2 + DF^2$ is equal to $AF^2$; therefore
$AF^2$ is equal to $AD^2 + DB^2$; and $AF^2$ has been proved
equal to $2AC^2 + 2CD^2$. \emph{Therefore $AD^2 + DB^2$ is equal
to $2AC^2 + 2CD^2$.}\par\medskip


\begin{footnotesize}
\begin{flalign*}
&\textit{\indent Or thus:}  &&  AD = CD + AC, &\phantom{\textit{Or thus:}}\\
  &&&  BD = CD-AC.
\end{flalign*}


Square and add, and we get $AD^2 + BD^2 = 2CD^2 + 2AC^2$.
\end{footnotesize}

The following enunciations include Propositions \textsc{ix}.
and \textsc{x}.:---

1.~\textit{The square on the sum of any two lines plus the
square on their difference equal twice the sum of their
squares}.


 2. \textit{The sum of the squares on any two lines it equal to
twice the square on half the sum plus twice the square on
half the difference of the lines}.

 3. \textit{If a line be cut into two unequal parts, and also
into two equal parts, the sum of the squares on the two
unequal parts exceeds the sum of the squares on the two
equal parts by the sum of the squares of the two differences
between the equal and unequal parts}.

\exhead{Exercises}.

\begin{footnotesize}
1.~Given the sum or the difference of any two lines, and the
sum of their squares; find the lines.

2.~The sum of the squares on two sides $AC$, $CB$ of a triangle is
equal to twice the square on half the base $AB$, and twice the
square on the median which bisects $AB$.

3.~If the base of a triangle be given both in magnitude and
position, and the sum of the squares on the sides in magnitude,
the locus of the vertex is a circle.

4.~If in the $\triangle\:ABC$ a point $D$ in the base $BC$ be such that
\[
BA^2+BD^2=CA^2+CD^2;
\]
prove that the middle point of $AD$ is equally distant from $B$
and $C$.

5.~The sum of the squares on the sides of a parallelogram is
equal to the sum of the squares on its diagonals.
\par\end{footnotesize}


\mypropflow{PROP\@.~XI\@.---Problem.}{To divide a given finite line $(AB)$ into two segments
$($in $H)$, so that the rectangle $(AB\,.\,BH)$ contained by
the whole line and one segment may be equal to the
square on the other segment.}{95}{12}{f093}

\textbf{Sol.}---On $AB$ describe the square $ABDC$ [I.\ \textsc{xlvi}.].
Bisect $AC$ in $E$. Join $BE$. Produce $EA$ to $F$, and
make $EF$ equal to $EB$. On $AF$ describe the square
$AFGH$. $H$ \emph{is the point required}.



\textbf{Dem}.---Produce $GH$ to $K$. Then because $CA$ is bisected
in $E$, and divided externally
in $F$, the rectangle $CF\,.\,AF$, together
with the square on $EA$, is equal to the
square on $EF$ [\textsc{vi}.]; but $EF$ is equal
to $EB$ (const.); therefore the rectangle
$CF\,.\,AF$, together with $EA^2$, is
equal to $EB^2$---that is [I.\ \textsc{xlvii}.] equal
to $EA^2 + AB^2$. Rejecting $EA^2$, which
is common, we get the rectangle
$CF\,.\,AF$ equal to $AB^2$. Again, since
$AF$ is equal to $FG$, being the sides of
a square, the rectangle $CF\,.\,AF$ is equal to $CF\,.\,FG$---that
is, to the figure $CG$; and $AB^{2}$ is equal to the figure
$AD$; therefore $CG$ is equal to $AD$. Reject the part
$AK$, which is common, and we get the figure $FH$ equal
to the figure $HD$; but $HD$ is equal to the rectangle
$AB\,.\,BH$, because $BD$ is equal to $AB$, and $FH$ is the
square on $AH$. \emph{Therefore the rectangle $AB\,.\,BH$ is
equal to the square on $AH$.}

\textsc{Def}.---A line divided as in this Proposition is said to
be divided in ``\emph{extreme and mean ratio}.''

\emph{Cor.}\ 1.---The line C$F$ is divided in ``extreme and
mean ratio'' at $A$.

\emph{Cor.}\ 2.---If from the greater segment $CA$ of $CF$ we
take a segment equal to $AF$, it is evident that $CA$ will
be divided into parts respectively equal to $AH$, $HB$.
Hence, if a line be divided in extreme and mean ratio,
the greater segment will be cut in the same manner
by taking on it a part equal to the less; and the less
will be similarly divided by taking on it a part equal
to the difference, and so on, \&c.


\imgflow{138}{3}{f094}

\emph{Cor.}\ 3.---Let $AB$ be divided in ``extreme and mean
ratio'' in $C$, then it is evident
(\emph{Cor.}\ 2) that $AC$ is greater
than $CB$. Cut off $CD = CB$;
then (\emph{Cor.}\ 2) $AC$ is cut in ``extreme and mean ratio'' at
$D$, and $CD$ is greater than $AD$. Next, cut off $DE$ equal

to $AD$, and in the same manner we have $DE$ greater
than $EC$, and so on. Now since $CD$ is greater than $AD$,
it is evident that $CD$ is not a common measure of $AC$
and $CB$, and therefore not a common measure of $AB$
and $AC$. In like manner $AD$ is not a common measure
of $AC$ and $CD$, and therefore not a common measure of
$AB$ and $AC$. Hence, no matter how far we proceed
we cannot arrive at any remainder which will be a
common measure of $AB$ and $AC$. \textit{Hence, the parts of a
line divided in ``extreme and mean ratio''\index{Ratio!extreme and mean}
are incommensurable.}\index{Lines!Incommensurable}

\exhead{Exercises.}

\begin{footnotesize}
1.~Cut a line externally in ``extreme and mean ratio.''

2.~The difference between the squares on the segments of a
line divided in ``extreme and mean ratio'' is equal to their
rectangle.

3.~In a right-angled triangle, if the square on one side be equal
to the rectangle contained by the hypotenuse and the other side,
the hypotenuse is cut in ``extreme and mean ratio'' by the
perpendicular on it from the right angle.

4.~If $AB$ be cut in ``extreme and mean ratio'' at $C$, prove
that
\begin{align*}
  (1) &\,\ AB^2 + BC^2 = 3AC^2.  \\
  (2) &\  (AB + BC)^2 = 5AC^2.
\end{align*}

5.~The three lines joining the pairs of points $G$,~$B$; $F$,~$D$;
$A$,~$K$, in the construction of Proposition \textsc{xi}., are parallel.

6.~If $CH$ intersect $BE$ in $O$, $AO$ is perpendicular to $CH$.

7.~If $CH$ be produced, it meets $BF$ at right angles.

8.~$ABC$ is a right-angled triangle having $AB = 2AC$: if $AH$
be made equal to the difference between $BC$ and $AC$, $AB$ is
divided in ``extreme and mean ratio'' at $H$.
\par\end{footnotesize}


\mypropl{PROP\@.~XII\@.---Theorem.}{In an obtuse-angled triangle $(ABC)$, the square on the
side $(AB)$ subtending the obtuse angle exceeds the sum of
the squares on the sides $(BC,\ CA)$ containing the obtuse
angle, by twice the rectangle contained by either of them
$(BC)$, and its continuation $(CD)$ to meet a perpendicular
$(AD)$ on it from the opposite angle.}


\imgflow{127}{8}{f095}

\textbf{Dem.}---Because $BD$ is divided into two parts in $C$,
we have
\begin{flalign*}
&&BD^2=BC^2 &+ CD^2 + 2BC\,.\,CD \text{\ [\textsc{iv.}]} \\
&\text{and }&& AD^2 = AD^2.
\end{flalign*}
Hence, adding, since [I.~\textsc{xlvii.}]
$BD^2 + AD^2 = AB^2$, and $CD^2 + AD^2
= CA^2$, we get
\[
AB^2 = BC^2 + CA^2 + 2BC\,.\,CD.
\]
\textit{Therefore $AB^2$ is greater than $BC^2 + CA^2$ by $2BC\,.\,CD$.}\par\medskip


\imgflow{145}{16}{f096}

\begin{footnotesize}
The foregoing proof differs from Euclid's only in the use of
symbols. I have found by experience that pupils more readily
understand it than any other method.\par\medskip

\textit{Or thus:} By the First Book:
Describe squares on the three
sides. Draw $AE$, $BF$, $CG$ perpendicular
to the sides of the
squares. Then it can be proved
exactly as in the demonstration
of [I.~\textsc{xlvii}.], that the rectangle

$BG$ is equal to $BE$, $AG$ to $AF$,
and $CE$ to $CF$. \textit{Hence the sum
of the two squares on $AC$, $CB$ is
less than the square on $AB$ by
twice the rectangle $CE$; that is,
by twice the rectangle $BC\,.\,CD$.}

\textit{Cor.}~1.---If perpendiculars
from $A$ and $B$ to the opposite
sides meet them in $H$ and $D$,
the rectangle $AC\,.\,CH$ is equal
to the rectangle $BC\,.\,CD$.
\par\end{footnotesize}



\exhead{Exercises.}

\begin{footnotesize}
1.~If the angle $ACB$ of a triangle be equal to twice the angle
of an equilateral triangle, \ $AB^{2} = BC^{2} + CA^{2} + BC\,.\,CA$.

2.~$ABCD$ is a quadrilateral whose opposite angles $B$ and $D$ are
right, and $AD$, $BC$ produced meet in $E$; prove $AE\,.\,DE = BE\,.\,CE$.

3.~$ABC$ is a right-angled triangle, and $BD$ is a perpendicular
on the hypotenuse $AC$; Prove $AB\,.\,DC = BD\,.\,BC$.

4.~If a line $AB$ be divided in $C$ so that $AC^{2} = 2CB^{2}$; prove that
$AB^{2} + BC^{2} = 2AB\,.\,AC$.

5.~If $AB$ be the diameter of a semicircle, find a point $C$ in $AB$
such that, joining $C$ to a fixed point $D$ in the circumference, and
erecting a perpendicular $CE$ meeting the circumference in $E$,
$CE^{2} - CD^{2}$ may be equal to a given square.

6.~If the square of a line $CD$, drawn from the angle $C$ of an
equilateral triangle $ABC$ to a point $D$ in the side $AB$ produced,
be equal to $2AB^{2}$; prove that $AD$ is cut in ``extreme and mean
ratio'' at $B$.
\par\end{footnotesize}

\mypropl{PROP\@.~XIII\@.---Theorem.}{In any triangle $(ABC)$, the square on any side subtending
an acute angle $(C)$ is less than the sum of the squares
on the sides containing that angle, by twice the rectangle
$(BC,\ CD)$ contained by either of them $(BC)$ and the intercept
$(CD)$ between the acute angle and the foot of the perpendicular
on it from the opposite angle.}


\imgflow{118}{7}{f097}

\textbf{Dem.}---Because $BC$ is divided into two segments
in $D$,
\[
BC^{2} + CD^{2} = BD^{2} + 2BC\,.\,CD \text{\ [\textsc{vii.}];}
\]
\begin{flalign*}
&\text{and }&& AD^{2} = AD^{2}.  &&\phantom{and }
\end{flalign*}
Hence, adding, since
\begin{flalign*}
&&CD^{2} + AD^{2} = AC^{2} &\text{\ [I.~\textsc{xlvii}.],}\\
&\text{and }&     BD^{2} + AD^{2} &= AB^{2},  &&\phantom{we\ get\ }\\
&\text{we get }&  BC^{2} + AC^{2} &= AB^{2} + 2BC\,.\,CD.  &&
\end{flalign*}
\textit{Therefore $AB^{2}$ is less than $BC^{2} + AC^{2}$ by $2BC\,.\,CD$.}



\imgcent{206}{f098}

\textit{Or thus:} Describe squares on the sides. Draw $AE$,
$BF$, $CG$ perpendicular to the sides; then, as in the
demonstration of [I.~\textsc{xlvii.}], the rectangle $BG$ is equal
to $BE$; $AG$ to $AF$, and $CE$ to $CF$. \textit{Hence the sum of
the squares on $AC$, $CB$ exceeds the square on $AB$ by twice
$CE$---that is, by $2BC\,.\,CD$}.\par\medskip

\begin{footnotesize}
\textbf{Observation.}---By comparing the proofs of the pairs of
Props. \textsc{iv.}\ and \textsc{vii.; v.}\ and \textsc{vi.; ix.}\ and \textsc{x.; xii.}\ and \textsc{xiii.}, it
will be seen that they are virtually identical\index{Propositions!identical}. In order to render
this identity more apparent, we have made some slight alterations
in the usual proofs. The pairs of Propositions thus grouped are
considered in Modern Geometry not as distinct, but each pair is
regarded as one Proposition.
\par\end{footnotesize}

\exhead{Exercises.}

\begin{footnotesize}
1.~If the angle $C$ of the $\triangle~ACB$ be equal to an angle of an
equilateral $\triangle$, $AB^2 = AC^2 + BC^2 - AC\,.\,BC$.

2.~The sum of the squares on the diagonals of a quadrilateral,
together with four times the square on the line joining their
middle points, is equal to the sum of the squares on its sides.

3.~Find a point $C$ in a given line $AB$ produced, so that
$AC^2 + BC^2 = 2AC\,.\,BC$.
\par\end{footnotesize}


\myprop{PROP\@.~XIV\@.---Problem.}{To construct a square equal to a given rectilineal figure $(X)$.}


\imgcent{225}{f099}

\textbf{Sol.}---Construct [I.~\textsc{xlv.}] the rectangle $AC$ equal to
$X$. Then, if the adjacent sides $AB$, $BC$ be equal, $AC$
is a square, and the problem is solved; if not, produce
$AB$ to $E$, and make $BE$ equal to $BC$; bisect $AE$ in $F$;
with $F$ as centre and $FE$ as radius, describe the semicircle
$AGE$; produce $CB$ to meet it in $G$. \textit{The square
described on $BG$ will be equal to $X$.}

\textbf{Dem.}---Join $FG$. Then because $AE$ is divided
equally in $F$ and unequally in $B$, the rectangle $AB\,.\,BE$,
together with $FB^{2}$ is equal to $FE^{2}$ [\textsc{v.}], that is, to $FG^{2}$;
but $FG^{2}$ is equal to $FB^{2} + BG^{2}$ [I.~\textsc{xlvii.}]. Therefore
the rectangle $AB\,.\,BE + FB^{2}$ is equal to $FB^{2} + BG^{2}$.
Reject $FB^{2}$, which is common, and we have the rectangle
$AB\,.\,BE = BG^{2}$; but since $BE$ is equal to $BC$,
the rectangle $AB\,.\,BE$ is equal to the figure $AC$.
Therefore $BG^{2}$ is equal to the figure $AC$, \textit{and therefore
equal to the given rectilineal figure $(X)$.}

\textit{Cor.}---The square on the perpendicular from any
point in a semicircle on the diameter is equal to the
rectangle contained by the segments of the diameter.



\exhead{Exercises.}

\begin{footnotesize}
1.~Given the difference of the squares on two lines and their
rectangle; find the lines.

2.~Divide a given line, so that the rectangle contained by
another given line and one segment may be equal to the square
on the other segment.
\par\end{footnotesize}

\addcontentsline{toc}{section}{Questions for Examination,}
\exhead{Questions for Examination on Book II\@.}

\begin{footnotesize}
1.~What is the subject-matter of Book II\@.? \emph{Ans.}\ Theory of
rectangles.

2.~What is a rectangle? A gnomon?

3.~What is a square inch? A square foot? A square perch?
A square mile? \emph{Ans.}\ The square described on a line whose
length is an inch, a foot, a perch, \&c.

4.~What is the difference between linear and superficial measurement?
\emph{Ans.}\ Linear measurement has but one dimension;
superficial has two.

5.~When is a line said to be divided internally? When externally?

6.~How is the area of a rectangle found?

7.~How is a line divided so that the rectangle contained by its
segments may be a maximum?

8.~How is the area of a parallelogram found?

9.~What is the altitude of a parallelogram whose base is 65
metres and area 1430 square metres?

10.~How is a line divided when the sum of the squares on its
segments is a minimum?

11.~The area of a rectangle is 108.60 square metres and its
perimeter is 48.20 linear metres; find its dimensions.

12.~What Proposition in Book II\@.\ expresses the distributive
law of multiplication?

13.~On what proposition is the rule for extracting the square
root founded?

14.~Compare I.~\textsc{xlvii.}\ and II\@.~\textsc{xii.} and \textsc{xiii}.

15.~If the sides of a triangle be expressed by $x^2 + 1$, $x^2 - 1$,
and $2x$ linear units, respectively; prove that it is right-angled.


16.~How would you construct a square whose area would be
exactly an acre? Give a solution by I.~\textsc{xlvii}.

17.~What is meant by incommensurable lines? Give an example
from Book II\@.

18.~Prove that a side and the diagonal of a square are incommensurable.

19.~The diagonals of a lozenge are 16 and 30 metres respectively;
find the length of a side.

20.~The diagonal of a rectangle is 4.25 perches, and its area is
7.50 square perches; what are its dimensions?

21.~The three sides of a triangle are 8, 11, 15; prove that it
has an obtuse angle.

22.~The sides of a triangle are 13, 14, 15; find the lengths of
its medians; also the lengths of its perpendiculars, and prove
that all its angles are acute.

23.~If the sides of a triangle be expressed by $m^2 + n^2$, $m^2 - n^2$,
and $2mn$ linear units, respectively; prove that it is right-angled.

24.~If on each side of a square containing 5.29 square perches
we measure from the corners respectively a distance of 1.5 linear
perches; find the area of the square formed by joining the points
thus found.
\par\end{footnotesize}

\addcontentsline{toc}{section}{Exercises,}
\exhead{Exercises on Book II\@.}

\begin{footnotesize}
1.~The squares on the diagonals of a quadrilateral are together
double the sum of the squares on the lines joining the middle
points of opposite sides.

2.~If the medians of a triangle intersect in $O$, $AB^{2} + BC^{2} + CA^{2}
=3(OA^{2} + OB^{2} + OC^{2})$.

3.~Through a given point $O$ draw three lines $OA$, $OB$, $OC$ of
given lengths, such that their extremities may be collinear, and
that $AB = BC$.

4.~If in any quadrilateral two opposite sides be bisected, the
sum of the squares on the other two sides, together with the sum
of the squares on the diagonals, is equal to the sum of the squares
on the bisected sides, together with four times the square on the
line joining the points of bisection.

5.~If squares be described on the sides of any triangle, the
sum of the squares on the lines joining the adjacent corners is
equal to three times the sum of the squares on the sides of the
triangle.



6.~Divide a given line into two parts, so that the rectangle
contained by the whole and one segment may be equal to any
multiple of the square on the other segment.

7.~If $P$ be any point in the diameter $AB$ of a semicircle, and
$CD$ any parallel chord, then
\[
CP^2 + PD^2 = AP^2 + PB^2.
\]

8.~If $A$, $B$, $C$, $D$ be four collinear points taken in order,
\[
AB\,.\,CD + BC\,.\,AD = AC\,.\,BD.
\]

9.~Three times the sum of the squares on the sides of any
pentagon exceeds the sum of the squares on its diagonals, by four
times the sum of the squares on the lines joining the middle
points of the diagonals.

10.~In any triangle, three times the sum of the squares on the
sides is equal to four times the sum of the squares on the medians.

11.~If perpendiculars be drawn from the angular points of a
square to any line, the sum of the squares on the perpendiculars
from one pair of opposite angles exceeds twice the rectangle
of the perpendiculars from the other pair by the area of the
square.

12.~If the base $AB$ of a triangle be divided in $D$, so that $mAD
= nBD$, then
\[
mAC^2 + nBC^2 = mAD^2 + nDB^2 + (m + n) CD^2.
\]

13.~If the point $D$ be taken in $AB$ produced, so that $mAD
= nDB$, then
\[
mAC^2 - nBC^2 = mAD^2 - nDB^2 + (m - n) CD^2.
\]

14.~Given the base of a triangle in magnitude and position,
and the sum or the difference of $m$ times the square on one side
and $n$ times the square on the other side, in magnitude, the locus
of the vertex is a circle.

15.~Any rectangle is equal to half the rectangle contained by
the diagonals of squares described on its adjacent sides.

16.~If $A$, $B$, $C$. \&c., be any number of fixed points, and $P$ a
variable point, find the locus of $P$, if $AP^2 + BP^2 + CP^2 +$ \&c., be
given in magnitude.

17.~If the area of a rectangle be given, its perimeter is a minimum
when it is a square.



18.~If a transversal cut in the points $A$, $C$, $B$ three lines
issuing from a point $D$, prove that
\[
BC\,.\,AD^2 + AC\,.\,BD^2 - AB\,.\,CD^2 = AB\,.\,BC\,.\,CA.
\]

19.~Upon the segments $AC$, $CB$ of a line $AB$ equilateral triangles
are described: prove that if $D$, $D'$ be the centres of circles
described about these triangles, $6DD'^2 = AB^2 + AC^2 + CB^2$.

20.~If $a$, $b$, $p$ denote the sides of a right-angled triangle about
the right angle, and the perpendicular from the right angle on
the hypotenuse, $\dfrac{1}{a^2} + \dfrac{1}{b^2} = \dfrac{1}{p^2}$.

21.~If, upon the greater segment $AB$ of a line $AC$, divided in
extreme and mean ratio, an equilateral triangle $ABD$ be described,
and $CD$ joined, $CD^2 = 2AB^2$.

22.~If a variable line, whose extremities rest on the circumferences
of two given concentric circles, subtend a right angle at
any fixed point, the locus of its middle point is a circle.
\par\end{footnotesize}




\mybook{BOOK III}{Theory of the Circle}

\textsc{i}. Equal circles are those whose radii are equal.\par\medskip

\begin{footnotesize}
This is a theorem, and not a definition. For if two circles
have equal radii, they are evidently congruent figures, and therefore
equal. From this way of proving this theorem Props. \textsc{xxvi.--xxix}.~follow
as immediate inferences.
\par\end{footnotesize}\medskip

\textsc{ii}.\ A chord\index{Circle!chord of} of a circle is the line joining two points
in its circumference.\par\medskip

\begin{footnotesize}
If the chord be produced both ways, the whole line is called a
secant, and each of the parts into which a secant divides the circumference
is called an arc\index{Arc of a circle}\index{Circle!arc of}---the greater the \emph{major conjugate arc},
and the lesser the \emph{minor conjugate arc}.---\textsc{Newcomb}.\index{Newcomb}
\par\end{footnotesize}\medskip


\imgflow{126}{10}{f100}

\textsc{iii}.\ A right line is said to
touch a circle when it meets
the circle, and, being produced
both ways, does not cut it;
the line is called a \emph{tangent}\index{Tangent}
to the circle, and the point
where it touches it the \emph{point
of contact.}\par\medskip

\begin{footnotesize}
In Modern Geometry a curve is considered as made up of an
infinite number of points, which are placed in order along the
curve, and then the secant through two consecutive points is a

tangent. Euclid's definition for a tangent is quite inadequate for
any curve but the circle, and those derived from it by projection
(the conic sections); and even for these the modern definition is
better.
\par\end{footnotesize}\medskip


\imgflow{160}{9}{f101}

\textsc{iv}.\ Circles are said to
touch one another when
they meet, but do not
intersect. There are two
species of contact\index{Circle!contact of}:---\\
1.~When each circle is
external to the other.\\
2.~When one is inside
the other.\par\smallskip

\begin{footnotesize}
The following is the modern definition of curve-contact:---
\emph{When two curves have two, three, four, \&c., consecutive points
common, they have contact of the first, second, third, \&c., orders.}
\par\end{footnotesize}\smallskip

\textsc{v.} A segment\index{Circle!segment of} of a circle is a figure bounded by a
chord and one of the arcs into which it divides the
circumference.

\textsc{vi.} Chords are said to be equally distant from the
centre when the perpendiculars drawn to them from
the centre are equal.

\textsc{vii.} The angle contained by two lines, drawn from
any point in the circumference of a segment to the
extremities of its chord, is called an \emph{angle in the segment}.

\textsc{viii.} The angle of a segment is the angle contained
between its chord and the tangent at either extremity.\par\smallskip

\begin{footnotesize}
A theorem is tacitly assumed in this Definition, namely, that
the angles which the chord makes with the tangent at its extremities
are equal. We shall prove this further on.
\par\end{footnotesize}\smallskip

\textsc{ix.} An angle in a segment is said to \emph{stand} on its
conjugate arc.


\imgflow{95}{6}{f102}

\textsc{x.} Similar segments of circles are
those that contain equal angles.

\textsc{xi.} A sector\index{Circle!sector of} of a circle is formed
of two radii and the arc included
between them.\par\smallskip

\begin{footnotesize}
To a pair of radii may belong either of the two conjugate arcs
into which their ends divide the circle.---\textsc{Newcomb}.\index{Newcomb}
\par\end{footnotesize}\smallskip



\textsc{xii}.\ Concentric circles\index{Circle!concentric} are those that have the same
centre.

\textsc{xiii}.\ Points which lie on the circumference of a
circle are said to be \emph{concyclic}.\index{Points!concyclic}

\textsc{xiv}.\ A \emph{cyclic quadrilateral}\index{Quadrilateral!cyclic} is one which is inscribed
in a circle.

\textsc{xv}.\ It will be proper to give here an explanation of
the extended meaning of the word \emph{angle}\index{Angle!extension of meaning} in Modern
Geometry. This extension is necessary in Trigonometry,
in Mechanics---in fact, in every application of
Geometry, and has been partly given in I.\ Def.\ \textsc{ix}.


\imgcent{308}{f103}

Thus, if a line $OA$ revolve about the point $O$, as in
figures 1, 2, 3, 4, until it comes into the position $OB$,
the amount of the \emph{rotation} from $OA$ to $OB$ is called an
\emph{angle}. From the diagrams we see that in fig.~1 it is
less than two right angles; in fig.~2 it is equal to two
right angles; in fig.~3 greater than two right angles,
but less than four; and in fig.~4 it is greater than four
right angles. The arrow-heads denote the direction or
\emph{sense}, as it is technically termed, in which the line $OA$
turns. It is usual to call the direction indicated in the
above figures \emph{positive}, and the opposite \emph{negative.} A line
such as $OA$, which turns about a fixed point, is called
a \emph{ray}\index{Ray}, and then we have the following definition:---

\textsc{xvi}.\ A ray which turns in the sense \emph{opposite} to the
hands of a watch describes a \emph{positive angle}, and one
which turns in the \emph{same} direction as the hands, a
\emph{negative angle}.


\addcontentsline{toc}{section}{Propositions \textsc{i.--xxxvii.,}}
\myprop{PROP\@.~I.---Problem.}{To find the centre of a given circle $(ADB)$.}


\imgflow{103}{9}{f104}

\textbf{Sol}.---Take any two points $A$, $B$ in the circumference.
Join $AB$. Bisect it in $C$. Erect
$CD$ at right angles to $AB$. Produce
$DC$ to meet the circle again in $E$.
Bisect $DE$ in $F$. \emph{Then $F$ is the
centre.}

\textbf{Dem}.---If possible, let any other
point $G$ be the centre. Join $GA$,
$GC$, $GB$. Then in the triangles
$ACG$, $BCG$ we have $AC$ equal to
$CB$ (const.), $CG$ common, and the
base $GA$ equal to $GB$, because they are drawn from $G$,
which is, by hypothesis, the centre, to the circumference.
Hence [I.\ \textsc{viii}.] the angle $ACG$ is equal to
the adjacent angle $BCG$, and therefore [I.\ Def.\ \textsc{xiii}.]
each is a right angle; but the angle $ACD$ is right
(const.); therefore $ACD$ is equal to $ACG$---a part equal
to the whole---which is absurd. Hence no point can
be the centre which is not in the line $DE$. \emph{Therefore $F$,
the middle point of $DE$, must be the centre.}

The foregoing proof may be abridged as follows:---\\
Because $ED$ bisects $AB$ at right angles, every point
equally distant from, the points $A$, $B$ must lie in $ED$
[I.\ \textsc{x.}\ Ex.\ 2]; but the centre is equally distant from
$A$ and $B$; hence the centre must be in $ED$; \emph{and since
it must be equally distant from $E$ and $D$, it must be the
middle point of $DE$.}

\textit{Cor}.~1.---The line which bisects any chord of a circle
perpendicularly passes through the centre of the circle.

\textit{Cor}.~2.---The locus of the centres of the circles which
pass through two fixed points is the line bisecting at
right angles that connecting the two points.

\textit{Cor}.~3.---If $A$, $B$, $C$ be three points in the circumference
of a circle, the lines bisecting perpendicularly
the chords $AB$, $BC$ intersect in the centre.




\mypropl{PROP\@.~II\@.---Theorem.}{If any two points $(A,\ B)$ be taken in the circumference
of a circle---\textrm{1.} The segment $(AB)$ of the indefinite line
through these points which lies between them falls within
the circle. \textrm{2.} The remaining\label{remainiug} 
parts of the line are without
the circle.}


\imgflow{140}{9}{f105}

\textbf{Dem.}---1. Let $C$ be the centre. Take any point $D$
in $AB$. Join $CA, CD, CB$.
Now the angle $ADC$ is
[I.\ \textsc{xvi.}] greater than $ABC$;
but the angle $ABC$ is equal
to $CAB$ [I.\ \textsc{v.}], because the
triangle $CAB$ is isosceles;
therefore the angle $ADC$ is
greater than $CAD$. Hence
$AC$ is greater than $CD$ [I.\ \textsc{xix.}]; therefore $CD$ is less
than the radius of the circle, consequently the point $D$
must be within the circle (note on I.\ Def.\ \textsc{xxiii.}).

In the same manner \textit{every other point between $A$ and
$B$ lies within the circle.}\par\medskip

2.~Take any point $E$ in $AB$ produced either way.
Join $CE$. Then the angle $ABC$ is greater than $AEC$
[I.\ \textsc{xvi.}]; therefore $CAB$ is greater than $AEC$. Hence
$CE$ is greater than $CA$, \textit{and the point $E$ is without the
circle.}\par\medskip

\begin{footnotesize}
We have added the second part of this Proposition. The indirect
proof given of the first part in several editions of Euclid
is very inelegant; it is one of those absurd things which give
many students a dislike to Geometry.
\par\end{footnotesize}\medskip

\textit{Cor}. 1.---Three collinear points cannot be concyclic.\par\smallskip

\textit{Cor}. 2.---A line cannot meet a circle in more than
two points.\par\smallskip

\textit{Cor}. 3.---The circumference of a circle is everywhere
concave towards the centre.



\mypropl{PROP\@.~III\@.--Theorem.}{If a line $(AB)$ passing through the centre of a circle
bisect a chord $(CD)$, which does not pass through the centre,
it cuts it at right angles. 2.\ If it cuts it at right angles,
it bisects it.}


\imgflow{117}{10}{f106}

\textbf{Dem}.---1. Let $O$ be the centre of the circle. Join
$OC$, $OD$. Then the triangles
$CEO$, $DEO$ have $CE$ equal to $ED$
(hyp.), $EO$ common, and $OC$ equal
to $OD$, because they are radii of the
circle; hence [I.\ \textsc{viii}.] the angle
$CEO$ is equal to $DEO$, and they
are adjacent angles. Therefore
[I.\ Def.\ \textsc{xiii}.] each is a right
angle. \emph{Hence $AB$ cuts $CD$ at
right angles.}

2.~The same construction being made: because $OC$
is equal to $OD$, the angle $OCD$ is equal to $ODC$ [I.\ \textsc{v}.],
and $CEO$ is equal to $DEO$ (hyp.), because each is
right. Therefore the triangles $CEO$, $DEO$ have two
angles in one respectively equal to two angles in the
other, and the side $EO$ common. Hence [I.\ \textsc{xxvi}.] the
side $CE$ is equal to $ED$. \emph{Therefore $CD$ is bisected in $E$.}\par\medskip

2.~May be proved as follows:---
\begin{flalign*}
&&OC^2 = OE^2 + &EC^2\ \text{[I.\ \textsc{xlvii}.], and } OD^2 = OE^2 + ED^2; \\
&\text{but}&OC^2 = &OD^2;\ \therefore\ OE^2 + EC^2 = OE^2 + ED^2. \\
&\text{Hence}& &EC^2 = ED^2, \text{\ and } EC = ED.& \phantom{\text{Hence}}
\end{flalign*}

\begin{footnotesize}
\textsf{Observation}.---The three theorems, namely, \emph{Cor}.\ 1., Prop.\ \textsc{i}.,
and Parts 1, 2, of Prop. \textsc{iii}., are so related, that any one being
proved directly, the other two follow by the Rule of Identity.
\par\end{footnotesize}\medskip

\emph{Cor}.\ 1.---The line which bisects perpendicularly one
of two parallel chords of a circle bisects the other perpendicularly.



\textit{Cor.} 2.---The locus of the middle points of a system
of parallel chords of a circle is the diameter of the
circle perpendicular to them all.

\textit{Cor.} 3.---If a line intersect two concentric circles,
its intercepts between the circles are equal.

\textit{Cor.} 4.---The line joining the centres of two intersecting
circles bisects their common chord perpendicularly.


\exhead{Exercises.}

\begin{footnotesize}
1.~If a chord of a circle subtend a right angle at a given
point, the locus of its middle point is a circle.

2.~Every circle passing through a given point, and having its
centre on a given line, passes through another given point.

3.~Draw a chord in a given circle which shall subtend a right
angle at a given point, and be parallel to a given line.
\par\end{footnotesize}


\myprop{PROP\@.~IV\@.---Theorem.}{Two chords of a circle $(AB,\ CD)$ which are not both
diameters cannot bisect each other, though either may
bisect the other.}


\imgflow{115}{9}{f107}

\textbf{Dem}.---Let $O$ be the centre. Let $AB$, $CD$ intersect
in $E$; then since $AB$, $CD$ are not
both diameters, join $OE$. If possible
let $AE$ be equal to $EB$, and
$CE$ equal to $ED$. Now, since $OE$
passing through the centre bisects
$AB$, which does not pass through
the centre, it is at right angles to
it; therefore the angle $AEO$ is
right. In like manner the angle
$CEO$ is right. Hence $AEO$ is equal to $CEO$---that is,
a part equal to the whole---which is absurd. \textit{Therefore
$AB$ and $CD$ do not bisect each other.}

\textit{Cor.}---If two chords of a circle bisect each other,
they are both diameters.



\myprop{PROP\@.~V.---Theorem.}{If two circles $(ABC,\ ABD)$ cut one another in any point
$(A)$, they are not concentric.}


\imgflow{118}{8}{f108}

\textbf{Dem}.---If possible let them have a common centre at
$O$. Join $OA$, and draw any other
line $OD$, cutting the circles in $C$
and $D$ respectively. Then because
$O$ is the centre of the circle $ABC$,
$OA$ is equal to $OC$. Again, because
$O$ is the centre of the circle
$ABD$, $OA$ is equal to $OD$. Hence
$OC$ is equal to $OD$---a part equal
to the whole---which is absurd.
\emph{Therefore the circles
are not concentric.}

\exhead{Exercises.}

\begin{footnotesize}
1.~If two non-concentric circles intersect in one point, they must
intersect in another point. For, let $O$, $O'$ be the centres, $A$ the
point of intersection; from $A$ let fall the $\perp AC$ on the line $OO'$.
Produce $AC$ to $B$, making $BC = CA$: then $B$ is another point of
intersection.

2.~Two circles cannot have three points in common without
wholly coinciding.
\par\end{footnotesize}


\mypropflow{PROP\@.~VI\@.---Theorem.}{If one circle $(ABC)$ touch another circle $(ADE)$ internally
in any point $(A)$, it is not concentric with it.}{110}{9}{f109}

\textbf{Dem}.---If possible let the circles be concentric, and
let $O$ be the centre of each. Join
$OA$, and draw any other line $OD$,
cutting the circles in the points
$B$, $D$ respectively. Then because $O$
is the centre of each circle (hyp.),
$OB$ and $OD$ are each equal to $OA$;
therefore $OB$ is equal to $OD$, which
is impossible. \emph{Hence the circles
cannot have the same centre.}



\mypropl{PROP\@.~VII\@.---Theorem.}{If from any point $(P)$ within a circle, which is not the
centre, lines $(PA,$\ $PB,$\ $PC,\ $ \textrm{\&c.}$)$, one of which passes
through the centre, be drawn to the circumference, then---\textrm{1.}
The greatest is the line $(PA)$ which passes through the
centre. \textrm{2.} The production $(PE)$ of this in the opposite
direction is the least. \textrm{3.} Of the others, that which is
nearest to the line through\label{throuyh} 
the centre is greater than every
one more remote. \textrm{4.} Any two lines making equal angles
with the diameter on opposite sides are equal. \textrm{5.} More
than two equal right lines cannot be drawn from the given
point $(P)$ to the circumference.}

\textbf{Dem}.---1. Let $O$ be the centre. Join $OB$. Now
since $O$ is the centre, $OA$ is
equal to $OB$: to each add $OP$,
and we have $AP$ equal to the
sum of $OB$, $OP$; but the sum
of $OB$, $OP$ is greater than
$PB$ [I.\ \textsc{xx}.]. \emph{Therefore $PA$ is
greater than $PB$.}


\imgflow{127}{12}{f110}

2.~Join $OD$. Then [I.\ \textsc{xx}.]
the sum of $OP$, $PD$ is greater
than $OD$; but $OD$ is equal to
$OE$ [I.\ Def.\ \textsc{xxx}.]. Therefore
the sum of $OP$, $PD$ is greater than $OE$. Reject $OP$,
which is common, and we have $PD$ \emph{greater than $PE$.}

3.~Join $OC$; then two triangles $POB$, $POC$ have
the side $OB$ equal to $OC$ [I.\ Def.\ \textsc{xxx}.], and $OP$ common;
but the angle $POB$ is greater than $POC$; therefore
[I.\ \textsc{xxiv}.] the base $PB$ is greater than $PC$. In
like manner $PC$ is \emph{greater than $PD$.}

4.~Make at the centre $O$ the angle $POF$ equal to
$POD$. Join $PF$. Then the triangles $POD$, $POF$ have
the two sides $OP$, $OD$ in one respectively equal to the
sides $OP$, $OF$ in the other, and the angle $POD$ equal to

the angle $POF$; hence $PD$ is equal to $PF$ [I.\ \textsc{iv.}], and
the angle $OPD$ equal to the angle $OPF$. \emph{Therefore $PD$
and $PF$ make equal angles with the diameter.}

5.~A third line cannot be drawn from $P$ equal to
either of the equal lines $PD$, $PF$. If possible let $PG$
be equal to $PD$, then $PG$ is equal to $PF$---that is, the
line which is nearest to the one through the centre is
equal to the one which is more remote, which is impossible.
\emph{Hence three equal lines cannot be drawn from
$P$ to the circumference.}

\textit{Cor.}~1.---If two equal lines $PD$, $PF$ be drawn from
a point $P$ to the circumference of a circle, the diameter
through $P$ bisects the angle $DPF$ formed by these
lines.

\textit{Cor.}~2.---If $P$ be the common centre of circles whose
radii are $PA$, $PB$, $PC$, \&c., then---1.\ The circle whose
radius is the maximum line $(PA)$ lies outside the circle
$ADE$, and touches it in $A$ [Def.\ \textsc{iv.}]. 2.~The circle
whose radius is the minimum line $(PE)$ lies inside the
circle $ADE$, and touches it in $E$. 3.~A circle having
any of the remaining lines $(PD)$ as radius cuts $ADE$
in two points $(D,\ F)$.\par\medskip

\begin{footnotesize}
\textsf{Observation.}---Proposition \textsc{vii.}\ affords a good illustration of
the following important definition (see \emph{Sequel to Euclid,} p.~13):---If
a geometrical magnitude varies its position continuously according
to any law, and if it retains the same value throughout,
it is said to be a constant, such as the radius of a circle revolving
round the centre; but if it goes on increasing for some time, and
then begins to decrease, it is said to be \emph{a maximum}\index{Maximum} at the end of
the increase. Thus, in the foregoing figure, $PA$, supposed to
revolve round $P$ and meet the circle, is a maximum. Again, if it
decreases for some time, and then begins to increase, it is \emph{a minimum}\index{Minimum}
at the commencement of the increase. Thus $PE$, supposed
as before to revolve round $P$ and meet the circle, is a minimum.
Proposition \textsc{viii.}\ will give other illustrations.
\par\end{footnotesize}


\mypropflow{PROP\@.~VIII\@.---Theorem.}{If from any point $(P)$ outside a circle, lines $(PA$,\ $PB$,\ $PC$,\ \textrm{\&c.}$)$
be drawn to the concave circumference,
then---\textrm{1.}~The maximum is that which passes through the centre.
\textrm{2.}~Of the others, that which is nearer to the one through
the centre is greater than the one more remote. Again, if
lines be drawn to the convex circumference---\textrm{3.}~The minimum
is that whose production passes through the centre.
\textrm{4.}~Of the others, that which is nearer to the minimum is
less than one more remote. \textrm{5.}~From the given point $(P)$
there can be drawn two equal lines to the concave or the
convex circumference, both of which make equal angles with
the line passing through the centre. \textrm{6.}~More than two
equal lines cannot be drawn from the given point $(P)$ to
either circumference.}{165}{18}{f111}

\textbf{Dem.}---1.~Let $O$ be the centre. Join $OB$. Now
since $O$ is the centre, $OA$ is equal to $OB$: to each add
$OP$, and we have $AP$ equal to the sum of $OB$, $OP$; but
the sum of $OB$, $OP$ is greater than $BP$ [I.\ \textsc{xx.}]. \textit{Therefore
$AP$ is greater than $BP$.}

2.~Join $OC$, $OD$. The two triangles $BOP$, $COP$
have the side $OB$ equal
to $OC$, and $OP$ common,
and the angle $BOP$
greater than $COP$; \textit{therefore
the base $BP$ is greater
than $CP$} [I.\ \textsc{xxiv.}]. \textit{In
like manner $CP$ is greater
than $DP$}, \&c.

3.~Join $OF$. Now in
the triangle $OFP$ the
sum of the sides $OF$,
$FP$ is greater than $OP$
[I.\ \textsc{xx.}]; but $OF$ is equal
to $OE$ [I.\ Def.\ \textsc{xxx.}].
Reject them, and $FP$ \textit{will
remain greater than $EP$}.

4.~Join $OG$, $OH$. The two triangles $GOP$, $FOP$

have two sides $GO$, $OP$ in one respectively equal to
two sides $FO$, $OP$ in the other; but the angle $GOP$ is
greater than $FOP$; \emph{therefore} [I.\ \textsc{xxiv.}] 
\emph{the base $GP$ is
greater than $FP$. In like manner $HP$ is greater than
$GP$.}

5.~Make the angle $POI$ equal $POF$ [I.\ \textsc{xxiii}.]. Join
$IP$. Now the triangles $IOP$, $FOP$ have two sides $IO$,
$OP$ in one respectively equal to two sides $FO$, $OP$ in
the other, and the angle $IOP$ equal to $FOP$ (const.);
\emph{therefore} [I.\ \textsc{iv}.] \emph{$IP$ is equal to $FP$.}

6.~\emph{A third line cannot be drawn from $P$ equal to either
of the lines $IP$, $FP$.} For if possible let $PK$ be equal to
$PF$; then $PK$ is equal to $PI$---that is, one which is
nearer to the minimum equal to one more remote---\emph{which
is impossible}.

\emph{Cor}.\ 1.---If $PI$ be produced to meet the circle again
in $L$, $PL$ is equal to $PB$.

\emph{Cor}.\ 2.---If two equal lines be drawn from $P$ to
either the convex or concave circumference, the diameter
through $P$ bisects the angle between them,
and the parts of them intercepted by the circle are
equal.

\emph{Cor}.\ 3.---If $P$ be the common centre of circles whose
radii are lines drawn from $P$ to the circumference of
$HDE$, then---1. The circle whose radius is the minimum
line $(PE)$ has contact of the \emph{first kind} with $ADE$
[Def.\ \textsc{iv}.]. 2. The circle whose radius is the maximum
line $(PA)$ has contact of the \emph{second kind}. 3. A
circle having any of the remaining lines $(PF)$ as radius
intersects $HDE$ in two points $(F,\ I)$.




\mypropflow{PROP\@.~IX\@.---Theorem.}{A point $(P)$ within a circle $(ABC)$, from which more
than two equal lines $(PA,\ PB,\ PC,\ $\textrm{\&c.}$)$ can be drawn to
the circumference, is the centre.}{108}{10}{f112}

\textbf{Dem.}---If $P$ be not the centre, let $O$ be the centre.
Join $OP$, and produce it to meet the
circle in $D$ and $E$; then $DE$ is the
diameter, and $P$ is a point in it
which is not the centre: therefore
[\textsc{vii.}] only two equal lines can be
drawn from $P$ to the circumference;
but three equal lines are drawn
(hyp.), which is absurd. \emph{Hence $P$
must be the centre.}\par\medskip

\begin{footnotesize}
\emph{Or thus:} Since the lines $AP$, $BP$ are equal, the line bisecting
the angle $APB$ [\textsc{vii.}~\emph{Cor.}~1] must pass through the centre: in like
manner the line bisecting the angle $BPC$ must pass through the
centre. \emph{Hence the point of intersection of these bisectors, that is,
the point $P$, must be the centre.}
\par\end{footnotesize}

\myprop{PROP\@.~X.---Theorem.}{If two circles have more than two points common, they
must coincide.}


\imgflow{120}{10}{f113}

\textbf{Dem.}---Let $X$ be one of the circles; and if possible
let another circle $Y$ have three
points, $A$, $B$, $C$, in common with
$X$, without coinciding with it.
Find $P$, the centre of $X$. Join
$PA$, $PB$, $PC$. Then since $P$ is the
centre of $X$, the three lines $PA$,
$PB$, $PC$ are equal to one another.

Again, since $Y$ is a circle and
$P$ a point, from which three equal
lines $PA$, $PB$, $PC$ can be drawn to its circumference,
$P$ must be the centre of $Y$. \emph{Hence $X$ and $Y$ are concentric,
which} [\textsc{v.}] \emph{is impossible.}

\emph{Cor.}---Two circles not coinciding cannot have more
than two points common. Compare I., Axiom~\textsc{x}., that
two right lines not coinciding cannot have more than
one point common.



\myprop{PROP\@.~XI\@.---Theorem.}{If one circle $(CPD)$ touch another circle $(APB)$ internally
at any point $P$, the line joining the centres must pass through that point.}


\imgflow{112}{10}{f114}

\textbf{Dem.}---Let $O$ be the centre of $APB$. Join $OP$. I
say the centre of the smaller
circle is in the line $OP$. If not,
let it be in any other position such
as $E$. Join $OE$, $EP$, and produce
$OE$ through $E$ to meet the circles
in the points $C$, $A$. Now since $E$
is a point in the diameter of the
larger circle between the centre
and $A$, $EA$ is less than $EP$
[\textsc{vii.}~2]; but $EP$ is equal to $EC$
(hyp.), being radii of the smaller circle. Hence $EA$
is less than $EC$; which is impossible; consequently the
centre of the smaller circle must be in the line $OP$.
Let it be $H$; then we see \textit{that the line joining the centres
passes through the point $P$.}

\textit{Or thus:} Since $EP$ is a line drawn from a point
within the circle $APB$ to the circumference, but not
forming part of the diameter through $E$, the circle
whose centre is $E$ and radius $EP$ cuts [\textsc{vii.}, \textit{Cor.}~2]
$APB$ in $P$, but it touches it (hyp.) also in $P$, which is
impossible. \textit{Hence the centre of the smaller circle $CPD$
must be in the line $OP$}.

\myprop{PROP\@.~XII\@.---Theorem.}{If two circles $(PCF,\ PDE)$ have external contact at any
point $P$, the line joining their centres must pass through
that point.}


\imgflow{150}{7}{f115}

\textbf{Dem.}---Let $A$ be the centre of one of the circles.
Join $AP$, and produce it to meet the second circle again
in $E$. I say the centre of the second circle is in the

line $PE$. If not, let it be elsewhere, as at $B$. Join
$AB$, intersecting the circles
in $C$ and $D$, and join $BP$.
Now since $A$ is the centre
of the circle $PCF$, $AP$ is
equal to $AC$; and since $B$
is the centre of the circle
$PDE$, $BP$ is equal to $BD$.
Hence the sum of the lines $AP$, $BP$ is equal to the
sum of the lines $AC$, $DB$; but $AB$ is greater than the
sum of $AC$ and $DB$; therefore $AB$ is greater than the
sum of $AP$, $PB$---that is, one side of a triangle greater
than the sum of the other two--which [I.~\textsc{xx.}] is impossible.
Hence the centre of the second circle must be
in the line $PE$. Let it be $G$, and we see that \textit{the line
through the centres passes through the point $P$.}

\textit{Or thus:} Since $BP$ is a line drawn from a point without
the circle $PCF$ to its circumference, and when produced
does not pass through the centre, the circle
whose centre is $B$ and radius $BP$ must cut the circle
$PCF$ in $P$ [\textsc{viii.}, \textit{Cor.}~3]; but it touches it (hyp.) also
in $P$, which is impossible. \textit{Hence the centre of the
second circle must be in the line $PE$.}\par\medskip

\begin{footnotesize}
\textsf{Observation.}---Propositions \textsc{xi, xii.}, may both be included in
one enunciation as follows:---``If two circles touch each other at
any point, the centres and that point are collinear.'' And this
latter Proposition is a limiting case of the theorem given in Proposition \textsc{iii.},
\textit{Cor.}~4, that ``The line joining the centres of two
intersecting circles bisects the common chord perpendicularly.''


\imgcent{313}{f116}

Suppose the circle whose centre is $O$ and one of the points of
intersection $A$ to remain fixed, while the second circle turns

round that point in such a manner that the second point of
intersection $B$ becomes ultimately consecutive to $A$; then, since
the line $OO'$ always bisects $AB$, we see that when $B$ ultimately
becomes consecutive to $A$, the line $OO'$ passes through $A$. In
consequence of the motion, the common chord will become in
the limit a tangent to each circle, as in the second diagram.---\textsc{Comberousse}\index{Comberousse},
\textit{G\'eom\'etrie Plane}, page~57.

\textit{Cor.}~1.---If two circles touch each other, their point of contact
is the union of two points of intersection. Hence a contact
counts for two intersections.

\newpage

\imgflow{100}{9}{f117}
\textit{Cor.}~2.---If two circles touch each other at any point, they
cannot have any other common point. For, since two circles
cannot have more than two points
common [\textsc{x.}], and that the point of contact
is equivalent to two common points,
circles that touch cannot have any
other point common. The following is
a formal proof of this Proposition:---Let
$O$, $O'$ be the centres of the two circles,
$A$ the point of contact, and let $O'$ lie
between $O$ and $A$; take any other point
$B$ in the circumference of $O$. Join $O'B$;
then [\textsc{vii.}] $O'B$ is greater than $O'A$; therefore the point $C$ is
outside the circumference of the smaller circle. Hence $B$ cannot
be common to both circles. In like manner, they cannot have
any other common point but $A$.
\par\end{footnotesize}

\myprop{PROP\@.~XIII\@.---Theorem.}{Two circles cannot have double contact, that it, cannot
touch each other in two points.}


\imgflow{120}{9}{f118}

\textbf{Dem.}---1. If possible let two circles touch each
other at two points $A$ and $B$.
Now since the two circles touch
each other in $A$, the line joining
their centres passes through $A$
[\textsc{xi.}]. In like manner, it passes
through $B$. Hence the centres
and the points $A$, $B$ are in one
right line; therefore $AB$ is a
diameter of each circle. Hence,
if $AB$ be bisected in $E$, $E$ must be the centre of each

circle---that is, the circles are concentric---\emph{which} [\textsc{v.}] \emph{is
impossible.}

2.~If two circles touched each other externally in
two distinct points, then [\textsc{xii.}] the line joining the
centres should pass through each point, \emph{which is impossible.}

\emph{Or thus:} Draw a line bisecting $AB$ at right angles.
Then this line [\textsc{i.}, \emph{Cor.}~1]\label{Icor1} must pass through the centre
of each circle, and therefore [\textsc{xi.~xii.}] must pass through
each point of contact, which is impossible. \emph{Hence two
circles cannot have double contact.}\par\medskip

\begin{footnotesize}
This Proposition is an immediate inference from the theorem
[\textsc{xii.}, \emph{Cor.}~1], that a point of contact counts for two intersections,
for then two contacts would be equivalent to four intersections;
but there cannot be more than two intersections [\textsc{x.}]. It also
follows from Prop.~\textsc{xii.}, \emph{Cor.}~2, that if two circles touch each
other in a point $A$, they cannot have any other point common;
\emph{hence they cannot touch again in $B$.}
\par\end{footnotesize}

\exhead{Exercises.}

\begin{footnotesize}
1.~If a variable circle touch two fixed circles externally, the
difference of the distances of its centre from the centres of the
fixed circles is equal to the difference or the sum of their radii,
according as the contacts are of the same or of opposite species
(Def.~\textsc{iv.}).

2.~If a variable circle be touched by one of two fixed circles
internally, and touch the other fixed circle either externally or
internally, the sum of the distances of its centre from the centres
of the fixed circles is equal to the sum or the difference of their
radii, according as the contact with the second circle is of the first
or second kind.

3.~If through the point of contact of two touching circles any
secant be drawn cutting the circles again in two points, the radii
drawn to these points are parallel.

4.~If two diameters of two touching circles be parallel, the
lines from the point of contact to the extremities of one diameter
pass through the extremities of the other.
\par\end{footnotesize}


\myprop{PROP\@.~XIV\@.---Theorem.}{In equal circles---$1$. equal chords $(AB,\ CD)$ are equally
distant from the centre. $2$. chords which are equally
distant from the centre are equal.}


\imgflow{108}{10}{f119}

\textbf{Dem.}---1. Let $O$ be the centre. Draw the perpendiculars
$OE$, $OF$. Join $AO$, $CO$.
Then because $AB$ is a chord in a
circle, and $OE$ is drawn from the
centre cutting it at right angles,
it bisects it [\textsc{iii.}]; therefore $AE$
is the half of $AB$. In like manner,
$CF$ is the half of $CD$; but $AB$
is equal to $CD$ (hyp.). Therefore
$AE$ is equal to $CF$ [I., Axiom~\textsc{vii.}].
And because $E$ is a right angle, $AO^{2}$ is equal to $AE^{2}
+ EO^{2}$. In like manner, $CO^{2}$ is equal to $CF^{2} + FO^{2}$;
but $AO^{2}$ is equal to $CO^{2}$. Therefore $AE^{2} + EO^{2}$ is
equal to $CF^{2} + FO^{2}$; and $AE^{2}$ has been proved equal to
$CF^{2}$. Hence $EO^{2}$ is equal to $FO^{2}$; therefore $EO$ is
equal to $FO$. \textit{Hence $AB$, $CD$ are} (Def.~\textsc{vi.}) \textit{equally
distant from the centre.}

2.~Let $EO$ be equal to $FO$, it is required to prove $AB$
equal to $CD$. The same construction being made, we
have, as before, $AE^{2} + EO^{2}$ equal to $CF^{2} + FO^{2}$; but $EO^{2}$
is equal to $FO^{2}$ (hyp.). Hence $AE^{2}$ is equal to $CF^{2}$,
and $AE$ is equal to $CF$; but $AB$ is double of $AE$, and
$CD$ double of $CF$. \emph{Therefore $AB$ is equal to $CD$.}

\exhead{Exercise.}

\begin{footnotesize}
If a chord of given length slide round a fixed circle---1. the
locus of its middle point is a circle; 2. the locus of any point
fixed in the chord is a circle.
\par\end{footnotesize}


\mypropl{PROP\@.~XV\@.---Theorem.}{The diameter $(AB)$ is the greatest chord in a circle;
and of the others, the chord $(CD)$ which is nearer to
the centre is greater than $(EF)$ one more remote, and
the greater is nearer to the centre than the less.}


\imgflow{120}{9}{f120}

\textbf{Dem.}---1. Join $OC$, $OD$, $OE$, and draw the perpendiculars
$OG$, $OH$; then because
$O$ is the centre, $OA$ is equal to
$OC$ [I., Def.~\textsc{xxxii.}]\label{Idefxxii}, and $OB$ is
equal to $OD$. Hence $AB$ is equal
to the sum of $OC$ and $OD$; but
the sum of $OC$, $OD$ is greater
than $CD$ [I.~\textsc{xx.}]. \emph{Therefore $AB$
is greater than $CD$.}

2.~Because the chord $CD$ is nearer to the centre
than $EF$, $OG$ is less than $OH$; and since the triangles
$OGC$, $OHE$ are right-angled, we have $OC^{2} = OG^{2} + GC^{2}$,
and $OE^{2} = OH^{2} + HE^{2}$; therefore $OG^{2} + GC^{2} = OH^{2} + HE^{2}$;
but $OG^{2}$ is less than $OH^{2}$; therefore $GC^{2}$ is greater
than $HE^{2}$, and $GC$ is greater than $HE$, but $CD$ and $EF$
are the doubles of $GC$ and $HE$. \emph{Hence $CD$ is greater
than $EF$.}

3.~Let $CD$ be greater than $EF$, it is required to
prove that $OG$ is less than $OH$.

As before, we have $OG^{2} + GC^{2}$ equal to $OH^{2} + HE^{2}$;
but $CG^{2}$ is greater than $EH^{2}$; therefore $OG^{2}$ is less
than $OH^{2}$. \emph{Hence $OG$ is less than $OH$.}

\exhead{Exercises.}

\begin{footnotesize}
1.~The shortest chord which can be drawn through a given
point within a circle is the perpendicular to the diameter which
passes through that point.

2.~Through a given point, within or without a given circle,
draw a chord of length equal to that of a given chord.


3.~Through one of the points of intersection of two circles
draw a secant---1.\ the sum of whose segments intercepted by the
circles shall be a maximum; 2.\ which shall be of any length
less than that of the maximum.

4.~Three circles touch each other externally\index{Circle!contact of} at $A$, $B$, $C$; the
chords $AB$, $AC$ of two of them are produced to meet the third
again in the points $D$ and $E$; prove that $DE$ is a diameter of the
third circle, and parallel to the line joining the centres of the
others.
\par\end{footnotesize}

\mypropl{PROP\@.~XVI\@.---Theorem.}{$1$. The perpendicular $(BI)$ to the diameter $(AB)$ of a
circle at its extremity $(B)$ touches the circle at that point.
$2$. Any other line $(BH)$ through the same point cuts the
circle.}


\imgflow{125}{10}{f121}

\textbf{Dem.}---1. Take any point $I$, and join it to the centre
$C$. Then because the angle $CBI$
is a right angle, $CI^{2}$ is equal to
$CB^{2} + BI^{2}$ [I.~\textsc{xlvii}.]; therefore
$CI^{2}$ is greater than $CB^{2}$. Hence
$CI$ is greater than $CB$, and the
point $I$ [note on I., Def.~\textsc{xxxii.}]
is without the circle. In like
manner, every other point in
$BI$, except $B$, is without the
circle. \emph{Hence, since $BI$ meets
the circle at $B$, but does not cut it, it must touch it.}

2.~To prove that $BH$, which is not perpendicular to
$AB$, cuts the circle. Draw $CG$ perpendicular to $HB$.
Now $BC^{2}$ is equal to $CG^{2} + GB^{2}$. Therefore $BC^{2}$ is
greater than $CG^{2}$, and $BC$ is greater than $CG$. Hence
[note on I., Def.~\textsc{xxxii.}] the point $G$ must be within
the circle, and consequently the line $BG$ produced
must meet the circle again, \emph{and must therefore cut it.}

This Proposition may be proved as follows:

\emph{At every point on a circle the tangent is perpendicular
to the radius.}



\Needspace*{4\baselineskip}

\imgflow{96}{7}{f122}
Let $P$ and $Q$ be two consecutive points on the circumference.
Join $CP$, $CQ$, $PQ$;
produce $PQ$ both ways. Now since
$P$ and $Q$ are consecutive points, $PQ$
is a tangent (Def.~\textsc{iii.}). Again, the
sum of the three angles of the triangle
$CPQ$ is equal to two right
angles; but the angle $C$ is infinitely
small, and the others are equal.
Hence each of them is a right angle. \textit{Therefore the
tangent is perpendicular to the diameter.}

\textit{Or thus:} A tangent is a limiting position of a secant,
namely, when the secant moves out until the two
points of intersection with the circle become consecutive;
but the line through the centre which bisects
the part of the secant within the circle [\textsc{iii.}] is perpendicular
to it. \textit{Hence, in the limit the tangent is
perpendicular to the line from the centre to the point
of contact.}

\textit{Or again:} The angle $CPR$ is always equal to $CQS$;
hence, when $P$ and $Q$ come together each is a right
angle, \textit{and the tangent is perpendicular to the radius.}

\exhead{Exercises.}

\begin{footnotesize}
1.~If two circles be concentric, all chords of the greater which
touch the lesser are equal.

2.~Draw a parallel to a given line to touch a given circle.

3.~Draw a perpendicular to a given line to touch a given
circle.

4.~Describe a circle having its centre at a given point---1.~and
touching a given line; 2.~and touching a given circle. How
many solutions of this case?

5.~Describe a circle of given radius that shall touch two given
lines. How many solutions?

6.~Find the locus of the centres of a system of circles touching
two given lines.

7.~Describe a circle of given radius that shall touch a given
circle and a given line, or that shall touch two given circles.
\par\end{footnotesize}


\myprop{PROP\@.~XVII\@.---Problem.}{From a given point $(P)$ without a given circle $(BCD)$ to
draw a tangent to the circle.}


\imgcent{295}{f123}

\textbf{Sol.}---Let $O$ (fig.~1) be the centre of the given circle.
Join $OP$, cutting the circumference in $C$. With $O$ as
centre, and $OP$ as radius, describe the circle $APE$.
Erect $CA$ at right angles to $OP$. Join $OA$, intersecting
the circle $BCD$ in $B$. Join $BP$; \emph{it will be the
tangent required.}

\textbf{Dem.}---Since $O$ is the centre of the two circles,
we have $OA$ equal to $OP$, and $OC$ equal to $OB$.
Hence the two triangles $AOC$, $POB$ have the sides
$OA$, $OC$ in one respectively equal to the sides $OP$, $OB$
in the other, and the contained angle common to both.
Hence [I.~\textsc{iv.}] the angle $OCA$ is equal to $OBP$; but
$OCA$ is a right angle (const.); therefore $OBP$ is a
right angle, and [\textsc{xvi.}] \emph{$PB$ touches the circle at $B$.}

\emph{Cor.}---If $AC$ (fig.~2) be produced to $E$, $OE$ joined,
cutting the circle $BCD$ in $D$, and the line $DP$ drawn,
\emph{$DP$ will be another tangent from $P$.}


\exhead{Exercises.}

\begin{footnotesize}
1.~The two tangents $PB$, $PD$ (fig.~2) are equal to one another,
because the square of each is equal to the square of $OP$ minus the
square of the radius.

2.~If two circles be concentric, all tangents to the inner from
points on the outer are equal.

3.~If a quadrilateral be circumscribed to a circle, the sum
of one pair of opposite sides is equal to the sum of the other
pair.

4.~If a parallelogram be circumscribed to a circle it must be
a lozenge, and its diagonals intersect in the centre.

5.~If $BD$ be joined, intersecting $OP$ in $F$, $OP$ is perpendicular
to $BD$.

6.~The locus of the intersection of two equal tangents to two
circles is a right line (\emph{called the radical axis\index{Axis!radical} of the two circles}).

7.~Find a point such that tangents from it to three given circles
shall be equal. (\emph{This point is called the radical centre\index{Centre!radical} of the three
circles.})

8.~The rectangle $OF\,.\,OP$ is equal to the square of the radius.

\textsc{Def.} \emph{Two points, such as $F$ and $P$, the rectangle of whose
distances $OF$, $OP$ from the centre is equal to the square of the
radius, are called inverse points\index{Points!inverse} with respect to the circle.}

9.~The intercept made on a variable tangent by two fixed tangents
subtends a constant angle at the centre.

10.~Draw a common tangent to two circles. Hence, show how
to draw a line cutting two circles, so that the intercepted chords
shall be of given lengths.
\par\end{footnotesize}


\myprop{PROP\@.~XVIII\@.---Theorem}{If a line $(CD)$ touch a circle, the line $(OC)$ from the centre
to the point of contact is perpendicular to it.}


\imgflow{135}{11}{f124}

\textbf{Dem.}---If not, suppose another line $OG$ drawn from
the centre to be perpendicular
to $CD$. Let $OG$ cut the
circle in $F$. Then because
the angle $OGC$ is right (hyp.)
the angle $OCG$ [I.~\textsc{xvii.}]
must be acute. Therefore
[I.~\textsc{xix.}] $OC$ is greater than
$OG$; but $OC$ is equal to $OF$
[I.~Def.~\textsc{xxxii.}]; therefore
$OF$ is greater than $OG$---that
is, a part greater than the
whole, which is impossible. \emph{Hence $OC$ must be perpendicular
to $CD$.}\par\smallskip

\begin{footnotesize}
\emph{Or thus:} Since the perpendicular must be the shortest line
from $O$ to $CD$, and $OC$ is evidently the shortest line; therefore
$OC$ must be perpendicular to $CD$.
\end{footnotesize}

\myprop{PROP\@.~XIX\@.---Theorem.}{If a line $(AB)$ be a tangent to a circle, the line $(AC)$
drawn at right angles to it from the point of contact passes
through the centre.}


\imgflow{130}{11}{f125}

If the centre be not in $AC$, let $O$ be the centre.
Join $AO$. Then because $AB$
touches the circle, and $OA$ is
drawn from the centre to the
point of contact, $OA$ is at right
angles to $AB$ [\textsc{xviii.}]; therefore
the angle $OAB$ is right,
and the angle $CAB$ is right
(hyp.); therefore $OAB$ is equal
to $CAB$---a part equal to the
whole, which is impossible.
\emph{Hence the centre must be in the line $AC$.}



\emph{Cor.}---If a number of circles touch the same line at
the same point, the locus of their centres is the perpendicular
to the line at the point.\par\medskip

\begin{footnotesize}
\textsf{Observation.}---Propositions \textsc{xvi., xviii., xix.,} are so related
that any two can be inferred from the third by the ``Rule of
Identity.'' Hence it would, in strict logic, be sufficient to prove
any one of the three, and the others would follow. Again, these
three theorems are limiting cases of Proposition~\textsc{i.}, \emph{Cor.}~1., and
Parts 1, 2, of Proposition \textsc{iii.}, namely, when the points in which
the chord cuts the circle become consecutive.
\end{footnotesize}

\myprop{PROP\@.~XX\@.---Theorem.}{The angle $(AOB)$ at the centre $(O)$ of a circle is double
the angle $(ACB)$ at the circumference standing on the
same arc.}


\imgcent{325}{f126}

\textbf{Dem.}---Join $CO$, and produce it to $E$. Then because
$OA$ is equal to $OC$, the angle $ACO$ is equal to $OAC$;
but the angle $AOE$ is equal to the sum of the two
angles $OAC$, $ACO$. Hence the angle $AOE$ is double
the angle $ACO$. In like manner the angle $EOB$ is
double the angle $OCB$. Hence (by adding in figs.~($\alpha$),
($\beta$), and subtracting in ($\gamma$)), \emph{the angle $AOB$ is double of
the angle $ACB$.}\par\smallskip

\emph{Cor.}---If $AOB$ be a straight line, $ACB$ will be a
right angle---\emph{that is, the angle in a semicircle is a right
angle} (compare \textsc{xxxi.}).


\myprop{PROP\@.~XXI\@.---Theorem.}{The angles $(ACB,\ ADB)$ in the same segment of a circle
are equal.}


\imgcent{235}{f127}

\textbf{Dem.}---Let $O$ be the centre. Join $OA$, $OB$. Then
the angle $AOB$ is double of the angle $ACB$ [\textsc{xx.}], and
also double of the angle $ADB$. \textit{Therefore the angle
$ACB$ is equal to the angle $ADB$.}

The following is the proof of the second part---that
is, when the arc $AB$ is not greater than a semicircle,
without using angles greater than two right angles:---


\imgflow{115}{9}{f128}

Let $O$ be the centre. Join $CO$, and produce it to meet
the circle again in $E$. Join $DE$.
Now since $O$ is the centre, the
segment $ACE$ is greater than a
semicircle; hence, by the first
case, fig.~($\alpha$), the angle $ACE$ is
equal to $ADE$. In like manner
the angle $ECB$ is equal to $EDB$.
\textit{Hence the whole angle $ACB$ is
equal to the whole angle $ADB$.}

\emph{Cor.}~1.---If two triangles $ACB$, $ADB$ on the same
base $AB$, and on the same side of it, have equal vertical
angles, the four points $A$, $C$, $D$, $B$ are concyclic\label{concylic1}.

\emph{Cor.}~2.---If $A$, $B$ be two fixed points, and if $C$ varies

its position in such a way that the angle $ACB$ retains
the same value throughout, the locus of $C$ is a circle.

In other words---\emph{Given the base of a triangle and the
vertical angle, the locus of the vertex is a circle.}

\exhead{Exercises.}

\begin{footnotesize}
1.~Given the base of a triangle and the vertical angle, find the
locus---

{\parindent= 3em
\hangindent= 5em
(1) of the intersection of its perpendiculars;

\hangindent= 5em
(2) of the intersection of the internal bisectors of its base
angles;

\hangindent= 5em
(3) of the intersection of the external bisectors of the base
angles;

\hangindent= 5em
(4) of the intersection of the external bisector of one base
angle and the internal bisector of the other.
}

2.~If the sum of the squares of two lines be given, their sum
is a maximum when the lines are equal.

3.~Of all triangles having the same base and vertical angle, the
sum of the sides of an isosceles triangle is a maximum.

4.~Of all triangles inscribed in a circle, the equilateral triangle
has the maximum perimeter.

5.~Of all concyclic figures having a given number of sides, the
area is a maximum when the sides are equal.
\par\end{footnotesize}

\myprop{PROP\@.~XXII\@.---Theorem.}{The sum of the opposite angles of a quadrilateral $(ABCD)$
inscribed in a circle is two right angles.}


\imgcent{225}{f129}

\textbf{Dem.}---Join $AC$, $BD$. The angle $ABD$ is equal to
$ACD$, being in the same segment $ABCD$ [\textsc{xxi.}]; and

the angle $DBC$ is equal to $DAC$, because they are in
the same segment $DABC$. Hence the whole angle $ABC$
is equal to the sum of the two angles $ACD$, $DAC$. To
each add the angle $CDA$, and we have the sum of the
two angles $ABC$, $CDA$ equal to the sum of the three
angles $ACD$, $DAC$, $CDA$ of the triangle $ACD$; but the
sum of the three angles of a triangle is equal to two
right angles [I.~\textsc{xxxii.}]. \emph{Therefore the sum of $ABC$,
$CDA$ is two right angles.}

\emph{Or thus:} Let $O$ be the centre of the circle. Join
$OA$, $OC$ (\emph{see} fig.~2). Now the angle $AOC$ is double of
$CDA$ [\textsc{xx.}], and the angle $COA$ is double of $ABC$.
Hence the sum of the angles [I.~Def.~\textsc{ix.}, note]
$AOC$, $COA$ is double of the sum of the angles $CDA$,
$ABC$; but the sum of two angles $AOC$, $COA$ is four
right angles. \emph{Therefore the sum of the angles $CDA$,
$ABC$ is two right angles.}

\emph{Or again:} Let $O$ be the centre (fig.~2). Join $OA$, $OB$
$OC$, $OD$. Then the four triangles $AOB$, $BOC$, $COD$,
$DOA$ are each isosceles. Hence the angle $OAB$ is
equal to the angle $OBA$, and the angle $OAD$ equal to
the angle $ODA$; therefore the angle $BAD$ is equal to
the sum of the angles $OBA$, $ODA$. In like manner
the angle $BCD$ is equal to the sum of the angles $OBC$,
$ODC$. Hence the sum of the two angles $BAD$, $BCD$
is equal to the sum of the two angles $ABC$, $ADC$, \emph{and
hence each sum is two right angles.}

\emph{Cor.}---If a parallelogram be inscribed in a circle it is
a rectangle.

\exhead{Exercises.}

\begin{footnotesize}
1.~If the opposite angles of a quadrilateral be supplemental, it
is cyclic.

2.~If a figure of six sides be inscribed in a circle, the sum of
any three alternate angles is four right angles.

3.~A line which makes equal angles with one pair of opposite
sides of a cyclic quadrilateral, makes equal angles with the remaining
pair and with the diagonals.


4.~If two opposite sides of a cyclic quadrilateral be produced to
meet, and a perpendicular be let fall on the bisector of the angle
between them from the point of intersection of the diagonals, this
perpendicular will bisect the angle between the diagonals.

5.~If two pairs of opposite sides of a cyclic hexagon be respectively
parallel to each other, the remaining pair of sides are also
parallel.

6.~If two circles intersect in the points $A$, $B$, and any two lines
$ACD$, $BFE$, be drawn through $A$ and $B$, cutting one of the circles
in the points $C$, $E$, and the other in the points $D$, $F$, the line $CE$
is parallel to $DF$.

7.~If equilateral triangles be described on the sides of any
triangle, the lines joining the vertices of the original triangle
to the opposite vertices of the equilateral triangles are concurrent.

8.~In the same case prove that the centres of the circles described
about the equilateral triangles form another equilateral
triangle.

9.~If a quadrilateral be described about a circle, the angles at
the centre subtended by the opposite sides are supplemental.

10.~The perpendiculars of a triangle are concurrent.

11.~If a variable tangent meets two parallel tangents it subtends
a right angle at the centre.

12.~The feet of the perpendiculars let fall on the sides of a
triangle from any point in the circumference of the circumscribed
circle are collinear (\textsc{Simson})\index{Simson}.

\textsc{Def.}---\emph{The line of collinearity is called Simson's line.}

13.~If a hexagon be circumscribed about a circle, the sum of
the angles subtended at the centre by any three alternate sides is
equal to two right angles.
\par\end{footnotesize}

\myprop{PROP\@.~XXIII---Theorem.}{Two similar segments of circles which do not coincide
cannot be constructed on the same chord $(AB)$, and on the
same side of that chord.}


\imgflow{145}{5}{f130}

\textbf{Dem.}---If possible, let $ACB$, $ADB$, be two similar
segments constructed on the same side of $AB$. Take

any point $D$ in the inner one. Join $AD$, and produce
it to meet the outer one in $C$.
Join $BC$, $BD$. Then since
the segments are similar,
the angle $ADB$ is equal to
$ACB$ (Def.~\textsc{x.}), which is impossible
[I.~\textsc{xvi.}]. \emph{Hence two similar segments not coinciding
cannot be described on the same chord and on the
same side of it.}

\myprop{PROP\@.~XXIV\@.---Theorem.}{Similar segments of circles $(AEB,\ CFD)$ on equal chords
$(AB,\ CD)$ are equal to one another.}


\imgcent{300}{f131}

\textbf{Dem.}---Since the lines are equal, if $AB$ be applied
to $CD$, so that the point $A$ will coincide with $C$, and
the line $AB$ with $CD$, the point $B$ shall coincide with
$D$; and because the segments are similar, they must
coincide [\textsc{xxiii.}]. \emph{Hence they are equal.}\par\medskip

\begin{footnotesize}
This demonstration may be stated as follows:---Since the chords
are equal, they are congruent; and therefore the segments, being
similar, must be congruent.
\par\end{footnotesize}

\myprop{PROP\@.~XXV\@.---Problem.}{An arc $(ABC)$ of a circle being given, it is required to
describe the whole circle.}

\textbf{Sol.}---Take any three points $A$, $B$, $C$ in the arc.
Join $AB$, $BC$. Bisect $AB$ in $D$, and $BC$ in $E$. Erect
$DF$, $EF$ at right angles to $AB$, $BC$; then $F$, the point
of intersection, will be the centre of the circle.




\imgflow{108}{6}{f132}

\textbf{Dem.}---Because $DF$ bisects the chord $AB$ and is perpendicular
to it, it passes through
the centre [\textsc{i.}, \emph{Cor.}~1]\label{Icor1again}. In like
manner $EF$ passes through the
centre. \emph{Hence the point $F$ must be
the centre; and the circle described
from $F$ as centre, with $FA$ as radius,
will be the circle required.}

\begin{center}
\textsc{PROP\@.~XXVI\@.---Theorem.}
\end{center}

\begin{footnotesize}
The four Propositions \textsc{xxvi.--xxix.} are so like in their enunciations
that students frequently substitute one for another. The
following scheme will assist in remembering them:---

\indent\begin{tabular}{@{}c@{\ } *{6}{l@{\ }} c}
In Proposition & \textsc{xxvi.}\ are & given & angles & $=$, to & prove & arcs & $=$, \\
  ,, & \textsc{xxvii.}  & ,, & arcs   & $=$, & ,, & angles & $=$, \\
  ,, & \textsc{xxviii.} & ,, & chords & $=$, & ,, & arcs   & $=$, \\
  ,, & \textsc{xxix.}   & ,, & arcs   & $=$, & ,, & chords & $=$;
\end{tabular} \\[.5ex]
so that Proposition \textsc{xxvii.} is the converse of \textsc{xxvi.}, and \textsc{xxix.} of
\textsc{xxviii}.
\par\end{footnotesize}\medskip

\begin{center}
\emph{In equal circles $(ACB,\ DFE)$, equal angles at the
centres $(AOB,\ DHE)$ or at the circumferences $(ACB,\ DFE)$
stand upon equal arcs.}
\end{center}


\imgflow{190}{9}{f133}

\textbf{Dem.}---1. Suppose the angles at the centres to be
given equal. Now because the circles are equal their
radii are equal (Def.~\textsc{i}.).
Therefore the
two triangles $AOB$,
$DHE$ have the sides
$AO$, $OB$ in one respectively
equal to
the sides $DH$, $HE$
in the other, and the
angle $AOB$ equal to
$DHE$ (hyp.). Therefore [I.~\textsc{iv.}] the base $AB$ is equal
to $DE$.

Again, since the angles $ACB$, $DFE$ are [\textsc{xx.}] the
halves of the equal angles $AOB$, $DHE$, they are equal

[I.~Axiom~\textsc{vii.}]. Therefore (Def.~\textsc{x.}) the segments
$ACB$, $DFE$ are similar, and their chords $AB$, $DE$ have
been proved equal; therefore [\textsc{xxiv.}] the segments are
equal. And taking these equals from the whole circles,
which are equal (hyp.), the remaining segments $AGB$,
$DKE$ are equal. \emph{Hence the arcs $AGB$, $DKE$ are equal.}

2.~The demonstration of this case is included in the
foregoing.

\emph{Cor.}~1.---If the opposite angles of a cyclic quadrilateral
be equal, one of its diagonals must be a diameter
of the circumscribed circle.

\emph{Cor.}~2.---Parallel chords in a circle intercept equal
arcs.

\emph{Cor.}~3.---If two chords intersect at any point within
a circle, the sum of the opposite arcs which they intercept
is equal to the arc which parallel chords intersecting
on the circumference intercept. 2.~If they
intersect without the circle, the difference of the arcs
they intercept is equal to the arc which parallel chords
intersecting on the circumference intercept.

\emph{Cor.}~4.---If two chords intersect at right angles, the
sum of the opposite arcs which they intercept on the
circle is a semicircle.

\mypropl{PROP\@.~XXVII\@.---Theorem.}{In equal circles $(ACB,\ DFE)$, angles at the centres
$(AOB,\ DHE)$, or at the circumferences $(ACB,\ DFE)$,
which stand on equal arcs $(AB,\ DE)$, are equal.}


\imgflow{190}{9}{f134}

\textbf{Dem.}---If possible let one of them, such as $AOB$,
be greater than the
other, $DHE$; and
suppose a part such
as $AOL$ to be equal
to $DHE$. Then since
the circles are equal,
and the angles $AOL$,
$DHE$ at the centres
are equal (hyp.), the
arc $AL$ is equal to $DE$ [\textsc{xxvi.}]; but $AB$ is equal to $DE$

(hyp.). Hence $AL$ is equal to $AB$---that is, a part
equal to the whole, which is absurd. \emph{Therefore the
angle $AOB$ is equal to $DHE$.}

2.~\emph{The angles at the circumference, being the halves of
the central angles, are therefore equal.}

\mypropflow{PROP\@.~XXVIII\@.---Theorem.}{In equal circles $(ACB,$\ $DFE)$, equal chords $(AB,\ DE)$
divide the circumferences into arcs, which are equal each
to each---that is, the lesser to the lesser, and the greater to
the greater.}{190}{9}{f135}

\textbf{Dem.}---If the equal chords be diameters, the Proposition
is evident. If
not, let $O$, $H$ be the
centres. Join $AO$,
$OB$, $DH$, $HE$; then
because the circles
are equal their radii
are equal (Def.~\textsc{i.}).
Hence the two triangles
$AOB$, $DHE$
have the sides $AO$, $OB$ in one respectively equal to
the sides $DH$, $HE$ in the other, and the base $AB$ is
equal to $DE$ (hyp.). Therefore [I.~\textsc{viii.}] the angle
$AOB$ is equal to $DHE$. \emph{Hence the arc $AGB$ is equal
to $DKE$} [\textsc{xxvi.}]; and since the whole circumference
$AGBC$ is equal to the whole circumference $DKEF$,
\emph{the remaining arc $ACB$ is equal to the remaining arc
$DFE$.}

\exhead{Exercises.}

\begin{footnotesize}
1.~The line joining the feet of perpendiculars from any point
in the circumference of a circle, on two diameters given in position,
is given in magnitude.

2.~If a line of given length slide between two lines given in
position, the locus of the intersection of perpendiculars to the
given lines at its extremities is a circle. (This is the converse
of 1.)
\par\end{footnotesize}


\myprop{PROP\@.~XXIX\@.---Theorem.}{In equal circles $(ACB,\ DFE)$, equal arcs $(AGB,\ DCK)$
are subtended by equal chords.}

\textbf{Dem.}---Let $O$, $H$ be the centres (\emph{see} last fig.). Join
$AO$, $OB$, $DH$, $HE$; then because the circles are equal,
the angles $AOB$, $DHE$ at the centres, which stand on
the equal arcs $AGB$, $DKE$, are equal [\textsc{xxvii.}]. Again,
because the triangles $AOB$, $DHE$ have the two sides
$AO$, $OB$ in one respectively equal to the two sides $DH$,
$HE$ in the other, and the angle $AOB$ equal to the angle
$DHE$, \emph{the base $AB$ of one is equal to the base $DE$ of the
other.}\par\medskip

\begin{footnotesize}
\textsf{Observation.}---Since the two circles in the four last Propositions
are equal, they are congruent figures, and the truth of the
Propositions is evident by superposition.
\par\end{footnotesize}

\myprop{PROP\@.~XXX\@.---Problem.}{To bisect a given arc $ACB$.}

\textbf{Sol.}---Draw the chord $AB$; bisect it in $D$; erect $DC$
at right angles to $AB$, meeting the arc in $C$; then the
arc is bisected in $C$.


\imgflow{120}{6}{f136}

\textbf{Dem.}---Join $AC$, $BC$. Then
the triangles $ADC$, $BDC$ have
the side $AD$ equal to $DB$ (const.),
and $DC$ common to both, and
the angle $ADC$ equal to the
angle $BDC$, each being right. Hence the base $AC$ is
equal to the base $BC$. Therefore [\textsc{xxviii.}] the arc $AC$
is equal to the arc $BC$. \emph{Hence the arc $AB$ is bisected
in $C$.}

\exhead{Exercises.}

\begin{footnotesize}
1.~$ABCD$ is a semicircle whose diameter is $AD$; the chord $BC$
produced meets $AD$ produced in $E$: prove that if $CE$ is equal to
the radius, the arc $AB$ is equal to three times $CD$.


2.~The internal and the external bisectors of the vertical angle
of a triangle inscribed in a circle meet the circumference again in
points equidistant from the extremities of the base.

3.~If from $A$, one of the points of intersection of two given
circles, two chords $ACD$, $AC'D'$ be drawn, cutting the circles in
the points $C$, $D$; $C'$, $D'$, the triangles $BCD$, $BC'D'$, formed by
joining these to the second point $B$ of intersection of the circles,
are equiangular.

4.~If the vertical angle $ACB$ of a triangle inscribed in a circle
be bisected by a line $CD$, which meets the circle again in $D$, and
from $D$ perpendiculars $DE$, $DF$ be drawn to the sides, one of
which must be produced: prove that $EA$ is equal to $BF$, and
hence show that $CE$ is equal to half the sum of $AC$, $BC$.
\par\end{footnotesize}


\mypropl{PROP\@.~XXXI\@.---Theorem.}{In a circle---$(1)$. The angle in a semicircle is a right
angle. $(2)$. The angle in a segment greater than a semicircle
is an acute angle. $(3)$. The angle in a segment less
than a semicircle is an obtuse angle.}


\imgflow{140}{13}{f137}

\textbf{Dem.}---(1). Let $AB$ be the diameter, $C$ any point in
the semicircle. Join $AC$, $CB$.
\textit{The angle $ACB$ is a right angle.}

For let $O$ be the centre.
Join $OC$, and produce $AC$ to
$F$. Then because $AO$ is equal
to $OC$, the angle $ACO$ is equal
to the angle $OAC$. In like
manner, the angle $OCB$ is
equal to $CBO$. Hence the
angle $ACB$ is equal to the
sum of the two angles $BAC$,
$CBA$; but [I.\ \textsc{xxxii.}] the
angle $FCB$ is equal to the sum of the two interior
angles $BAC$, $CBA$ of the triangle $ABC$. Hence the
angle $ACB$ is equal to its adjacent angle $FCB$, \textit{and
therefore it is a right angle} [I.\ Def.\ \textsc{xiii.}].

(2). Let the arc $ACE$ be greater than a semicircle.
Join $CE$. Then the angle $ACE$ is evidently less than
$ACB$; but $ACB$ is right; \textit{therefore $ACE$ is acute}.



(3). Let the arc $ACD$ be less than a semicircle; then
evidently, from (1), \emph{the angle $ACD$ is obtuse.}

\emph{Cor.}~1.---If a parallelogram be inscribed in a circle,
its diagonals intersect at the centre of the circle.

\emph{Cor.}~2.---Find the centre of a circle by means of a
carpenter's square.

\emph{Cor.}~3.---From a point outside a circle draw two
tangents to the circle.

\mypropl{PROP\@.~XXXII\@.---Theorem.}{If a line $(EF)$ be a tangent to a circle, and from the
point of contact $(A)$ a chord $(AC)$ be drawn cutting the
circle, the angles made by this line with the tangent are
respectively equal to the angles in the alternate segments
of the circle.}


\imgflow{143}{11}{f138}

\textbf{Dem.}---(1). If the chord passes through the centre,
the Proposition is evident,
for the angles are right
angles; but if not, from the
point of contact $A$ draw $AB$
at right angles to the tangent.
Join $BC$. Then because $EF$
is a tangent to the circle,
and $AB$ is drawn from the
point of contact perpendicular
to $EF$, $AB$ passes through
this centre [\textsc{xix.}]. Therefore
the angle $ACB$ is right [\textsc{xxxi.}]. Hence the sum of the
two remaining angles $ABC$, $CAB$ is one right angle;
but the angle $BAF$ is right (const.); therefore the sum
of the angles $ABC$, $BAC$ is equal to $BAF$. Reject
$BAC$, which is common, and we get \emph{the angle $ABC$
equal to the angle $FAC$}.

(2). Take any point $D$ in the arc $AC$. It is required
to prove that the angle $CAE$ is equal to $CDA$.


\imgflow{130}{12}{f139}

Since the quadrilateral $ABCD$ is cyclic, the sum of
the opposite angles $ABC$, $CDA$ is two right angles
[\textsc{xxii.}], and therefore equal to the sum of the angles

$FAC$, $CAE$; but the angles $ABC$, $FAC$ are equal (1).
Reject them, and we get the \emph{angle $CDA$ equal to
$CAE$.}

\emph{Or thus:} Take any point
$G$ in the semicircle $AGB$.
Join $AG$, $GB$, $GC$. Then the
angle $AGB = FAB$, each being
right, and $CGB = CAB$ [\textsc{xxi.}].
Therefore the remaining angle
$AGC = FAC$. Again, join $BD$,
$CD$. The angle $BDA = BAE$,
each being right, and $CDB
= CAB$ [\textsc{xxi.}]. \emph{Hence the angle
$CDA = CAE$.}---\textsc{Lardner}.\index{Lardner}


\imgflow{120}{9}{f140}

\textit{Or by the method of limits}, see \textsc{Townsend's}\index{Townsend} \emph{Modern
Geometry}, vol.~i., page~14.

The angle $BAC$ is equal to
$BDC$ [\textsc{xxi.}]. Now let the point
$B$ move until it becomes consecutive
to $A$; then $AB$ will be
a tangent, and $BD$ will coincide
with $AD$, and the angle $BDC$
with $ADC$. Hence, if $AX$ be
a tangent at $A$, $AC$ any chord,
\emph{the angle which the tangent makes with the chord is equal
to the angle in the alternate segment.}

\exhead{Exercises.}

\begin{footnotesize}
1.~If two circles touch, any line drawn through the point of
contact will cut off similar segments.

2.~If two circles touch, and any two lines be drawn through the
point of contact, cutting both circles again, the chord connecting
their points of intersection with one circle is parallel to the chord
connecting their points of intersection with the other circle.

3.~$ACB$ is an arc of a circle, $CE$ a tangent at $C$, meeting the
chord $AB$ produced in $E$, and $AD$ a perpendicular to $AB$ in $D$:
prove, if $DE$ be bisected in $C$, that the arc $AC = 2CB$.

4.~If two circles touch at a point $A$, and $ABC$ be a chord
through $A$, meeting the circles in $B$ and $C$: prove that the tangents
at $B$ and $C$ are parallel to each other, and that when one
circle is within the other, the tangent at $B$ meets the outer circle
in two points equidistant from $C$.


5.~If two circles touch externally, their common tangent at
either side subtends a right angle at the point of contact, and its
square is equal to the rectangle contained by their diameters.
\par\end{footnotesize}

\myprop{PROP\@.~XXXIII\@.---Problem.}{On a given right line $(AB)$ to describe a segment of a
circle which shall contain an angle equal to a given rectilineal
angle $(X)$.}

\textbf{Sol.}---If $X$ be a right angle, describe a semicircle on
the given line, and the thing required is done; for the
angle in a semicircle is a right angle.


\imgcent{290}{f141}

If not, make with the given line $AB$ the angle $BAE$
equal to $X$. Erect $AC$ at right angles to $AE$, and $BC$
at right angles to $AB$. On $AC$ as diameter describe a
circle: \emph{it will be the circle required.}

\textbf{Dem.}---The circle on $AC$ as diameter passes through
$B$, since the angle $ABC$ is right [\textsc{xxxi.}] and touches
$AE$, since the angle $CAE$ is right [\textsc{xvi.}]. Therefore
the angle $BAE$ [\textsc{xxxii.}] is equal to the angle in the
alternate segment; but the angle $BAE$ is equal to the
angle $X$ (const.). \emph{Therefore the angle $X$ is equal to
the angle in the segment described on $AB$.}

\exhead{Exercises.}

\begin{footnotesize}
1.~Construct a triangle, being given base, vertical angle, and
any of the following data:---1. Perpendicular. 2. The sum or
difference of the sides. 3. Sum or difference of the squares of
the sides. 4. Side of the inscribed square on the base. 5. The
median that bisects the base.


2.~If lines be drawn from a fixed point to all the points of the
circumference of a given circle, the locus of all their points of
bisection is a circle.

3.~Given the base and vertical angle of a triangle, find the
locus of the middle point of the line joining the vertices of
equilateral triangles described on the sides.

4.~In the same case, find the loci of the angular points of a
square described on one of the sides.
\par\end{footnotesize}

\myprop{PROP\@.~XXXIV\@.---Problem.}{To cut off from a given circle $(ABC)$ a segment which
shall contain an angle equal to a given angle $(X)$.}


\imgflow{140}{9}{f142}

\textbf{Sol.}---Take any point $A$ in the circumference. Draw
the tangent $AD$, and make the
angle $DAC$ equal to the given
angle $X$. $AC$ will cut off the
required segment.

\textbf{Dem.}---Take any point $B$
in the alternate segment.
Join $BA$, $BC$. Then the
angle $DAC$ is equal to $ABC$
[\textsc{xxxii.}]; but $DAC$ is equal
to $X$ (const.). \emph{Therefore the angle $ABC$ is equal to $X$.}

\myprop{PROP\@.~XXXV\@.---Theorem.}{If two chords $(AB,\ CD)$ of a circle intersect in a point
$(E)$ within the circle, the rectangles $(AE\,.\,EB,\ CE\,.\,ED)$
contained by the segments are equal.}


\imgflow{115}{9}{f143}

\textbf{Dem.}---1. If the point of intersection be the centre,
each rectangle is equal to the square of the radius.
\emph{Hence they are equal.}

2.~Let one of the chords $AB$ pass through the centre
$O$, and cut the other chord $CD$, which does not pass
through the centre, at right angles. Join $OC$. Now

because $AB$ passes through the centre, and cuts the
other chord $CD$, which does not
pass through the centre at right
angles, it bisects it [\textsc{iii.}]. Again,
because $AB$ is divided equally in
$O$ and unequally in $E$, the rectangle
$AE\,.\,EB$, together with $OE^{2}$,
is equal to $OB^{2}$---that is, to $OC^{2}$
[II\@.~\textsc{v.}]; but $OC^{2}$ is equal to $OE^{2}
+ EC^{2}$ [I.~\textsc{xlvii.}] Therefore
\[
AE\,.\,EB + OE^{2} = OE^{2} + EC^{2}.
\]
Reject $OE^{2}$, which is common, and we have $AE\,.\,EB
= EC^{2}$; but $CE^{2}$ is equal to the rectangle $CE\,.\,ED$,
since $CE$ is equal to $ED$. \emph{Therefore the rectangle
$AE\,.\,EB$ is equal to the rectangle $CE\,.\,ED$.}


\imgflow{110}{8}{f144}

3.~Let $AB$ pass through the centre, and cut $CD$,
which does not pass through the
centre obliquely. Let $O$ be the
centre. Draw $OF$ perpendicular to
$CD$ [I.~\textsc{xi.}]. Join $OC$, $OD$. Then,
since $CD$ is cut at right angles by
$OF$, which passes through the centre,
it is bisected in $F$ [\textsc{iii.}], and divided
unequally in $E$. Hence
\begin{flalign*}
&&           CE\,.\,ED + FE^{2} &= FD^{2} \text{\ [II\@.~\textsc{v.}]},  &&\\
&\text{and }&        OF^{2} &= OF^{2}.  &\phantom{and}&
\end{flalign*}
Hence, adding, since $FE^{2} + OF^{2} = OE^{2}$ [I.~\textsc{xlvii.}], and
$FD^{2} + OF^{2} = OD^{2}$, we get
\[
CE\,.\,ED + OE^{2} = OD \text{\ or } OB^{2}.
\]
Again, since $AB$ is bisected in $O$ and divided unequally
in $E$,
\begin{flalign*}
&&           AE\,.\,EB + OE^{2} &= OB^{2} \text{\ [II\@.~\textsc{v.}].}  &&\\
&\text{Therefore }&  CE\,.\,ED + OE^{2} &= AE\,.\,EB + OE^{2}.  &\phantom{Therefore}&\\
&\text{Hence }&               CE\,.\,ED &= AE\,.\,EB.  &&
\end{flalign*}




\imgflow{112}{10}{f145}

4.~Let neither chord pass through the centre.
Through the\label{thopoint} point $E$, where they
intersect, draw the diameter $FG$.
Then by 3, the rectangle $FE\,.\,EG$
is equal to the rectangle $AE\,.\,EB$,
and also to the rectangle $CE\,.\,ED$.
\emph{Hence the rectangle $AE\,.\,EB$ is
equal to the rectangle $CE\,.\,ED$.}

\emph{Cor.}~1.---If a chord of a circle
be divided in any point within
the circle, the rectangle contained by its segments is
equal to the difference between the square of the
radius and the square of the line drawn from the
centre to the point of section.

\emph{Cor.}~2.---If the rectangle contained by the segments
of one of two intersecting lines be equal to the rectangle
contained by the segments of the other, the four
extremities are concyclic.

\textit{Cor.}~3.---\textit{If two triangles be equiangular, the rectangle
contained by the non-corresponding
sides about any two equal angles are
equal.}

\Needspace*{4\baselineskip}

\imgflow{100}{9}{f146}
Let $ABO$, $DCO$ be the equiangular
triangles, and let them be
placed so that the equal angles at
$O$ may be vertically opposite, and
that the non-corresponding sides
$AO$, $CO$ may be in one line; then
the non-corresponding sides $BO$, $OD$ shall be in one
line. Now, since the angle $ABD$ is equal to $ACD$, the
points $A$, $B$, $C$, $D$ are concyclic\label{concylic2} [\textsc{xxi.}, \emph{Cor.}~1]. \emph{Hence
the rectangle $AO\,.\,OC$ is equal to the rectangle $BO\,.\,OD$}
[\textsc{xxxv}.].

\exhead{Exercises.}

\begin{footnotesize}
1.~In any triangle, the rectangle contained by two sides is
equal to the rectangle contained by the perpendicular on the third
side and the diameter of the circumscribed circle.

\textbf{Def.}---\emph{The supplement of an arc is the difference between it and
a semicircle.}


2.~The rectangle contained by the chord of an arc and the
chord of its supplement is equal to the rectangle contained by
the radius and the chord of twice the supplement.

3.~If the base of a triangle be given, and the sum of the sides,
the rectangle contained by the perpendiculars from the extremities
of the base on the external bisector of the vertical angle is given.

4.~If the base and the difference of the sides be given, the
rectangle contained by the perpendiculars from the extremities
of the base on the internal bisector is given.

5.~Through one of the points of intersection of two circles
draw a secant, so that the rectangle contained by the intercepted
chords may be given, or a maximum.


\imgflow{135}{16}{f147}

6.~\emph{If the sum of two arcs, $AC$, $CB$ of a circle be less than a
semicircle, the rectangle $AC\,.\,CB$
contained by their chords is equal
to the rectangle contained by the
radius, and the excess of the chord
of the supplement of their difference
above the chord of the supplement of
their sum.}---\textsc{Catalan.}

\textbf{Dem.}---Draw $DE$, the diameter
which is perpendicular to $AB$, and
draw the chords $CF$, $BG$ parallel
to $DE$. Now it is evident that the
difference between the arcs $AC$, $CB$
is equal to $2 CD$, and therefore
$= CD + EF$. Hence the arc $CBF$
is the supplement of the difference, and $CF$ is the chord of that
supplement. Again, since the angle $ABG$ is right, the arc $ABG$
is a semicircle. Hence $BG$ is the supplement of the sum of the
arcs $AC$, $CB$; therefore the line $BG$ is the chord of the supplement
of the sum. Now (Ex.~1), the rectangle $AC\,.\,CB$ is equal
to the rectangle contained by the diameter and $CI$, and therefore
equal to the rectangle contained by the radius and $2 CI$; but the
difference between $CF$ and $BG$ is evidently equal to $2 CI$. \emph{Hence
the rectangle $AC\,.\,CB$ is equal to the rectangle contained by the
radius and the difference between the chords $CF$, $BG$.}

7.~If we join $AF$, $BF$ we find, as before, the rectangle $AF\,.\,FB$
equal to the rectangle contained by the radius and $2 FI$---that
is, equal to the rectangle contained by the radius and the sum of
$CF$ and $BG$. Hence---\emph{If the sum of two arcs of a circle be greater
than a semicircle, the rectangle contained by their chords is equal to
the rectangle contained by the radius, and the sum of the chords of
the supplements of their sum and their difference.}

8.~Through a given point draw a transversal cutting two lines
given in position, so that the rectangle contained by the segments
intercepted between it and the line may be given.
\par\end{footnotesize}


\mypropl{PROP\@.~XXXVI\@.---Theorem.}{If from any point $(P)$ without a circle two lines be drawn
to it, one of which $(PT)$ is a tangent, and the other $(PA)$ a
secant, the rectangle $(AP,\ BP)$ contained by the segments
of the secant is equal to the square of the tangent.}


\imgflow{160}{9}{f148}

\textbf{Dem.}---1. Let $PA$ pass through the centre $O$. Join
$OT$. Then because $AB$ is
bisected in $O$ and divided
externally in $P$, the rectangle
$AP\,.\,BP + OB^{2}$ is
equal to $OP^{2}$ [II\@.~\textsc{vi.}].
But since $PT$ is a tangent,
and $OT$ drawn from the
centre to the point of
contact, the angle $OTP$
is right [\textsc{xviii.}]. Hence $OT^{2} + PT^{2}$ is equal to $OP^{2}$.
\begin{flalign*}
&\text{\indent Therefore\quad}&
  AP\,.\,BP + OB^{2} &= OT^{2} + PT^{2};
&\phantom{\indent Therefore\quad}&
\\
&\text{but }&  OB^{2} &= OT^{2}.  &&\\
&\multispan{2}{\emph{Hence the rectangle\quad \hfill
             $AP\,.\,BP$}} &= PT^{2}.  &&
\end{flalign*}


\imgflow{167}{9}{f149}

2.~If $AB$ does not pass through the centre $O$, let fall
the perpendicular $OC$
on $AB$. Join $OT$, $OB$,
$OP$. Then because $OC$,
a line through the centre,
cuts $AB$, which does
not pass through the
centre at right angles,
it bisects it [\textsc{iii.}].
Hence, since $AB$ is bisected
in $C$ and divided
externally in $P$, the rectangle
\begin{flalign*}
&&  AP\,.\,BP + CB^{2} &= CP^{2} \text{\ [II\@.~\textsc{vi.}]};  &&\\
&\text{and }&  OC^{2} &= OC^{2}.  &\phantom{and }&
\end{flalign*}
Hence, adding, since $CB^{2} + OC^{2} = OB^{2}$ [I.~\textsc{xlvii.}], and
$CP^{2} + OC^{2} = OP^{2}$, we get
\begin{flalign*}
&\text{rectangle }&  AP\,.\,BP + OB^{2} &= OP^{2};  &&\\
&\text{but }&  OT^{2} + PT^{2} &= OP^{2} \text{\ [I.~\textsc{xlvii.}].} &&\\
&\text{\indent Therefore}&  AP\,.\,BP + OB^2 &= OT^2 + PT^2;
\end{flalign*}


\noindent and rejecting the equals $OB^2$ and $OT^2$, \emph{we have the
rectangle}
\[
AP\,.\,BP = PT^2.
\]

\begin{footnotesize}
The two Propositions \textsc{xxxv., xxxvi.}, may be included in one
enunciation, as follows:---\emph{The rectangle $AP\,.\,BP$ contained by the
segments of any chord of a given circle passing through a fixed point
$P$, either within or without the circle, is constant.} For let $O$ be
the centre: join $OA$, $OB$, $OP$. Then $OAB$ is an isosceles triangle,
and $OP$ is a line drawn from its vertex to a point $P$ in the base,
or base produced. Then the rectangle $AP\,.\,BP$ is equal to the
difference of the squares of $OB$ and $OP$, and is therefore constant.
\par\end{footnotesize}

\emph{Cor}.~1.---If two lines $AB$, $CD$ produced meet in $P$,
and if the rectangle $AP\,.\,BP = CP\,.\,DP$, the points
$A$, $B$, $C$, $D$ are concyclic\label{concylic3} (compare \textsc{xxxv}., \emph{Cor.}~2).

\emph{Cor.}~2.---Tangents to two circles from any point in
their common chord are equal (compare \textsc{xvii}., Ex.~6).

\emph{Cor.}~3.---The common chords of any three intersecting
circles are concurrent (compare \textsc{xvii.}, Ex.~7).

\exhead{Exercise.}

\begin{footnotesize}
If from the vertex $A$ of a $\triangle\ ABC$, $AD$ be drawn, meeting
$CB$ produced in $D$, and making the angle $BAD = ACB$, prove
$DB\,.\,DC = DA^2$.
\par\end{footnotesize}

\mypropl{PROP\@.~XXXVII\@.---Theorem.}{If the rectangle $(AP\,.\,BP)$ contained by the segments of
a secant, drawn from any point $(P)$ without a circle, be
equal to the square of a line $(PT)$ drawn from the same
point to meet the circle, the line which meets the circle is
a tangent.}


\imgflow{175}{10}{f150}

\textbf{Dem.}---From $P$ draw $PQ$ touching the circle [\textsc{xvii.}].
Let $O$ be the centre.
Join $OP$, $OQ$, $OT$.
Now the rectangle
$AP\,.\,BP$ is equal to the
square on $PT$ (hyp.),
and equal to the square
on $PQ$ [\textsc{xxxvi.}]. Hence
$PT^2$ is equal to $PQ^2$,
and therefore $PT$ is
equal to $PQ$. Again,

the triangles $OTP$, $OQP$ have the side $OT$ equal $OQ$,
$TP$ equal $QP$, and the base $OP$ common; hence [I.~\textsc{viii.}]
the angle $OTP$ is equal to $OQP$; but $OQP$ is a right
angle, since $PQ$ is a tangent [\textsc{xviii.}]; hence $OTP$ is
right, \emph{and therefore} [\textsc{xvi.}] \emph{$PT$ is a tangent.}

\exhead{Exercises.}

\begin{footnotesize}
1.~Describe a circle passing through two given points, and
fulfilling either of the following conditions: 1, touching a given
line; 2, touching a given circle.

2.~Describe a circle through a given point, and touching two
given lines; or touching\label{touchlng} a given file and a given circle.

3.~Describe a circle passing through a given point, having its
centre on a given line and touching a given circle.

4.~Describe a circle through two given points, and intercepting
a given arc on a given circle.

5.~$A$, $B$, $C$, $D$ are four collinear points, and $EF$ is a common
tangent to the circles described upon $AB$, $CD$ as diameters:
prove that the triangles $AEB$, $CFD$ are equiangular.

6.~The diameter of the circle inscribed in a right-angled triangle
is equal to half the sum of the diameters of the circles
touching the hypotenuse, the perpendicular from the right angle
of the hypotenuse, and the circle described about the right-angled
triangle.
\par\end{footnotesize}


\addcontentsline{toc}{section}{Questions for Examination,}
\exhead{Questions for Examination on Book III\@.}

\begin{footnotesize}
1.~What is the subject-matter of Book III\@.?

2.~Define equal circles.

3.~What is the difference between a chord and a secant?

4.~When does a secant become a tangent?

5.~What is the difference between a segment of a circle and
a sector?

6.~What is meant by an angle in a segment?

7.~If an arc of a circle be one-sixth of the whole circumference,
what is the magnitude of the angle in it?

8.~What are linear segments?

9.~What is meant by an angle standing on a segment?

10.~What are concyclic points?

11.~What is a cyclic quadrilateral?

12.~How many intersections can a line and a circle have?

13.~What does the line become when the points of intersection
become consecutive?

14.~How many points of intersection can two circles have?

15.~What is the reason that if two circles touch they cannot
have any other common point?

16.~Give one enunciation that will include Propositions \textsc{xi.,
xii.}\ of Book~III\@.

17.~What Proposition is this a limiting case of?

18.~Explain the extended meaning of the word angle.

19.~What is Euclid's limit of an angle?

20.~State the relations between Propositions \textsc{xvi., xviii., xix.}

21.~What Propositions are these limiting cases of?

22.~How many common tangents can two circles have?

23.~What is the magnitude of the rectangle of the segments of
a chord drawn through a point 3.65 metres distant from the centre
of a circle whose radius is 4.25 metres?


24.~The radii of two circles are 4.25 and 1.75 feet respectively,
and the distance between their centres 6.5 feet; find the lengths
of their direct and their transverse common tangents.

25.~If a point be $h$ feet outside the circumference of a circle
whose diameter is 7920 miles, prove that the length of the tangent
drawn from it to the circle is $\sqrt{\dfrac{3h}{2}}$ miles.

26.~Two parallel chords of a circle are 12 perches and 16 perches
respectively, and their distance asunder is 2 perches; find the
length of the diameter.

27.~What is the locus of the centres of all circles touching a 
given circle in a given point?

28.~What is the condition that must be fulfilled that four
points may be concyclic?

29.~If the angle in a segment of a circle be a right angle and
a-half, what part of the whole circumference is it?

30.~Mention the converse Propositions of Book~III\@. which are
proved directly.


\label{ex21}31. What is the locus of the middle points of equal chords in a
circle?

32.~The radii of two circles are 6 and 8, and the distance between
their centres 10; find the length of their common chord.

33.~If a figure of any even number of sides be inscribed in a
circle, prove that the sum of one set of alternate angles is equal
to the sum of the remaining angles.
\par\end{footnotesize}


\addcontentsline{toc}{section}{Exercises,}
\exhead{Exercises on Book III\@.}

\begin{footnotesize}
1.~If two chords of a circle intersect at right angles, the sum
of the squares on their segments is equal to the square on the
diameter.

2.~If a chord of a given circle subtend a right angle at a fixed
point, the rectangle of the perpendiculars on it from the fixed
point and from the centre of the given circle is constant. Also
the sum of the squares of perpendiculars on it from two other
fixed points (which may be found) is constant.

3.~If through either of the points of intersection of two equal
circles any line be drawn meeting them again in two points,
these points are equally distant from the other intersection of the
circles.

4.~Draw a tangent to a given circle so that the triangle formed
by it and two fixed tangents to the circle shall be---1, a maximum;
2, a minimum.

5.~If through the points of intersection $A$, $B$ of two circles any
two lines $ACD$, $BEF$ be drawn parallel to each other, and meeting
the circles again in $C$, $D$, $E$, $F$; then $CD = EF$.

6.~In every triangle the bisector of the greatest angle is the
least of the three bisectors of the angles.

7.~The circles whose diameters are the four sides of any cyclic
quadrilateral intersect again in four concyclic points.

8.~The four angular points of a cyclic quadrilateral determine
four triangles whose orthocentres (the intersections of their perpendiculars)
form an equal quadrilateral.

9.~If through one of the points of intersection of two circles
we draw two common chords, the lines joining the extremities
of these chords make a given angle with each other.

10.~The square on the perpendicular from any point in the circumference
of a circle, on the chord of contact of two tangents, is
equal to the rectangle of the perpendiculars from the same point
on the tangents.

11.~Find a point in the circumference of a given circle, the
sum of the squares on whose distances from two given points may
be a maximum or a minimum.

12.~Four circles are described on the sides of a quadrilateral as
diameters. The common chord of any two on adjacent sides is
parallel to the common chord of the remaining two.


13.~The rectangle contained by the perpendiculars from any
point in a circle, on the diagonals of an inscribed quadrilateral, is
equal to the rectangle contained by the perpendiculars from the
same point on either pair of opposite sides.

14.~The rectangle contained by the sides of a triangle is
greater than the square on the internal bisector of the vertical
angle, by the rectangle contained by the segments of the base.

15.~If through $A$, one of the points of intersection of two
circles, we draw any line $ABC$, cutting the circles again in
$B$ and $C$, the tangents at $B$ and $C$ intersect at a given angle.

16.~If a chord of a given circle pass through a given point, the
locus of the intersection of tangents at its extremities is a right
line.

17.~The rectangle contained by the distances of the point where
the internal bisector of the vertical angle meets the base, and the
point where the perpendicular from the vertex meets it from the
middle point of the base, is equal to the square on half the difference
of the sides.

18.~State and prove the Proposition analogous to 17 for the
external bisector of the vertical angle.

19.~The square on the external diagonal of a cyclic quadrilateral
is equal to the sum of the squares on the tangents from
its extremities to the circumscribed circle.

20.~If a variable circle touch a given circle and a given line,
the chord of contact passes through a given point.

21.~If $A$, $B$, $C$ be three points in the circumference of a circle,
and $D$, $E$ the middle points of the arcs $AB$, $AC$; \label{theifthen}then if the line

$DE$ intersect the chords $AB$, $AC$ in the points $F$, $G$, $AF$ is equal
to $AG$.

22.~Given two circles, $O$, $O'$; then if any secant cut $O$ in the
points $B$, $C$, and $O'$ in the points $B'$, $C'$, and another secant cuts
them in the points $D$, $E$; $D'$, $E'$ respectively; the four chords
$BD$, $CE$, $B'D'$, $C'E'$ form a cyclic quadrilateral.

23.~If a cyclic quadrilateral be such that a circle can be
inscribed in it, the lines joining the points of contact are perpendicular
to each other.

24.~If through the point of intersection of the diagonals of a
cyclic quadrilateral the minimum chord be drawn, that point will
bisect the part of the chord between the opposite sides of the
quadrilateral.


25.~Given the base of a triangle, the vertical angle, and either
the internal or the external bisector at the vertical angle; construct
it.

26.~If through the middle point $A$ of a given arc $BAC$ we
draw any chord $AD$, cutting $BC$ in $E$, the rectangle $AD\,.\,AE$ is
constant.

27.~The four circles circumscribing the four triangles formed
by any four lines pass through a common point.

28.~If $X$, $Y$, $Z$ be any three points on the three sides of a triangle
$ABC$, the three circles about the triangles $YAZ$, $ZBX$, $XCY$
pass through a common point.

29.~If the position of the common point in the last question be
given, the three angles of the triangle $XYZ$ are given, and conversely.

30.~Place a given triangle so that its three sides shall pass
through three given points.

31.~Place a given triangle so that its three vertices shall lie on
three given lines.

32.~Construct the greatest triangle equiangular to a given one
whose sides shall pass through three given points.

33.~Construct the least triangle equiangular to a given one
whose vertices shall lie on three given lines.

34.~Construct the greatest triangle equiangular to a given one
whose sides shall touch three given circles.

35.~If two sides of a given triangle pass through fixed points,
the third touches a fixed circle.

36.~If two sides of a given triangle touch fixed circles, the
third touches a fixed circle.

37.~Construct an equilateral triangle having its vertex at a
given point, and the extremities of its base on a given circle.

38.~Construct an equilateral triangle having its vertex at a
given point, and the extremities of its base on two given circles.

39.~Place a given triangle so that its three sides shall touch
three given circles.

40.~Circumscribe a square about a given quadrilateral.

41.~Inscribe a square in a given quadrilateral.

42.~Describe circles---(1) orthogonal\index{Orthogonal} (cutting at right angles) to
a given circle and passing through two given points; (2) orthogonal
to two others, and passing through a given point; (3) orthogonal
to three others.


43.~If from the extremities of a diameter $AB$ of a semicircle
two chords $AD$, $BE$ be drawn, meeting in $C$, $AC\,.\,AD + BC\,.\,BE
= AB^{2}$.

44.~If $ABCD$ be a cyclic quadrilateral, and if we describe any
circle passing through the points $A$ and $B$, another through $B$
and $C$, a third through $C$ and $D$, and a fourth through $D$ and $A$;
these circles intersect\label{circlesin} 
successively in four other points $E$, $F$, $G$, $H$,
forming another cyclic quadrilateral.

45.~If $ABC$ be an equilateral triangle, what is the locus of the
point $M$, if $MA = MB + MC$?

46.~In a triangle, given the sum or the difference of two sides
and the angle formed by these sides both in magnitude and position,
the locus of the centre of the circumscribed circle is a right line.

47.~Describe a circle---(1) through two given points which
shall bisect the circumference of a given circle; (2) through
one given point which shall bisect the circumference of two
given circles.

48.~Find the locus of the centre of a circle which bisects the
circumferences of two given circles.

49.~Describe a circle which shall bisect the circumferences of
three given circles.

50.~$AB$ is a diameter of a circle; $AC$, $AD$ are two chords meeting
the tangent at $B$ in the points $E$, $F$ respectively: prove that
the points $C$, $D$, $E$, $F$ are concyclic.

51.~$CD$ is a perpendicular from any point $C$ in a semicircle on
the diameter $AB$; $EFG$ is a circle touching $DB$ in $E$, $CD$ in $F$,
and the semicircle in $G$; prove---(1) that the points $A$, $F$, $G$ are
collinear; (2) that $AC = AE$.

52.~Being given an obtuse-angled triangle, draw from the
obtuse angle to the opposite side a line whose square shall be
equal to the rectangle contained by the segments into which it
divides the opposite side.

53.~$O$ is a point outside a circle whose centre is $E$; two perpendicular
lines passing through $O$ intercept chords $AB$, $CD$ on
the circle; then $AB^{2} + CD^{2} + 4OE^{2} = 8R^{2}$.

54.~The sum of the squares on the sides of a triangle is equal
to twice the sum of the rectangles contained by each perpendicular
and the portion of it comprised between the corresponding vertex
and the orthocentre; also equal to $12R^{2}$ minus the sum of the
squares of the distances of the orthocentre from the vertices.

55.~If two circles touch in $C$, and if $D$ be any point outside the
circles at which their radii through $C$ subtend equal angles, if
$DE$, $DF$ be tangent from $D$, $DE\,.\,DF=DC^{2}$.
\end{footnotesize}





\mybook{BOOK IV}{Inscription and Circumscription of Triangles and of Regular Polygons in and about Circles}

\textsc{i.} If two rectilineal figures be so related that the
angular points of one lie on the sides of the other---1,
the former is said to be inscribed\index{Inscribed} in the latter;
2, the latter is said to be described about the former.

\textsc{ii.} A rectilineal figure is said to be inscribed in a
circle when its angular points are on the circumference.
\emph{Reciprocally}, a rectilineal figure is said to be circumscribed\index{Circle!circumscribed}
to a circle when each side touches the circle.

\textsc{iii.} A circle is said to be inscribed in a rectilineal
figure when it touches each side of the figure. \emph{Reciprocally},
a circle is said to be circumscribed to a rectilineal
figure when it passes through each angular point
of the figure.

\textsc{iv.} A rectilineal figure which is both equilateral and
equiangular is said to be regular.\par\medskip

\begin{footnotesize}
\textsf{Observation.}---The following summary of the contents of the
Fourth Book will assist the student in remembering it:---

1.~It contains sixteen Propositions, of which four relate to
triangles, four to squares, four to pentagons, and four miscellaneous
Propositions.

2.~Of the four Propositions occupied with triangles---

($\alpha$) One is to inscribe a triangle in a circle.

($\beta$) Its reciprocal, to describe a triangle about a circle.

($\gamma$) To inscribe a circle in a triangle.

($\delta$) Its reciprocal, to describe a circle about a triangle.


3.~If we substitute in ($\alpha$), ($\beta$), ($\gamma$), ($\delta$) squares for triangles,
and pentagons for triangles, we have the problems for squares
and pentagons respectively.

4.~Every Proposition in the fourth Book is a problem.
\par\end{footnotesize}

\addcontentsline{toc}{section}{Propositions \textsc{i.--xvi.,}}
\myprop{PROP\@.~I.---Problem.}{In a given circle $(ABC)$ to place a chord equal to a given
line $(D)$ not greater than the diameter.}


\imgflow{180}{9}{f151}

\textbf{Sol.}---Draw any diameter $AC$ of the circle; then, if
$AC$ be equal to $D$,
the thing required is
done; if not, from $AC$
cut off the part $AE$
equal to $D$ [I.~\textsc{iii.}];
and with $A$ as centre
and $AE$ as radius, describe
the circle $EBF$,
cutting the circle $ABC$
in the points $B$, $F$. Join $AB$. \textit{Then $AB$ is the chord
required.}

\textbf{Dem.}---Because $A$ is the centre of the circle $EBF$,
$AB$ is equal to $AE$ [I.~Def.~\textsc{xxxii.}]; but $AE$ is equal
to $D$ (const.); \emph{therefore $AB$ is equal to $D$}.

\myprop{PROP\@.~II\@.---Problem.}{In a given circle $(ABC)$ to inscribe a triangle equiangular
to a given triangle $(DEF)$.}


\imgflow{170}{9}{f152}

\textbf{Sol.}---Take any point $A$ in the circumference, and
at it draw the tangent
$GH$; then make the
angle $HAC$ equal to
$E$, and $GAB$ equal to
$F$ [I.~\textsc{xxiii.}] Join $BC$.
\textit{$ABC$ is a triangle fulfilling
the required conditions.}

\textbf{Dem.}---The angle $E$ is equal to $HAC$ (const.), and

$HAC$ is equal to the angle $ABC$ in the alternate segment
[III\@.~\textsc{xxxii.}]. Hence the angle $E$ is equal to
$ABC$. In like manner the angle $F$ is equal to $ACB$.
Therefore [I.~\textsc{xxxii.}] the remaining angle $D$ is equal to
$BAC$. \textit{Hence the triangle $ABC$ inscribed in the given
circle is equiangular to $DEF$.}

\myprop{PROP\@.~III\@.---Problem.}{About a given circle $(ABC)$ to describe a triangle equiangular
to a given triangle $(DEF)$.}

\textbf{Sol.}---Produce any side $DE$ of the given triangle
both ways to $G$ and $H$, and from the centre $O$ of the
circle draw any radius $OA$; make the angle $AOB$
equal to $GEF$ [I.~\textsc{xxiii.}], and the angle $AOC$ equal to
$HDF$. At the points $A$, $B$, $C$ draw the tangents $LM$,
$MN$, $NL$ to the given circle. \emph{$LMN$ is a triangle fulfilling
the required conditions.}


\imgcent{230}{f153}

\textbf{Dem.}---Because $AM$ touches the circle at $A$, the
angle $OAM$ is right. In like manner, the angle $MBO$
is right; but the sum of the four angles of the quadrilateral
$OAMB$ is equal to four right angles. Therefore
the sum of the two remaining angles $AOB$, $AMB$
is two right angles; and [I.~\textsc{xiii.}] the sum of the two
angles $GEF$, $FED$ is two right angles. Therefore the
sum of $AOB$, $AMB$ is equal to the sum of $GEF$, $FED$;
but $AOB$ is equal to $GEF$ (const.). Hence $AMB$ is

equal to $FED$. In like manner, $ALC$ is equal to
$EDF$; therefore [I.~\textsc{xxxii.}] the remaining angle $BNC$
is equal to $DFE$. \textit{Hence the triangle $LMN$ is equiangular
to $DEF$.}

\myprop{PROP\@.~IV\@.---Problem.}{To inscribe a circle in a given triangle $(ABC)$.}

\textbf{Sol.}---Bisect any two angles $A$, $B$ of the given triangle
by the lines $AO$, $BO$; \emph{then $O$, their point of
intersection, is the centre of the required circle.}


\imgflow{120}{11}{f154}

\textbf{Dem.}---From $O$ let fall the perpendiculars $OD$, $OE$,
$OF$ on the sides of the triangle. 
Now, in the triangles $OAE$,
$OAF$ the angle $OAE$ is equal to
$OAF$ (const.), and the angle
$AEO$ equal to $AFO$, because
each is right, and the side $OA$
common. Hence [I.~\textsc{xxvi.}] the
side $OE$ is equal to $OF$. In
like manner $OD$ is equal to $OF$;
therefore the three lines $OD$,
$OE$, $OF$ are all equal. And the
circle described with $O$ as centre and $OD$ as radius will
pass through the points $E$, $F$; and since the angles
$D$, $E$, $F$ are right, it will [III\@.~\textsc{xvi.}] touch the three
sides of the triangle $ABC$; \textit{and therefore the circle $DEF$
is inscribed in the triangle $ABC$.}

\exheadin{Exercises.}


\imgflow{155}{19}{f155}

\begin{footnotesize}
1.~If the points $O$, $C$ be joined, the angle $C$ is bisected.
Hence ``the bisectors of the angles of a triangle are concurrent''
(compare I.~\textsc{xxvi.}, Ex.~7).

2.~If the sides $BC$, $CA$, $AB$ of the triangle $ABC$ be denoted by
$a$, $b$, $c$, and half their sum by $s$, the distances of the vertices
$A$, $B$, $C$ of the triangle from the points of contact of the inscribed
circle are respectively $s-a$, $s-b$, $s-c$.



3.~If the external angles of the triangle $ABC$ be bisected as in
the annexed diagram, the three
angular points $O'$, $O''$, $O'''$, of
the triangle formed by the three
bisectors will be the centres of
three circles, each touching one
side externally, and the other
two produced. These three
circles are called the \emph{escribed}\index{Circle!escribed}
circles of the triangle $ABC$.

4.~The distances of the vertices
$A$, $B$, $C$ from the points
of contact of the escribed circle
which touches $AB$ externally
are $s-b$, $s-a$, $s$.

5.~The centre of the inscribed
circle, the centre of each
escribed circle, and two of the
angular points of the triangle,
are concyclic. Also any two of the escribed centres are concyclic
with the corresponding two of the angular points of the triangle.

6.~Of the four points $O$, $O'$, $O''$, $O'''$, any one is the orthocentre
of the triangle formed by the remaining three.

7.~The three triangles $BCO'$, $CAO''$, $ABO'''$ are equiangular.

8.~The rectangle $CO\,.\,CO''' = ab$; $AO\,.\,AO' = bc$; $BO\,.\,BO''
= ca$.

9.~Since the whole triangle $ABC$ is made up of the three triangles
$AOB$, $BOC$, $COA$, we see that the rectangle contained by
the sum of the three sides, and the radius of the inscribed circle,
is equal to twice the area of the triangle. Hence, if $r$ denote the
radius of the inscribed circle, $rs =$ area of the triangle.

10.~If $r'$ denote the radius of the escribed circle which touches
the side $a$ externally, it may be shown in like manner that
$r'(s-a) =$ area of the triangle.

11.~$rr' = s-b\,.\,s-c$.

12.~Square of area $= s\,.\,s-a\,.\,s-b\,.\,s-c$.

13.~Square of area $= r\,.\,r'\,.\,r''\,.\,r'''$.

14.~If the triangle $ABC$ be right-angled, having the angle $C$
right,
\[
r = s-c;\ r' = s-b;\ r'' = s-a;\ r''' = s.
\]

15.~Given the base of a triangle, the vertical angle, and the
radius of the inscribed, or any of the escribed circles: construct
it.
\par\end{footnotesize}


\myprop{PROP\@.~V.---Problem.}{To describe a circle about a given triangle $(ABC)$.}


\imgflow{108}{10}{f156}

\textbf{Sol.}---Bisect any two sides $BC$, $AC$ in the points $D$, $E$.
Erect $DO$, $EO$ at right angles to $BC$,
$CA$; \emph{then $O$, the point of intersection
of the perpendiculars, is the centre
of the required circle.}

\textbf{Dem.}---Join $OA$, $OB$, $OC$. The
triangles $BDO$, $CDO$ have the side
$BD$ equal $CD$ (const.), and $DO$ common,
and the angle $BDO$ equal to
the angle $CDO$, because each is
right. Hence [I.~\textsc{iv.}] $BO$ is equal to $OC$. In like
manner $AO$ is equal to $OC$. Therefore the three lines
$AO$, $BO$, $CO$ are equal, and the circle described with
$O$ as centre, and $OA$ as radius, will pass through the
points $A$, $B$, $C$, \emph{and be described about the triangle $ABC$.}

\emph{Cor.}~1.---Since the perpendicular from $O$ on $AB$ bisects
it [III\@.~\textsc{iii.}], we see that the perpendiculars at the
middle points of the sides of a triangle are concurrent.

\textsc{Def.}---\textit{The circle $ABC$ is called the circumcircle\index{Circumcircle}, its
radius the circumradius, and its centre the circumcentre\index{Circumcentre}
of the triangle.}

\exhead{Exercises.}

\begin{footnotesize}
1.~The three perpendiculars of a triangle $(ABC)$ are concurrent.


\imgflow{103}{10}{f157}

\textbf{Dem.}---Describe a circle about the triangle. Let fall the perpendicular
$CF$. Produce $CF$ to meet the
circle in $G$. Make $FO = FG$. Join $AG$,
$AO$. Produce $AO$ to meet $BC$ in $D$.
Then the triangles $GFA$, $OFA$ have the
sides $GF$, $FA$ in one equal to the sides
$OF$, $FA$ in the other, and the contained
angles equal. Hence [I.~\textsc{iv}.] the angle
$GAF$ equal $OAF$; but $GAF = GCB$
[III\@.~\textsc{xxi}.]; hence $OAF = OCD$, and $FOA
= DOC$; hence $OFA = ODC$; but $OFA$
is right, hence $ODC$ is right. In like
manner, if $BO$ be joined to meet $AC$ in $E$, $BE$ will be perpendicular
to $AC$. \emph{Hence the three perpendiculars pass through $O$,

and are concurrent.} This Proposition may be proved simply as
follows:---


\imgflow{150}{13}{f158}

Draw parallels to the sides of the original triangle $ABC$ through
its vertices, forming a new triangle
$A'B'C'$ described about
$ABC$; then the three perpendiculars
at the middle points of the
sides of $A'B'C'$ are concurrent [\textsc{v.}~\emph{Cor.}~1],
and these are evidently
the perpendiculars from the vertices
on the opposite sides of the
triangle $ABC$ (compare Ex.~16,
Book~I.).

\textbf{Def.}---\textit{The point $O$ is called the
orthocentre\index{Orthocentre} of the triangle $ABC$.}

2.~The three rectangles $OA\,.\,OP$, $OB\,.\,OQ$, $OC\,.\,OR$ are equal.

\textbf{Def.}---\textit{The circle round $O$ as centre, the square of whose radius
is equal $OA\,.\,OP = OB\,.\,OQ = OC\,.\,OR$, is called the polar circle\index{Circle!polar}
of the triangle $ABC$.}

\textsf{Observation.}---If the orthocentre of the triangle $ABC$ be
within the triangle, the rectangles $OA\,.\,OP$, $OB\,.\,OQ$, $OC\,.\,OR$ are
negative, because the lines $OA\,.\,OP$, \&c., are measured in opposite
directions, and have contrary signs; hence the \emph{polar circle} is
imaginary; but it is real when the point $O$ is without the triangle---that
is, when the triangle has an obtuse angle.

3.~If the perpendiculars of a triangle be produced to meet the
circumscribed circle, the intercepts between the orthocentre and
the circle are bisected by the sides of the triangle.


\imgflow{145}{16}{f159}

4.~The point of bisection $(I)$\label{bisection1} of the line $(OP)$ joining the orthocentre
$(O)$ to the circumference $(P)$
of any triangle is equally distant
from the feet of the perpendiculars,
from the middle points of
the sides, and from the middle
points of the distances of the vertices
from the orthocentre.

\textbf{Dem.}---Draw the perpendicular
$PH$; then, since $OF$, $PH$ are
perpendiculars on $AB$, and $OP$ is
bisected in $I$, it is easy to see that
$IH = IF$. Again, since $OP$, $OG$
are bisected in $I$, $F$; $IF = \frac{1}{2}PG$---that
is, $IF = \frac{1}{2}$ the radius. Hence
the distance of $I$ from the foot of each perpendicular, and from
the middle point of each side, is $= \frac{1}{2}$ the radius. In like manner,
if $OC$ be bisected in $K$, then $IK = \frac{1}{2}$ the radius. Hence we have

the following theorem:---\emph{The nine points made up of the feet of the
perpendiculars, the middle points of the sides, and the middle points
of the lines from the vertices to the orthocentre, are concyclic.}

\textbf{Def.}---\emph{The circle through these nine points is called the ``nine
points circle''\index{Circle!nine-points} of the triangle.}

5.~The circumcircle of a triangle is the ``nine points circle''
of each of the four triangles formed by joining the centres of the
inscribed and escribed circles.

6.~The distances between the vertices of a triangle and its
orthocentre are respectively the doubles of the perpendiculars
from the circumcentre on the sides.

7.~The radius of the ``nine points circle'' of a triangle is equal
to half its circumradius.
\par\end{footnotesize}

\myprop{PROP\@.~VI\@.---Problem.}{In a given circle $(ABCD)$ to inscribe a square.}


\imgflow{120}{10}{f160}

\textbf{Sol.}---Draw any two diameters $AC$, $BD$ at right
angles to each other. Join $AB$,
$BC$, $CD$, $DA$. \emph{$ABCD$ is a square.}

\textbf{Dem.}---Let $O$ be the centre.
Then the four angles at $O$, being
right angles, are equal. Hence
the arcs on which they stand are
equal [\textsc{III\@.~xxvi.}], and hence the
four chords are equal [\textsc{III\@.~xxix.}].
Therefore the figure $ABCD$ is
equilateral.

Again, because $AC$ is a diameter, the angle $ABC$
is right [\textsc{III\@.~xxxi.}]. In like manner the remaining
angles are right. \emph{Hence $ABCD$ is a square.}

\myprop{PROP\@.~VII\@.---Problem.}{About a given circle $(ABCD)$ to describe a square.}

\textbf{Sol.}---Through the centre $O$ draw any two diameters
at right angles to each other, and draw at the points
$A$, $B$, $C$, $D$ the lines $HE$, $EF$, $FG$, $GH$ touching the
circle. \emph{$EFGH$ is a square.}



\imgflow{120}{10}{f161}

\textbf{Dem.}---Because $AE$ touches the circle at $A$, the angle
$EAO$ is right [\textsc{III\@.~xviii.}], and
therefore equal to $BOC$, which
is right (const.). Hence $AE$ is
parallel to $OB$. In like manner
$EB$ is parallel to $AO$; and since
$AO$ is equal to $OB$, the figure
$AOBE$ is a lozenge, and the angle
$AOB$ is right; hence $AOBE$ is a
square. In like manner each of
the figures $BC$, $CD$, $DA$ is a
square. \emph{Hence the whole figure is a square.}

\emph{Cor.}---The circumscribed square is double of the inscribed
square.

\myprop{PROP\@.~VIII\@.---Problem.}{In a given square $(ABCD)$ to inscribe a circle.}

\textbf{Sol}.---Bisect (\emph{see} last diagram) two adjacent sides
$EH$, $EF$ in the points $A$, $B$, and through $A$, $B$ draw
the lines $AC$, $BD$, respectively parallel to $EF$, $EH$;
\emph{then $O$, the point of intersection of these parallels, is the
centre of the required circle.}

\textbf{Dem.}---Because $AOBE$ is a parallelogram, its opposite
sides are equal; therefore $AO$ is equal to $EB$; but
$EB$ is half the side of the given square; therefore $AO$
is equal to half the side of the given square; and so in
like manner is each of the lines $OB$, $OC$, $OD$; therefore
the four lines $OA$, $OB$, $OC$, $OD$ are all equal; and since
they are perpendicular to the sides of the given square,
\emph{the circle described with $O$ as centre, and $OA$ as radius,
will be inscribed in the square.}

\myprop{PROP\@.~IX\@.---Problem.}{About a given square\label{circlesquare} $(ABCD)$ to describe a circle.}

\textbf{Sol.}---Draw the diagonals $AC$, $BD$ intersecting in $O$
(\emph{see} diagram to Proposition \textsc{vi.}). \emph{$O$ is the centre of the
required circle.}

\textbf{Dem.}---Since $ABC$ is an isosceles triangle, and the

angle $B$ is right, each of the other angles is half a right
angle; therefore $BAO$ is half a right angle. In like
manner $ABO$ is half a right angle; hence the angle
$BAO$ equal $ABO$; therefore [\textsc{I.~vi.}] $AO$ is equal to $OB$.
In like manner $OB$ is equal to $OC$, and $OC$ to $OD$.
\emph{Hence the circle described, with $O$ as centre and $OA$ as
radius, will pass through the points $B$, $C$, $D$, and be
described about the square.}

\myprop{PROP\@.~X.---Problem.}{To construct an isosceles triangle having each base angle
double the vertical angle.}


\imgflow{130}{11}{f162}

\textbf{Sol.}---Take any line $AB$. Divide it in $C$, so that the
rectangle $AB\,.\,BC$ shall be equal
to $AC^2$ [\textsc{II\@.~xi.}]. With $A$ as centre,
and $AB$ as radius, describe
the circle $BDE$, and in it place
the chord $BD$ equal to $AC$ [\textsc{i.}].
Join $AD$. \emph{$ADB$ is a triangle fulfilling
the required conditions.}

\textbf{Dem.}---Join $CD$. About the
triangle $ACD$ describe the circle
$CDE$ [\textsc{v.}]. Then, because the
rectangle $AB\,.\,BC$ is equal to
$AC^2$ (const.), and that $AC$ is equal to $BD$ (const.);
therefore the rectangle $AB\,.\,BC$ is equal to $BD^2$. Hence
[\textsc{III\@.~xxxii.}] $BD$ touches the circle $ACD$. Hence the
angle $BDC$ is equal to the angle $A$ in the alternate segment
[\textsc{III\@.~xxxii.}]. To each add $CDA$, and we have the
angle $BDA$ equal to the sum of the angles $CDA$ and $A$;
but the exterior angle $BCD$ of the triangle $ACD$ is
equal to the sum of the angles $CDA$ and $A$. Hence the
angle $BDA$ is equal to $BCD$; but since $AB$ is equal to
$AD$, the angle $BDA$ is equal to $ABD$; therefore the
angle $CBD$ is equal to $BCD$. Hence [\textsc{I.~vi.}] $BD$ is
equal to $CD$; but $BD$ is equal to $AC$ (const.); therefore
$AC$ is equal to $CD$, and therefore [\textsc{I.~v.}] the angle
$CDA$ is equal to $A$; but $BDA$ has been proved to be

equal to the sum of $CDA$ and $A$. Hence $BDA$ is double
of $A$. \textit{Hence each of the base angles of the triangle $ABD$
is double of the vertical angle.}

\exhead{Exercises.}

\begin{footnotesize}
1.~Prove that $ACD$ is an isosceles triangle whose vertical angle
is equal to three times each of the base angles.

2.~Prove that $BD$ is the side of a regular decagon inscribed in
the circle $BDE$.

3.~If $DB$, $DE$, $EF$ be consecutive sides of a regular decagon
inscribed in a circle, prove $BF - BD =$ radius of circle.

4.~If $E$ be the second point of intersection of the circle $ACD$
with $BDE$, $DE$ is equal to $DB$; and if $AE$, $BE$, $CE$, $DE$ be
joined, each of the triangles $ACE$, $ADE$ is congruent with $ABD$.

5.~$AC$ is the side of a regular pentagon inscribed in the circle
$ACD$, and $EB$ the side of a regular pentagon inscribed in the
circle $BDE$.

6.~Since $ACE$ is an isosceles triangle, $EB^{2} - EA^{2} = AB\,.\,BC$---that
is = $BD^{2}$; therefore $EB^{2} - BD^{2} = EA^{2}$---that is, \textit{the square
of the side of a pentagon inscribed in a circle exceeds the square of
the side of the decagon inscribed in the same circle by the square of
the radius.}
\par\end{footnotesize}

\myprop{PROP\@.~XI\@.---Problem.}{To inscribe a regular pentagon in a given circle $(ABCDE)$.}


\imgflow{130}{11}{f163}

\textbf{Sol.}---Construct an isosceles triangle [\textsc{x.}], having
each base angle double the vertical
angle, and inscribe in the
given circle a triangle $ABD$
equiangular to it. Bisect the
angles $DAB$, $ABD$ by the lines
$AC$, $BE$. Join $EA$, $ED$, $DC$,
$CB$; \emph{then the figure $ABCDE$ is
a regular pentagon.}

\textbf{Dem.}---Because each of the
base angles $BAD$, $ABD$ is
double of the angle $ADB$, and the lines $AC$, $BE$ bisect
them, the five angles $BAC$, $CAD$, $ADB$, $DBE$, $EBA$ are
all equal; therefore the arcs on which they stand are
equal; and therefore the five chords, $AB$, $BC$, $CD$, $DE$,
$EA$ are equal. Hence the figure $ABCDE$ is equilateral.

Again, because the arcs $AB$, $DE$ are equal, adding

the arc $BCD$ to both, the arc $ABCD$ is equal to the arc
$BCDE$, and therefore [III\@.~\textsc{xxvii.}] the angles $AED$,
$BAE$, which stand on them, are equal. In the same
manner it can be proved that all the angles are equal;
therefore the figure $ABCDE$ is equiangular. \emph{Hence it
is a regular pentagon.}

\exhead{Exercises.}

\begin{footnotesize}
1.~The figure formed by the five diagonals of a regular pentagon
is another regular pentagon.

2.~If the alternate sides of a regular pentagon be produced to
meet, the five points of meeting form another regular pentagon.

3.~Every two consecutive diagonals of a regular pentagon
divide each other in extreme and mean ratio.

4.~Being given a side of a regular pentagon, construct it.

5.~Divide a right angle into five equal parts.
\par\end{footnotesize}

\myprop{PROP\@.~XII\@.---Problem.}{To describe a regular pentagon about a given circle $(ABCDE)$.}

\textbf{Sol.}---Let the five points $A$, $B$, $C$, $D$, $E$ on the circle
be the vertices of any inscribed regular pentagon: at
these points draw tangents $FG$, $GH$, $HI$, $IJ$, $JF$: \emph{the
figure $FGHIJ$ is a circumscribed regular pentagon.}


\imgflow{140}{12}{f164}

\textbf{Dem.}---Let $O$ be the centre of the circle. Join $OE$,
$OA$, $OB$. Now, because the
angles $A$, $E$ of the quadrilateral
$AOEF$ are right angles
[III\@.~\textsc{xviii.}], the sum of the
two remaining angles $AOE$,
$AFE$ is two right angles. In
like manner the sum of the
angles $AOB$, $AGB$ is two
right angles; therefore the
sum of $AOE$, $AFE$ is equal
to the sum of $AOB$, $AGB$;
but the angles $AOE$, $AOB$ are equal, because they
stand on equal arcs $AE$, $AB$ [III\@.~\textsc{xxvii.}]. Hence the
angle $AFE$ is equal to $AGB$. In like manner the
remaining angles of the figure $FGHIJ$ are equal.
\emph{Therefore it is equiangular.}

Again, join $OF$, $OG$. Now the triangles $EOF$, $AOF$

have the sides $AF$, $FE$ equal [III\@.~\textsc{xvii.}, Ex.~1], and $FO$
common, and the base $AO$ equal to the base $EO$. Hence
the angle $AFO$ is equal to $EFO$ [I.~\textsc{viii.}]. Therefore
the angle $AFO$ is half the angle $AFE$. In like manner
$AGO$ is half the angle $AGB$; but $AFE$ has been
proved equal to $AGB$; hence $AFO$ is equal to $AGO$,
and $FAO$ is equal to $GAO$, each being right, and $AO$
common to the two triangles $FAO$, $GAO$; hence
[I.~\textsc{xxvi.}] the side $AF$ is equal to $AG$; therefore $GF$ is
double $AF$. In like manner $JF$ is double $EF$; but $AF$
is equal to $EF$; hence $GF$ is equal to $JF$. In like
manner the remaining sides are equal; therefore the
figure $FGHIJ$ is equilateral, and it has been proved
equiangular. \emph{Hence it is a regular pentagon.}\par\smallskip

\begin{footnotesize}
This Proposition is a particular case of the following general
theorem, of which the proof is the same as the foregoing:---

``If tangents be drawn to a circle, at the angular points of an
inscribed polygon of any number of sides, they will form a regular
polygon of the same number of sides circumscribed to the circle.''
\par\end{footnotesize}

\myprop{PROP\@.~XIII\@.---Problem.}{To inscribe a circle in a regular pentagon $(ABCDE)$.}

\textbf{Sol.}---Bisect two adjacent angles $A$, $B$ by the lines
$AO$, $BO$; \textit{then $O$, the point of intersection of the bisectors,
is the centre of the required circle.}

\textbf{Dem.}---Join $CO$, and let fall perpendiculars from $O$
on the five sides of the pentagon.
Now the triangles
$ABO$, $CBO$ have the side $AB$
equal to $BC$ (hyp.), and $BO$
common, and the angle $ABO$
equal to $CBO$ (const.). Hence
the angle $BAO$ is equal to
$BCO$ [I.~\textsc{iv.}]; but $BAO$ is half
$BAE$ (const.). Therefore $BCO$
is half $BCD$, and therefore $CO$
bisects the angle $BCD$. In
like manner it may be proved that $DO$ bisects the
angle $D$, and $EO$ the angle $E$.




\imgflow{133}{11}{f165}

Again, the triangles $BOF$, $BOG$ have the angle $F$
equal to $G$, each being right; and $OBF$ equal to $OBG$,
because $OB$ bisects the angle $ABC$ (const.), and $OB$
common; hence [\textsc{I.~xxvi.}] $OF$ is equal to $OG$. In like
manner all the perpendiculars from $O$ on the sides of
the pentagon are equal; hence the circle whose centre
is $O$, and radius $OF$, will touch all the sides of the
pentagon, \emph{and will therefore be inscribed in it.}

In the same manner a circle may be inscribed in
any regular polygon.

\myprop{PROP\@.~XIV\@.---Problem.}{To describe a circle about a regular pentagon $(ABCDE)$.}

\textbf{Sol.}---Bisect two adjacent angles $A$, $B$ by the lines
$AO$, $BO$. \emph{Then $O$, the point of intersection of the bisectors,
is the centre of the required circle.}


\imgflow{110}{10}{f166}

\textbf{Dem.}---Join $OC$, $OD$, $OE$. Then the triangles\label{traingles} $ABO$,
$CBO$ have the side $AB$ equal to
$BC$ (hyp.), $BO$ common, and the
angle $ABO$ equal to $CBO$ (const.).
Hence the angle $BAO$ is equal to
$BCO$ [\textsc{I.~iv.}]; but the angle $BAE$
is equal to $BCD$ (hyp.); and since
$BAO$ is half $BAE$ (const.), $BCO$
is half $BCD$. Hence $CO$ bisects
the angle $BCD$. In like manner
it may be proved that $DO$ bisects
$CDE$, and $EO$ the angle $DEA$. Again, because the
angle $EAB$ is equal to $ABC$, their halves are equal.
Hence $OAB$ is equal to $OBA$; therefore [\textsc{I.~vi.}] $OA$ is
equal to $OB$. In like manner the lines $OC$, $OD$, $OE$
are equal to one another and to $OA$. Therefore the
circle described with $O$ as centre, and $OA$ as radius,
will pass through the points $B$, $C$, $D$, $E$, \emph{and be described
about the pentagon.}

In the same manner a circle may be described about
any regular polygon.\par\smallskip



\begin{footnotesize}
Propositions \textsc{xiii.}, \textsc{xiv.} are particular cases of the following
theorem:---

``A regular polygon of any number of sides has one circle inscribed
in it, and another described about it, and both circles are
concentric.''
\end{footnotesize}

\myprop{PROP\@.~XV\@.---Problem.}{In a given circle $(ABCDEF)$ to inscribe a regular hexagon.}

\textbf{Sol.}---Take any point $A$ in the circumference, and
join it to $O$, the centre
of the given circle; then
with $A$ as centre, and $AO$
as radius, describe the
circle $OBF$, intersecting
the given circle in the
points $B$, $F$. Join $OB$,
$OF$, and produce $AO$, $BO$,
$FO$ to meet the given
circle again in the points $D$, $E$, $C$. Join $AB$, $BC$, $CD$,
$DE$, $EF$, $FA$; \emph{$ABCDEF$ is the required hexagon.}


\imgflow{156}{8}{f167}

\textbf{Dem.}---Each of the triangles $AOB$, $AOF$ is equilateral
(\emph{see} Dem., \textsc{I.~i.}). Hence the angles $AOB$, $AOF$
are each one-third of two right angles; therefore $EOF$
is one-third of two right angles. Again, the angles
$BOC$, $COD$, $DOE$ are [\textsc{I.~xv.}] respectively equal to the
angles $EOF$, $FOA$, $AOB$. Therefore the six angles at
the centre are equal, because each is one-third of two
right angles. Therefore the six chords are equal [\textsc{III\@.~xxix.}].
\emph{Hence the hexagon is equilateral.}

Again, since the arc $AF$ is equal to $ED$, to each add
the arc $ABCD$; then the whole arc $FABCD$ is equal to
$ABCDE$; therefore the angles $DEF$, $EFA$ which stand
on these arcs are equal [\textsc{III\@.~xxvii.}]. In the same
manner it may be shown that the other angles of the
hexagon are equal. Hence it is equiangular, \emph{and is
therefore a regular hexagon inscribed in the circle.}



\emph{Cor}.~1.---The side of a regular hexagon inscribed in
a circle is equal to the radius.

\emph{Cor}.~2.---If three alternate angles of a hexagon be
joined, they form an inscribed equilateral triangle.

\exhead{Exercises.}

\begin{footnotesize}
1.~The area of a regular hexagon inscribed in a circle is equal to
twice the area of an equilateral triangle inscribed in the circle;
and the square of the side of the triangle is three times the square
of the side of the hexagon.

2.~If the diameter of a circle be produced to $C$ until the produced
part is equal to the radius, the two tangents from $C$ and
their chord of contact form an equilateral triangle.

3.~The area of a regular hexagon inscribed in a circle is half
the area of an equilateral triangle, and three-fourths of the area
of a regular hexagon circumscribed to the circle.
\par\end{footnotesize}

\myprop{PROP\@.~XVI\@.---Problem.}{To inscribe a regular polygon of fifteen sides in a given circle.}

\textbf{Sol.}---Inscribe a regular pentagon $ABCDE$ in the
circle [\textsc{xi.}], and also an equilateral triangle $AGH$ [\textsc{ii.}].
Join $CG$. \emph{$CG$ is a side of the required polygon.}


\imgflow{110}{9}{f168}

\textbf{Dem.}---Since $ABCDE$ is a regular pentagon, the arc
$ABC$ is $\frac{2}{5}$ths of the circumference;
and since $AGH$ is an equilateral
triangle, the arc $ABG$ is $\frac{1}{3}$rd of
the circumference. Hence the arc
$GC$, which is the difference between
these two arcs, is equal to
$\frac{2}{5}\text{ths} - \frac{1}{3}$rd, or $\frac{1}{15}$th of the entire
circumference; \emph{and therefore, if
chords equal to $GC$} [\textsc{i.}] \emph{be placed
round the circle, we shall have a
regular polygon of fifteen sides, or quindecagon, inscribed
in it.}\par\smallskip



\begin{footnotesize}
\textit{Scholium}.---Until the year 1801 no regular polygon could be
described by constructions employing the line and circle only,
except those discussed in this Book, and those obtained from
them by the continued bisection of the arcs of which their sides
are the chords; but in that year the celebrated Gauss proved that
if $2^n+1$ be a prime number, regular polygons of $2^n+1$ sides are
inscriptable by elementary geometry. For the case $n=4$, which
is the only figure of this class except the pentagon for which
a construction has been given, see Note at the end of this work.
\par\end{footnotesize}

\addcontentsline{toc}{section}{Questions for Examination,}
\exhead{Questions for Examination on Book IV\@.}

\begin{footnotesize}
1.~What is the subject-matter of Book IV\@.?

2.~When is one rectilineal figure said to be inscribed in
another?

3.~When circumscribed?

4.~When is a circle said to be inscribed in a rectilineal figure?

5.~When circumscribed about it?

6.~What is meant by reciprocal propositions? \textit{Ans}. In reciprocal
propositions, to every line in one there corresponds a point
in the other; and, conversely, to every point in one there corresponds
a line in the other.

7.~Give instances of reciprocal propositions in Book IV\@.

8.~What is a regular polygon?

9.~What figures can be inscribed in, and circumscribed about,
a circle by means of Book~IV\@.?

10.~What regular polygons has Gauss proved to be inscriptable
by the line and circle?

11.~What is meant by escribed circles?

12.~How many circles can be described to touch three lines
forming a triangle?

13.~What is the centroid of a triangle?

14.~What is the orthocentre?

15.~What is the circumcentre?

16.~What is the polar circle?

17.~When is the polar circle imaginary?

18.~What is the ``nine-points circle''?

19.~Why is it so called?

20.~Name the special nine points through which it passes.


21.~What three regular figures can be used in filling up the
space round a point? \textit{Ans}. Equilateral triangles, squares, and
hexagons.

22.~If the sides of a triangle be 13, 14, 15, what are the values
of the radii of its inscribed and escribed circles?

23.~What is the radius of the circumscribed circle?

24.~What is the radius of its nine-points circle?

25.~What is the distance between the centres of its inscribed
and circumscribed circles?

26.~If $r$ be the radius of a circle, what is the area of its
inscribed equilateral triangle?---of its inscribed square?---its
inscribed pentagon?---its inscribed hexagon?---its inscribed octagon?---its
inscribed decagon?

27.~With the same hypothesis, find the sides of the same regular
figures.
\par\end{footnotesize}

\addcontentsline{toc}{section}{Exercises,}
\exhead{Exercises on Book IV\@.}

\begin{footnotesize}
1.~If a circumscribed polygon be regular, the corresponding
inscribed polygon is also regular, and conversely.

2.~If a circumscribed triangle be isosceles, the corresponding
inscribed triangle is isosceles, and conversely.

3.~If the two isosceles triangles in Ex.~2 have equal vertical
angles, they are both equilateral.

4.~Divide an angle of an equilateral triangle into five equal
parts.

5.~Inscribe a circle in a sector of a given circle.

6.~The line $DE$ is parallel to the base $BC$ of the triangle $ABC$:
prove that the circles described about the triangles $ABC$, $ADE$
touch at $A$.

7.~The diagonals of a cyclic quadrilateral intersect in $E$: prove
that the tangent at $E$ to the circle about the triangle $ABE$ is
parallel to $CD$.

8.~Inscribe a regular octagon in a given square.

9.~A line of given length slides between two given lines: find
the locus of the intersection of perpendiculars from its extremities
to the given lines.

10.~If the perpendicular to any side of a triangle at its middle
point meet the internal and external bisectors of the opposite
angle in the points $D$ and $E$; prove that $D$, $E$ are points on the
circumscribed circle.


11.~Through a given point $P$ draw a chord of a circle so that
the intercept $EF$ may subtend a given angle $X$.

12.~In a given circle inscribe a triangle having two sides passing
through two given points, and the third parallel to a given
line.

13.~Given four points, no three of which are collinear; describe
a circle which shall be equidistant from them.

14.~In a given circle inscribe a triangle whose three sides shall
pass through three given points.

15.~Construct a triangle, being given---
\begin{changemargin}{0.3in}{0in}
\begin{enumerate}
\item[1.] The radius of the inscribed circle, the vertical angle,
and the perpendicular from the vertical angle on the
base.

\item[2.] The base, the sum or difference of the other sides, and
the radius of the inscribed circle, or of one of the
escribed circles.

\item[3.] The centres of the escribed circles.
\end{enumerate}
\end{changemargin}

16.~If $F$ be the middle point of the base of a triangle, $DE$ the
diameter of the circumscribed circle which passes through $F$, and
$L$ the point where a parallel to the base through the vertex meets
$DE$: prove $DL\,.\,FE$ is equal to the square of half the sum, and
$DF\,.\,LE$ equal to the square of half the difference of the two remaining
sides.

17.~If from any point within a regular polygon of $n$ sides perpendiculars
be let fall on the sides, their sum is equal to $n$ times
the radius of the inscribed circle.

18.~The sum of the perpendiculars let fall from the angular
points of a regular polygon of $n$ sides on any line is equal to
$n$ times the perpendicular from the centre of the polygon on the
same line.

19.~If $R$ denotes the radius of the circle circumscribed about a
triangle $ABC$, $r$, $r'$, $r''$, $r'''$ the radii of its inscribed and escribed
circles, $\delta$, $\delta'$, $\delta''$ the perpendiculars from its circumcentre on the
sides; $\mu$, $\mu'$, $\mu''$ the segments of these perpendiculars between the
sides and circumference of the circumscribed circle, we have the
relations---
\begin{alignat}{3}
 r' &\,+\, &r'' &\,+\, &r''' &= 4R + r,  \\
 \mu &\,+\, &\mu' &\,+\, &\mu'' &= 2R - r,  \\
 \delta &\,+\, &\delta' &\,+\, &\delta'' &= \phantom{0}R + r.
\end{alignat}
The relation (3) supposes that the circumcentre is inside the
triangle.


20.~Through a point $D$, taken on the side $BC$ of a triangle
$ABC$, is drawn a transversal $EDF$, and circles described about
the triangles $DBF$, $ECD$. The locus of their second point of
intersection is a circle.

21.~In every quadrilateral circumscribed about a circle, the
middle points of its diagonals and the centre of the circle are
collinear.

22.~Find on a given line a point $P$, the sum or difference of
whose distances from two given points may be given.

23.~Find a point such that, if perpendiculars be let fall from it
on four given lines, their feet may be collinear.

24.~The line joining the orthocentre of a triangle to any point
$P$, in the circumference of its circumscribed circle, is bisected by
the line of collinearity of perpendiculars from $P$ on the sides of
the triangle.

25.~The orthocentres of the four triangles formed by any four
lines are collinear.

26.~If a semicircle and its diameter be touched by any circle,
either internally or externally, twice the rectangle contained by
the radius of the semicircle, and the radius of the tangential circle,
is equal to the rectangle contained by the segments of any secant
to the semicircle, through the point of contact of the diameter and
touching circle.

27.~If $\rho$, $\rho'$ be the radii of two circles, touching each other at
the centre of the inscribed circle of a triangle, and each touching
the circumscribed circle, prove
\[
\frac{ 1 }{\rho} + \frac{1 }{\rho'}  = \frac{2 }{r} ,
\]
and state and prove corresponding theorems for the escribed
circles.

28.~If from any point in the circumference of the circle, circumscribed
about a regular polygon of $n$ sides, lines be drawn to
its angular points, the sum of their squares is equal to $2n$ times
the square of the radius.

29.~In the same case, if the lines be drawn from any point in
the circumference of the inscribed circle, prove that the sum of
their squares is equal to $n$ times the sum of the squares of the
radii of the inscribed and the circumscribed circles.

30.~State the corresponding theorem for the sum of the squares
of the lines drawn from any point in the circumference of any
concentric circle.


31.~If from any point in the circumference of any concentric
circle perpendiculars be let fall on all the sides of any regular
polygon, the sum of their squares is constant.

32.~For the inscribed circle, the constant is equal to $\dfrac{3n}{ 2}$ times
the square of the radius.

33.~For the circumscribed circle, the constant is equal to $n$
times the square of the radius of the inscribed circle, together with
$\frac{ 1}{ 2} n$ times the square of the radius of the circumscribed circle.

34.~If the circumference of a circle whose radius is $R$ be divided
into seventeen equal parts, and $AO$ be the diameter drawn from
one of the points of division $(A)$, and if $\rho_{1}$, $\rho_{2} \ldots \ldots \rho_{8}$ denote
the chords from $O$ to the points of division, $A_{1}$, $A_{2} \ldots \ldots A_{8}$
on one side of $AO$, then
\[
  \rho_1 \rho_2 \rho_4 \rho_8 = R^4; \text{\ and }
  \rho_3 \rho_5 \rho_6 \rho_7 = R^4.\text{---\textsc{Catalan.}\index{Catalan}}
\]

\textbf{Dem.}---Let the supplemental chords corresponding to $\rho_{1}$, $\rho_{2}$,
\&c., be denoted by $r_{1}$, $r_{2}$, \&c.; then [III\@.~\textsc{xxxv.}\ Ex.~2], we
have
\begin{flalign*}
&&  \rho_1 r_1 &= R r_2,  &&\qquad\phantom{Hence }\\
&&  \rho_2 r_2 &= R r_4,  &&\\
&&  \rho_4 r_4 &= R r_8,  &&\\
&&  \rho_8 r_8 &= R r_1,  &&\\
&\text{Hence }&
  \rho_1 \rho_2 \rho_4 \rho_8 &= R^4.  &&
\end{flalign*}
And it may be proved in the same manner that
\begin{flalign*}
&&  \rho_1 \rho_2 \rho_3 \rho_4 \rho_5 \rho_6 \rho_7 \rho_8 &= R^8. &&\phantom{Therefore }\\
&\text{Therefore }&
  \rho_3 \rho_5 \rho_6 \rho_7 &= R^4.  &&
\end{flalign*}

35.~If from the middle point of the line joining any two of
four concyclic points a perpendicular be let fall on the line joining
the remaining two, the six perpendiculars thus obtained are
concurrent.

36.~The greater the number of sides of a regular polygon circumscribed
about a given circle, the less will be its perimeter.

37.~The area of any regular polygon of more than four sides
circumscribed about a circle is less than the square of the diameter.

38.~Four concyclic points taken three by three determine four
triangles, the centres of whose nine-points circles are concyclic.


39.~If two sides of a triangle be given in position, and if their
included angle be equal to an angle of an equilateral triangle, the
locus of the centre of its nine-points circle is a right line.

40.~If, in the hypothesis and notation of Ex.~34, $\alpha$, $\beta$ denote
any two suffixes whose sum is less than 8, and of which $\alpha$ is the
greater,
\[
\rho_\alpha \rho_\beta = R(\rho_{\alpha - \beta} + \rho_{\alpha + \beta}).
\]

For instance, \quad $\rho_1\rho_4 = R(\rho_3 + \rho_5)$ [\textsc{III\@.~xxxv.}, Ex.~7].

In the same case, if the suffixes be greater than 8,
\[
\rho_\alpha \,.\, \rho_\beta = R(\rho_{\alpha - \beta} - \rho_{17 - \alpha - \beta}).
\]

For instance, \quad $\rho_8 \rho_2 = R(\rho_6 - \rho_7)$ [\textsc{III\@.~xxxv.}, Ex.~6].

41.~Two lines are given in position: draw a transversal through
a given point, forming with the given lines a triangle of given
perimeter.

42.~Given the vertical angle and perimeter of a triangle, construct
it with either of the following data: 1.~The bisector of
the vertical angle; 2.~the perpendicular from the vertical angle
on the base; 3.~the radius of the inscribed circle.

43.~In a given circle inscribe a triangle so that two sides may
pass through two given points, and that the third side may be a
maximum or a minimum.

44.~If $s$ be the semiperimeter of a triangle, $r'$, $r''$, $r'''$, the
radii of its escribed circles,
\[
r'r'' + r''r''' + r'''r' = s^2.
\]

45.~The feet of the perpendiculars from the extremities of the
base on either bisector of the vertical angle, the middle point of
the base, and the foot of the perpendicular from the vertical
angle on the base, are concyclic.

46.~Given the base of a triangle and the vertical angle; find
the locus of the centre of the circle passing through the centres
of the escribed circles.

47.~The perpendiculars from the centres of the escribed circles
of a triangle on the corresponding sides are concurrent.

48.~If $AB$ be the diameter of a circle, and $PQ$ any chord cutting
$AB$ in $O$, and if the lines $AP$, $AQ$ intersect the perpendicular
to $AB$ at $O$, in $D$ and $E$ respectively, the points $A$, $B$, $D$, $E$ are
concyclic.


49.~If the sides of a triangle be in arithmetical progression,
and if $R$, $r$ be the radii of the circumscribed and inscribed circles;
then $6Rr$ is equal to the rectangle contained by the greatest and
least sides.

50.~Inscribe in a given circle a triangle having its three sides
parallel to three given lines.

51.~If the sides $AB$, $BC$, \&c., of a regular pentagon be bisected
in the points $A'$, $B'$, $C'$, $D'$, $E'$, and if the two pairs of alternate
sides, $BC$, $AE$; $AB$, $DE$, meet in the points $A''$, $E''$, respectively,
prove
\[
\triangle\, A''AE'' - \triangle\, A'AE' = \text{pentagon}\ A'B'C'D'E'.
\]

52.~In a circle, prove that an equilateral inscribed polygon is
regular, and also an\label{enforan} equilateral circumscribed polygon, if the

number of sides be odd.

53.~Prove also that an equiangular circumscribed polygon is
regular, and an equiangular inscribed polygon, if the number of
sides be odd.

54.~The sum of the perpendiculars drawn to the sides of an
equiangular polygon from any point inside the figure is constant.

55.~Express the sides of a triangle in terms of the radii of its
escribed circles.
\par\end{footnotesize}




\mybook{BOOK V}{Theory of Proportion}

\addcontentsline{toc}{section}{Introduction,}
\textbf{Introduction.}---Every proposition in the theories of
ratio and proportion is true for all descriptions of magnitude.
Hence it follows that the proper treatment is
the Algebraic. It is, at all events, the easiest and the
most satisfactory. Euclid's proofs of the propositions,
in the \textit{Theory of Proportion}, possess at present none
but a historical interest, as no student reads them now.
But although his demonstrations are abandoned, his
propositions are quoted by every writer, and his nomenclature
is universally adopted. For these reasons
it appears to us that the best method is to state Euclid's
definitions, explain them, or prove them when necessary,
for some are theorems under the guise of definitions,
and then supply simple algebraic proofs of his
propositions.

\textsc{i.} A less magnitude is said to be a \emph{part} or \emph{submultiple}\index{Submultiple}
of a greater magnitude, when the less measures
the greater---that is, when the less is contained a certain
number of times exactly in the greater.

\textsc{ii.} A greater magnitude is said to be a \emph{multiple}\index{Multiples} of a
less when the greater is measured by the less---that is,
when the greater contains the less a certain number of
times exactly.



\textsc{iii.} Ratio\index{Ratio} is the mutual relation of two magnitudes
of the same kind with respect to quantity.

\textsc{iv.} Magnitudes are said to have a \emph{ratio} to one
another when the less can be multiplied so as to
exceed the greater.

These definitions require explanation, especially
Def.~\textsc{iii.}, which has the fault of conveying no precise
meaning---being, in fact, unintelligible.

The following annotations will make them explicit:---\par\smallskip


\imgflow{153}{2}{f256}

\begin{footnotesize}
1.~If an integer be divided into any number of equal parts,
one, or the sum of any number of these parts, is called a \emph{fraction}.
Thus, if the line $AB$ represent
the integer, and if it be divided
into four equal parts in the
points $C$, $D$, $E$, then $AC$ is $\frac{1}{4}$;
$AD$, $\frac{2}{4}$; $AE$, $\frac{3}{4}$. Thus, a fraction is denoted by two numbers
parted by a horizontal line; the lower, called the \emph{denominator},
denotes the number of equal parts into which the integer is divided;
and the upper, called the \emph{numerator}, denotes the number
of these equal parts which are taken. Hence it follows, that if
the numerator be less than the denominator, the fraction is less
than unity. If the numerator be equal to the denominator, the
fraction is equal to unity; and if greater than the denominator, it
is greater than unity. It is evident that a fraction is an abstract
quantity---that is, that its value is independent of the nature of
the integer which is divided.

2.~If we divide each of the equal parts $AC$, $CD$, $DE$, $EB$ into
two equal parts, the whole, $AB$, will be divided into eight equal
parts; and we see that $AC = \frac{2}{8}$; $AD = \frac{4}{8}$; $AE = \frac{6}{8}$; $AB = \frac{8}{8}$;
Now, we saw in 1, that $AE = \frac{3}{4}$ of the integer, and we have just
shown that it is equal to $\frac{6}{8}$. Hence $\frac{3}{4}=\frac{6}{8}$; but $\frac{6}{8}$ would be got
from $\frac{3}{4}$ by multiplying its terms (numerator and denominator)
by $2$. Hence we infer generally that multiplying the terms of
any fraction by $2$ does not alter its value. In like manner it
may be shown that multiplying the terms of a fraction by any
whole number does not alter its value. Hence it follows conversely,
that dividing the terms of a fraction by a whole number
does not alter the value. Hence we have the following important
and fundamental theorem:---\textit{Two transformations can be made on
any fraction without changing its value; namely, its terms can be
either multiplied or divided by any whole number, and in either
case the value of the new fraction is equal to the value of the
original one.}



3.~If we take any number, such as $3$, and multiply it by any
whole number, the product is called a \emph{multiple} of $3$. Thus $6$, $9$,
$12$, $15$, \&c., are multiples of $3$; but $10$, $13$, $17$, \&c., are not, because
the multiplication of $3$ by any whole number will not produce
them. Conversely, $3$ is a \emph{submultiple}, or \emph{measure}, or \emph{part}
of $6$, $9$, $12$, $15$, \&c., because it is contained in each of these without
a remainder; but not of $10$, $13$, $17$, \&c., because in each case
it leaves a remainder.


\imgflow{150}{6}{f257}

4.~If we consider two magnitudes of the same kind, such as
two lines $AB$, $CD$, and if we suppose that $AB$ is equal to $\frac{3}{4}$ of
$CD$, it is evident, if $AB$ be divided into $3$ equal parts, and $CD$ into
$4$ equal parts, that one of the
parts into which $AB$ is divided
is equal to one of the parts into
which $CD$ is divided. And as
there are $3$ parts in $AB$, and $4$
in $CD$, we express this relation
by saying that $AB$ has to
$CD$ the ratio of $3$ to $4$; and we denote it thus, $3:4$. Hence the
ratio $3:4$ expresses the same idea as the fraction $\frac{3}{4}$. In fact,
both are different ways of expressing and writing the same thing.
When written $3:4$ it is called a ratio\index{Ratio}, and when $\frac{3}{4}$ a fraction.
\emph{In the same manner it can be shown that every ratio whose terms
are commensurable can be converted into a fraction; and, conversely,
every fraction can be turned into a ratio.}\par\smallskip

From this explanation we see that the ratio of any two commensurable
magnitudes is the same as the ratio of the numerical
quantities which denote these magnitudes. Thus, the ratio of
two commensurable lines is the ratio of the numbers which express
their lengths, measured with the same unit. And this may
be extended to the case where the lines are incommensurable\index{Lines!Incommensurable}.
Thus, if $a$ be the side and $b$ the diagonal of a square, the ratio
of $a:b$ is
\[
\frac{a}{b}, \text{\ or } \frac{1}{\sqrt{2}}.
\]

When two quantities are incommensurable, such as the diagonal
and the side of a square, although their ratio is not equal to
that of any two commensurable numbers, yet a series of pairs of
fractions can be found whose difference is continually diminishing,
and which ultimately becomes indefinitely small; such that
the ratio of the incommensurable quantities is greater than one,
and less than the other fraction of each pair. These fractions
are called convergents. By their means we can approximate as
nearly as we please to the exact value of the ratio. In the case

of the diagonal and the side of a square, the following are the
pairs of convergents:---
\[
  \frac{14}{10},    \ \frac{15}{10};    \ \
  \frac{141}{100},  \ \frac{142}{100};  \ \
  \frac{1414}{1000},\ \frac{1415}{1000};\ \ \&c.,
\]
and the ratio is intermediate to each pair. It is evident we may
continue the series as far as we please. Now if we denote the
first of any of the foregoing pairs of fractions by $\dfrac{m}{n}$, the second
will be $\dfrac{m+1}{n}$; and in general, in the case of two incommensurable
quantities, two fractions $\dfrac{m}{n}$ and $\dfrac{m+1}{n}$ can always be
found, where $n$ can be made as large as we please, one of which
is less and the other greater than the true value of the ratio.
For let $a$ and $b$ be the incommensurable quantities; then, evidently,
we cannot find two multiples $na$, $mb$, such that $na = mb$.
In this case, take any multiple of $a$, such as $na$, then this quantity
must lie between some two consecutive multiples of $b$, such as $mb$,
and $(m + 1)b$; therefore $\dfrac{na}{mb}$ is greater than unity, and $\dfrac{na}{(m+1)b}$
less than unity. Hence $\dfrac{a}{b}$ lies between $\dfrac{m}{n}$ and $\dfrac{m+1}{n}$. Now,
since the difference between $\dfrac{m}{n}$ and $\dfrac{m+1}{n}$ namely, $\dfrac{1}{n}$ becomes
small as $n$ increases, we see that the difference between the ratio
of two incommensurable quantities and that of two commensurable
numbers $m$ and $n$ can be made as small as we please.
Hence, ultimately, the ratio of incommensurable quantities may
be regarded as the limit of the ratio of commensurable quantities.

5.~The two terms of a ratio are called the \emph{antecedent}\index{Antecedent of a ratio}\index{Ratio!antecedent of a} and the
\emph{consequent}\index{Ratio!consequent of a}. These correspond to the numerator and the denominator
of a fraction. Hence we have the following definition:---``\emph{A
ratio is the fraction got by making the antecedent the numerator
and the consequent the denominator.}''

6.~The \emph{reciprocal}\index{Ratio!reciprocal of} of a ratio is the ratio obtained by interchanging
the antecedent and the consequent. Thus, $4 : 3$ is the reciprocal
of the ratio $3 : 4$. Hence we have the following theorem:---``\emph{The
product of a ratio and its reciprocal is unity.}''

7.~If we multiply any two numbers, as $5$ and $7$, by any number
such as $4$, the products $20$, $28$ are called \emph{equimultiples} of $5$ and $7$.
In like manner, $10$ and $15$ are equimultiples of $2$ and $3$, and $18$
and $30$ of $3$ and $5$, \&c.
\par\end{footnotesize}


\textsc{v.} The first of four magnitudes has to the second
the same ratio which the third has to the fourth, when
any equimultiples whatsoever of the first and third
being taken, and any equimultiples whatsoever of the
second and fourth, if, according as the multiple of the
first is greater than, equal to, or less than the multiple
of the second, the multiple of the third is greater than,
equal to, or less than the multiple of the fourth.

\textsc{vi.} Magnitudes which have the same ratio are called
proportionals. When four magnitudes are proportionals,
it is usually expressed by saying, ``The first is to
the second as the third is to the fourth.''

\textsc{viii.} Analogy or proportion is the similitude of
ratios.\par\smallskip

\begin{footnotesize}
We have given the foregoing definitions in the order of Euclid,
as given by Simson, Lardner, and others;\footnote{Except that \textsc{viii.}\ is put before \textsc{vii.}, because it relates, as \textsc{v}.\ and
\textsc{vi.}, to the equality of ratios, whereas \textsc{vii.}\ is a test of their
inequality.} but it is evidently an
inverted order; for \textsc{vi.}\ \textsc{viii.}\ are definitions of proportion, and \textsc{v}.\ is
only a test of proportion, and is not a definition but a theorem,
and one which, instead of being taken for granted, requires
proof. The following explanations will give the student clear
conceptions of their meaning:---

1.~If we take two ratios, such as $6 : 9$ and $10 : 15$, which are
each equal to the same thing (in this example each is equal to $\frac{2}{3}$),
they are equal to one another (I.\ Axiom~\textsc{i.}). Then we may
write it thus---
\[
6 : 9 = 10 : 15.
\]
This would be the most intelligible way, but it is not the usual
one, which is as follows:---$6 : 9 :: 10 : 15$. In this form it is
called a \emph{proportion}\index{Proportion}. Hence a proportion consists of two ratios
which are asserted by it to be equal. Its four terms consist of
two antecedents and two consequents. The 1st and 3rd terms
are the \emph{antecedents}, and the 2nd and 4th the \emph{consequents}. Also
the first and last terms are called the \emph{extremes}\index{Proportion!extremes of}, and the two
middle terms the \emph{means}\index{Proportion!means of}.

2.~Since a proportion consists of two equal ratios, and each

ratio can be written as a fraction, whenever we have a proportion
such as
\[
a : b :: c : d,
\]
we can write it in the form of two equal fractions. Thus:
\[
\frac{a}{b} = \frac{c}{d}.
\]
Conversely, an equation between two fractions can be put into a
proportion. By means of these simple principles all the various
properties of proportion can be proved in the most direct and easy
manner.

3.~If we take the proportion $a : b :: c : d$, and multiply the
first and third terms, each by $m$, and second and fourth, each by
$n$, we get the four multiples, $ma$, $nb$, $mc$, $nd$; and we want to
prove that if $ma$ is greater than $nb$, $mc$ is greater than $nd$; if
equal, equal; and if less, less.
\begin{flalign*}
&\text{\indent\textbf{Dem.}---Since }&
  a : b &:: c : d,
&\phantom{\indent\textbf{Dem.}---Since }&\\
&\text{we have }&
  \frac{a}{b} &= \frac{c}{d}.  &&\\
\intertext{Hence, multiplying each by $\dfrac{m}{n}$ we get}
&&  \frac{ma}{nb} &= \frac{mc}{nd}.  &&
\end{flalign*}
Now, it is evident that if $\dfrac{ma}{nb}$ is greater than unity, $\dfrac{mc}{nd}$ is greater
than unity; but if $\dfrac{ma}{nb}$ is greater than unity, $ma$ is greater than
$nb$; and if $\dfrac{mc}{nd}$ is greater than unity, $mc$ is greater than $nd$. In
like manner, if $ma$ be equal to $nb$, $mc$ is equal to $nd$; and if less,
less.

The foregoing is an easy proof of the converse of the theorem
which is contained in Euclid's celebrated Fifth Definition.

Next, to prove Euclid's theorem---that if, according as the
multiple of the first of four magnitudes is greater than, equal to,
or less than the multiple of the second, the multiple of the third
is greater than, equal to, or less than the multiple of the fourth;
the ratio of the first to the second is equal to the ratio of the
third to the fourth.


\textbf{Dem.}---Let, $a$, $b$, $c$, $d$ be the four magnitudes. First suppose
that $a$ and $b$ are commensurable, then it is evident that we can
take multiples $na$, $mb$, such that $na = mb$. Hence, by hypothesis,
$nc = md$. Thus,
\[
\frac{na}{mb} = 1,\ \frac{nc}{md} = 1;
\]
\begin{flalign*}
&\text{therefore }&&
  \frac{a}{b} = \frac{c}{d}.
&\phantom{therefore }&
\end{flalign*}
Next, suppose $a$ and $b$ are incommensurable. Then, as in a recent
note, we can find two numbers $m$ and $n$, such that $\dfrac{na}{mb}$ is greater
than unity, but $\dfrac{na}{(m+1)b}$ less than unity. Hence $\dfrac{a}{b}$ lies between $\dfrac{m}{n}$
and $\dfrac{m+1}{n}$. Now, since by hypothesis, when $\dfrac{na}{mb}$ is greater than
unity, $\dfrac{nc}{md}$ is greater than unity; and when $\dfrac{na}{(m+1)b}$ is less than
unity, $\dfrac{nc}{(m+1)d}$ is less than unity. Hence, since $\dfrac{a}{b}$ lies between
$\dfrac{m}{n}$ and $\dfrac{m+1}{n}$, $\dfrac{c}{d}$ lies between the same quantities. Therefore
the difference between $\dfrac{a}{b}$ and $\dfrac{c}{d}$ is less than $\dfrac{1}{n}$; and since $n$ may
be as large as we please, the difference is nothing;
therefore
\[
\frac{a}{b} = \frac{c}{d}.
\]
\par\end{footnotesize}

\textsc{vii}. When of the multiples of four magnitudes
(taken as in Def.~\textsc{v.}) the multiple of the first is greater
than that of the second, but the multiple of the third
not greater than that of the fourth, the first has to
the second a greater ratio than the third has to the
fourth.\par\smallskip

\begin{footnotesize}
This, instead of being a definition, is a theorem. We have
altered the last clause from that given in Simson's Euclid, which
runs thus:---``The first is \emph{said} to have to the second a greater
ratio than the third has to the fourth.'' This is misleading, as it
implies that it is, by convention, that the first ratio is greater
than the second, whereas, in fact, such is not the case; for it
follows from the hypothesis that the first ratio is greater than
the second; and if it did not, it could not be made so by definition.

We have made a similar change in the enunciation of the
Fifth Definition.

Let $a$, $b$, $c$, $d$ be the four magnitudes, and $m$ and $n$ the multiples
taken, it is required to prove, that if $ma$ be greater than $nb$,
but $mc$ not greater than $nd$, that the ratio $a : b$ is greater than the
ratio $c : d$.

\medskip
\textbf{Dem.}---Since $ma$ is greater than $nb$, but $mc$ not greater than
$nd$, it is evident that
\begin{flalign*}
&& \frac{ma}{nb}&\ \text{is greater than}\ \frac{mc}{nd}; &&\\
&\text{therefore }&
   \frac{a}{b}&\ \text{is greater than}\ \frac{c}{d}; &\phantom{therefore }&
\end{flalign*}
that is, the ratio $a : b$ is greater than the ratio $c : d$.
\par\end{footnotesize}

\textsc{ix}. Proportion consists of three terms at least.

\smallskip
\begin{footnotesize}
This has the same fault as some of the others---it is not a
definition, but an inference. It occurs when the means in a
proportion are equal, so that, in fact, there are four terms. As
an illustration, let us take the numbers $4$, $6$, $9$. Here the ratio of
$4 : 6$ is $\frac{2}{3}$, and the ratio of $6 : 9$ is $\frac{2}{3}$, so that $4$, $6$, $9$ are continued
proportionals; but, in reality, there are four terms, for the full
proportion is $4 : 6 :: 6 : 9$.
\par\end{footnotesize}

\textsc{x}. When three magnitudes are continual proportionals,
the first is said to have to the third the
\emph{duplicate}\index{Ratio!duplicate} ratio of that which it has to the second.

\textsc{xi}. When four magnitudes are continual proportionals,
the first is said to have to the fourth the
\emph{triplicate} ratio of that which it has to the second.

\textsc{xii}. When there is any number of magnitudes of the
same kind greater than two, the first is said to have to
the last the ratio compounded\index{Ratio!compound} of the ratios of the first
to the second, of the second to the third, of the third
to the fourth, \&c.

\smallskip
\begin{footnotesize}
We have placed these definitions in a group; but their order
is inverted, and, as we shall see, Def.~\textsc{xii}.\ is a theorem, and \textsc{x}.\
and \textsc{xi}.\ are only inferences from it.

1.~If we have two ratios, such as $5 : 7$ and $3 : 4$, and if we
convert each ratio into a fraction, and multiply these fractions
together, we get a result which is called the ratio compounded of

the two ratios; viz.~in this example it is $\frac{15}{28}$, or $15 : 28$. It is
evident we get the same result if we multiply the two antecedents
together for a new antecedent, and the two consequents for a
new consequent. Hence we have the following definition:---``\emph{The
ratio compounded\index{Ratio!compound} of any number of ratios it the ratio of
the product of all the antecedents to the product of all the consequents.}''

2.~To prove the theorem contained in Def.~\textsc{xii}.

Let the magnitudes be $a$, $b$, $c$, $d$. Then the ratio of
\begin{align*}
\text{1st $:$ 2nd} & = \frac{a}{b}, \\
\text{2nd $:$ 3rd} & = \frac{b}{c}, \\
\text{3rd $:$ 4th} & = \frac{c}{d}.
\end{align*}
Hence the ratio compounded of the ratio of 1st $:$ 2nd, of 2nd $:$ 3rd,
of 3rd $:$ 4th
\[
= \frac{abc}{bcd} = \frac{a}{d} = \text{ratio of 1st $:$ 4th}.
\]

3.~If three magnitudes be proportional, the ratio of the 1st $:$ 3rd
is equal to the square of the ratio of the 1st $:$ 2nd. For the ratio
of the 1st $:$ 3rd is compounded of the ratio of the 1st $:$ 2nd, and
of the ratio of the 2nd $:$ 3rd; and since these ratios are equal,
the ratio compounded of them will be equal to the square of one
of them.

\emph{Or thus}: Let the proportionals be $a$, $b$, $c$, that is, let $a : b :: b : c$;
hence we have
\[
\frac{b}{c} = \frac{a}{b}.
\]
And multiplying each by $\dfrac{a}{b}$, we get
\[
\frac{a}{c} = \frac{a^2}{b^2};
\]
or $a : c :: a^2 : b^2$---that is, 1st $:$ 3rd $::$ square of 1st $:$ square of 2nd.
Now, the ratio of 1st $:$ 3rd is, by Def.~\textsc{x}., the duplicate ratio of
1st $:$ 2nd. Hence the duplicate ratio\index{Ratio!duplicate} of two magnitudes means
the square of their ratio, or, what is the same thing, the ratio of
their squares (see Book VI\@.\ \textsc{xx}.).


4.~If four magnitudes be continual proportionals, the ratio of
1st $:$ 4th is equal to the cube of the ratio of 1st $:$ 2nd. This may
be proved exactly like 3. Hence we see that what Euclid calls
triplicate ratio of two magnitudes is the ratio of their cubes, or
the cube of their ratio.

We also see that there is no necessity to introduce extraneous
magnitudes for the purpose of defining duplicate and triplicate
ratios, as Euclid does. In fact, the definitions by squares and
cubes are more explicit.
\par\end{footnotesize}

\textsc{xiii}. In proportionals, the antecedent terms are
called \emph{homologous}\index{Homologous terms} to one another; as also the consequents
to one another.\par\smallskip

\begin{footnotesize}
If one proportion be given, from it an indefinite number of
other proportions can be inferred, and a great part of the theory
of proportion consists in proving the truth of these derived proportions.
Geometers make use of certain technical terms to
denote the most important of these processes. We shall indicate
these terms by including them in parentheses in connexion with
the Propositions to which they refer. They are useful as indicating,
by one word, the whole enunciation of a theorem.
\par\end{footnotesize}

Every Proposition in the Fifth Book is a Theorem.


\addcontentsline{toc}{section}{Propositions \textsc{i.--xxv.,}}
\mypropl{PROP\@.~I.---Theorem.}{If any number of magnitudes of the same kind $(a,\ b,\ c,\ \textrm{\&c}.)$,
be equimultiples of as many others $(a',\ b',\ c',\ \textrm{\&c}.)$,
then the sum of the first magnitudes $(a + b + c,\ \textrm{\&c}.)$ shall
be the same multiple of the sum of the second which any
magnitude of the first system is of the corresponding magnitude
of the second system.}

\textbf{Dem.}---Let $m$ denote the multiple which the magnitudes
of the first system are of those of the second
system.
\begin{flalign*}
&\text{\indent Then we have }&
   & a = ma' \ (\text{hyp}.),
&\phantom{\indent Then\ we\ have\ }&\\
&& & b = mb',  &&\\
&& & c = mc'.  &&\\
&& & \text{\&c.,\ \&c.}  &&
\end{flalign*}
Hence, by addition,
\[
(a + b + c + \text{\&c.}) = m(a' + b' + c',\ \text{\&c.}).
\]


\mypropl{PROP\@.~II\@.---Theorem.}{If two magnitudes of the same kind $(a,\ b)$ be the same
multiples of another $(c)$ which two corresponding magnitudes
$(a',\ b')$ are of another $(c')$, then the sum of the two
first is the same multiple of their submultiple which the
sum of their corresponding magnitudes is of their submultiple.}

\textbf{Dem.}---Let $m$ and $n$ be the multiples which $a$ and $b$
are of $c$.

Then we have
\begin{equation*}
\begin{alignedat}{2}
a &= mc & \text{\ and } a' &= mc', \\
b &= nc & \text{\ and } b' &= nc'.
\end{alignedat}
\end{equation*}
Therefore
\begin{equation*}
   (a + b) = (m + n) c \text{, and\ }
  (a' + b') = (m + n) c'.
\end{equation*}
Hence $a+b$ is the same multiple of $c$ that $a' + b'$ is of $c'$.

This Proposition is evidently true for any number of
multiples.

\mypropl{PROP\@.~III\@.---Theorem.}{If two magnitudes $(a,\ b)$ be equimultiples of two others
$(a',\ b')$; then any equimultiples of the first magnitudes
$(a,\ b)$ will be also equimultiples of the second magnitudes
$(a',\ b')$.}

\textbf{Dem.}---Let $m$ denote the multiples which $a$, $b$ are of
$a'$, $b'$; then we have
\begin{equation*}
  a = ma', \quad b = mb'.
\end{equation*}
Hence, multiplying each equation by $n$, we get
\begin{equation*}
  na = mna', \quad nb = mnb'.
\end{equation*}
Hence, $na$, $nb$ are equimultiples of $a'$, $b'$.


\mypropl{PROP\@.~IV\@.---Theorem.}{If four magnitudes be proportional, and if any equimultiples
of the first and third be taken, and any other
equimultiples of the second and fourth; then the multiple
of the first $:$ the multiple of the second $::$ the multiple of
the third $:$ the multiple of the fourth.}

Let $a : b :: c : d$; then $ma : nb :: mc : nd$.

\textbf{Dem.}---We have\quad $a : b :: c : d$ (hyp.);
\begin{flalign*}
&\text{therefore }&
   \frac{a}{b} &= \frac{c}{d}.  &\phantom{therefore }&
\intertext{Hence, multiplying each fraction by $\dfrac{m}{n}$, we get}
&& \frac{ma}{nb} &= \frac{mc}{nd};
\end{flalign*}
therefore\hfill $ ma : nb :: mc : nd $. \hfill\phantom{therefore }

\mypropl{PROP\@.~V.---Theorem.}{If two magnitudes of the same kind $(a,\ b)$ be the same
multiples of another $(c)$ which two corresponding magnitudes
$(a',\ b')$ are of another $(c')$, then the difference of the
two first is the same multiple of their submultiple $(c)$,
which the difference of their corresponding magnitudes is
of their submultiple $(c')$ \textrm{(compare Proposition~\textsc{ii.}).}}

\textbf{Dem.}---Let $m$ and $n$ be the multiples which $a$ and $b$
are of $c$.
\begin{flalign*}
&\text{\indent Then we have }&
   &  a = mc, \text{\ and\ }  a' = mc',
&\phantom{\indent Then\ we \ have\ }&
\\
&& &  b = nc, \text{\ and\ }  b' = nc'.  &&
\end{flalign*}

Therefore $(a-b) = (m-n)c$, and $(a'-b') = (m-n)c'$.
\textit{Hence $a-b$ is the same multiple of $c$ that $a'-b'$ is of $c'$.}

\emph{Cor.}---If $a-b = c$,\quad $a'-b' = c'$;\quad  for if $a-b = c$,
$m-n = 1$.


\mypropl{PROP\@.~VI\@.---Theorem.}{If a magnitude $(a)$ be the same multiple of another $(b)$,
which a magnitude $(a')$ taken from the first is of a magnitude
$(b')$ taken from the second, the remainder is the
same multiple of the remainder that the whole is of the
whole \textrm{(compare Proposition~\textsc{i.}).}}

\textbf{Dem.}---Let $m$ denote the multiples which the magnitudes
$a$, $a'$ are of $b$, $b'$; then we have
\begin{flalign*}
&&  a  &= mb ,  &\phantom{\indent Hence }&\\
&&  a' &= mb'.  &&\\
&\text{\indent Hence }& (a-a') &= m(b-b').  &&
\end{flalign*}

\mypropl{Prop.~A.---Theorem (Simson).}{If two ratios be equal, then according as the antecedent
of the first ratio is greater than, equal to, or less than its
consequent, the antecedent of the second ratio is greater
than, equal to, or less than its consequent.}
\textbf{Dem.}---Let \quad $a : b :: c : d$; \\
then\hfill $ \dfrac{a}{b} = \dfrac{c}{d}; $ \hfill\phantom{then }\\
and if $a$ be greater than $b$, $\dfrac{a}{b}$ is greater than unity;
therefore $\dfrac{c}{d}$ is greater than unity, and $c$ is greater
than $d$.

In like manner, if $a$ be equal to $b$, $c$ is equal to $d$,
and if less, less.



\myprop{Prop.~B.---Theorem (Simson).}{If two ratios are equal their reciprocals are equal $($invertendo$)$\index{Invertendo}.}

Let \hfill$
  a: b :: c : d, \text{\ then } b : a :: d : c. $
\hfill\phantom{\indent Let}
\begin{flalign*}
&\text{\indent \textbf{Dem.}---Since }&
  a : b &:: c : d;  &\phantom{\indent Dem.---Since }&
\\
&\text{then }&
  \frac{a}{b} &= \frac{c}{d}; &&
\\
&\text{therefore }&
  1 \div \frac{a}{b} &= 1 \div \frac{c}{d},  &&
\\
&\text{or }&
  \frac{b}{a} &= \frac{d}{c}  &&
\\
&\text{Hence }&
  b : a &:: d : c.  &&
\end{flalign*}

\myprop{Prop.~C.---Theorem (Simson).}{If the first of four magnitudes be the same multiple of
the second which the third is of the fourth, the first is to
the second as the third is to the fourth.}

Let \hfill
$a = mb$, $c = md$; then $a : b :: c : d$.
\hfill\phantom{\indent Let }

\textbf{Dem.}---Since \hfill $a = mb$, we have $\dfrac{a}{b} = m$.
\hfill\phantom{\indent Dem.---Since }

In like manner, $\dfrac{c}{d} = m$; therefore $\dfrac{a}{b} = \dfrac{c}{d}$.
\begin{flalign*}
&\text{Hence }&&
 a : b :: c : d.  &\phantom{Hence }&
\end{flalign*}


\mypropl{Prop.~D.---Theorem (Simson).}{If the first be to the second as to the third is to the
fourth, and if the first be a multiple or submultiple of
the second, the third is the same multiple or submultiple
of the fourth.}

1.~Let $a : b :: c : d$, and let $a$ be a multiple of $b$,
then $c$ is the same multiple of $d$.

\textbf{Dem.}---Let $a = mb$, then $\dfrac{a}{b} = m$;\\
but \hfill$\dfrac{a}{b} = \dfrac{c}{d}$; therefore $\dfrac{c}{d} = m$, and $c = md$.\hfill\phantom{but }

2.~Let $a = \dfrac{b}{n}$, then $\dfrac{a}{b} = \dfrac{1}{n}$;
\begin{flalign*}
&\text{therefore }& \frac{c}{d} &= \frac{1}{n}, &&\phantom{therefore }
\\
&\text{Hence }&     c &= \frac{d}{n}.  &&
\end{flalign*}

\myprop{PROP\@.~VII\@.---Theorem.}{\textrm{1.} Equal magnitudes have equal ratios to the same
magnitude.\\\textrm{2.} The same magnitude has equal ratios to equal magnitudes.}

Let $a$ and $b$ be equal magnitudes, and $c$ any other
magnitude.
\begin{flalign*}
\text{\indent Then 1. }&&
  a : c &:: b : c,
&&\phantom{\indent Then\ 1.}
\\
\text{\ 2. }&&
  c : a &:: c : b.  &&
\end{flalign*}



\textbf{Dem.}---Since $a = b$, dividing each by $c$, we have
\[
\frac{a}{c} = \frac{b}{c};
\]
therefore \hfill$a : c :: b : c$.\hfill\phantom{therefore}

Again, since $a = b$, dividing $c$ by each, we have
\[
\frac{c}{a} = \frac{c}{b};
\]
therefore \hfill$ c : a :: c : b$.\hfill\phantom{therefore}\par\smallskip

\emph{Observation.}---2 follows at once from 1 by Proposition~B.

\mypropl{PROP\@.~VIII\@.---Theorem.}{$1$. Of two unequal magnitudes, the greater has a greater
ratio to any third magnitude than the less has; $2$. any
third magnitude has a greater ratio to the less of two unequal
magnitudes than it has to the greater.}

1.~Let $a$ be greater than $b$, and let $c$ be any other
magnitude of the same kind, then the ratio $a : c$ is
greater than the ratio $b : c$.

\textbf{Dem.}---Since $a$ is greater than $b$, dividing each by $c$,
\[
\frac{a}{c} \text{\ is greater than } \frac{b}{c};
\]
therefore the ratio $a : c$ is greater than the ratio $b : c$.

2.~To prove that the ratio $c : b$ is greater than the
ratio $c : a$.

\textbf{Dem.}---Since $b$ is less than $a$, the quotient which is
the result of dividing any magnitude by $b$ is greater
than the quotient which is got by dividing the same
magnitude by $a$;
\begin{flalign*}
&\text{therefore }& \frac{c}{b}\text{\ is greater than }\frac{c}{a}. &\phantom{therefore}&
\end{flalign*}
Hence the ratio $c : b$ is greater than the ratio $c : a$.


\mypropl{PROP\@.~IX\@.---Theorem.}{Magnitudes which have equal ratios to the same magnitude
are equal to one another; $2$. magnitudes to which
the same magnitude has equal ratios are equal to one
another.}

\medskip
1.~If $a : c :: b : c$, to prove $a = b$.
\begin{flalign*}
&\text{\indent \textbf{Dem.}---Since}&
a : c & :: b : c,
&&\phantom{\indent Dem.---Since}\\
&&  \frac{a}{c} & = \frac{b}{c}.  &&
\end{flalign*}
Hence, multiplying each by $c$, we get $a = b$.

\medskip
2.~If $c : a :: c : b$, to prove $a = b$.
\begin{flalign*}
&\text{\indent \textbf{Dem.}---Since }&
  c : a &:: c : b,
&&\phantom{\indent Dem.---Since}\\
&\text{by inversion, }&
  a : c &:: b : c; &&  \\
&\text{therefore }&
      a &= b. \rlap{\qquad [1].} &&
\end{flalign*}

\mypropl{PROP\@.~X.---Theorem.}{Of two unequal magnitudes, that which has the greater
ratio to any third is the greater of the two; and that to
which any third has the greater ratio is the less of the
two.}

\medskip
1.~If the ratio $a : c$ be greater than the ratio $b: c$,
to prove $a$ greater than $b$.

\medskip
\textbf{Dem.}---Since the ratio $a : c$ is greater than the ratio
$b : c$,
\[
\frac{a}{c} \ \text{is greater than} \  \frac{b}{c}.
\]
Hence, multiplying each by $c$, we get $a$ greater than $b$.



2.~If the ratio $c : b$ is greater than the ratio $c : a$, to
prove $b$ is less than $a$.

\textbf{Dem.}---Since the ratio $c : b$ is greater than the ratio
$c : a$,
\[
\frac{c}{b} \text{\ is greater than } \frac{c}{a}.
\]
\begin{flalign*}
&\text{Hence }&
1 \div \frac{c}{b} &\text{\ is less than } 1 \div \frac{c}{a},  &&\phantom{that\ is, }\\
&\text{that is, }&
\frac{b}{c} &\text{\ is less than } \frac{a}{c}.  &&
\end{flalign*}
Hence, multiplying each by $c$, we get
\[
b \text{\ less than } a.
\]

\myprop{PROP\@.~XI\@.---Theorem.}{Ratios that are equal to the same ratio are equal to one another.}

Let $a : b :: e : f$, and $c : d :: e : f$, to prove $a : b :: c : d$.

\textbf{Dem.}---Since \hfill$a : b :: e : f$, \hfill\phantom{\indent Dem.---Since}
\begin{flalign*}
&&  \frac{a}{b} &= \frac{e}{f}.  &&\phantom{In\ like\ manner, }\\
&\text{In like manner, }&
    \frac{c}{d} &= \frac{e}{f}.  &&\\
&\text{Hence }&
    \frac{a}{b} &= \frac{c}{d} \rlap{\qquad [I., Axiom \textsc{i}.],}  &&\\
&\text{and }&
a :{}& b :: c : d.
\end{flalign*}


\mypropl{PROP\@.~XII\@.---Theorem.}{If any number of ratios be equal to one another, any
one of these equal ratios is equal to the ratio of the sum
of all the antecedents to the sum of all the consequents.}

Let the ratios $a : b$, $c : d$, $e : f$, be all equal to one
another; it is required to prove that any of these
ratios is equal to the ratio $a + c + e : b + d + f$.

\textbf{Dem.}---By hypotheses,
\[
\frac{a}{b} = \frac{c}{d} = \frac{e}{f}.
\]
Since these fractions are all equal, let their common
value be $r$; then we have
\begin{flalign*}
&&  \frac{a}{b} = r,\quad \frac{c}{d} &= r,\quad \frac{e}{f} = r; &&\\
&\text{therefore }&
    a &= br,  &&\phantom{therefore }\\
&&  c &= dr,  &&\\
&&  e &= fr\,;  &&\\
&\text{therefore }&  a + c + e &= (b + d + f)r.  &&
\end{flalign*}
\begin{flalign*}
&\text{Hence }&
&  \frac{a + c + e}{b + d + f} = r\,;  &&\phantom{therefore }\\
&\text{therefore }&
&  \frac{a}{b} = \frac{a + c + e}{b + d +f},  &&\\
&\text{and }&
   a : b ::{}& a + c + e : b + d + f.  &&
\end{flalign*}

\emph{Cor.}---With the same hypotheses, if $l$, $m$, $n$ be any
three multipliers, $a : b :: la + mc + ne : lb + md + nf$.



\myprop{PROP\@.~XIII\@.---Theorem.}{If two ratios are equal, and if one of them be greater
than any third ratio, then the other is also greater than
that third ratio.}

If $a : b :: c : d$, but the ratio of $c : d$ greater than
the ratio of $e : f$; then the ratio of $a : b$ is greater than
the ratio of $e : f$.

\textbf{Dem.}---Since the ratio of $c : d$ is greater than the
ratio of $e : f$,
\[
\frac{c}{d} \text{\ is greater than } \frac{e}{f}.
\]
\begin{flalign*}
&\text{Again, since }&
   a &: b :: c : d,  &&\phantom{Again, since }\\
&& &\frac{a}{b} = \frac{c}{d};  &&
\end{flalign*}
\begin{flalign*}
&\text{therefore }&
&\frac{a}{b} \text{\ is greater than } \frac{e}{f}. &&\phantom{therefore }
\end{flalign*}
or the ratio of $a : b$ is greater than the ratio of $e : f$.

\mypropl{PROP\@.~XIV\@.---Theorem.}{If two ratios be equal, then, according as the antecedent
of the first ratio is greater than, equal to, or less than the
antecedent of the second, the consequent of the first is
greater than, equal to, or less than the consequent of the
second.}

Let $a : b :: c : d$; then if $a$ be greater than $c$, $b$ is
greater than $d$; if equal, equal; if less, less.
\begin{flalign*}
&\text{\indent\textbf{Dem.}---Since }&
a : b &:: c : d. &&\phantom{\indent Dem.}\\
&\text{we have }&
\frac{a}{b} &= \frac{c}{d},  &&
\end{flalign*}

and multiplying each by $\dfrac{b}{c}$ we get
\begin{flalign*}
&&    \frac{a}{b} \times \frac{b}{c}
= {}& \frac{c}{d} \times \frac{b}{c},      &&\\
&\text{or }&        \frac{a}{c} = {}& \frac{b}{d};  &&\\
&\text{therefore }&      a : c :: {}& b : d.  &&\phantom{therefore }
\end{flalign*}
Hence, Proposition [A], if $a$ be greater than $c$, $b$ is
greater than $d$; if equal, equal; and if less, less.

\myprop{PROP\@.~XV\@.---Theorem.}{Magnitudes have the same ratio which all equimultiples of them have.}

Let $a$, $b$ be two magnitudes, then the ratio $a : b$ is
equal to the ratio $ma : mb$.

\textbf{Dem.}---The ratio $a : b = \dfrac{a}{b}$, and the ratio of $ma : mb
= \dfrac{ma}{mb}$; but since the value of a fraction is not altered
by multiplying its numerator and denominator by the
same number,
\begin{flalign*}
&&  \frac{a}{b} &=  \frac{ma}{mb};  &&\\
&\text{therefore }&
          a : b &:: ma : mb.  &&\phantom{therefore }
\end{flalign*}

\myprop{PROP\@.~XVI---Theorem.}{If four magnitudes of the same kind be proportionals they
are also proportionals by alternation $($alternando$)$\index{Alternando}.}

Let \hfill$a : b :: c : d$, then $a : c :: b : d$. \hfill\phantom{indent Let }



\textbf{Dem.}---Since \hfill $a : b :: c : d$, \hfill\phantom{\indent Dem.---Since }
\[
\frac{a}{b} = \frac{c}{d},
\]
and multiplying each by $\dfrac{b}{c}$, we get
\begin{flalign*}
&&  \frac{a}{b} \,.\, \frac{b}{c}
 &= \frac{c}{d} \,.\, \frac{b}{c},  &&\phantom{therefore }\\
&\text{or }&         \frac{a}{c} &= \frac{b}{d};  &&\\
&\text{therefore }&        a : c &:: b : d.  &&
\end{flalign*}

\mypropl{PROP\@.~XVII\@.---Theorem.}{If four magnitudes be proportional, the difference between
the first and second $:$ the second $::$ the difference between
the third and fourth $:$ the fourth $($dividendo$)$\index{Dividendo}.}

Let \hfill$
  a : b :: c : d:$ then $a-b : b :: c-d : d$;
\hfill\phantom{\indent Let }
\begin{flalign*}
&\indent\text{\textbf{Dem.}---Since }&
                     a : b           &:: c : d,  &&\phantom{\indent Dem.---Since }\\
&&                   \frac{a}{b}     &= \frac{c}{d};      &&\\
&\text{therefore }&  \frac{a}{b} - 1 &= \frac{c}{d} - 1,  &&\\
&\text{or }&         \frac{a-b}{b}   &= \frac{c-d}{d};    &&\\
&\text{therefore }&  a-b : b         &:: c-d : d.         &&
\end{flalign*}

\myprop{PROP\@.~XVIII\@.---Theorem.}{If four magnitudes be proportionals, the sum of the
first and second $:$ the second $::$ the sum of the third and
fourth $:$ the fourth $($componendo$)$.\index{Componendo}}

Let \hfill $a : b :: c : d$; then $a+b : b :: c+d : d$. \hfill\phantom{\indent Let }

\begin{flalign*}
&\indent\text{\textbf{Dem.}---Since }&
    a: b ::{}& c : d,  &&\phantom{\indent Dem.---Since }\\
&&  \frac{a}{b} ={}& \frac{c}{d};  &&\\
&\text{therefore }&
    \frac{a}{b} + 1 ={}& \frac{c}{d} + 1,  &&\\
&\text{or }&
    \frac{a+b}{b} ={}& \frac{c+d}{d};  &&\\
&\text{therefore }&
    a+b : b ::{}& c+d : d.  &&
\end{flalign*}

\mypropl{PROP\@.~XIX\@.---Theorem.}{If a whole magnitude be to another whole at a magnitude
taken from the first it to a magnitude taken from
the second, the first remainder $:$ the second remainder $::$
the first whole $:$ the second whole.}

Let $a : b :: c : d$, $c$ and $d$ being less than $a$ and $b$;\\
then \hfill$ a-c : b-d :: a : b$. \hfill\phantom{then }
\begin{flalign*}
&\text{\indent\textbf{Dem.}---Since }&
    a : b &:: c : d,  &&\phantom{\indent Dem.---Since }\\
&\text{then }&
    a : c &:: b : d \rlap{\ [\emph{alternando}],}  &&\\
&\text{and }&
    c : a &:: d : b \rlap{\ [\emph{invertendo}];}  &&\\
&\text{therefore }&
    \frac{c}{a} &= \frac{d}{b},  &&\\
&\text{and }&
    1 - \frac{c}{a} &= 1 - \frac{d}{b},  &&\\
&\text{or }&
    \frac{a-c}{a} &= \frac{b-d}{b}.  &&\\
&\text{Hence }&
    a-c :{}& b-d :: a : b.  &&

\end{flalign*}

\myprop{Prop.~E.---Theorem (Simson).}{If four magnitudes be proportional, the first $:$ its excess
above the second $::$ the third $:$ its excess above the fourth
$($convertendo$)$\index{Convertendo}.}

Let \quad $a : b :: c : d$; then $a : a-b :: c : c-d$.

\begin{flalign*}
&\indent\text{\textbf{Dem.}---Since }&
    a : b &:: c : d,  &&\phantom{\indent Dem.---Since }\\
&&  \frac{a}{b} &= \frac{c}{d};  &&\\
&\text{therefore }&
    \frac{a-b}{b} &= \frac{c-d}{d} \qquad\ \rlap{\ [Dem.\ of \textsc{xvii.}],}  &&\\
&\text{therefore }&
    \frac{a}{b} \div \frac{a-b}{b} &= \frac{c}{d} \div \frac{c-d}{d},  &&\\
&\text{or }&
    \frac{a}{a-b} &= \frac{c}{c-d},  &&\\
&\text{therefore }&
    a : a-b &:: c : c-d.  &&
\end{flalign*}

\mypropl{PROP\@.~XX\@.---Theorem.}{If there be two sets of three magnitudes, which taken
two by two in direct order have equal ratios, then if the
first of either set be greater than the third, the first of
the other set is greater than the third; if equal, equal;
and if less, less.}

Let $a$, $b$, $c$; $a'$, $b'$, $c'$ be the two sets of magnitudes,
and let the ratio $a : b = a' : b'$, and $b : c = b' : c'$; then,
if $a$ be greater than, equal to, or less than $c$, $a'$ will be
greater than, equal to, or less than $c'$.
\begin{flalign*}
&\text{\indent\textbf{Dem.}---Since }&
   a &: b :: a' : b',  &&\phantom{\indent Dem.---Since }\\
&\text{we have }&
&  \frac{a}{b} = \frac{a'}{b'},  &&\\
&\text{In like manner, }&
&  \frac{b}{c} = \frac{b'}{c'},  &&\\
&\text{Hence }&
   \frac{a}{b} \times{}& \frac{b}{c} = \frac{a'}{b'} \times \frac{b'}{c'},  &&\\
&\text{or }&
&  \frac{a}{c} = \frac{a'}{c'}.  &&
\end{flalign*}

\textit{Therefore if $a$ be greater than $c$, $a'$ is greater than $c'$;
if equal, equal; and if less, less.}



\mypropl{PROP\@.~XXI\@.---Theorem.}{If there be two sets of three magnitudes, which taken
two by two in transverse order have equal ratios; then,
if the first of either set be greater than the third, the first
of the other set is greater than the third; if equal, equal;
and if less, less.}

Let $a$, $b$, $c$; $a'$, $b'$, $c'$ be the two sets of magnitudes,
and let the ratio $a : b = b' : c'$, and $b : c = a' : b'$. Then,
if $a$ be greater than, equal to, or less than $c$, $a'$ will
be greater than, equal to, or less than $c'$.
\begin{flalign*}
&\text{\indent\textbf{Dem.}---Since }&
  a : b ::{}& b' : c',  &&\phantom{\indent Dem.---Since }\\
&\text{we have }&
  \frac{a}{b} ={}& \frac{b'}{c'}.  &&\\
&\text{In like manner, }&
  \frac{b}{c} ={}& \frac{a'}{b'}.  &&\\
&\text{Hence, multiplying }&
  \frac{a}{c} ={}& \frac{a'}{c'}.  &&
\end{flalign*}

\textit{Therefore, if $a$ be greater than $c$, $a'$ is greater than $c'$;
if equal, equal; if less, less.}

\mypropl{PROP\@.~XXII\@.---Theorem.}{If there be two sets of magnitudes, which, taken two by
two in direct order, have equal ratios, then the first $:$ the
last of the first set $::$ the first $:$ the last of the second set
$($``ex aequali,''\index{Ex aequali} or ``ex aequo''$)$.}

Let $a$, $b$, $c$; $a'$, $b'$, $c'$ be the two sets of magnitudes,
and if $a : b :: a' : b'$, and $b : c :: b' : c'$, then $a : c :: a' : c'$.
\begin{flalign*}
&\text{\indent\textbf{Dem.}---Since }&
  a : b ::{}& a' : b',  &&\phantom{\indent Dem.---Since }\\
&\text{we have }&
  \frac{a}{b} ={}& \frac{a'}{b'}  &&\\

&\text{In like manner, }&
  \frac{b}{c} ={}& \frac{b'}{c'}.  &&\\
&\text{Hence, multiplying, }&
  \frac{a}{c} ={}& \frac{a'}{c'}.  &&\\
&\text{Therefore }&
  a : c ::{}& a': c',  &&
\end{flalign*}
and similarly for any number of magnitudes in each
set.

\emph{Cor.}~1.---If the ratio $b : c$ be equal to the ratio $a : b$,
then $a$, $b$, $c$ will be in continued proportion, and so will
$a'$, $b'$, $c'$. Hence [Def.~\textsc{xii.}\ Annotation~3],
\[
\frac{a}{c} = \frac{a^{2}}{b^{2}} \text{\ and }
\frac{a'}{c'} = \frac{a'^{2}}{b'^{2}};
\]
\begin{flalign*}
&\text{but }&
  \frac{a}{c} &= \frac{a'}{c'}.  &&\rlap{[\textsc{xxii.}]}\phantom{Therefore }
\\
&\text{Therefore }&
  \frac{a^{2}}{b^{2}} &= \frac{a'^{2}}{b'^{2}}.  &&
\end{flalign*}
\begin{flalign*}
&\text{Hence, if }&
    a     &: b :: a' : b',  &&\phantom{Hence,\ if }\\
&&  a^{2} &: b^{2}  :: a'^{2} : b'^{2}  &&
\end{flalign*}
\textit{Or if four magnitudes be proportional, their squares are
proportional.}

\emph{Cor.}~2.---If four magnitudes be proportional, their
cubes are proportional.

\mypropl{PROP\@.~XXIII\@.---Theorem.}{If there be two sets of magnitudes, which, taken two
by two in transverse order, have equal ratios; then the
first $:$ the last of the first set $::$ the first $:$ the last of the
second set $($``ex aequo perturbato''$)$\index{Ex aequo perturbato}.}

Let $a$, $b$, $c$; $a'$, $b'$, $c'$ be the two sets of magnitudes,
and let the ratio $a : b = b' : c'$, and $b : c = a' : b'$; then
$a : c :: a' : c'$.

\begin{flalign*}
&\text{\indent\textbf{Dem.}---Since }&
  a : b ::{}& b': c',  &&\phantom{Hence,\ multiplying,\ }\\
&\text{we have }&
  \frac{a}{b} ={}& \frac{b'}{c'}.  &&\\
&\text{In like manner,}&
  \frac{b}{c} ={}& \frac{a'}{b'}.  &&\\
&\text{Hence, multiplying, }&
  \frac{a}{c} ={}& \frac{a'}{c'};  &&\\
&\text{therefore }&
  a : c ::{}& a' : c',  &&
\end{flalign*}
and similarly for any number of magnitudes in each
set.

This Proposition and the preceding one may be included
in one enunciation, thus: ``\textit{Ratios compounded
of equal ratios are equal.}''

\mypropl{PROP\@.~XXIV\@.---Theorem.}{If two magnitudes of the same kind $(a,\ b)$ have to
a third magnitude $(c)$ ratios equal to those which two
other magnitudes $(a',\ b')$ have to a third $(c')$, then the sum
$(a + b)$ of the first two has the same ratio to their third $(c)$
which the sum $({a' + b'})$ of the other two magnitudes has to
their third $(c')$.}\vspace{-2\baselineskip}
\begin{flalign*}
&\text{\indent\textbf{Dem.}---Since }&
  a : c &:: a' : c',  &&\phantom{therefore,\ adding,\ }\\
&\text{we have }&
  \frac{a}{c} &= \frac{a'}{c'}.  &&\\
&\text{In like manner, }&
  \frac{b}{c} &= \frac{b'}{c'};  &&\\
&\text{therefore, adding,}&
  \frac{a+b}{c} &= \frac{a'+b'}{c'}.  &&\\
&\text{Hence }&
  a + b : c &:: a' + b' : c'.  &&
\end{flalign*}


\label{NoProblem}
\myprop{PROP\@.~XXV\@.---Theorem.}{If four magnitudes of the same kind be proportionals,
the sum of the greatest and least is greater than the sum
of the other two.}

Let $a:b::c:d$; then, if $a$ be the greatest, $d$ will
be the least [\textsc{xiv}.\ and \textsc{a}]. It is required to prove that
$a + d$ is greater than $b + c$.
\begin{flalign*}
&\text{\indent\textbf{Dem.}---Since }&
  &a : b :: c : d,  &&\phantom{therefore }\\
&&
  &a : c :: b : d \  \text{[\emph{alternando}]};  &&\\
&\text{therefore }&
  &a : a - c :: b : b - d \  \text{[\textbf{E}]}.;  &&\\
&\text{but }&
  &a \  \text{is greater than} \  b \  \text{(hyp.)},  &&\\
&\text{therefore }&
  &a - c \  \text{is greater than} \  b - d  \qquad \text{[\textsc{xiv}.]}. &&\\
&\text{Hence }&
  &a + d \  \text{is greater than} \  b + c.  &&
\end{flalign*}


\addcontentsline{toc}{section}{Questions for Examination,}
\exhead{Questions for Examination on Book V.}

\begin{footnotesize}
1.~What is the subject-matter of this book?

2.~When is one magnitude said to be a multiple of another?

3.~What is a submultiple or measure?

4.~What are equimultiples?

5.~What is the ratio of two commensurable magnitudes?

6.~What is meant by the ratio of incommensurable magnitudes?

7.~Give an Illustration of the ratio of incommensurables.

8.~What are the terms of a ratio called?

9.~What is a ratio of greater inequality?

10.~What is a ratio of lesser inequality?

11.~What is the product of two ratios called? \textit{Ans}. The ratio
compounded of these ratios.



12.~What is \emph{duplicate ratio}?

13.~What is Euclid's definition of duplicate ratio?

14.~Give another definition.

15.~Define triplicate ratio.

16.~What is proportion?  \emph{Ans.}\ equality of ratios.

17.~Give Euclid's definition of proportion.

18.~How many ratios in a proportion?

19.~What are the Latin terms in use to denote some of the
Propositions of Book~V.?

20.~When is a line divided harmonically\index{Line!divided harmonically}?

21.~When a line is divided harmonically, what are corresponding
pairs of points called? \emph{Ans.}\ Harmonic conjugates\index{Harmonic conjugates}.

22.~What are reciprocal ratios?

23.~Give one enunciation that will include Propositions~\textsc{xxii.},
\textsc{xxiii.}\ of Book~V.
\par\end{footnotesize}

\addcontentsline{toc}{section}{Exercises,}
\exhead{Exercises on Book V.}

\begin{footnotesize}
\textsc{Def.~I.}---A ratio whose antecedent is greater than its consequent
is called \emph{a ratio of greater inequality;}\index{Ratio!of greater inequality} and a ratio whose
antecedent is less than its consequent, \emph{a ratio of lesser inequality.}\index{Ratio!of lesser inequality}

\textsc{Def.~II\@.}---A right line is said to be cut \emph{harmonically} when it is
divided internally and externally in any ratios that are equal in
magnitude.

1.~A ratio of greater inequality is increased by diminishing its
terms by the same quantity, and diminished by increasing its
terms by the same quantity.

2.~A ratio of lesser inequality is diminished by diminishing its
terms by the same quantity, and increased by increasing its terms
by the same quantity.

3.~If four magnitudes be proportionals, the sum of the first and
second is to their difference as the sum of the third and fourth is
to their difference (\emph{componendo et dividendo}).

4.~If two sets of four magnitudes be proportionals, and if we
multiply corresponding terms together, the products are proportionals.


5.~If two sets of four magnitudes be proportionals, and if we
divide corresponding terms, the quotients are proportionals.

6.~If four magnitudes be proportionals, their squares, cubes,
\&c., are proportionals.

7.~It two proportions have three terms of one respectively
equal to three corresponding terms of the other, the remaining
term of the first is equal to the remaining term of the second.

8.~If three magnitudes be continual proportionals, the first is
to the third as the square of the difference between the first and
second is to the square of the difference between the second and
third.

9.~If a line $AB$, cut harmonically in $C$ and $D$, be bisected in
$O$; prove $OC$, $OB$, $OD$ are continual proportionals.

10.~In the same case, if $O'$ be the middle point of $CD$; prove
$OO'^2 = OB^2 + O'D^2$.

11.~And \,$AB (AC + AD) = 2AC\,.\,AD$, \ or
$\dfrac{1}{AC}+\dfrac{1}{AD}=\dfrac{2}{AB}$.

12.~And $CD (AD + BD) = 2AD\,.\,BD$, or
$\dfrac{1}{BD}+\dfrac{1}{AD}=\dfrac{2}{CD}$.

13.~And $AB\,.\,CD = 2AD\,.\,CB$.
\par\end{footnotesize}



\mybook{BOOK VI}{Application of the Theory of Proportion}

\textsc{i}. \emph{Similar Rectilineal Figures}\index{Figures!similar} are those whose several
angles are equal, each to each, and whose sides about
the equal angles are proportional.

\smallskip
\begin{footnotesize}
Similar figures agree in shape; if they agree also in size, they
are congruent.

1.~When the shape of a figure is given, it is said to be given
in \emph{species}\index{Figures!given in species}. Thus a triangle whose angles are given is given in
species. Hence similar figures are of the same species.

2.~When the size of a figure is given, it is said to be given
in \emph{magnitude}\index{Figures!given in magnitude}; for instance, a square whose side is of given
length.

3.~When the place which a figure occupies is known, it is said
to be given in \emph{position}\index{Figures!given in position}.
\par\end{footnotesize}

\textsc{ii}. A right line is said to be cut at a point in
extreme and mean ratio when the whole line is to
the greater segment as the greater segment is to the
less.

\textsc{iii}. If three quantities of the same kind be in continued
proportion, the middle term is called a \emph{mean
proportional}\index{Proportional, mean} between the other two.

Magnitudes in continued proportion are also said to
be in \emph{geometrical progression}.

\textsc{iv}. If four quantities of the same kind be in continued
proportion, the two middle terms are called
two \emph{mean proportionals}\index{Proportionals!two mean} between the other two.



\textsc{v.} The \emph{altitude}\index{Altitude} of any figure is the length of the
perpendicular from its highest point to its base.

\textsc{vi.} Two corresponding angles of two figures have
the sides about them reciprocally proportional, when
a side of the first is to a side of the second as the
remaining side of the second is to the remaining side
of the first.

\smallskip
\begin{footnotesize}
This is evidently equivalent to saying that a side of the first is
to a side of the second in the reciprocal ratio of the remaining
side of the first to the remaining side of the second.
\par\end{footnotesize}

\addcontentsline{toc}{section}{Propositions \textsc{i.--xxxiii.,}}
\myprop{PROP\@.~I.---Theorem.}{Triangles $(ABC,\ ACD)$ and parallelograms $(EC,\ CF)$
which have the same altitude are to one another as their
bases $(BC,\ CD)$.}


\imgflow{190}{6}{f169}

\textbf{Dem.}---Produce $BD$ both ways, and cut off any number
of parts $BG$, $GH$,
\&c., each equal to
$CB$, and any number
$DK$, $KL$, each
equal to $CD$. Join
$AG$, $AH$, $AK$, $AL$.

Now, since the several bases $CB$, $BG$, $GH$ are all
equal, the triangles $ACB$, $ABG$, $AGH$ are also all equal
[I.~\textsc{xxxviii.}]. Therefore the triangle $ACH$ is the
same multiple of $ACB$ that the base $CH$ is of the base
$CB$. In like manner, the triangle $ACL$ is the same
multiple of $ACD$ that the base $CL$ is of the base $CD$;
and it is evident that [I.~\textsc{xxxviii.}] if the base $HC$ be
greater than $CL$, the triangle $HAC$ is greater than
$CAL$; if equal, equal; and if less, less. Now we
have four magnitudes: the base $BC$ is the first, the
base $CD$ the second, the triangle $ABC$ the third, and
the triangle $ACD$ the fourth. We have taken equimultiples
of the first and third, namely, the base
$CH$, and the triangle $ACH$; also equimultiples of the

second and fourth, namely, the base $CL$, and the triangle
$ACL$; and we have proved that according as the
multiple of the first is greater than, equal to, or less
than the multiple of the second, the multiple of the
third is greater than, equal to, or less than the multiple
of the fourth. \textit{Hence} [V.\ Def.~\textsc{v.}] \textit{the base
$BC : CD ::$  the triangle  $ABC : ACD$.}

2.~The parallelogram $EC$ is double of the triangle $ABC$
[I.~\textsc{xxxiv.}], and the parallelogram $CF$ is double of the
triangle $ACD$. Hence [V.~\textsc{xv.}] $EC : CF ::$ the triangle
$ABC : ACD$; but $ABC : ACD :: BC : CD$ (Part~I.).
\textit{Therefore} [V.~\textsc{xi.}] \textit{$EC : CF ::$ the base $BC : CD$.}

\smallskip
\begin{footnotesize}
\emph{Or thus:} Let $A$, $A'$ denote the areas of the triangles $ABC$, $ACD$,
respectively, and $P$ their common altitude; then [II\@.~\textsc{i.}, \emph{Cor.}~1],
\begin{flalign*}
&&  &\, A = \tfrac{1}{2} P\,.\,BC, \quad A' = \tfrac{1}{2} P\,.\,CD.  &&\phantom{Hence }\\
&\text{Hence }&
    &\frac{A}{A'} = \frac{BC}{CD},\text{\ or } A : A' :: BC : CD.  &&
\end{flalign*}

In extending this proof to parallelograms we have only to use
$P$ instead of $\frac{1}{2} P$.
\par\end{footnotesize}

\mypropl{PROP\@.~II\@.---Theorem.}{If a line $(DE)$ be parallel to a side $(BC)$ of a triangle
$(ABC)$, it divides the remaining sides, measured from the
opposite angle $(A)$, proportionally; \textrm{and, conversely,} If
two sides of a triangle, measured from an angle, be cut
proportionally, the line joining the points of section is
parallel to the third side.}


\imgflow{95}{8}{f170}

1.~It is required to prove that
$AD : DB :: AE : EC$.

\label{VIi}
\textbf{Dem.}---Join $BE$, $CD$. The triangles
$BDE$, $CED$ are on the same
base $DE$, and between the same parallels
$BC$, $DE$. Hence [I.~\textsc{xxxvii.}] they
are equal, and therefore [V.~\textsc{vii.}] the
triangle $ADE : BDE :: ADE : CDE$;

\begin{flalign*}
&\text{but }&
  ADE &: BDE :: AD : DB   \ \text{\ [\textsc{i.}]},  &&\phantom{Hence }\\
&\text{and }&
  ADE &: CDE :: AE : EC \,\ \text{\ [\textsc{i.}]}.  &&\\
&\text{\emph{Hence} }&
  AD &: DB :: AE : EC.  &&
\end{flalign*}

2.~If $AD : DB :: AE : EC$, it is required to prove
that $DE$ is parallel to $BC$.

\textbf{Dem.}---Let the same construction be made;
\begin{flalign*}
&\text{then }&
&  AD : DB :: \text{the triangle } ADE : BDE   \ \text{\ [\textsc{i.}].}  &&\phantom{Hence }\\
&\text{and }&
&  AE : EC :: \text{the triangle } ADE : CDE \,\ \text{\ [\textsc{i.}]};  &&\\
&\text{but }&
&  AD : DB :: AE : EC \text{\ (hyp.)}.  &&\\
&\text{\emph{Hence} }&
&  ADE : BDE :: ADE : CDE.  &&
\end{flalign*}
Therefore [V.~\textsc{ix.}] the triangle $BDE$ is equal to $CDE$,
and they are on the same base $DE$, and on the same
side of it; hence they are between the same parallels
[I.~\textsc{xxxix.}]. \textit{Therefore $DE$ is parallel to $BC$.}

\smallskip
\begin{footnotesize}
\textsf{Observation.}---The line $DE$ may cut the sides $AB$, $AC$ produced
through $B$, $C$, or through the angle $A$; but evidently a
separate figure for each of these cases is unnecessary.
\par\end{footnotesize}

\exhead{Exercise.}

\begin{footnotesize}
If two lines be cut by three or more parallels, the intercepts
on one are proportional to the corresponding intercepts on the
other.
\par\end{footnotesize}

\mypropl{PROP\@.~III\@.---Theorem.}{If a line $(AD)$ bisect any angle $(A)$ of a triangle $(ABC)$,
it divides the opposite side $(BC)$ into segments proportional
to the adjacent sides. \textrm{Conversely,} If the segments
$(BD,\ DC)$ into which a line $(AD)$ drawn from any angle
$(A)$ of a triangle divides the opposite side be proportional
to the adjacent sides, that line bisects the angle $(A)$.}


\imgflow{108}{8}{f171}

\textbf{Dem.}---1. Through $C$ draw $CE$ parallel to $AD$, to
meet $BA$ produced in $E$. Because $BA$ meets the

parallels $AD$, $EC$, the angle $BAD$ [I.~\textsc{xxix}.] is equal
to $AEC$; and because $AC$ meets the parallels $AD$, $EC$,
the angle $DAC$ is equal to $ACE$;
but the angle $BAD$ is equal to
$DAC$ (hyp.); therefore the angle
$ACE$ is equal to $AEC$; therefore
$AE$ is equal to $AC$ [I.~\textsc{vi}.]. Again,
because $AD$ is parallel to $EC$, one
of the sides of the triangle $BEC$,
$BD : DC :: BA : AE$ [\textsc{ii.}];\label{VIii} but
$AE$ has been proved equal to $AC$.  \emph{Therefore
$BD : DC :: BA : AC$.}

2.~If $BD : DC :: BA : AC$, the angle $BAC$ is bisected.

\textbf{Dem.}---Let the same construction be made.

Because $AD$ is parallel to $EC$, $BA: AE :: BD : DC$
\textsc{[ii.]}; but $BD : DC:: BA : AC$ (hyp.). Therefore
[V.~\textsc{xi}.] $BA : AE :: BA : AC$, and hence [V.~\textsc{ix}.] $AE$
is equal to $AC$; therefore the angle $AEC$ is equal to
$ACE$; but $AEC$ is equal to $BAD$ [I.~\textsc{xxix}.], and $ACE$
to $DAC$; hence $BAD$ is equal to $DAC$, \textit{and the line $AD$
bisects the angle $BAC$.}

\exhead{Exercises.}


\imgflow{142}{9}{f172}

\begin{footnotesize}
1.~If the line $AD$ bisect the external vertical angle $CAE$,
$BA : AC :: BD : DC$, and conversely.

\textbf{Dem.}---Cut off $AE = AC$. Join
$ED$. Then the triangles $ACD$,
$AED$ are evidently congruent;
therefore the angle $EDB$ is bisected;
hence [\textsc{iii}.] $BA : AE ::
BD : DE$; or $BA : AC :: BD : DC$.

2.~Exercise 1 has been proved
by quoting Proposition~\textsc{iii}. Prove
it independently, and prove \textsc{iii}.\ as an inference from it.

3.~The internal and the external bisectors of the vertical angle
of a triangle divide the base harmonically (\textit{see} Definition, p.~191).

4.~Any line intersecting the legs of any angle is cut harmonically
by the internal and external bisectors of the angle.


5.~Any line intersecting the legs of a right angle is cut harmonically
by any two lines through its vertex which make equal
angles with either of its sides.

6.~If the base of a triangle be given in magnitude and position,
and the ratio of the sides, the locus of the vertex is a circle which
divides the base harmonically in the ratio of the sides.

7.~If $a$, $b$, $c$ denote the sides of a triangle $ABC$, and $D$, $D'$ the
points where the internal and external bisectors of $A$ meet $BC$;
prove
\[
DD'=\frac{2abc}{b^2-c^2}.
\]

8.~In the same case, if $E$, $E'$, $F$, $F'$ be points similarly determined
on the sides $CA$, $AB$, respectively; prove

\begin{align*}
&\frac{1}{DD'}+\frac{1}{EE'}+\frac{1}{FF'}=0,\\
and &\frac{a^2}{DD'}+\frac{b^2}{EE'}+\frac{c^2}{FF'}=0.
\end{align*}
\par\end{footnotesize}

\mypropl{PROP\@.~IV\@.---Theorem.}{The sides about the equal angles of equiangular triangles
$(BAC,\ CDE)$ are proportional, and those which
are opposite to the equal angles are homologous.}


\imgflow{125}{8}{f173}

\textbf{Dem.}---Let the sides $BC$, $CE$, which are opposite to
the equal angles $A$ and $D$, be conceived to be placed so
as to form one continuous line,
the triangles being on the same
side, and so that the equal
angles $BCA$, $CED$ may not
have a common vertex.

Now, the sum of the angles
$ABC$, $BCA$ is less than two
right angles; but $BCA$ is equal
to $BED$ (hyp.). Therefore the sum of the angles $ABE$,
$BED$ is less than two right angles; hence [I., Axiom
\textsc{xii.}] the lines $AB$, $ED$ will meet if produced. Let them
meet in $F$. Again, because the angle $BCA$ is equal
to $BEF$, the line $CA$ [I.\ \textsc{xxviii.}] is parallel to $EF$. In

like manner, $BF$ is parallel to $CD$; therefore the figure
$ACDF$ is a parallelogram; hence $AC$ is equal to $DF$,
and $CD$ is equal to $AF$. Now, because $AC$ is parallel
to $FE$, $BA:AF :: BC:CE$ \textsc{[ii.]}; but $AF$ is equal to
$CD$, therefore $BA:CD :: BC:CE$; hence [V.~\textsc{xvi.}];
$AB:BC :: DC:CE$. Again, because $CD$ is parallel
to $BF$, $BC:CE :: FD:DE$; but $FD$ is equal to
$AC$, therefore $BC:CE :: AC:DE$; hence [V.~\textsc{xvi.}]
$BC:AC :: CE:DE$. Therefore we have proved that
$AB:BC :: DC:CE$, and that $BC:CA :: CE:ED$.
Hence (\emph{ex aequali}) $AB:AC :: DC:DE$. \emph{Therefore
the sides about the equal angles are proportional}.

\smallskip
\begin{footnotesize}
This Proposition may also be proved very simply by superposition.
Thus (\emph{see} fig., Prop.~\textsc{ii.}): let the two triangles be $ABC$,
$ADE$; let the second triangle $ADE$ be conceived to be placed on
$ABC$, so that its two sides $AD$, $AE$ may fall on the sides $AB$,
$AC$; then, since the angle $ADE$ is equal to $ABC$, the side $DE$
is parallel to $BC$. Hence \textsc{[ii.]} $AD:DB :: AE:EC$; hence
$AD:AB :: AE:AC$, and [V.~\textsc{xvi.}] $AD:AE :: AB:AC$.
\textit{Therefore the sides about the equal angles $BAC$, $DAE$ are
proportional, and similarly for the others.}

It can be proved by this Proposition that two lines which meet
at infinity are parallel. For, let $I$ denote the point at infinity
through which the two given lines pass, and draw any two parallels
intersecting them in the points $A$, $B$; $A'$, $B'$; then the
triangles $AIB$, $A'IB'$ are equiangular; therefore $AI:AB :: A'I:A'B'$;
but the first term of the proportion is equal to the third;
therefore [V.~\textsc{xiv.}] the second term $AB$ is equal to the fourth
$A'B'$, and, being parallel to it, the lines $AA'$, $BB'$ [I.~\textsc{xxxiii.}]
are parallel.
\par\end{footnotesize}

\exhead{Exercises.}

\begin{footnotesize}
1.~If two circles intercept equal chords $AB$, $A'B'$ on any secant,
the tangents $AT$, $A'T$ to the circles at the points of intersection
are to one another as the radii of the circles.

2.~If two circles intercept on any secant chords that have a
given ratio, the tangents to the circles at the points of intersection
have a given ratio, namely, the ratio compounded of the
direct ratio of the radii and the inverse ratio of the chords.

3.~Being given a circle and a line, prove that a point may be
found, such that the rectangle of the perpendiculars let fall on the
line from the points of intersection of the circle with any chord
through the point shall be given.


4.~$AB$ is the diameter of a semicircle $ADB$; $CD$ a perpendicular
to $AB$; draw through $A$ a chord $AF$ of the semicircle meeting $CD$
in $E$, so that the ratio $CE:EF$ may be given.
\par\end{footnotesize}

\mypropl{PROP\@.~V.--Theorem.}{If two triangles $(ABC,\ DEF)$ have their sides proportional
$(BA:AC :: ED:DF;\ AC:CB :: DF:FE)$ they
are equiangular, and those angles are equal which are
subtended by\label{bg} the homologous sides.}


\imgflow{150}{12}{f174}

\textbf{Dem.}---At the points $D$,
$E$ make the angles $EDG$,
$DEG$ equal to the angles
$A$, $B$ of the triangle $ABC$.
Then [I.~\textsc{xxxii.}] the triangles
$ABC$, $DEG$ are equiangular.
\begin{flalign*}
&\text{Therefore }&
  BA : AC &:: ED : DG \text{\ [\textsc{iv.}]}; \\
&\text{but }&
  BA : AC &:: ED : DF \text{\ (hyp.)}.
\end{flalign*}
Therefore $DG$ is equal to $DF$. In like manner it
may be proved that $EG$ is equal to $EF$. Hence the
triangles $EDF$, $EDG$ have the sides $ED$, $DF$ in one
equal to the sides $ED$, $DG$ in the other, and the base
$EF$ equal to the base $EG$. Hence [I.~\textsc{viii.}] they are
equiangular; but the triangle $DEG$ is equiangular to
$ABC$. \textit{Therefore the triangle $DEF$ is equiangular to
$ABC$.}\smallskip

\begin{footnotesize}
\textsf{Observation.}---In VI\@.~Def.~\textsc{i.}\ two conditions are laid down
as necessary for the similitude of rectilineal figures. 1.~The
equality of angles; 2.~The proportionality of sides. Now, from
Propositions \textsc{iv.}\ and \textsc{v.}, we see that if two triangles possess either
condition, they also possess the other. Triangles are unique in
this respect. In all other rectilineal figures one of the conditions
may exist without the other. Thus, two quadrilaterals may
have their sides proportional without having equal angles, or
\emph{vice vers{\^a}}.
\par\end{footnotesize}



\mypropl{PROP\@.~VI\@.---Theorem.}{If two triangles $(ABC,\ DEF)$ have one angle $(A)$ in
one equal to one angle $(D)$ in the other, and the sides about
these angles proportional $(BA:AC::ED:DF)$, the triangles
are equiangular, and have those angles equal which
are opposite to the homologous sides.}

\textbf{Dem}.---Make the same construction as in the last
Proposition; then the triangles $ABC$, $DEG$ are equiangular.
\begin{flalign*}
&\text{Therefore }&
  BA : AC &:: ED : DG \text{\ [\textsc{iv.}]}; &&\phantom{Therefore }\\
&\text{but }&
  BA : AC &:: ED : DF \text{\ (hyp.)}.  &&
\end{flalign*}
Therefore $DG$ is equal to $DF$. Again, because the
angle $EDG$ is equal to $BAC$ (const.), and $BAC$ equal
to $EDF$ (hyp.), the angle $EDG$ is equal to $EDF$; and
it has been proved that $DG$ is equal to $DF$, and $DE$
is common; hence the triangles $EDG$ and $EDF$ are
equiangular; but $EDG$ is equiangular to $BAC$. \textit{Therefore
$EDF$ is equiangular to $BAC$.}

\smallskip
\begin{footnotesize}
[It is easy to see, as in the case of Proposition \textsc{iv.}, that an
immediate proof of this Proposition can also be got from Proposition
\textsc{ii.}].
\par\end{footnotesize}

\smallskip
\emph{Cor.}~1.---If the ratio of two sides of a triangle be
given, and the angle between them, the triangle is
given in species.

\mypropl{PROP\@.~VII\@.---Theorem.}{If two triangles $(ABC,\ DEF)$ have one angle $(A)$ one
equal to one angle $(D)$ in the other, the sides about two
other angles $(B,\ E)$ proportional $(AB:BC::DE:EF)$,
and the remaining angles $(C,\ F)$ of the same species
\textrm{(i.~e.} either both acute or both not acute\textrm{)}, the triangles
are similar.}




\imgflow{148}{9}{f175}

\textbf{Dem.}---If the angles $B$ and $E$ are not equal, one
must be greater than the
other. Suppose $ABC$ to be
the greater, and that the
part $ABG$ is equal to $DEF$,
then the triangles $ABG$,
$DEF$ have two angles in
one equal to two angles
in the other, and are
[I.~\textsc{xxxii.}] equiangular.
\begin{flalign*}
&\text{Therefore }&
  AB : BG &:: DE : EF \text{\ [\textsc{iv.}]}; \\
&\text{but }&
  AB : BC &:: DE : EF \text{\ (hyp.)}.
\end{flalign*}
Therefore $BG$ is equal to $BC$. Hence the angles
$BCG$, $BGC$ must be each acute [I.~\textsc{xvii.}]; therefore
$AGB$ must be obtuse; hence $DFE$, which is equal
to it, is obtuse; and it has been proved that $ACB$ is
acute; therefore the angles $ACB$, $DFE$ are of different
species; but (hyp.) they are of the same species, which
is absurd. Hence the angles $B$ and $E$ are not unequal,
that is, they are equal. \textit{Therefore the triangles are equiangular.}

\emph{Cor.}~1.---If two triangles $ABC$, $DEF$ have two sides
in one proportional to two sides in the other, $AB:BC
::DE:EF$, and the angles $A$, $D$ opposite one pair of
homologous sides equal, the angles $C$, $F$ opposite the
other are either equal or supplemental. This Proposition
is nearly identical with \textsc{vii.}

\emph{Cor.}~2.---If either of the angles $C$, $F$ be right, the
other must be right.

\mypropl{PROP\@.~VIII\@.---Theorem.}{The triangles $(ACD,\ BCD)$ into which a right-angled
triangle $(ACB)$ is divided, by the perpendicular $(CD)$ from
the right angle $(C)$ on the hypotenuse, are similar to the
whole and to one another.}



\textbf{Dem.}---Since the two triangles $ADC$, $ACB$ have the
angle $A$ common, and the angles
$ADC$, $ACB$ equal, each being
right, they are [I.~\textsc{xxxii.}] equiangular;
hence [\textsc{iv}.] they are
similar. In like manner it may
be proved that $BDC$ is similar
to $ABC$. Hence $ADC$, $CDB$ \textit{are
each similar to $ACD$, and therefore they are similar to one
another.}


\imgflow{110}{6}{f176}

\emph{Cor.}~1.---The perpendicular $CD$ is a mean proportional
between the segments $AD$, $DB$ of the hypotenuse.

For, since the triangles $ADC$, $CDB$ are equiangular,
we have $AD:DC::DC:DB$; hence $DC$ is a mean
proportional between $AD$, $DB$ (Def.~\textsc{iii.}).

\emph{Cor.}~2.---$BC$ is a mean proportional between $AB$,
$BD$; and $AC$ between $AB$, $AD$.

\emph{Cor.}~3.---The segments $AD$, $DB$ are in the duplicate
of $AC:CB$, or in other words, $AD:DB::AC^2:CB^2$,

\emph{Cor.}~4.---$BA:AD$ in the duplicate ratios of $BA:AC$;
and $AB:BD$ in the duplicate ratio of $AB:BC$.

\myprop{PROP\@.~IX\@.---Problem.}{From a given right line $(AB)$ to cut off any part required
\textrm{(i.e.} to cut off any required submultiple$)$}.

\textbf{Sol.}---Let it be required, for instance, to cut off the
fourth part. Draw $AF$, making any angle with $AB$,
and in $AF$ take any point $C$, and cut off (I.~\textsc{iii.}) the
parts $CD$, $DE$, $EF$ each equal to $AC$. Join $BF$, and
draw $CG$ parallel to $BF$. $AG$ is the fourth part of
$AB$.




\imgflow{115}{6}{f177}

\textbf{Dem.}---Since $CG$ is parallel to the side $BF$ of the
triangle $ABF$, $AC : AF :: AG
: AB$ [\textsc{ii.}]; but $AC$ is the fourth
part of $AF$ (const.). Hence $AG$
is the fourth part of $AB$ [V., \textsc{d.}].

\textit{In the same manner, any other required
submultiple may be cut off.}

Proposition \textsc{x.}, Book I., is a particular case of this
Proposition.

\myprop{PROP\@.~X.---Problem.}{To divide a given undivided line $(AB)$ similarly to a given
divided line $(CD)$.}


\imgflow{140}{8}{f178}

\textbf{Sol.}---Draw $AG$, making any angle with $AB$, and
cut off the parts $AH$, $HI$, $IG$
respectively equal to the
parts $CE$, $EF$, $FD$ of the
given divided line $CD$. Join
$BG$, and draw $HK$, $IL$, each
parallel to $BG$. $AB$ will be
divided similarly to $CD$.

\textbf{Dem.}---Through $H$ draw
$HN$ parallel to $AB$, cutting $IL$ in $M$. Now in the
triangle $ALI$, $HK$ is parallel to $IL$. Hence [\textsc{ii.}]
$AK : KL :: AH : HI$, that is $:: CE : EF$ (const.).
Again, in the triangle $HNG$, $MI$ is parallel to $NG$.
Therefore [\textsc{ii.}] $HM : MN :: HI : IG$; but [I.~\textsc{xxxiv.}]
$HM$ is equal to $KL$, $MN$ is equal to $LB$, $HI$ is equal
to $EF$, and $IG$ is equal to $FD$ (const.). Therefore
$KL : LB :: EF : FD$. \textit{Hence the line $AB$ is divided
similarly to the line $CD$.}

\exhead{Exercises.}

\begin{footnotesize}
1.~To divide a given line $AB$ \emph{internally} or \emph{externally} in the
ratio of two given lines, $m$, $n$.



\imgcent{250}{f179}

\textbf{Sol.}---Through $A$ and $B$ draw any two parallels $AC$ and $BD$ in
opposite directions. Cut off $AC = m$, and $BD = n$, and join
$CD$; the joining line will divide $AB$ internally at $E$ in the ratio
of $m : n$.

2.~If $BD'$ be drawn in the same direction with $AC$, as denoted
by the dotted line, then $CD'$ will cut $AB$ externally at $E'$ in the
ratio of $m : n$.

\emph{Cor.}---The two points $E$, $E'$ divide $AB$ harmonically.

This problem is manifestly equivalent to the following:---Given
the sum or difference of two lines and their ratio, to find
the lines.

3.~Any line $AE'$, through the middle point $B$ of the base $DD'$
of a triangle $DCD'$, is cut harmonically by the sides of the
triangle and a parallel to the base through the vertex.

4.~Given the sum of the squares on two lines and their ratio;
find the lines.

5.~Given the difference of the squares on two lines and their
ratio; find the lines.

6.~Given the base and ratio of the sides of a triangle; construct
it when any of the following data is given:---1, the area;
2, the difference on the squares of the sides; 3, the sum of the
squares on the sides; 4, the vertical angle; 5, the difference of
the base angles.
\par\end{footnotesize}

\myprop{PROP\@.~XI\@.---Problem.}{To find a third proportional\index{Proportionals!third} to two given lines $(X,\ Y)$.}

\textbf{Sol.}---Draw any two lines $AC$, $AE$ making an angle.
Cut off $AB$ equal $X$, $BC$ equal $Y$, and $AD$ equal $Y$.
Join $BD$, and draw $CE$ parallel to $BD$, then $DE$ is the
third proportional required.




\imgflow{175}{7}{f180}

\textbf{Dem.}---In the triangle $CAE$, $BD$ is parallel to
$CE$; therefore $AB : BC
:: AD : DE$ [\textsc{ii.}]; but
$AB$ is equal to $X$, and
$BC$, $AD$ each equal to
$Y$. Therefore $X : Y
:: Y : DE$. \textit{Hence $DE$
is a third proportional to $X$ and $Y$.}

\begin{footnotesize}
Another solution can be inferred from Proposition~\textsc{viii.} For if
$AD$, $DC$ in that Proposition be respectively equal to $X$ and $Y$,
then $DB$ will be the third proportional. Or again, if in the diagram,
Proposition~\textsc{viii.}, $AD = X$, and $AC = Y$, $AB$ will be the
third proportional. Hence may be inferred a method of continuing
the proportion to any number of terms.
\par\end{footnotesize}

\exhead{Exercises.}

\begin{footnotesize}

\imgflow{180}{7}{f181}

1.~If $AO\Omega$ be a triangle, having the side $A\Omega$ greater than $AO$;
then if we cut off $AB
= AO$, draw $BB'$ parallel
to $AO$, cut off $BC = BB'$,
\&c., the series of lines
$AB$, $BC$, $CD$, \&c., are
in continual proportion.

2.~$AB - BC : AB ::
AB : A\Omega$.  This is evident
by drawing through $B'$ a parallel to $A\Omega$.
\par\end{footnotesize}

\myprop{PROP\@.~XII\@.---Problem.}{To find a fourth proportional\index{Proportionals!fourth} to three given lines
$(X,\ Y,\ Z)$.}


\imgflow{180}{8}{f182}

\textbf{Sol.}---Draw any two lines $AC$, $AE$, making an
angle; then cut off
$AB$ equal $X$, $BC$
equal $Y$, $AD$ equal
$Z$.  Join $BD$, and
draw $CE$ parallel
to $BD$.  \emph{$DE$ will
be the fourth proportional
required.}



\textbf{Dem.}---Since $BD$ is parallel to $CE$, we have [\textsc{ii.}]
$AB:BC::AD:DE$; therefore $X:Y::Z:DE$.
\textit{Hence $DE$ is a fourth proportional to $X$, $Y$, $Z$.}


\imgflow{110}{7}{f183}

\emph{Or thus:} Take two lines $AD$,
$BC$ intersecting in $O$. Make $OA
=X$, $OB=Y$, $OC=Z$, and describe
a circle through the points
$A$, $B$, $C$ [IV\@.~\textsc{v.}] cutting $AD$ in $D$.
$OD$ \emph{will be the fourth proportional
required.}

The demonstration is evident
from the similarity of the triangles $AOB$ and $COD$.

\myprop{PROP\@.~XIII\@.---Problem.}{To find a mean proportional\index{Proportional, mean} between two given lines.
$(X,\ Y)$.}


\imgflow{210}{7}{f184}

\textbf{Sol.}---Take on any line $AC$ parts $AB$, $BC$ respectively
equal to $X$, $Y$.
On $AC$ describe a
semicircle $ADC$.
Erect $BD$ at right
angles to $AC$,
meeting the semicircle
in $D$.
$BD$ will be the mean proportional required.

\textbf{Dem.}---Join $AD$, $DC$. Since $ADC$ is a semicircle,
the angle $ADC$ is right [III\@.~\textsc{xxxi.}]. Hence, since
$ADC$ is a right-angled triangle, and $BD$ a perpendicular
from the right angle on the hypotenuse, $BD$ is a
mean proportional [\textsc{viii.}~\emph{Cor.}~1] between $AB$, $BC$; that
is, $BD$ \textit{is a mean proportional between $X$ and $Y$.}


\exhead{Exercises.}

\begin{footnotesize}
1.~Another solution may be inferred from Proposition~\textsc{viii.},
\emph{Cor.}~2.

2.~If through any point within a circle the chord be drawn,
which is bisected in that point, its half is a mean proportional
between the segments of any other chord passing through the
same point.


3.~The tangent to a circle from any external point is a mean
proportional between the segments of any secant passing through
the same point.

4.~If through the middle point $C$ of any arc of a circle any
secant be drawn cutting the chord of the arc in $D$, and the circle
again in $E$, the chord of half the arc is a mean proportional between
$CD$ and $CE$.

5.~If a circle be described touching another circle internally
and two parallel chords, the perpendicular from the centre of the
former on the diameter of the latter, which bisects the chords,
is a mean proportional between the two extremes of the three
segments into which the diameter is divided by the chords.

6.~If a circle be described touching a semicircle and its diameter,
the diameter of the circle is a harmonic mean between the segments
into which the diameter of the semicircle is divided at the
point of contact.

7.~State and prove the Proposition corresponding to Ex.~5, for
external contact of the circles.
\par\end{footnotesize}


\mypropflow{PROP\@.~XIV\@.---Theorem.}{$1$. Equiangular parallelograms $(AB,$\ $CD)$ which are
equal in area have the sides about the equal angles reciprocally
proportional---$AC:CE::GC:CB$.\\\indent $2$. Equiangular parallelograms which have the sides
about the equal angles reciprocally proportional are equal
in area.}{138}{9}{f185}

\textbf{Dem.}---Let $AC$, $CE$ be so placed as to form one right
line, and that the equal angles
$ACB$, $ECG$ may be vertically
opposite. Now, since the
angle $ACB$ is equal to $ECG$,
to each add $BCE$, and we have
the sum of the angles $ACB$,
$BCE$ equal to the sum of the
angles $ECG$, $BCE$; but the
sum of $ACB$, $BCE$ is [I.~\textsc{xiii.}]
two right angles. Therefore the sum of $ECG$, $BCE$ is

two right angles. Hence [I.~\textsc{xiv.}] $BC$, $CG$ form one
right line. Complete the parallelogram $BE$.

Again, since the parallelograms $AB$, $CD$ are equal
(hyp.),
\begin{flalign*}
&&  AB : CF &:: CD : CF \text{\ [V.~\textsc{vii.}]};  &&\\
&\text{but }&
    AB : CF &:: AC : CE \text{\ [\textsc{i.}]};  &&\\
&\text{and }&
  CD : CF &:: GC : CB \text{\ [\textsc{i.}]};  &&\\
&\text{therefore }&
  AC : CE &:: GC:  CB;  &&\phantom{therefore }
\end{flalign*}
that is, \emph{the sides about the equal angles are reciprocally
proportional.}

2.~Let $AC : CE :: GC : CB$, to prove the parallelograms
$AB$, $CD$ are equal.

\textbf{Dem.}---Let the same construction be made, we have
\begin{flalign*}
&&  AB : CF ::{}& AC : CE \text{\ [\textsc{i.}]};  &&\\
&\text{and }&
  CD : CF ::{}& GC : CB \text{\ [\textsc{i.}]};  &&\\

&\text{but }&
  AC : CE ::{}& GC : CB \text{\ (hyp.)}.  &&\\
&\text{Therefore }&
  AB : CF ::{}& CD : CF.  &&\phantom{therefore }\\
&\text{Hence }&
  AB &= CD \quad \text{\ [V.~\textsc{ix.}]};  &&
\end{flalign*}
that is, \emph{the parallelograms are equal.}

\smallskip
\begin{footnotesize}
\emph{Or thus:} Join $HE$, $BE$, $HD$, $BD$. The $\pgram\ HC =$ twice the
$\triangle\ HBE$, and the $\pgram\ CD =$ twice the $\triangle\ BDE$. Therefore the
$\triangle\ HBE = BDE$, and [I.~\textsc{xxxix.}] $HD$ is parallel to $BE$. Hence
\[
  HB : BF :: DE : EF; \text{\ that is,}\quad AC : CE :: GC : CB.
\]

2.~May be proved by reversing this demonstration.

\smallskip
Another demonstration of this Proposition may be got by producing
the lines $HA$ and $DG$ to meet in $I$. Then [I.~\textsc{xliii.}] the
points $I$, $C$, $F$ are collinear, and the Proposition is evident.
\par\end{footnotesize}




\mypropflow{PROP\@.~XV\@.---Theorem.}{$1$. Two triangles equal in area $(ACB,\ DCE)$, which
have one angle $(C)$ in one equal to one angle $(C)$ in the
other, have the sides about these angles reciprocally proportional.\\
\indent $2$. Two triangles, which have one angle in one equal to
one angle in the other, and the sides about these angles
reciprocally proportional, are equal in area.}{100}{8}{f186}


\textbf{Dem.}---1. Let the equal angles be so placed as to be
vertically opposite, and that $AC$,
$CD$ may form one right line; then
it may be demonstrated, as in the
last Proposition, that $BC$, $CE$ form
one right line. Join $BD$.

Now since the triangles $ACB$,
$DCE$ are equal,
\begin{flalign*}
&&  ACB &: BCD :: DCE : BCD \text{\ [V.~\textsc{vii.}]};  &&\\
&\text{but }&
    ACB &: BCD :: AC : CD \text{\ [\textsc{i.}]},  &&\\
&\text{and }&
    DCE &: BCD :: EC : CB \text{\ [\textsc{i.}].}  &&\\
&\text{Therefore }&
     AC &: CD :: EC : CB;  &&\phantom{Therefore }
\end{flalign*}
that is, \emph{the sides about the equal angles are reciprocally
proportional.}

2.~If $AC : CD :: EC : CB$, to prove the triangle $ACB$
equal to $DCE$.

\textbf{Dem.}---Let the same construction be made, then we
have
\begin{flalign*}
&& AC : CD &:: \text{triangle } ACB : BCD \text{\ [\textsc{i.}]},  &&\\
&\text{and }&
   EC : CB &:: \text{triangle } DCE : BCD \text{\ [\textsc{i.}]};  &&\\
&\text{but }&
   AC : CD &:: EC : CB \text{\ (hyp.)}.  &&\phantom{but }
\end{flalign*}



Therefore the triangle
\[
ACB : BCD :: DCE : BCD.
\]
Hence the triangle $ACB = DCE$ [V.~\textsc{ix.}]---that is, \emph{the
triangles are equal.}

\smallskip
\begin{footnotesize}
This Proposition might have been appended as a \emph{Cor.}\ to the
preceding, since the triangles are the halves of equiangular parallelograms,
or it may be proved by joining $AE$, and showing that
it is parallel to $BD$.
\par\end{footnotesize}

\mypropl{PROP\@.~XVI\@.---Theorem.}{$1$. If four right lines $(AB,\ CD,\ E,\ F)$ be proportional,
the rectangle $(AB\,.\,F)$ contained by the extremes is equal
to the rectangle $(CD\,.\,E)$ contained by the means.\\
\indent $2$. If the rectangle contained by the extremes of four
right lines be equal to the rectangle contained by the
means, the four lines are proportional.}


\imgflow{175}{9}{f187}

\textbf{Dem.}---1. Erect $AH$, $CI$ at right angles to $AB$ and
$CD$, and equal to $F$
and $E$ respectively,
and complete the rectangles.
Then because
$AB : CD :: E : F$
(hyp.), and that $E$ is
equal to $CI$, and $F$ to
$AH$ (const.), we have
$AB : CD:: CI : AH$.
Hence the parallelograms $AG$, $CK$ are equiangular, and
have the sides about their equal angles reciprocally
proportional. Therefore they are [\textsc{xiv.}] equal; but
since $AH$ is equal to $F$, $AG$ is equal to the rectangle
$AB\,.\,F$. In like manner, $CK$ is equal to the rectangle
$CD\,.\,E$. Hence $AB\,.\,F = CD\,.\,E$; that is, \emph{the rectangle
contained by the extremes is equal to the rectangle contained
by the means.}

2.~If $AB\,.\,F = CD\,.\,E$, to prove $AB :CD :: E : F$.

The same construction being made, because $AB\,.\,F
= CD\,.\,E$, and that $F$ is equal to $AH$, and $E$ to $CI$, we

have the parallelogram $AG = CK$; and since these parallelograms
are equiangular, the sides about their equal
angles are reciprocally proportional. Therefore
\[
AB : CD :: CI : AH;\ \text{that is,}\ AB : CD :: E : F.
\]


\imgflow{100}{7}{f188}

\begin{footnotesize}
\emph{Or thus:} Place the four lines in a concurrent
position so that the extremes may
form one continuous line, and the means
another. Let the four lines so placed be
$AO$, $BO$, $OD$, $OC$. Join $AB$, $CD$. Then
because $AO : OB :: OD : OC$, and the
angle $AOB = DOC$, the triangles $AOB$,
$COD$ are equiangular. Hence the four
points $A$, $B$, $C$, $D$ are concyclic. Therefore
[III\@.~\textsc{xxxv.}] $AO\,.\,OC = BO\,.\,OD$.
\par\end{footnotesize}

\mypropl{PROP\@.~XVII\@.---Theorem}{$1$. 
If three right lines $(A,\ B,\ C)$ be proportional, the
rectangle $(A\,.\,C)$ contained by the extremes is equal to the
square $(B^{2})$ of the mean.\\
\indent $2$. If the rectangle contained by the extremes of three
right lines be equal to the square of the mean, the three
lines are proportional.}


\imgflow{118}{6}{f189}

\textbf{Dem.}---1. Assume a line $D = B$; then because $A : B
:: B : C$, we have $A : B :: D : C$.
Therefore [\textsc{xvi.}]\label{XVI} $AC = BD$; but
$BD = B^{2}$. Therefore $AC = B^{2}$;
that is, \emph{the rectangle contained
by the extremes is equal to the
square of the mean.}

2.~The same construction being made, since $AC = B^{2}$,
we have $A\,.\,C = B\,.\,D$; therefore $A : B :: D : C$; but
$D = B$. Hence $A : B :: B : C$; that is, \emph{the three lines
are proportionals.}

\smallskip
\begin{footnotesize}
This Proposition may be inferred as a \emph{Cor.}\ to the last, which
is one of the fundamental Propositions in Mathematics.
\par\end{footnotesize}



\exhead{Exercises.}


\imgflow{100}{12}{f190}

\begin{footnotesize}
1.~If a line $CD$ bisect the vertical angle $C$ of any triangle $ACB$,
its square added to the rectangle $AD\,.\,DB$
contained by the segments of the base is
equal to the rectangle contained by the
sides.

\textbf{Dem.}---Describe a circle about the triangle,
and produce $CD$ to meet it in $E$;
then it is easy to see that the triangles
$ACD$, $ECB$ are equiangular. Hence [\textsc{iv.}]
$AC:CD::CE:CB$; therefore $AC\,.\,CB
= CE\,.\,CD = CD^2 + CD\,.\,DE = CD^2 +
AD\,.\,DB$ [III\@.~\textsc{xxxv.}].

2.~If the line $CD'$ bisect the external vertical angle of any triangle
$ACB$, its square subtracted from the rectangle $AD'\,.\,D'B$
is equal to $AC\,.\,CB$.

3.~The rectangle contained by the diameter of the circumscribed
circle, and the radius of the inscribed circle of any triangle,
is equal to the rectangle contained by the segments of any
chord of the circumscribed circle passing through the centre of
the inscribed.

\textbf{Dem.}---Let $O$ be the centre of the inscribed circle. Join $OB$
(\emph{see} foregoing fig.); let fall the perpendicular $OG$, draw the diameter
$EF$ of the circumscribed circle. Now the angle $ABE = ECB$
[III\@.~\textsc{xxvii.}], and $ABO = OBC$; therefore $EBO =$ sum of $OCB$,
$OBC = EOB$. Hence $EB = EO$. Again, the triangles $EBF$, $OGC$
are equiangular, because $EFB$, $ECB$ are equal, and $EBF$, $OGC$
are each right. Therefore, $EF:EB::OC:OG$; therefore
$EF\,.\,OG = EB\,.\,OC = EO\,.\,OC$.

4.~Ex.~3 may be extended to each of the escribed circles of the
triangle $ACB$.


\imgcent{100}{f191}

5.~The rectangle contained by two sides of a triangle is equal
to the rectangle contained by the perpendicular
and the diameter of the circumscribed
circle. For, let $CE$ be the diameter.
Join $AE$. Then the triangles $ACE$, $DCB$
are equiangular; hence $AC:CE::CD:CB$;
therefore $AC\,.\,CB = CD\,.\,CE$.


\imgflow{115}{13}{f192}

6.~If a circle passing through one of
the angles $A$ of a parallelogram $ABCD$
intersect the two sides $AB$, $AD$ again in
the points $E$, $G$ and the diagonal $AC$ again in $F$; then $AB\,.\,AE
+ AD\,.\,AG = AC\,.\,AF$.



\textbf{Dem.}---Join $EF$, $FG$, and make the angle $ABH = AFE$. Then
the triangles $ABH$, $AFE$ are equiangular.
Therefore $AB : AH :: AF : AE$.
Hence $AB\,.\,AE = AF\,.\,AH$. Again, it
is easy to see that the triangles $BCH$,
$FAG$ are equiangular; therefore $BC : CH
:: AF : AG$; hence $BC\,.\,AG = AF\,.\,CH$,
or $AD\,.\,AG = AF\,.\,CH$; but we have
proved $AB\,.\,AE = AF\,.\,AH$. Hence
$AD\,.\,AG + AB\,.\,AE = AF\,.\,AC$.

7.~If $DE$, $DF$ be parallels to the sides
of a triangle $ABC$ from any point $D$ in
the base, then $AB\,.\,AE + AC\,.\,AF =
AD^{2} + BD\,.\,DC$. This is an easy deduction from 6.


\imgflow{120}{12}{f193}

8.~If through a point $O$ within a
triangle $ABC$ parallels $EF$, $GH$, $IK$
to the sides be drawn, the sum of the
rectangles of their segments is equal
to the rectangle contained by the
segments of any chord of the circumscribing
circle passing through $O$.\\[-\baselineskip]
\begin{flalign*}
&&&\text{\hspace{-2.5em}\textbf{Dem.}---}AO\,.\,AL = AB\,.\,AK + AC\,.\,AE.  && \tag{6} \\
&\text{But }&   & AO^{2} = AG\,.\,AK + AH\,.\,AE - GO\,.\,OH.  && \tag{7} \\
&\text{Hence }& & AO\,.\,OL  = BG\,.\,AK + CH\,.\,AE + GO\,.\,OH,  && \\
&\text{or }&    & AO\,.\,OL  = EO\,.\,OF + IO\,.\,OK + GO\,.\,OH.  &&
\end{flalign*}

9.~The rectangle contained by the side of an inscribed square
standing on the base of a triangle, and the sum of the base and
altitude, is equal to twice the area of the triangle.

10.~The rectangle contained by the side of an escribed square
standing on the base of a triangle, and the difference between the
base and altitude, is equal to twice the area of the triangle.

11.~If from any point $P$ in the circumference of a circle a perpendicular
be drawn to any chord, its square is equal to the
rectangle contained by the perpendiculars from the extremities of
the chord on the tangent at $P$.

12.~If $O$ be the point of intersection of the diagonals of a cyclic
quadrilateral $ABCD$, the four rectangles $AB\,.\,BC$, $BD\,.\,CD$,
$CD\,.\,DA$, $DA\,.\,AB$, are proportional to the four lines $BO$, $CO$,
$DO$, $AO$.



\imgflow{108}{11}{f194}

13.~The sum of the rectangles of the opposite sides of a cyclic
quadrilateral $ABCD$ is equal to the rectangle
contained by its diagonals.

\textbf{Dem.}---Make the angle $DAO = CAB$;
then the triangles $DAO$, $CAB$ are equiangular;
therefore $AD : DO :: AC : CB$;
therefore $AD\,.\,BC = AC\,.\,DO$. Again,
the triangles $DAC$, $OAB$ are equiangular,
and $CD : AC :: BO : AB$; therefore
$AC\,.\,CD = AC\,.\,BO$. Hence $AD\,.\,BC
+ AB\,.\,CD = AC\,.\,BD$.\footnote
 {This Proposition is known as Ptolemy's theorem\index{Ptolemy's theorem}.}

14.~If the quadrilateral $ABCD$ is not cyclic, prove that the
three rectangles $AB\,.\,CD$, $BC\,.\,AD$, $AC\,.\,BD$ are proportional to
the three sides of a triangle which has an angle equal to the sum
of a pair of opposite angles of the quadrilateral.

15.~Prove by using Theorem 11 that if perpendiculars be let
fall on the sides and diagonals of a cyclic quadrilateral, from any
point in the circumference of the circumscribed circle, the rectangle
contained by the perpendiculars on the diagonals is equal
to the rectangle contained by the perpendiculars on either pair of
opposite sides.

16.~If $AB$ be the diameter of a semicircle, and $PA$, $PB$ chords
from any point $P$ in the circumference, and if a perpendicular to
AB from any point $C$ meet $PA$, $PB$ in $D$ and $E$, and the semicircle
in $F$, $CF$ is a mean proportional between $CD$ and $CE$.
\par\end{footnotesize}

\mypropl{PROP\@.~XVIII\@.---Problem.}{On a given right line $(AB)$ to construct a rectilineal
figure similar to a given one $(CDEFG)$, and similarly
placed as regards any side $(CD)$ of the latter.}

\textsc{Def.}---Similar figures are said to be \emph{similarly described
upon given right lines}, when these lines are homologous
sides of the figures.


\imgcent{270}{f195}



\textbf{Sol.}---Join $CE$, $CF$, and construct a triangle $ABH$
on $AB$ equiangular to $CDE$, and similarly placed as
regards $CD$; that is, make, the angle $ABH$ equal to
$CDE$, and $BAH$ equal to $DCE$. In like manner construct
the triangle $HAI$ equiangular to $ECF$, and similarly
placed, and lastly, the triangle $IAJ$ equiangular
and similarly placed with $FCG$. \emph{Then $ABHIJ$ is the
figure required.}

\textbf{Dem.}---From the construction it is\label{isis} 
evident that the
figures are equiangular, and it is only required to prove
that the sides about the equal angles are proportional.
Now because the triangle $ABH$ is equiangular to $CDE$,
$AB : BH :: CD : DE$ [\textsc{iv.}]; hence the sides about the
equal angles $B$ and $D$ are proportional. Again, from
the same triangles we have $BH : HA :: DE : EC$, and
from the triangles $IHA$, $FEC$; $HA : HI :: EC : EF$;
therefore (\emph{ex {\ae}quali}) $BH : HI :: DE : EF$; that is, the
sides about the equal angles $BHI$, $DEF$ are proportional,
and so in like manner are the sides about the
other equal angles. \emph{Hence} (Def.~\textsc{i.}) \emph{the figures are
similar.}

\smallskip
\begin{footnotesize}
\textsf{Observation.}---In the foregoing construction, the line $AB$ is
homologous to $CD$, and it is evident that we may take $AB$ to
be homologous to any other side of the given figure $CDEFG$.
Again, in each case, it the figure $ABHIJ$ be turned round the line
$AB$ until it falls on the other side, it will still be similar to the
figure $CDEFG$. \emph{Hence on a given line $AB$ there can be constructed
two figures each similar to a given figure $CDEFG$, and having the
given line $AB$ homologous to any given side $CD$ of the given figure.}

The first of the figures thus constructed is said to be \emph{directly}
similar\index{Figures!directly similar}, and the second \emph{inversely} similar\index{Figures!inversely similar} to the given figure.
These technical terms are due to Hamilton\index{Hamilton}: \emph{see} ``Elements of
Quaternions,'' page~112.
\par\end{footnotesize}

\emph{Cor.}~1.---Twice as many polygons may be constructed
on $AB$ similar to a given polygon $CDEFG$ as
that figure has sides.

\emph{Cor.}~2.---If the figure $ABHIJ$ be applied to $CDEFG$
so that the point $A$ will coincide with $C$, and that the
line $AB$ may be placed along $CD$, then the points $H$, $I$, $J$
will be respectively on the lines $CE$, $CF$, $CG$; also the

sides $BH$, $HI$, $IJ$ of the one polygon will be respectively
parallel to their homologous sides $DE$, $EF$, $FG$
of the other.

\emph{Cor.}~3.---If lines drawn from any point $O$ in the
plane of a figure to all its angular points be divided in
the same ratio, the lines joining the points of division
will form a new figure similar to, and having every
side parallel to, the homologous side of the original.

\myprop{PROP\@.~XIX\@.---Theorem.}{Similar triangles $(ABC,\ DEF)$ have their areas to
one another in the duplicate ratio of their homologous
sides.}


\imgflow{178}{9}{f196}

\textbf{Dem.}---Take $BG$ a third proportional to $BC$, $EF$
[\textsc{xi.}]. Join $AG$. Then
because the triangles
$ABC$, $DEF$ are similar,
$AB : BC :: DE : EF$;
hence (alternately) $AB
: DE :: BC : EF$; but
$BC : EF :: EF : BG$
(const.); therefore [V.~\textsc{xi.}]
$AB : DE :: EF :
BG$; hence the sides of the triangles $ABG$, $DEF$ about
the equal angles $B$, $E$ are reciprocally proportional;
therefore the triangles are equal. Again, since the
lines $BC$, $EF$, $BG$ are continual proportionals, $BC : BG$
in the duplicate ratio of $BC : EF$ [V.~Def.~\textsc{x.}]; but
$BC : BG ::$ triangle $ABC : ABG$. Therefore $ABC : ABG$
in the duplicate ratio of $BC : EF$; but it has been
proved that the triangle $ABG$ is equal to $DEF$. \emph{Therefore
the triangle $ABC$ is to the triangle $DEF$ in the
duplicate ratio of $BC : EF$.}

This is the first Proposition in Euclid in which the
technical term ``duplicate ratio'' occurs. My experience
with pupils is, that they find it very difficult to understand

either Euclid's proof or his definition. On this
account I submit the following alternative proof, which,
however, makes use of a new definition of the duplicate
ratio of two lines, viz.\ the ratio of the squares (\emph{see}
Annotations on V.~Def.~\textsc{x.}) described on these lines.

On $AB$ and $DE$ describe squares, and through $C$ and
$F$ draw lines parallel to $AB$ and $DE$, and complete
the rectangles $AI$, $DN$.

Now, the triangles $JAC$, $ODF$ are evidently equiangular.


\imgcent{210}{f197}

\begin{flalign*}
&\text{\indent Hence }&
  JA\, &: AC :: OD : DF\ \text{[\textsc{iv.}]};  &&\phantom{therefore }\\
&\text{but }&
  AC   &: AB :: DF : DE\ \text{[\textsc{iv.}]}.  &&\\
&\text{Hence }&
  JA\, &: AB :: OD : DE\ \text{(\emph{ex {\ae}quali})};  &&\\
&\text{but }&
  AB   &= AG,\ \text{and}\ DE = DL;  &&\\
&\text{therefore }&
  JA\, &: AG :: OD : DL.  &&\\
&\text{Again, }&
  JA\, &: AG :: \pgram AI :\ \text{square}\ AH\ \text{[\textsc{i.}]},  &&\\
&\text{and }&
  OD   &: DL :: \pgram DN :\ \text{square}\ DM\ \text{[\textsc{i.}]}.  &&\\
&\text{Hence }&
  AI\; &: AH :: DN : DM;  &&\\
&\text{therefore }&
  AI\; &: DN :: AH : DM\ \text{[V.~\textsc{xvi.}]};  &&\\
&\text{\emph{hence }}& \multispan{2}{$
  \triangle ABC : \triangle DEF :: AB^{2} : DE^{2}$.\hfill}  &&
\end{flalign*}



\exhead{Exercises.}

\begin{footnotesize}
1.~If one of two similar triangles has its sides 50 per cent.\
longer than the homologous sides of the other; what is the ratio
of their areas?

2.~When the inscribed and circumscribed regular polygons of
any common number of sides to a circle have more than four
sides, the difference of their areas is less than the square of the
side of the inscribed polygon.
\par\end{footnotesize}

\mypropl{PROP\@.~XX\@.---Theorem.}{Similar polygons may be divided $(1)$ into the same
number of similar triangles; $(2)$ the corresponding triangles
have the same ratio to one another which the
polygons have; $(3)$ the polygons are to each other in the
duplicate ratio of their homologous sides.}

\textbf{Dem.}---Let $ABHIJ$, $CDEFG$ be the polygons, and
let the sides $AB$, $CD$ be homologous. Join $AH$, $AI$,
$CE$, $CF$.


\imgcent{263}{f198}

1.~The triangles into which the polygons are divided
are similar. For, since the polygons are similar, they
are equiangular, and have the sides about their equal
angles proportional [Def.~\textsc{i.}]; hence the angle $B$ is
equal to $D$, and $AB : BH :: CD : DE$; therefore [\textsc{vi.}]
the triangle $ABH$ is equiangular to $CDE$; hence the
angle $BHA$ is equal to $DEC$; but $BHI$ is equal to $DEF$
(hyp.); therefore the angle $AHI$ is equal to $CEF$.

Again, because the polygons are similar, $IH : HB ::
FE : ED$; and since the triangles $ABH$, $CDE$ are similar,
$HB : HA :: ED : EC$; hence (\emph{ex aequali}) $IH : HA
:: FE : EC$, and the angle $IHA$ has been proved to be
equal to the angle $FEC$; \emph{therefore the triangles $IHA$,
$FEC$ are equiangular}. In the same manner it can be
proved that \emph{the remaining triangles are equiangular}.

2.~Since the triangle $ABH$ is similar to $CDE$, we
have [\textsc{xix.}].
\[
  ABH : CDE \text{\ in the duplicate ratio of } AH : CE.
\]
In like manner,
\begin{flalign*}
  \multispan{6}{\hfill$
  AHI : CEF \text{\ in the duplicate ratio of } AH : CE;
  $\hfill}  \\
&\text{hence }&
  ABH : CDE &= AHI : CEF\ \text{[V.~\textsc{xi.}]}.  &&\\
&\text{Similarly, }&
  AHI : CEF &= AIJ : CFG.  &&\phantom{Similarly, }
\end{flalign*}
In these equal ratios, the triangles $ABH$, $AHI$, $AIJ$
are the antecedents, and the triangles $CDE$, $CEF$, $CFG$
the consequents, and [V.~\textsc{xii.}] any one of these equal
ratios is equal to the ratio of the sum of all the antecedents
to the sum of all the consequents; \emph{therefore
the triangle $ABH :$ the triangle $CDE ::$ the polygon
$ABHIJ :$ the polygon $CDEFG$.}

3.~The triangle $ABH : CDE$ in the duplicate ratio of
$AB : CD$ [\textsc{xix.}]. \emph{Hence $(2)$ the polygon $ABHIJ :$ the
polygon $CDEFG$ in the duplicate ratio of $AB : CD$.}

\emph{Cor.}~1.---The perimeters of similar polygons are to
one another in the ratio of their homologous sides.

\emph{Cor.}~2.---As squares are similar polygons, therefore
the duplicate ratio of two lines is equal to the ratio of
the squares described on them (compare Annotations,
V.~Def.~\textsc{x.}).

\emph{Cor.}~3.---Similar portions of similar figures bear the
same ratio to each other as the wholes of the figures.

\emph{Cor.}~4.---Similar portions of the perimeters of similar
figures are to each other in the ratio of the whole perimeters.



\exhead{Exercises.}

\textsc{Def.~i.}---\emph{Homologous points}\index{Points!homologous} in the planes of two
similar figures are such, that lines drawn from them
to the angular points of the two figures are proportional
to the homologous sides of the two figures.

\smallskip

\imgflow{190}{11}{f199}

\begin{footnotesize}
1.~If two figures be similar, to each point in the plane of one
there will be a corresponding
point in the
plane of the other.

\textbf{Dem.}---Let $ABCD$,
$A'B'C'D'$ be the two
figures, $P$ a point in
the plane of $ABCD$.
Join $AP$, $BP$, and construct
a triangle $A'P'B'$
on $A'B'$, similar to
$APB$; then it is easy to see that lines from $P'$ to the angular
points of $A'B'C'D'$ are proportional to the lines from $P$ to the
angular points of $ABCD$.


\imgflow{170}{12}{f200}

2.~If two figures be directly similar, and in the same plane,
there is in the plane a special point which, regarded as belonging
to either figure, is its own
homologous point with respect
to the other. For,
let $AB$, $A'B'$ be two homologous
sides of the figures,
$C$ their point of intersection.
Through the two triads of
points $A$, $A'$, $C$; $B$, $B'$, $C$
describe two circles intersecting
again in the point
$O$: $O$ will be the point required.
For it is evident
that the triangles $OAB$, $OA'B'$ are similar and that either may
be turned round the point $O$, so that the two bases, $AB$, $A'B'$,
will be parallel.
\par\end{footnotesize}

\textsc{Def.~ii.}---The point $O$ is called the \emph{centre of similitude\index{Centre!of similitude}
of the figures. It is also called their double point\index{Double point}.}

\smallskip
\begin{footnotesize}
3.~Two regular polygons of $n$ sides each have $n$ centres of
similitude.


4.~If any number of similar triangles have their corresponding
vertices lying on three given lines, they have a common centre of
similitude.

5.~If two figures be directly similar, and have a pair of homologous
sides parallel, every pair of homologous sides will be
parallel.
\par\end{footnotesize}

\textsc{Def.~iii.}---Two figures, such as those in 5, are said
to be \emph{homothetic}\index{Homothetic}.

\smallskip
\begin{footnotesize}
6.~If two figures be homothetic, the lines joining corresponding
angular points are concurrent, and the point of concurrence is the
centre of similitude of the figures.

7.~If two polygons be directly similar, either may be turned
round their centre of similitude until they become homothetic,
and this may be done in two different ways.

8.~Two circles are similar figures.


\imgflow{235}{12}{f201}

\textbf{Dem.}---Let $O$, $O'$ be their centres; let the angle $AOB$ be
indefinitely
small, so that
the arc $AB$
may be regarded
as a
right line;
make the
angle $A'O'B'$
equal to $AOB$;
then the triangles
$AOB$,
$A'O'B'$ are similar.

Again, make the angle $BOC$ indefinitely small, and make $B'O'C'$
equal to it; the triangles $BOC$, $B'O'C'$ are similar. Proceeding
in this way, we see that the circles can be divided into the same
number of similar elementary triangles. Hence the circles are
similar figures.

9.~Sectors of circles having equal central angles are similar
figures.

10.~As any two points of two circles may be regarded as homologous,
two circles have in consequence an infinite number of
centres of similitude; their locus is the circle, whose diameter is
the line joining the two points for which the two circles are
homothetic.


11.~The areas\index{Area} of circles are to one another as the squares of
their diameters. For they are to one another as the similar
elementary triangles into which they are divided, and these are
as the squares of the radii.

12.~The circumferences of circles are as their diameters
(\textit{Cor.}~1).

13.~The circumference of sectors having equal central angles
are proportional to their radii. Hence if $a$, $a'$ denote the arcs of
two sectors, which subtend equal angles at the centres, and if
$r$, $r'$ be their radii, $\dfrac{a}{r} = \dfrac{a'}{r'}$.

14.~The area of a circle is equal to half the rectangle contained
by the circumference and the radius. This is evident by
dividing the circle into elementary triangles, as in Ex.~8.

15.~The area of a sector of a circle is equal to half the rectangle
contained by the arc of the sector and the radius of the
circle.
\par\end{footnotesize}

\myprop{PROP\@.~XXI\@.---Theorem.}{Rectilineal figures $(A,\ B)$, which are similar to the same
figure $(C)$, are similar to one another.}

\textbf{Dem.}---Since the figures $A$ and $C$ are similar, they
are equiangular, and have the sides about their equal
angles proportional. In like manner $B$ and $C$ are
equiangular, and have the sides about their equal
angles proportional. Hence $A$ and $B$ are equiangular,
and have the sides about their equal angles proportional.
\emph{Therefore they are similar.}


\imgcent{230}{f202}

\textit{Cor.}---Two similar rectilineal figures which are homothetic
to a third are homothetic to one another.


\exhead{Exercise.}

\begin{footnotesize}
If three similar rectilineal figures be homothetic, two by two,
their three centres of similitudes are collinear.

\par\end{footnotesize}



\mypropl{PROP\@.~XXII---Theorem.}{If four lines $(AB,\ CD,\ EF,\ GH)$ be proportional, and
any pair of similar rectilineal figures $(ABK,\ CDL)$ be
similarly described on the first and second, and also any
pair $(EI,\ GJ)$ on the third and fourth, these figures are
proportional. \emph{Conversely}, if any rectilineal figure described
on the first of four right lines: the similar and
similarly described figure described on the second $::$ any
rectilineal figure on the third $:$ the similar and similarly
described figure on the fourth, the four lines are proportional.}


\imgcent{180}{f203}

\textbf{Dem.~1.}---$ABK : CDL :: AB^2 : CD^2$. [\textsc{xx.}];
\begin{flalign*}
&\text{and }&
    EI   &: GJ :: EF^2 : GH^2\ \text{[\textsc{xx.}]}. &&\phantom{therefore }\\
&\text{But since }&
    AB   &: CD :: EF : GH,  &&\\
&&  AB^2 &: CD^2 :: EF^2 : GH^2\ \text{[V.~\textsc{xxii.}, \textit{Cor.}~1]};  &&\\
&\text{therefore }&
    ABK  &: CDL :: EI : GJ.  &&
\end{flalign*}

If $ABK : CDL :: EI : GJ$, $AB: CD :: EF : GH$.
\begin{flalign*}
&\begin{alignedat}{3}
  &\text{\indent\textbf{Dem.}~2.---}&
      ABK  &: CDL :: AB^2 : CD^2  &&\quad \text{[\textsc{xx.}]},  \\
  &\text{and }&
      EI   &: GJ :: EF^2 : GH^2   &&\quad \text{[\textsc{xx.}]};  \\
  &\text{therefore }&
      AB^2 &: CD^2 :: EF^2 : GH^2.  &&\\
  &\text{Hence }&
      AB   &: CD :: EF : GH.  &&
  \end{alignedat}
&&\phantom{\indent Dem.\ s.---}
\end{flalign*}

The enunciation of this Proposition is wrongly stated
in Simson's \textit{Euclid}, and in those that copy it. As given
in those works, the four figures should be similar.



\myprop{PROP\@.~XXIII\@.---Theorem.}{Equiangular parallelograms $(AD,\ CG)$ are to each
other as the rectangles contained by their sides about a
pair of equal angles.}


\imgflow{117}{9}{f258}

\textbf{Dem.}---Let the two sides $AB$, $BC$ about the equal
angles $ABD$, $CBG$, be placed so
as to form one right line; then
it is evident, as in Prop.\ \textsc{xiv.},
that $GB$, $BD$ form one right
line. Complete the parallelogram
$BF$. Now, denoting the parallelograms
$AB$, $BF$, $CG$ by $X$,
$Y$, $Z$, respectively, we have---
\begin{flalign*}
&&               X &: Y  :: AB : BC \quad \text{[\textsc{i.}]}, &&\phantom{Hence }\\
&&               Y &: Z  :: BD : BG\quad \text{[\textsc{i.}]}. &&\\
&\text{Hence }& XY &: YZ :: AB\,.\,BD : BC\,.\,BG;  &&\\
&\text{or }&     X &: Z  :: AB\,.\,BD : BC\,.\,BG.  &&
\end{flalign*}

\begin{footnotesize}
\textsf{Observation.}---Since $AB\,.\,BD : BC\,.\,BG$ is compounded of the
two ratios $AB : BC$ and $BD : BG$ [V.\ Def.\ of compound ratio],
the enunciation is the same as if we said, ``in the ratio compounded
of the ratios of the sides,'' which is Euclid's; but it is
more easily understood as we have put it.
\par\end{footnotesize}

\exhead{Exercises.}

\begin{footnotesize}
1.~Triangles which have one angle of one equal or supplemental
to one angle of the other, are to one another in the ratio
of the rectangles of the sides about those angles.

2.~Two quadrilaterals whose diagonals intersect at equal
angles are to one another in the ratio of the rectangles of the
diagonals.

\par\end{footnotesize}



\myprop{PROP\@.~XXIV\@.---Theorem.}{In any parallelogram $(AC)$, every two parallelograms
$(AF,\ FC)$ which are about a diagonal are similar to the
whole and to one another.}


\imgflow{80}{8}{f204}

\textbf{Dem.}---Since the parallelograms $AC$, $AF$ have a
common angle, they are equiangular
[I.~\textsc{xxxiv.}], and all that is required to
be proved is, that the sides about the
equal angles are proportional. Now,
since the lines $EF$, $BC$ are parallel, the
triangles $AEF$, $ABC$ are equiangular;
therefore [\textsc{iv.}] $AE : EF :: AB : BC$,
and the other sides of the parallelograms
are equal to $AE$, $EF$; $AB$, $BC$: hence the sides
about the equal angles are proportional; \emph{therefore the
parallelograms $AF$, $AC$ are similar.} In the same manner
\emph{the parallelograms $AF$, $FC$ are similar.}

\emph{Cor.}---The parallelograms $AF$, $FC$, $AC$ are, two by
two, homothetic.

\myprop{PROP\@.~XXV\@.---Problem.}{To describe a rectilineal figure equal to a given one $(A)$,
and similar to another given one $(BCD)$.}

\textbf{Sol.}---On any side $BC$ of the figure $BCD$ describe the
rectangle $BE$ equal to $BCD$ [I.~\textsc{xlv.}], and on $CE$ describe
the rectangle $EF$ equal to $A$. Between $BC$, $CF$
find a mean proportional $GH$, and on it describe the
figure $GHI$ similar to $BCD$ [\textsc{xviii.}], so that $BC$ and
$GH$ may be homologous sides. \emph{$GHI$ is the figure required.}


\imgcent{240}{f205}

\textbf{Dem.}---The three lines $BC$, $GH$, $CF$ are in continued
proportion; therefore $BC : CF$ in the duplicate ratio
of $BC : GH$ [V.\ Def.~\textsc{x.}]; and since the figures $BCD$,
$GHI$ are similar, $BCD : GHI$ in the duplicate ratio

of $BC : GH$ [\textsc{xx.}]; also $BC : CF ::$ rectangle $BE :$
rectangle $EF$. Hence rectangle $BE : EF ::$ figure
$BCD : GHI$; but the rectangle $BE$ is equal to the
figure $BCD$; therefore the rectangle $EF$ is equal to
the figure $GHI$; but $EF$ is equal to $A$ (const.). \emph{Therefore
the figure $GHI$ is equal to $A$, and it is similar to
$BCD$. Hence it is the figure required.}

\begin{footnotesize}

\imgcent{260}{f206}

\emph{Or thus:} Describe the squares $EFJK$, $LMNO$ equal to the
figures $BCD$ and $A$ respectively [II\@.~\textsc{xiv.}]; then find $GH$ a fourth
proportional to $EF$, $LM$, and $BC$ [\textsc{xii.}]. On $GH$ describe the
rectilineal figure $GHI$ similar to the figure $BCD$ [\textsc{xviii.}], so that
$BC$ and $GH$ may be homologous sides. $GHI$ \emph{is the figure required.}


\textbf{Dem.}---Because $EF: LM:: BC: GH$ (const.), the figure
$EFJK: LMNO:: BCD: GHI$ [\textsc{xxii.}]; but $EFJK$ is equal to
$BCD$ (const.); therefore $LMNO$ is equal to $GHI$; but $LMNO$ is
equal to $A$ (const.). \emph{Therefore $GHI$ is equal to $A$, and it is similar
to $BCD$.}
\par\end{footnotesize}

\myprop{PROP\@.~XXVI\@.---Theorem.}{If two similar and similarly situated parallelograms
$(AEFG,\ ABCD)$ have a common angle, they are about
the same diagonal.}

\textbf{Dem.}---Draw the diagonals (\emph{see} fig., Prop.~\textsc{xxiv.})
$AF$, $AC$. Then because the parallelograms $AEFG$,
$ABCD$ are similar figures, they can be divided into
the same number of similar triangles [\textsc{xx.}]. Hence
the triangle $FAG$ is similar to $CAD$, and therefore
the angle $FAG$ is equal to the angle $CAD$. Hence
the line $AC$ must pass through the point $F$, and \emph{therefore
the parallelograms are about the same diagonal.}

\smallskip
\begin{footnotesize}
\textsf{Observation.}---Proposition \textsc{xxvi.}, being the converse of \textsc{xxiv.},
has evidently been misplaced. The following would be a simpler
enunciation:---``If two homothetic parallelograms have a common
angle, they are about the same diagonal.''
\par\end{footnotesize}

\myprop{PROP\@.~XXVII---Problem.}{To inscribe in a given triangle $(ABC)$ the maximum
parallelogram having a common angle $(B)$ with the triangle.}

\textbf{Sol.}---Bisect the side $AC$ opposite to the angle $B$, at
$P:$ through $P$ draw $PE$, $PF$ parallel to the other sides
of the triangle. \emph{$BP$ is the parallelogram required.}


\imgflow{140}{8}{f207}

\textbf{Dem.}---Take any other point $D$ in $AC:$ draw $DG$,

$DH$ parallel to the sides, and $CK$ parallel to $AB$;
produce $EP$, $GD$ to meet
$CK$ in $K$ and $J$, and produce $HD$ to meet $PK$ in $I$.

Now, since $AC$ is bisected
in $P$, $EK$ is also bisected
in $P$; hence [I.~\textsc{xxxvi.}] the
parallelogram $EO$ is equal
to $OK$; therefore $EO$ is
greater than $DK$; but $DK$ is equal to $FD$ [I.~\textsc{xliii.}];
hence $EO$ is greater than $FD$. To each add $BO$, and
we have the parallelogram $BP$ greater than $BD$.
\emph{Hence $BP$ is the maximum parallelogram which can be
inscribed in the given triangle}.

\emph{Cor}.~1.---The maximum parallelogram exceeds any
other parallelogram ab\-out the same angle in the triangle,
by the area of the similar parallelogram whose
diagonal is the line between the middle point $P$ of the
opposite side and the point $D$, which is the corner of
the other inscribed parallelogram.

\emph{Cor}.~2.---The parallelograms inscribed in a triangle,
and having one angle common with it, are proportional
to the rectangles contained by the segments of the
sides of the triangle, made by the opposite corners of
the parallelograms.

\emph{Cor}.~3.---The parallelogram $AC : GH :: AC^2 :
AD\,.\,DC$.

\mypropl{PROP\@.~XXVIII\@.---Problem.}{To inscribe in a given triangle $(ABC)$ a parallelogram
equal to a given rectilineal figure $(X)$ not greater than
the maximum inscribed parallelogram, and having an
angle $(B)$ common with the triangle.}

\textbf{Sol}.---Bisect the side $AC$ opposite to $B$, at $P$. Draw
$PF$, $PE$ parallel to the sides $AB$, $BC$; then [\textsc{xxvii}.] $BP$ is

the maximum parallelogram that can be inscribed in the
triangle $ABC$; and if $X$ be equal to it, the problem is
solved. If not, produce $EP$, and draw $CJ$ parallel to
$PF$; then describe the parallelogram $KLMN$ [\textsc{xxv.}]
equal to the difference between the figure $PJCF$ and
$X$, and similar to $PJCF$, and so that the sides $PJ$ and
$KL$ will be homologous; then cut off $PI$ equal to $KL$;
draw $IH$ parallel to $AB$, cutting $AC$ in $D$, and draw
$DG$ parallel to $BC$. \emph{$BD$ is the parallelogram required}.


\imgcent{200}{f208}

\textbf{Dem.}---Since the parallelograms $PC$, $PD$ are about
the same diagonal, they are similar [\textsc{xxiv.}]; but $PC$ is
similar to $KPT$ (const.); therefore $PD$ is similar to $KN$,
and (const.) their homologous sides, $PI$ and $KL$, are
equal; hence [\textsc{xx.}] $PD$ is equal to $KN$. Now, $PD$ is
the difference between $EF$ and $GH$ [\textsc{xxvii.}\ \textit{Cor}.~1], and
$KN$ is (const.) the difference between $PC$ and $X$; therefore
the difference between $PC$ and $X$ is equal to the
difference between $EF$ and $GH$; but $EF$ is equal to $PC$.
\emph{Hence $GH$ is equal to $X$}.

\mypropl{PROP\@.~XXIX\@.---Problem.}{To escribe to a given triangle $(ABC)$ a parallelogram
equal to a given rectilineal figure $(X)$, and having an
angle common with an external angle $(B)$ of the triangle.}

\textbf{Sol.}---The construction is the same as the last,

except that, instead of making the parallelogram $KN$
equal to the excess of the parallelogram $PC$ over the
rectilineal figure $X$, we make it equal to their sum;
and then make $PI$ equal to $KL$; draw $IH$ parallel to
$AB$, and the rest of the construction as before.


\imgcent{280}{f209}

\textbf{Dem.}---Now it can be proved, as in II\@.~\textsc{vi}., that
the parallelogram $BD$ is equal to the gnomon $OHJ$;
that is, equal to the difference between the parallelograms
$PD$ and $PC$, or the difference (const.) between
$KN$ and $PC$; that is (const.), equal to $X$, and $BD$ is
escribed to the triangle $ABC$, and has an angle common
with the external angle $B$. \textit{Hence the thing required
is done}.

\smallskip
\begin{footnotesize}
\textsf{Observation.}---The enunciations of the three foregoing Propositions
have been altered, in order to express them in modern
technical language. Some writers recommend the student to
omit them---we think differently. In the form we have given
them they are freed from their usual repulsive appearance. The
constructions and demonstrations are Euclid's, but slightly modified.
\par\end{footnotesize}



\myprop{PROP\@.~XXX\@.---Theorem.}{To divide a given line $(AB)$ in ``extreme and mean ratio.''}

\textbf{Sol.}---Divide $AB$ in $C$, so that the rectangle $AB\,.\,BC$
may be equal to the square on $AC$
[II\@.~\textsc{xi}.] \textit{Then $C$ is the point required}.


\imgflow{110}{2}{f259}

\textbf{Dem.}---Because the rectangle $AB\,.\,BC$ is equal to the
square on $AC, AB:AC::AC:BC$ [\textsc{xvii}.]. \textit{Hence $AB$
is cut in extreme and mean ratio in $C$} [Def.~\textsc{ii}.].

\exhead{Exercises.}

\begin{footnotesize}
1.~If the three sides of a right-angled triangle be in continued
proportion, the hypotenuse is divided in extreme and mean ratio
by the perpendicular from the right angle on the hypotenuse.

2.~In the same case the greater segment of the hypotenuse is
equal to the least side of the triangle.

3.~The square on the diameter of the circle described about the
triangle formed by the points $F$, $H$, $D$ (\textit{see} fig.~II\@.\ \textsc{xi}.), is equal to
six times the square on the line $FD$.
\par\end{footnotesize}

\mypropl{PROP\@.~XXXI\@.---Theorem.}{If any similar rectilineal figure be similarly described
on the three sides of a right-angled triangle $(ABC)$, the
figure on the hypotenuse is equal to the sum of those described
on the two other sides.}

\textbf{Dem.}---Draw the perpendicular $CD$ [I.~\textsc{xii}.]. Then
because $ABC$ is a right-angled triangle, and $CD$ is
drawn from the right angle perpendicular to the hypotenuse;
$BD:AD$ in the duplicate ratio of $BA:AC$
[\textsc{viii}.~\textit{Cor}.~4]. Again, because the figures described

on $BA$, $AC$ are similar, they are in the duplicate ratio
of $BA:AC$ [\textsc{xx}.]. Hence [V.~\textsc{xi}.] $BA : AD ::$ figure
described on $BA :$ figure described on AC\@. In like
manner, $AB : BD ::$ figure described on $AB :$ figure
described on $BC$. Hence [V.~\textsc{xxiv}.] $AB :$ sum of $AD$
and $BD ::$ figure described on the line $AB :$ sum of
the figures described on the lines $AC$, $BC$; but $AB$
is equal to the sum of $AD$ and $BD$.
\textit{Therefore} [V.~\textsc{a}.] \textit{the figure described on the line $AB$ is equal to the sum of
the similar figures described on the lines $AC$ and $BC$.}


\imgflow{150}{8}{f210}

\emph{Or thus:} Let us denote the sides by $a$, $b$, $c$, and the
figures by $\alpha$, $\beta$, $\gamma$; then because the figures are
similar, we have [\textsc{xx}.]\\[-\baselineskip]
\begin{flalign*}
&&  \alpha : \gamma &:: a^2 : c^2 \\
&\text{therefore}&
  \frac{\alpha}{\gamma} &= \frac{a^2}{c^2}. && \\
&\text{In like manner, }&
   \frac{\beta}{\gamma} &= \frac{b^2}{c^2};
&&\phantom{In }\\
&\text{therefore }&
   \frac{\alpha + \beta}{\gamma}
&= \frac{a^2 + b^2}{c^2};  &&
\end{flalign*}
but $a^2 + b^2 = c^2$ [I.~\textsc{xlvii}.]. Therefore $\alpha + \beta = \gamma$; that
is, \textit{the sum of the figures on the sides is equal to the figure
on the hypotenuse}.

\exhead{Exercise.}

\begin{footnotesize}
If semicircles be described on supplemental chords of a semicircle,
the sum of the areas of the two crescents thus formed is
equal to the area of the triangle whose sides are the supplemental
chords and the diameter.
\par\end{footnotesize}



\mypropl{PROP\@.~XXXII\@.---Theorem.}{If two triangles $(ABC,\ CDE)$ which have two sides of
one proportional to two sides of the other $(AB : BC :: CD : DE)$,
and the contained angles $(B,\ D)$ equal, be joined at
an angle $(C)$, so as to have their homologous sides parallel,
the remaining sides are in the same right line.}


\imgflow{130}{7}{f211}

\textbf{Dem.}---Because the triangles $ABC$, $CDE$ have the
angles $B$ and $D$ equal, and
the sides about these angles
proportional, viz., $AB : BC :: CD : DE$,
they are equiangular
[\textsc{vi}.]; therefore the
angle $BAC$ is equal to $DCE$.
To each add $ACD$, and we have the sum of the angles $BAC$, $ACD$ equal to the
sum of $DCE$ and $ACD$; but the sum of $BAC$, $ACD$ is
[I.~\textsc{xxix}.] two right angles; therefore the sum of $DCE$
and $ACD$ is two right angles. \textit{Hence} [I.~\textsc{xiv}.] \textit{$AC$, $CE$
are in the same right line.}

\mypropl{PROP\@.~XXXIII\@.--Theorem.}{In equal circles, angles $(BOC,\ EPF)$ at the centres or
$(BAC,\ EDF)$ at the circumferences have the same ratio to
one another as the arcs $(BC,\ EF)$ on which they stand,
and so also have the sectors $(BOC,\ EPF)$.}


\imgcent{240}{f212}

\textbf{Dem.}---1. Take any number of arcs $CG$, $GH$ in the
first circle, each equal to $BC$. Join $OG$, $OH$, and in

the second circle take any number of arcs $FI$, $IJ$, each
equal to $EF$. Join $IP$, $JP$. Then because the arcs
$BC$, $CG$, $GH$ are all equal, the angles $BOC$, $COG$, $GOH$,
are all equal [III\@.~\textsc{xxvii.}]. Therefore the arc $BH$ and
the angle $BOH$ are equimultiples of the arc $BC$ and
the angle $BOC$. In like manner it may be proved
that the arc $EJ$ and the angle $EPJ$ are equimultiples
of the arc $EF$ and the angle $EPF$. Again, since the
circles are equal, it is evident that the angle $BOH$
is greater than, equal to, or less than the angle $EPJ$,
according as the arc $BH$ is greater than, equal to, or
less than the arc $EJ$. Now we have four magnitudes,
namely, the arc $BC$, the arc $EF$, the angle $BOC$, and
the angle $EPF$; and we have taken equimultiples of the
first and third, namely, the arc $BH$, the angle $BOH$,
and other equimultiples of the second and fourth,
namely, the arc $EJ$ and the angle $EPJ$, and we have
proved that, according as the multiple of the first is
greater than, equal to, or less than the multiple of the
second, the multiple of the third is greater than, equal
to, or less than the multiple of the fourth. \textit{Hence}
[V.\ Def.~\textsc{v.}] \textit{$BC: EF::$ the angle $BOC: EPF$.}

Again, since the angle $BAC$ is half the angle $BOC$
[III\@.~\textsc{xx.}], and $EDF$ is half the angle $EPF$,
\begin{flalign*}
&&&&  BAC : EDF &:: BOC : EPF\ &&\text{[V. \textsc{xv}.]}; &&&&\phantom{Hence }\\
&\text{but }&&&
      BOC : EPF &:: BC : EF.   &&  &&&&\\
&\text{Hence }&&&
      BAC : EDF &:: BC : EF\   &&\text{[V. \textsc{xi}.]}.  &&&&
\end{flalign*}

2.~The sector $BOC :$ sector $EPF :: BC : EF$.

\textbf{Dem.}---The same construction being made, since the
arc $BC$ is equal to $CG$, the angle $BOC$ is equal to $COG$.
Hence the sectors $BOC$, $COG$ are congruent (\textit{see} Observation,
Proposition~\textsc{xxix}., Book~III\@.); therefore they
are equal. In like manner the sectors $COG$, $GOH$ are
equal. Hence there are as many equal sectors as there
are equal arcs; therefore the arc $BH$ and the sector
$BOH$ are equimultiples of the arc $BC$ and the sector

$BOC$. In the same manner it may be proved that the
arc $EJ$ and the sector $EPJ$ are equimultiples of the
arc $EF$ and the sector $EPF$; and it is evident, by
superposition, that if the arc $BH$ is greater than, equal
to, or less than the arc $EJ$, the sector $BOH$ is greater
than, equal to, or less than the sector $EPJ$. \textit{Hence}
[V.\ Def.~\textsc{v}.] \textit{the arc $BC : EF ::$ sector $BOC :$ sector
$EPF$.}

\smallskip
The second part may be proved as follows:---

\begin{footnotesize}
Sector $BOC = \tfrac{1}{2}$ rectangle contained by the arc $BC$, and the
radius of the circle $ABC$ [\textsc{xx.}\ Ex.~14] and sector $EPF = \tfrac{1}{2}$ rectangle
contained by the arc $EF$ and the radius of the circle $EDF$;
and since the circles are equal, their radii are equal. Hence,
sector $BOC :$ sector $EPF ::$ arc $BC :$ arc $EF$.
\par\end{footnotesize}

\addcontentsline{toc}{section}{Questions for Examination,}
\exhead{Questions for Examination on Book VI\@.}

\begin{footnotesize}
1.~What is the subject-matter of Book VI\@.? \textit{Ans}.\ Application
of the theory of proportion.

2.~What are similar rectilineal figures?

3.~What do similar figures agree in?

4.~How many conditions are necessary to define similar triangles?

5.~How many to define similar rectilineal figures of more
than three sides?

6.~When is a figure said to be given in species?

7.~When in magnitude?

8.~When in position?

9.~What is a mean proportional between two lines?

10.~Define two mean proportionals.

11.~What is the altitude of a rectilineal figure?

12.~If two triangles have equal altitudes, how do their areas
vary?

13.~How do these areas vary if they have equal bases but unequal altitudes?


14.~If both bases and altitudes differ, how do the areas vary?

15.~When are two lines divided proportionally?

16.~If in two lines divided proportionally a pair of homologous
points coincide with their point of intersection, what property
holds for the lines joining the other pairs of homologous
points?

17.~Define reciprocal proportion.

18.~If two triangles have equal areas, prove that their perpendiculars
are reciprocally proportional to the bases.

19.~What is meant by figures inversely similar?

20.~If two figures be inversely similar, how can they be
changed into figures directly similar?

21.~Give an example of two triangles inversely similar. \textit{Ans}.\
If two lines passing through any point $O$ outside a circle intersect
it in pairs of points $A$, $A'$; $B$, $B'$, respectively, the triangles
$OAB$, $OA'B'$, are inversely similar.

22.~What point is it round which a figure can be turned so as
to bring its sides into positions of parallelism with the sides of a
similar rectilineal figure. \textit{Ans}.\ The centre of similitude of the
two figures.

23.~How many figures similar to a given rectilineal figure of
sides can be described on a given line?

24.~How many centres of similitude can two regular polygons
of $n$ sides each have? \textit{Ans}.\ $n$ centres, which lie on a
circle.

25.~What are homothetic figures?

26.~How do the areas of similar rectilineal figures vary?

27.~What proposition is \textsc{xix}.\ a special case of?

28.~Define Philo's line.

29.~How many centres of similitude have two circles?
\par\end{footnotesize}


\exhead{Exercises on Book VI\@.}

\begin{footnotesize}
1.~If in a fixed triangle we draw a variable parallel to the
base, the locus of the points of intersection of the diagonals of
the trapezium thus cut off from the triangle is the median that
bisects the base.

2.~Find the locus of the point which divides in a given ratio
the several lines drawn from a given point to the circumference
of a given circle.

3.~Two lines $AB$, $XY$, are given in position: $AB$ is divided in
$C$ in the ratio $m:n$, and parallels $AA'$, $BB'$, $CC'$, are drawn in
any direction meeting $XY$ in the points $A'$, $B'$, $C'$; prove
\[
   (m + n) CC' = n AA' + m BB'.
\]

4.~Three concurrent lines from the vertices of a triangle $ABC$
meet the opposite sides in $A'$, $B'$, $C'$; prove
\[
   AB'\,.\,BC'\,.\,CA' = A'B\,.\,B'C\,.\,C'A.
\]

5.~If a transversal meet the sides of a triangle $ABC$ in the
points $A'$, $B'$, $C'$; prove
\[
   AB'\,.\,BC'\,.\,CA' = - A'B\,.\,B'C\,.\,C'A.
\]

6.~If on a variable line $AC$, drawn from a fixed point $A$ to any
point $B$ in the circumference of a given circle, a point $C$ be taken
such that the rectangle $AB\,.\,AC$ is constant, the locus of $C$ is a
circle.

7.~If $D$ be the middle point of the base $BC$ of a triangle $ABC$,
$E$ the foot of the perpendicular, $L$ the point where the bisector of
the angle $A$ meets $BC$, $H$ the point of contact of the inscribed
circle with $BC$; prove $DE\,.\,HL = HE\,.\,HD$.

8.~In the same case, if $K$ be the point of contact with $BC$
of the escribed circle, which touches the other sides produced,
$LH\,.\,BK = BD\,.\,LE$.

9.~If $R$, $r$, $r'$, $r''$, $r'''$ be the radii of the circumscribed, the inscribed,
and the escribed circles of a plane triangle, $d$, $d'$, $d''$, $d'''$
the distances of the centre of the circumscribed circle from the
centres of the others, then $R^2 = d^2 + 2 R r = d'^2 - 2 R r'$, \&c.

10.~In the same case, $12 R^2 = d^2 + d'^2 + d''^2 + d'''^2$.


11.~If $p'$, $p''$, $p'''$ denote the perpendiculars of a triangle, then
\begin{align*}
  (1)\ &\frac{1}{p'} + \frac{1}{p''} + \frac{1}{p'''} = \frac{1}{r} ; \\
  (2)\ &\frac{1}{p''} + \frac{1}{p'''} - \frac{1}{p'} = \frac{1}{r'} , \text{\ \&c.;} \\
  (3)\ &\frac{2}{p'} = \frac{1}{r} - \frac{1}{r'} , \text{\ \&c.;} \\
  (4)\ &\frac{2}{p'} = \frac{1}{r''} + \frac{1}{r'''} , \text{\ \&c.}
\end{align*}

12.~In a given triangle inscribe another of given form, and
having one of its angles at a given point in one of the sides of
the original triangle.

13.~If a triangle of given form move so that its three sides
pass through three fixed points, the locus of any point in its
plane is a circle.

14.~The angle $A$ and the area of a triangle $ABC$ are given in
magnitude: if the point $A$ be fixed in position, and the point $B$
move along a fixed line or circle, the locus of the point $C$ is a
circle.

15.~One of the vertices of a triangle of given form remains
fixed; the locus of another is a right line or circle; find the
locus of the third.

16.~Find the area of a triangle---(1) in terms of its medians;
(2) in terms of its perpendiculars.

17.~If two circles touch externally, their common tangent is a
mean proportional between their diameters.

18.~If there be given three parallel lines, and two fixed points
$A$, $B$; then if the lines of connexion of $A$ and $B$ to any variable
point in one of the parallels intersect the other parallels in the
points $C$ and $D$, $E$ and $F$, respectively, $CF$ and $DE$ pass each
through a fixed point.

19.~If a system of circles pass through two fixed points, any
two secants passing through one of the points are cut proportionally
by the circles.

20.~Find a point\label{poiat} $O$ in the plane of a triangle $ABC$, such that
the diameters of the three circles, about the triangles $OAB$, $OBC$,
$OCA$, may be in the ratios of three given lines.


21.~$ABCD$ is a cyclic quadrilateral: the lines $AB$, $AD$, and
the point $C$, are given in position; find the locus of the point
which divides $BD$ in a given ratio.

22.~$CA$, $CB$ are two tangents to a circle; $BE$ is perpendicular
to $AD$, the diameter through $A$; prove that $CD$ bisects $BE$.

23.~If three lines from the vertices of a triangle $ABC$ to any
interior point $O$ meet the opposite sides in the points $A'$, $B'$, $C'$;
prove
\[
   \frac{OA'}{AA'} + \frac{OB'}{BB'} + \frac{OC'}{CC'} = 1.
\]

24.~If three concurrent lines $OA$, $OB$, $OC$ be cut by two
transversals in the two systems of points $A$, $B$, $C$; $A'$, $B'$, $C'$,
respectively: prove
\[
  \frac{AB}{A'B'} \,.\, \frac{OC}{OC'}
= \frac{BC}{B'C'} \,.\, \frac{OA}{OA'}
= \frac{CA}{C'A'} c \frac{OB}{OB'}.
\]

25.~The line joining the middle points of the diagonals of a
quadrilateral circumscribed to a circle---
\begin{changemargin}{0.3in}{0in}
\begin{enumerate}
   \item[(1)] divides each pair of opposite sides into inversely proportional segments;
   \item[(2)] is divided by each pair of opposite lines into segments which, measured from the centre, are proportional to the sides;
   \item[(3)] is divided by both pairs of opposite sides into segments which, measured from either diagonal, have the same ratio to each other.
\end{enumerate}
\end{changemargin}

26.~If $CD$, $CD'$ be the internal and external bisectors of the
angle $C$ of the triangle $ACB$, the three rectangles $AD\,.\, DB$,
$AC\,.\, CB$, $AD\,.\, BD'$ are proportional to the squares of $AD$, $AC$,
$AD'$; and are---(1) in arithmetical progression if the difference of
the base angles be equal to a right angle; (2) in geometrical
progression if one base angle be right; (3) in harmonical
progression if the sum of the base angles be equal to a right angle.

27.~If a variable circle touch two fixed circles, the chord of
contact passes through a fixed point on the line connecting the
centres of the fixed circles.


\imgcent{304}{f213}

\textbf{Dem.}---Let $O$, $O'$ be the centres of the two fixed circles; $O''$
the centre of the variable circle; $A$, $B$ the points of contact. Let
$AB$ and $OO'$ meet in $C$, and cut the fixed circles again in the
points $A'$, $B'$ respectively. Join $A'O$, $AO$, $BO'$. Then $AO$, $BO'$

meet in $O''$ [III\@.~\textsc{xi.}]. Now, because the triangles $OAA'$, $O''AB$
are isosceles, the angle $O''BA = O''AB = OA'A$. Hence $OA'$ is
parallel to $O'B$; therefore $OC : O'C :: OA' : O'B$; that is, in a
given ratio. Hence $C$ is a given point.

28.~If $DD'$ be the common tangent to the two circles, $DD'^{2}
= AB' \,.\, A'B$.

29.~If $R$ denote the radius of $O''$ and $\rho$, $\rho'$, the radii of $O$, $O'$,
$DD'^{2} : AB^{2} :: (R \pm \rho)(R \pm \rho') : R^{2}$, the choice of sign depending
on the nature of the contacts. This follows from 28.

30.~If four circles be tangential to a fifth, and if we denote by
$\overline{12}$ the common tangent to the first and second, \&c., then
\[
  \overline{12} \,.\, \overline{34}
+ \overline{23} \,.\, \overline{14}
= \overline{13} \,.\, \overline{24}.
\]

31.~The inscribed and escribed circles of any triangle are all
touched by its nine-points circle.

32.~The four triangles which are determined by four points,
taken three by three, are such that their nine-points circles
have one common point.

33.~If $a$, $b$, $c$, $d$ denote the four sides, and $D$, $D'$ the diagonals
of a quadrilateral; prove that the sides of the triangle, formed by
joining the feet of the perpendiculars from any of its angular
points on the sides of the triangle formed by the three remaining
points, are proportional to the three rectangles $ac$, $bd$, $DD'$.

34.~Prove the converse of Ptolemy's theorem\index{Ptolemy's theorem!converse of}\index{Ptolemy's theorem!extension of} (\emph{see} \textsc{xvii.}, Ex.~13).

35.~Describe a circle which shall---(1) pass through a given
point, and touch two given circles; (2)~touch three given
circles.

36.~If a variable circle touch two fixed circles, the tangent to
it from their centre of similitude, through which the chord of contact
passes (27), is of constant length.


37.~If the lines $AD$, $BD'$ (\emph{see} fig., Ex.~27) be produced, they
meet in a point on the circumference of $O''$, and the line $O''P$ is
perpendicular to $DD'$.

38.~If $A$, $B$ be two fixed points on two lines given in position,
and $A'$, $B'$ two variable points, such that the ratio $AA' : BB'$ is
constant, the locus of the point dividing $A'B'$ in a given ratio is a
right line.

39.~If a line $EF$ divide proportionally two opposite sides of a
quadrilateral, and a line $GH$ the other sides, each of these is
divided by the other in the same ratio as the sides which determine
them.

40.~In a given circle inscribe a triangle, such that the triangle
whose angular points are the feet of the perpendiculars from the
extremities of the base on the bisector of the vertical angle, and
the foot of the perpendicular from the vertical angle on the base,
may be a maximum.

41.~In a circle, the point of intersection of the diagonals of any
inscribed quadrilateral coincides with the point of intersection of
the diagonals of the circumscribed quadrilateral, whose sides touch
the circle at the angular points of the inscribed quadrilateral.

42.~Through two given points describe a circle whose common
chord with another given circle may be parallel to a given line,
or pass through a given point.

43.~Being given the centre of a circle, describe it so as to cut
the legs of a given angle along a chord parallel to a given line.

44.~If concurrent lines drawn from the angles of a polygon of
an odd number of sides divide the opposite sides each into two
segments, the product of one set of alternate segments is equal to
the product of the other set.

45.~If a triangle be described about a circle, the lines from the
points of contact of its sides with the circle to the opposite angular
points are concurrent.

46.~If a triangle be inscribed in a circle, the tangents to the
circle at its three angular points meet the three opposite sides at
three collinear points.

47.~The external bisectors of the angles of a triangle meet the
opposite sides in three collinear points.

48.~Describe a circle touching a given line at a given point,
and cutting a given circle at a given angle.


\textsc{Def.}---\textit{The centre of mean position of any number of points
$A$, $B$, $C$, $D$}, \&c., \textit{is a point which may be found as follows}:---Bisect
the line joining any two points $A$, $B$, in $G$. Join $G$ to a
third point $C$; divide $GC$ in $H$, so that $GH = \frac{1}{3} GC$. Join $H$ to a
fourth point $D$, and divide $HD$ in $K$, so that $HK = \frac{1}{4} HD$, and so
on. The last point found will be the centre of mean position of
the given points.

49.~The centre of mean position of the angular points of a
regular polygon is the centre of figure of the polygon.

50.~The sum of the perpendiculars let fall from any system of
points $A$, $B$, $C$, $D$, \&c., whose number is $n$ on any line $L$, is equal
to $n$ times the perpendicular from the centre of mean position
on $L$.

51.~The sum of the squares of lines drawn from any system of
points $A$, $B$, $C$, $D$, \&c., to any point $P$, exceeds the sum of the
squares of lines from the same points to their centre of mean
position, $O$, by $nOP^{2}$.

52.~If a point be taken within a triangle, so as to be the centre
of mean position of the feet of the perpendiculars drawn from it
to the sides of the triangle, the sum of the squares of the perpendiculars
is a minimum.

53.~Construct a quadrilateral, being given two opposite angles,
the diagonals, and the angle between the diagonals.

54.~A circle rolls inside another of double its diameter; find
the locus of a fixed point in its circumference.

55.~Two points, $C$, $D$, in the circumference of a given circle are
on the same side of a given diameter; find a point $P$ in the circumference
at the other side of the given diameter, $AB$, such that
$PC$, $PD$ may cut $AB$ at equal distances from the centre.

56.~If the sides of any polygon be cut by a transversal, the
product of one set of alternate segments is equal to the product
of the remaining set.

57.~A transversal being drawn cutting the sides of a triangle,
the lines from the angles of the triangle to the middle points
of the segments of the transversal intercepted by those angles
meet the opposite sides in collinear points.

58.~If lines be drawn from any point $P$ to the angles of a
triangle, the perpendiculars at $P$ to these lines meet the opposite
sides of the triangle in three collinear points.



59.~Divide a given semicircle into two parts by a perpendicular
to the diameter, so that the radii of the circles inscribed in them
may have a given ratio.

60.~From a point within a triangle perpendiculars are let fall
on the sides; find the locus of the point, when the sum of the
squares of the lines joining the feet of the perpendiculars is
given.

61.~If a circle make given intercepts on two fixed lines, the
rectangle contained by the perpendiculars from its centre on the
bisectors of the angle formed by the lines is given.

62.~If the base and the difference of the base angles of a
triangle be given, the rectangle contained by the perpendiculars
from the vertex on two lines through the middle point of the
base, parallel to the internal and external bisectors of the vertical
angle, is constant.

63.~The rectangle contained by the perpendiculars from the
extremities of the base of a triangle, on the internal bisector of
the vertical angle, is equal to the rectangle contained by the
external bisector and the perpendicular from the middle of the
base on the internal bisector.

64.~State and prove the corresponding theorem for perpendiculars
on the external bisector.

65.~If $R$, $R'$ denote the radii of the circles inscribed in the
triangles into which a right-angled triangle is divided by the perpendicular
from the right angle on the hypotenuse; then, if $c$ be
the hypotenuse, and $s$ the semiperimeter, $R^{2} + R'^{2} = (s-c)^{2}$.

66.~If $A$, $B$, $C$, $D$ be four collinear points, find a point $O$ in
the same line with them such that $OA\,.\,OD = OB\,.\,OC$.

67.~The four sides of a cyclic quadrilateral are given; construct
it.

68.~Being given two circles, find the locus of a point such
that tangents from it to the circles may have a given ratio.

69.~If four points $A$, $B$, $C$, $D$ be collinear, find the locus of the
point $P$ at which $AB$ and $CD$ subtend equal angles.

70.~If a circle touch internally two sides, $CA$, $CB$, of a triangle
and its circumscribed circle, the distance from $C$ to the point of
contact on either side is a fourth proportional to the semiperimeter,
and $CA$, $CB$.

71.~State and prove the corresponding theorem for a circle
touching the circumscribed circle externally and two sides produced.



\imgflow{170}{28}{f214}

72.~\emph{Pascal's Theorem.}\index{Pascal}---If the opposite sides of an irregular
hexagon $ABCDEF$ inscribed in a circle be produced till they
meet, the three points of intersection $G$, $H$, $I$ are collinear.

\textbf{Dem.}---Join $AD$. Describe a circle about the triangle $ADI$,
cutting the lines $AF$, $CD$ produced, if necessary, in $K$ and $L$.
Join $IK$, $KL$, $LI$. Now,
the angles $KLG$, $FCG$ are
each [III\@.~\textsc{xxi.}] equal to
the angle $GAD$. Hence they
are equal. Therefore $KL$ is
parallel to $CF$. Similarly,
$LI$ is parallel to $CH$, and
$KI$ to $FH$; hence the triangles
$KLI$, $FCH$ are homothetic.
Hence the lines
joining corresponding vertices
are concurrent. Therefore
the points $I$, $H$, $G$ are
collinear.

73.~If two sides of a triangle
circumscribed to a
given circle be given in
position, but the third side
variable, the circle described
about the triangle touches a
fixed circle.

74.~If two sides of a triangle
be given in position,
and if the area be given in
magnitude, two points can be found, at each of which the base
subtends a constant angle.

75.~If $a$, $b$, $c$, $d$ denote the sides of a cyclic quadrilateral, and $s$
its semiperimeter, prove its area $= \sqrt{(s-a)(s-b)(s-c)(s-d)}$.

76.~If three concurrent lines from the angles of a triangle $ABC$
meet the opposite side in the points $A'$, $B'$, $C'$, and the points
$A'$, $B'$, $C'$ be joined, forming a second triangle $A'B'C'$,
\[
\triangle\ ABC : \triangle\ A'B'C' :: AB\,.\,BC\,.\,CA : 2AB'\,.\,BC'\,.\,CA'.
\]

77.~In the same case the diameter of the circle circumscribed
about the triangle $ABC = AB'\,.\,BC'\,.\,CA'$ divided by the area of
$A'B'C'$.


78.~If a quadrilateral be inscribed in one circle, and circumscribed
to another, the square of its area is equal to the product
of its four sides.

79.~If on the sides $AB$, $AC$ of a triangle $ABC$ we take two
points $D$, $E$, and on their line of connexion $F$, such that
\[
\frac{BD}{AD}=\frac{AE}{CE}=\frac{DF}{EF};
\]
prove the triangle $BFC=2ADE$.

80.~If through the middle points of each of the two diagonals
of a quadrilateral we draw a parallel to the other, the lines drawn
from their points of intersection to the middle points of the sides
divide the quadrilateral into four equal parts.

81.~$CE$, $DF$ are perpendiculars to the diameter of a semicircle,
and two circles are described touching $CE$, $DE$, and the semicircle,
one internally and the other externally; the rectangle contained by
the perpendiculars from their centres on $AB$ is equal to $CE\,.\,DF$.

82.~If lines be drawn from any point in the circumference of a
circle to the angular points of any inscribed regular polygon of an
odd number of sides, the sums of the alternate lines are equal.

83.~If at the extremities of a chord drawn through a given
point within a given circle tangents be drawn, the sum of the
reciprocals of the perpendiculars from the point upon the tangents
is constant.

84.~If a cyclic quadrilateral be such that three of its sides pass
through three fixed collinear points, the fourth side passes through
a fourth fixed point, collinear with the three given ones.

85.~If all the sides of a polygon be parallel to given lines, and
if the loci of all the angles but one be right lines, the locus of the
remaining angle is also a right line.

86.~If the vertical angle and the bisector of the vertical angle
be given, the sum of the reciprocals of the containing sides is
constant.

87.~If $P$, $P'$ denote the areas of two regular polygons of any
common number of sides, inscribed and circumscribed to a circle,
and $\Pi$, $\Pi'$ the areas of the corresponding polygons of double the
number of sides; prove $\Pi$ is a geometric mean between $P$ and $P'$,
and $\Pi'$ a harmonic mean between

$\Pi$ and $P'$.

88.~The difference of the areas of the triangles formed by joining
the centres of the circles described about the equilateral triangles
constructed---(1) outwards; (2) inwards---on the sides of
any triangle, is equal to the area of that triangle.


89.~In the same case, the sum of the squares of the sides of
the two new triangles is equal to the sum of the squares of the
sides of the original triangle.

90.~If $R$, $r$ denote the radii of the circumscribed and inscribed
circles to a regular polygon of any number of sides, $R'$, $r'$, corresponding
radii to a regular polygon of the same area, and double
the number of sides; prove
\[
R' = \sqrt{Rr},\ \text{and}\ r' = \sqrt{\frac{r(R+r)}{2}}.
\]

91.~If the altitude of a triangle be equal to its base, the sum
of the distances of the orthocentre from the base and from the
middle point of the base is equal to half the base.

92.~In any triangle, the radius of the circumscribed circle is to
the radius of the circle which is the locus of the vertex, when
the base and the ratio of the sides are given, as the difference of
the squares of the sides is to four times the area.

93.~Given the area of a parallelogram, one of its angles, and
the difference between its diagonals; construct the parallelogram.

94.~If a variable circle touch two equal circles, one internally
and the other externally, and perpendiculars be let fall from its
centre on the transverse tangents to these circles, the rectangle of
the intercepts between the feet of these perpendiculars and the
intersection of the tangents is constant.

95.~Given the base of a triangle, the vertical angle, and the
point in the base whose distance from the vertex is equal half
the sum of the sides; construct the triangle.

96.~If the middle point of the base $BC$ of an isosceles triangle
$ABC$ be the centre of a circle touching the equal sides, prove that
any variable tangent to the circle will cut the sides in points
$D$, $E$, such that the rectangle $BD\,.\,CE$ will be constant.

97.~Inscribe in a given circle a trapezium, the sum of whose
opposite parallel sides is given, and whose area is given.

98.~Inscribe in a given circle a polygon all whose sides pass
through given points.

99.~If two circles $X$, $Y$ be so related that a triangle may be
inscribed in $X$ and circumscribed about $Y$, an infinite number of
such triangles can be constructed.

100.~In the same case, the circle inscribed in the triangle
formed by joining the points of contact on $Y$ touches a given
circle.


101.~And the circle described about the triangle formed by
drawing tangents to $X$, at the angular points of the inscribed triangle,
touches a given circle.

102.~Find a point, the sum of whose distances from three
given points may be a minimum.

103.~A line drawn through the intersection of two tangents to
a circle is divided harmonically by the circle and the chord of
contact.

104.~To construct a quadrilateral similar to a given one whose
four sides shall pass through four given points.

105.~To construct a quadrilateral, similar to a given one,
whose four vertices shall lie on four given lines.

106.~Given the base of a triangle, the difference of the base
angles, and the rectangle of the sides; construct the triangle.

107.~$ABCD$ is a square, the side $CD$ is bisected in $E$, and the
line $EF$ drawn, making the angle $AEF = EAB$; prove that $EF$
divides the side $BC$ in the ratio of $2 : 1$.

108.~If any chord be drawn through a fixed point on a diameter
of a circle, and its extremities joined to either end of the
diameter, the joining lines cut off, on the tangent at the other
end, portions whose rectangle is constant.

109.~If two circles touch, and through their point of contact
two secants be drawn at right angles to each other, cutting the
circles respectively in the points $A$, $A'$; $B$, $B'$; then $AA'^2 + BB'^2$
is constant.

110.~If two secants at right angles to each other, passing through
one of the points of intersection of two circles, cut the circles
again, and the line through their centres in the two systems of
points $a$, $b$, $c$; $a'$, $b'$, $c'$ respectively, then $ab : bc : : a'b' : b'c'$.

111.~Two circles described to touch an ordinate of a semicircle,
the semicircle itself, and the semicircles on the segments
of the diameter, are equal to one another.

112.~If a chord of a given circle subtend a right angle at
a given point, the locus of the intersection of the tangents at
its extremities is a circle.

113.~The rectangle contained by the segments of the base of a
triangle, made by the point of contact of the inscribed circle, is
equal to the rectangle contained by the perpendiculars from the
extremities of the base on the bisector of the vertical angle.



114.~If $O$ be the centre of the inscribed circle of the triangle
prove
\[
\frac{OA^{2}}{bc} + \frac{OB^{2}}{ca} + \frac{OC^{2}}{ab} = 1.
\]

115.~State and prove the corresponding theorems for the
centres of the escribed circles.

116.~Four points $A$, $B$, $C$, $D$ are collinear; find a point $P$ at
which the segments $AB$, $BC$, $CD$ subtend equal angles.

117.~The product of the bisectors of the three angles of a triangle
whose sides are $a$, $b$, $c$, is
\[
  \frac{8 abc.s.\text{area} }
       {  (a+b)(b+c)(c+a)   }.
\]

118.~In the same case the product of the alternate segments of
the sides made by the bisectors of the angles is
\[
  \frac{ a^2 b^2 c^2 }
       {  (a+b)(b+c)(c+a)              }.
\]

119.~If three of the six points in which a circle meets the sides
of any triangle be such, that the lines joining them to the opposite
vertices are concurrent, the same property is true of the three
remaining points.

120.~If a triangle $A'B'C'$ be inscribed in another $ABC$, prove
\[
AB'\,.\,BC'\,.\,CA' + A'B\,.\,B'C\,.\,C'A
\]
is equal twice the triangle $A'B'C'$ multiplied by the diameter of
the circle $ABC$.

121.~Construct a polygon of an odd number of sides, being
given that the sides taken in order are divided in given ratios
by fixed points.

122.~If the external diagonal of a quadrilateral inscribed in a
given circle be a chord of another given circle, the locus of its
middle point is a circle.

123.~If a chord of one circle be a tangent to another, the line
connecting the middle point of each arc which it cuts off on the
first, to its point of contact with the second, passes through a given
point.

124.~From a point $P$ in the plane of a given polygon perpendiculars
are let fall on its sides; if the area of the polygon
formed by joining the feet of the perpendiculars be given, the
locus of $P$ is a circle.
\par\end{footnotesize}




\mybook{BOOK XI}{Theory of Planes, Coplanar Lines, and Solid Angles}

\textsc{i.} When two or more lines are in one plane they are
said to be \emph{coplanar}\index{Lines!Coplanar}.

\textsc{ii.} The angle which one plane makes with another is
called a \emph{dihedral} angle\index{Angle!dihedral}.

\textsc{iii.} A \emph{solid angle}\index{Angle!solid} is that which is made by more than
two plane angles, in different planes, meeting in a
point.

\textsc{iv.} The point is called the \emph{vertex} of the solid angle.

\textsc{v.} If a solid angle be composed of three plane angles
it is called a \emph{trihedral angle;}\index{Angle!trihedral}\index{Trihedral angle} if of four, a \emph{tetrahedral
angle;} and if of more than four, a \emph{polyhedral angle}.

\addcontentsline{toc}{section}{Propositions \textsc{i.--xxi.,}}
\myprop{PROP\@.~I.---Theorem.}{One part $(AB)$ of a right line cannot be in a plane $(X)$,
and another part $(BC)$ not in it.}


\imgflow{150}{6}{f215}

\textbf{Dem.}---Since $AB$ is in the plane $X$, it can be produced
in it [Bk.~I.\ Post.~\textsc{ii.}];
let it be produced to $D$.
Then, if $BC$ be not in $X$,
let any other plane passing
through $AD$ be turned
round $AD$ until it passes
through the point $C$. Now,
because the points $B$, $C$ are in this second plane, the line

$BC$ [I., Def.~\textsc{vi.}] is in it. Therefore the two right lines
$ABC$, $ABD$ lying in one plane have a common segment
$AB$, which is impossible. Therefore, \&c.


\mypropflow{PROP\@.~II\@.---Theorem.}{Two right lines $(AB,\ CD)$ which intersect one another
in any point $(E)$ are coplanar, and so also are any three
right lines $(EC,\ CB,\ BE)$ which form a triangle.}{113}{9}{f216}

\textbf{Dem.}---Let any plane pass through $EB$, and be turned
round it until it passes through
$C$. Then because the points
$E$, $C$ are in this plane, the right
line $EC$ is in it [I., Def.~\textsc{vi.}].
For the same reason the line $BC$
is in it. \textit{Therefore the lines $EC$,
$CB$, $BE$ are coplanar;} but $AB$
and $CD$ are two of these lines.
\textit{Hence $AB$ and $CD$ are coplanar.}

\myprop{PROP\@.~III\@.---Theorem.}{If two planes $(AB,\ BC)$ cut one another, their common
section $(BD)$ is a right line.}


\imgflow{135}{7}{f217}

\textbf{Dem.}---If not from $B$ to $D$, draw in the plane $AB$ the
right line $BED$, and in the
plane $BC$ the right line $BFD$.
Then the right lines $BED$,
$BFD$ enclose a space, which
[I., Axiom~\textsc{x.}] is impossible.
\textit{Therefore the common section
$BD$ of the two planes must be
a right line.}

\mypropl{PROP\@.~IV\@.---Theorem.}{If a right line $(EF)$ be perpendicular to each of two intersecting
lines $(AB,$\ $CD)$, it will be perpendicular to any
line $GH$, which is both coplanar and concurrent with them.}


\imgcent{158}{f218}

\textbf{Dem.}---Through any point $G$ in $GH$ draw a line $BC$

intersecting $AB$, $CD$, and so as to be bisected in $G$; and
join any point $F$ in $EF$ to $B$, $G$, $C$. Then, because
$EF$ is perpendicular to the lines $EB$, $EC$, we have
\begin{flalign*}
\multispan{6}{$\hfill
    BF^{2} = BE^{2} + EF^{2}, \quad \text{and}\quad
    CF^{2} = CE^{2} + EF^{2}; \hfill$}\\
&&  \therefore BF^{2} + CF^{2} &= BE^{2} + CE^{2} + 2EF^{2}.  &&\phantom{and }\\
&\rlap{\indent Again }&
    BF^{2} + CF^{2} &= 2BG^{2} + 2GF^{2}\ \text{[II\@.~\textsc{x}.~Ex.~2]},  &&\\
&\text{and }&
    BE^{2} + CE^{2} &= 2BG^{2} + 2GE^{2};  &&\\
&&  \therefore 2BG^{2} + 2GF^{2} &= 2BG^{2} + 2GE^{2} + 2EF^{2};  &&\\
&&  \therefore GF^{2}            &= GE^{2} + EF^{2}.  &&
\end{flalign*}

\textit{Hence the angle $GEF$ is right, and $EF$ is perpendicular
to $EG$.}

\smallskip
\textsc{Def.~vi.}---\textit{A line such as $EF$, which is perpendicular
to a system of concurrent and coplanar lines, is said to be
perpendicular to the plane of these lines, and is called a
\textsc{normal}\index{Lines!normal to a plane} to it.}

\textit{Cor.}~1.---The normal is the least line that may be
drawn from a given point to a given plane; and of all
others that may be drawn to it, the lines of any
system making equal angles with the normal are equal
to each other.

\textit{Cor.}~2.---A perpendicular to each of two intersecting
lines is normal to their plane.




\mypropflow{PROP\@.~V.---Theorem.}{If three concurrent lines $(BC,\ BD,\ BE)$ have a common
perpendicular $(AB)$, they are coplanar.}{100}{11}{f219}

\textbf{Dem.}---For if possible let $BC$ be not coplanar with
$BD$, $BE$, and let the plane of $AB$,
$BC$ intersect the plane of $BD$, $BE$
in the line $BF$. Then [XI\@.~\textsc{iii}.]
$BF$ is a right line; and, since it is
coplanar with $BD$, $BE$, which are
each perpendicular to $AB$, it is
[XI\@.~\textsc{iv}.] perpendicular to $AB$.
Therefore the angle $ABF$ is right;
and the angle $ABC$ is right (hyp.).
Hence $ABC$ is equal to $ABF$,
which is impossible [I., Axiom~\textsc{ix}.].
\textit{Therefore the lines $BC$, $BD$, $BE$ are coplanar.}

\myprop{PROP\@.~VI\@.---Theorem.}{If two right lines $(AB,\ CD)$ be normals to the same
plane $(X)$, they shall be parallel to one another.}


\imgflow{93}{11}{f220}

\textbf{Dem.}---Let $AB$, $CD$ meet the plane $X$ at the points
$B$, $D$. Join $BD$, and in the plane
$X$ draw $DE$ at right angles to $BD$;
take any point $E$ in $DE$. Join $BE$,
$AE$, $AD$. Then because $AB$ is
normal to $X$, the angle $ABE$ is
right. Therefore $AE^2 = AB^2 + BE^2
= AB^2 + BD^2 + DE^2$; because the
angle $BDE$ is right. But $AB^2 + BD^2
= AD^2$, because the angle $ABD$
is right. Hence $AE^2 = AD^2 + DE^2$.
Therefore the angle $ADE$ is right.
[I.~\textsc{xlviii}]. And since $CD$ is normal to the plane $X$,
$DE$ is perpendicular to $CD$. Hence $DE$ is a common
perpendicular to the three concurrent lines $CD$, $AD$, $BD$.

Therefore these lines are coplanar [XI\@.~\textsc{v.}]. But $AB$ is
coplanar with $AD$, $BD$ [XI\@.~\textsc{ii.}]. Therefore the lines
$AD$, $BD$, $CD$ are coplanar; and since the angles $ABD$,
$BDC$ are right, \textit{the line $AB$ is parallel to $CD$} [I.~\textsc{xxviii.}].

\smallskip
\textsc{Def.~vii.}---\textit{If from every point in a given line normals
be drawn to a given plane, the locus of their feet is called
the projection of the given line on the plane.}

\exhead{Exercises.}

\begin{footnotesize}
1.~The projection of any line on a plane is a right line.

2.~The projection on either of two intersecting planes of a
normal to the other plane is perpendicular to the line of intersection
of the planes.
\par\end{footnotesize}

\myprop{PROP\@.~VII\@.---Theorem.}{Two parallel lines $(AB,\ CD)$ and any line $(EF)$ intersecting
them are coplanar.}


\imgflow{150}{6}{f221}

\textbf{Dem.}---If possible let the intersecting line be out of
the plane, as $EGF$. And in
the plane, of the parallels
draw [I.~Post.~\textsc{ii}.] the
right line $EHF$.  Then
we have two right lines
$EGF$, $EHF$, enclosing a
space, which [I.~Axiom \textsc{x.}] is impossible. \textit{Hence the
two parallel right lines and the transversal are coplanar.}

\smallskip
\begin{footnotesize}
\emph{Or thus:} Since the points $E$, $F$ are in the plane of the parallels,
the line joining these points is in that plane [I.~Def.~\textsc{vi}].
\par\end{footnotesize}

\myprop{PROP\@.~VIII\@.---Theorem.}{If one $(AB)$ of two parallel right lines $(AB,\ CD)$, be
normal to a plane $(X)$, the other line $(CD)$ shall be normal
to the same plane.}


\imgflow{100}{11}{f222}

\textbf{Dem.}---Let $AB$, $CD$ meet the plane $X$ in the points

$B$, $D$. Join $BD$. Then the lines $AB$, $BD$, $CD$ are coplanar.
Now in the plane $X$, to
which $AB$ is normal, draw $DE$ at
right angles to $BD$. Take any
point $E$ in $DE$, and join $BE$, $AE$,
$AD$.

Then because $AB$ is normal to
the plane $X$, it is perpendicular to
the line $BE$ in that plane [XI\@.\ Def.~\textsc{vi.}].
Hence the angle $ABE$
is right; therefore $AE^{2} = AB^{2} + BE^{2}
= AB^{2} + BD^{2} + DE^{2}$ (because $BDE$
is right (const.)) $= AD^{2} + DE^{2}$ (because $ABD$ is right
(hyp.)). Therefore the angle $ADE$ is right. Hence
$DE$ is at right angles both to $AD$ and $BD$. Therefore
[XI\@.~\textsc{iv.}] $DE$ is perpendicular to $CD$, which is coplanar
and concurrent with $AD$ and $BD$. Again, since $AB$
and $CD$ are parallel, the sum of the angles $ABD$, $BDC$
is two right angles [I.~\textsc{xxix.}]; but $ABD$ is right (hyp.);
therefore $BDC$ is right. Hence $CD$ is perpendicular to
the two lines $DB$, $DE$, \textit{and therefore} [XI\@.~\textsc{iv.}] \textit{it is normal
to their plane, that is, it is normal to $X$.}

\myprop{PROP\@.~IX---Theorem.}{Two right lines $(AB,\ CD)$ which are each parallel to a
third line $(EF)$ are parallel to one another.}


\imgflow{133}{7}{f223}

\textbf{Dem.}---If the three lines be coplanar, the Proposition
is evidently the same
as I.~\textsc{xxx}. If they are not
coplanar, from any point $G$
in $EF$ draw in the planes of
$EF$, $AB$; $EF$, $CD$, respectively,
the lines $GH$, $GK$ each
perpendicular\label{perpendiclar}
to $EF$ [I.~\textsc{xi.}].
Then because $EF$ is perpendicular to each of the lines
$GH$, $GK$, it is normal to their plane [XI\@.~\textsc{iv.}]. And
because $AB$ is parallel to $EF$ (hyp.), and $EF$ is normal
to the plane $GHK$, $AB$ is normal to the plane $GHK$

[XI\@.~\textsc{viii.}]. In like manner $CD$ is normal to the plane
$HGK$. \emph{Hence, since $AB$} and $CD$ \emph{are normals to the same
plane, they are parallel to one another.}

\mypropl{PROP\@.~X.---Theorem.}{If two intersecting right lines $(AB,\ BC)$ be respectively
parallel to two other intersecting right lines $(DE,\ EF)$,
the angle $(ABC)$ between the former is equal to the angle
$(DEF)$ between the latter.}


\imgflow{120}{10}{f224}

\textbf{Dem.}---If both pairs of lines be coplanar, the proposition
is the same as I.~\textsc{xxix.},
Ex.~2. If not, take any points
$A$, $C$ in the lines $AB$, $BC$, and
cut off $ED = BA$, and $EF = BC$
[I.~\textsc{iii.}]. Join $AD$, $BE$, $CF$,
$AC$, $DF$. Then because $AB$ is
equal and parallel to $DE$, $AD$
is equal and parallel to $BE$
[I.~\textsc{xxxiii}]. In like manner
$CF$ is equal and parallel to $BE$.
Hence [XI\@.~\textsc{ix.}] $AD$ is equal
and parallel to $CF$. Hence [I.~\textsc{xxxiii.}] $AC$ is equal to
$DF$. Therefore the triangles $ABC$, $DEF$, have the
three sides of one respectively equal to the three sides
of the other. \emph{Hence} [I.~\textsc{viii.}] \emph{the angle $ABC$ is equal
to $DEF$.}

\textsc{Def.~viii.}---\emph{Two planes which meet are perpendicular
to each other, when the right lines drawn in one of them
perpendicular to their common section are normals to the
other.}

\textsc{Def.~ix.}---\emph{When two planes which meet are not perpendicular
to each other, their inclination is the acute angle
contained by two right lines drawn from any point of their
common section at right angles to it---one in one plane, and
the other in the other.}

\smallskip
\begin{footnotesize}
\textsf{Observation.}---These definitions tacitly assume the result of
Props.~\textsc{iii.}\ and \textsc{x.}\ of this book. On this account we have departed
from the usual custom of placing them at the beginning of the
book. We have altered the place of Definition \textsc{vi.} for a similar
reason.
\par\end{footnotesize}



\myprop{PROP\@.~XI\@.---Problem.}{To draw a normal to a given plane $(BH)$ from a given
point $(A)$ not in it.}


\imgflow{150}{9}{f225}

\textbf{Sol.}---In the given plane $BH$ draw any line $BC$, and
from $A$ draw $AD$ perpendicular to
$BC$ [I.~\textsc{xii}.];
then if $AD$ be perpendicular
to the plane, the
thing required is done. If
not, from $D$ draw $DE$ in
the plane $BH$ at right
angles to $BC$ [I.~\textsc{xi}.], and
from $A$ draw $AF$ [I.~\textsc{xii}.]
perpendicular to $DE$. $AF$ \emph{is normal to the plane $BH$.}

\textbf{Dem.}---Draw $GH$ parallel to $BC$. Then because $BC$
is perpendicular both to $ED$ and $DA$, it is normal to
the plane of $ED$, $DA$ [XI\@.~\textsc{iv}.]; and since $GH$ is parallel
to $BC$, it is normal to the same plane [XI\@.~\textsc{viii}.]. Hence
$AF$ is perpendicular to $GH$ [XI\@.~Def.~\textsc{vi}.], and $AF$ is
perpendicular to $DE$ (const.). \emph{Therefore $AF$ is normal
to the plane of $GH$ and $ED$---that is, to the plane $BH$.}


\mypropflow{PROP\@.~XII\@.---Problem.}{To draw a normal to a given plane from a given point $(A)$
in the plane.}{150}{10}{f226}

\textbf{Sol.}---From any point $B$ not in the plane draw
[XI\@.~\textsc{xi}.] $BC$ normal to it. If
this line pass through $A$ it is the
normal required. If not, from
$A$ draw $AD$ parallel to $BC$
[I.~\textsc{xxxi}.]. Then because $AD$
and $BC$ are parallel, and $BC$ is
normal to the plane, $AD$ is also
normal to it [XI\@.~\textsc{viii}.], and it
is drawn from the given point.
\emph{Hence it is the required normal.}




\mypropflow{PROP\@.~XIII\@.---Theorem.}{From the same point $(A)$ there can be but one normal
drawn to a given plane $(X)$.}{150}{8}{f227}

\textbf{Dem.}---1. Let $A$ be in the given plane, and if possible
let $AB$, $AC$ be both
normals to it, on the same
side. Now let the plane
of $BA$, $AC$ cut the given
plane $X$ in the line $DE$.
Then because $BA$ is a normal,
the angle $BAE$ is
right. In like manner
$CAE$ is right. Hence $BAE = CAE$, which is impossible.

2.~If the point be above the plane, there can be but
one normal; for, if there could be two, they would be
parallel [XI\@.~\textsc{vi.}] to one another, which is absurd.
\emph{Therefore from the same point there can be drawn but one
normal to a given plane.}


\mypropflow{PROP\@.~XIV\@.---Theorem.}{Planes $(CD,\ EF)$ which have a common normal $(AB)$ are
parallel to each other.}{155}{16}{f228}

\textbf{Dem.}---If the planes
be not parallel, they will
meet when produced. Let
them meet, their common
section being the line
$GH$, in which take any
point $K$. Join $AK$, $BK$.
Then because $AB$ is normal
to the plane $CD$, it
is perpendicular to the
line $AK$, which it meets
in that plane [XI\@. Def.~\textsc{vi.}].
Therefore the angle
$BAK$ is right. In like
manner the angle $ABK$
is right. Hence the
plane triangle $ABK$ has
two right angles, which is impossible. \emph{Therefore

the planes $CD$, $EF$ cannot meet---that is, they are parallel.}

\exhead{Exercises.}

\begin{footnotesize}
1.~The angle between two planes is equal to the angle between
two intersecting normals to these planes.

2.~If a line be parallel to each of two planes, the sections
which any plane passing through it makes with them are parallel.

3.~If a line be parallel to each of two intersecting planes, it is
parallel to their intersection.

4.~If two right lines be parallel, they are parallel to the common
section of any two planes passing through them.

5.~If the intersections of several planes be parallel, the normals
drawn to them from any point are coplanar.
\par\end{footnotesize}

\mypropl{PROP\@.~XV\@.---Theorem.}{Two planes $(AC,\ DF)$ are parallel, if two intersecting
lines $(AB,\ BC)$ on one of them be respectively parallel to
two intersecting lines $(DE,\ EF)$ on the other.}

\textbf{Dem.}---From $B$ draw $BG$ perpendicular to the plane
$DF$ [XI\@.~\textsc{xi.}], and let it meet that plane in $G$. Through
$G$ draw $GH$ parallel to $ED$, and $GK$ to $EF$. Now,
since $GH$ is parallel to $ED$ (const.), and $AB$ to $ED$
(hyp.), $AB$ is parallel to $GH$ [XI\@.~\textsc{ix.}]. Hence the sum
of the angles $ABG$, $BGH$ is two right angles [I.~\textsc{xxix}];

but $BGH$ is right (const.); therefore $ABG$ is right. In
like manner $CBG$ is right. Hence $BG$ is normal to
the plane $AC$ [XI\@.\ Def.~\textsc{vi.}], and it is normal to $DF$
(const.). Hence the planes $AC$, $DF$ have a common
normal $BG$; \emph{therefore they are parallel to one another.}


\imgcent{200}{f229}


\mypropflow{PROP\@.~XVI\@.---Theorem.}{If two parallel planes $(AB,\ CD)$ be cut by a third plane
$(EF,\ HG)$, their common sections $(EF,\ GH)$ with it are
parallel.}{172}{14}{f230}

\textbf{Dem.}---If the lines $EF$, $GH$ are not parallel, they
must meet at some
finite distance. Let
them meet in $K$. Now
since $K$ is a point in
the line $EF$, and $EF$
is in the plane $AB$,
$K$ is in the plane $AB$.
In like manner $K$ is a
point in the plane $CD$.
Hence the planes $AB$,
$CD$ meet in $K$, which
is impossible, since
they are parallel.
\emph{Therefore the lines $EF$,
$GH$ must be parallel.}


\exhead{Exercises.}

\begin{footnotesize}
1.~Parallel planes intercept equal segments on parallel lines.

2.~Parallel lines intersecting the same plane make equal angles
with it.

3.~A right line intersecting parallel planes makes equal angles
with them.
\par\end{footnotesize}


\mypropl{PROP\@.~XVII\@.---Theorem.}{If two parallel lines $(AB,\ CD)$ be cut by three parallel
planes $(GH,\ KL,$\ $MN)$ in two triads of points $(A,\ E,\ B;\ C,\ F,\ D)$,
their segments between those points are proportional.}


\imgflow{130}{15}{f231}

\textbf{Dem.}---Join $AC$, $BD$, $AD$. Let $AD$ meet the plane
$KL$ in $X$. Join $EX$, $XF$.
Then because the parallel
planes $KL$, $MN$ are cut by
the plane $ABD$ in the lines
$EX$, $BD$, these lines are
parallel [XI\@.~\textsc{xvi.}]. Hence
\[
AE:EB::AX:XD\ \text{[VI\@.~\textsc{ii.}].}
\]
In like manner,
\[
AX: XD:: CF: FD.
\]
Therefore [V.~\textsc{xi.}]
\[
AE:EB::CF:FD.
\]


\mypropl{PROP\@.~XVIII\@.---Theorem.}{If a right line $(AB)$ be normal to a plane $(CI)$, \emph{any}
plane $(DE)$ passing through it shall be perpendicular to
that plane.}


\imgflow{160}{9}{f232}

\textbf{Dem.}---Let $CE$ be the common section of the planes
$DE$, $CI$. From any point
$F$ in $CE$ draw $FG$ in
the plane $DE$ parallel
to $AB$ [I.~\textsc{xxxi.}]. Then
because $AB$ and $FG$ are
parallel, but $AB$ is normal,
to the plane $CI$;
hence $FG$ is normal to it
[XI\@.~\textsc{viii.}]. Now since
$FG$ is parallel to $AB$, the angles $ABF$, $BFG$ are equal

to two right angles [I.~\textsc{xxix.}]; but $ABF$ is right
(hyp.); therefore $BFG$ is right---that is, $FG$ is perpendicular
to $CE$. Hence every line in the plane $DE$,
drawn perpendicular to the common section of the
planes $DE$, $CI$, is normal to the plane $CI$. \emph{Therefore}
[XI\@.\ Def.~\textsc{viii.}] \emph{the planes $DE$, $CI$ are perpendicular to
each other.}



\mypropflot{PROP\@.~XIX\@.---Theorem.}{If two intersecting planes $(AB,\ BC)$ be each perpendicular
to a third plane $(ADC)$, their common section $(BD)$
shall be normal to that plane.}{115}{11}{f233}

\textbf{Dem.}---If not, draw from $D$ in the plane $AB$ the
line $DE$ perpendicular to $AD$,
the common section of the planes
$AB$, $ADC$; and in the plane $BC$
draw $BF$ perpendicular to the
common section $DC$ of the planes
$BC$, $ADC$. Then because the
plane $AB$ is perpendicular to
$ADC$, the line $DE$ in $AB$ is
normal to the plane $ADC$ [XI\@.\ Def.~\textsc{viii.}].
In like manner $DF$
is normal to it. Therefore from
the point $D$ there are two distinct normals to the
plane $ADC$, which [XI\@.~\textsc{xiii.}] is impossible. \emph{Hence
$BD$ must be normal to the plane $ADC$.}


\exhead{Exercises.}

\begin{footnotesize}
1.~If three planes have a common line of intersection, the normals
drawn to these planes from any point of that line are coplanar.

2.~If two intersecting planes be respectively perpendicular to
two intersecting lines, the line of intersection of the former is
normal to the plane of the latter.

3.~In the last case, show that the dihedral angle between the
planes is equal to the rectilineal angle between the normals.
\par\end{footnotesize}


\myprop{PROP\@.~XX\@.---Theorem.}{The sum of any two plane angles $(BAD,\ DAC)$ of a
trihedral angle $(A)$ is greater than the third $(BAC)$.}


\imgflow{150}{8}{f234}

\textbf{Dem.}---If the third angle $BAC$ be less than or equal
to either of the other
angles the proposition is
evident. If not, suppose it
greater: take any point
$D$ in $AD$, and at the
point $A$ in the plane $BAC$
make the angle $BAE$
equal $BAD$ [I.~\textsc{xxiii.}],
and cut off $AE$ equal $AD$. Through $E$ draw $BC$, cutting
$AB$, $AC$ in the points $B$, $C$. Join $DB$, $DC$.

Then the triangles $BAD$, $BAE$ have the two sides
$BA$, $AD$ in one equal respectively to the two sides
$BA$, $AE$ in the other, and the angle $BAD$ equal to
$BAE$; therefore the third side $BD$ is equal to $BE$.
But the sum of the sides $BD$, $DC$ is greater than $BC$;
hence $DC$ is greater than $EC$. Again, because the
triangles $DAC$, $EAC$ have the sides $DA$, $AC$ respectively
equal to the sides $EA$, $AC$ in the other, but the
base $DC$ greater than $EC$; therefore [I.~\textsc{xxv.}] the angle
$DAC$ is greater than $EAC$, but the angle $DAB$ is equal
to $BAE$ (const.). \emph{Hence the sum of the angles $BAD$,
$DAC$  is greater than the angle $BAC$.}


\mypropflot{PROP\@.~XXI\@.---Theorem.}{The sum of all the plane angles
$(BAC,$\ $CAD,$\ $\&\mathrm{c}.)$ forming
any solid angle $(A)$ is less than four right angles.}{128}{10}{f235}

\textbf{Dem.}---Suppose for simplicity that the solid angle\index{Angle!solid} $A$
is contained by five plane angles $BAC$, $CAD$, $DAE$,
$EAF$, $FAB$; and let the planes of these angles be cut

by another plane in the lines $BC$, $CD$, $DE$, $EF$, $FB$;
then we have [XI\@.~\textsc{xx.}],
\begin{center}
\begin{tabular}{r@{\ }c@{\ }l}
$\angle ABC + ABF$ & greater than & $FBC$, \\[1ex]
$\angle ACB + ACD$ &      ,,      & $BCD$, \&c.
\end{tabular}
\end{center}
Hence, adding, we get the sum
of the base angles of the five
triangles $BAC$, $CAD$, \&c.,
greater than the sum of the
interior angles of the pentagon
$BCDEF$ ---that is, greater than
six right angles. But the sum
of the base angles of the same
triangles, together with the
sum of the plane angles $BAC$,
$CAD$, \&c., forming the solid
angle $A$, is equal to twice as
many right angles as there are triangles $BAC$, $CAD$,
\&c.---that is, equal to ten right angles. \emph{Hence the sum
of the angles forming the solid angle is less than four
right angles.}

\smallskip
\begin{footnotesize}
\textsf{Observation.}---This Prop.\ may not hold if the polygonal base
$BCDEF$ contain \emph{re-entrant} angles.
\par\end{footnotesize}

\addcontentsline{toc}{section}{Exercises,}
\exhead{Exercises on Book XI\@.}

\begin{footnotesize}
1.~Any face angle of a trihedral angle is less than the sum, but
greater than the difference, of the supplements of the other two
face angles.

2.~A solid angle cannot be formed of equal plane angles which
are equal to the angles of a regular polygon of $n$ sides, except in
the case of $n = 3,\ 4, \text{\ or } 5$.

3.~Through one of two non-coplanar lines draw a plane parallel
to the other.

4.~Draw a common perpendicular to two non-coplanar lines,
and show that it is the shortest distance between them.

5.~If two of the plane angles of a tetrahedral angle be equal,
the planes of these angles are equally inclined to the plane of the
third angle, and conversely. If two of the planes of a trihedral
angle be equally inclined to the third plane, the angles contained
in those planes are equal.


6.~The three lines of intersection of three planes are either
parallel or concurrent.

7.~If a trihedral angle $O$ be formed by three right angles, and
$A$, $B$, $C$ be points along the edges, the orthocentre of the triangle
$ABC$ is the foot of the normal from $O$ on the plane $ABC$.

8.~If through the vertex $O$ of a trihedral angle $O$---$ABC$ any
line $OD$ be drawn interior to the angle, the sum of the rectilineal
angles $DOA$, $DOB$, $DOC$ is less than the sum, but greater than
half the sum, of the face angles of the trihedral.

9.~If on the edges of a trihedral angle $O$---$ABC$ three equal
lines $OA$, $OB$, $OC$ be taken, each of these is greater than the
radius of the circle described about the triangle $ABC$.

10.~Given the three angles of a trihedral angle, find, by a plane
construction, the angles between the containing planes.

11.~If any plane $P$ cut the four sides of a \emph{Gauche} quadrilateral
(a quadrilateral whose angular points are not coplanar) $ABCD$ in
four points, $a$, $b$, $c$, $d$, then the product of the four ratios
\[
  \frac{Aa}{aB},\quad
  \frac{Bb}{bC},\quad
  \frac{Cc}{cD},\quad
  \frac{Dd}{dA}
\]
is plus unity, and conversely, if the product
\[
  \frac{Aa}{aB} \,.\,
  \frac{Bb}{bC} \,.\,
  \frac{Cc}{cD} \,.\,
  \frac{Dd}{dA} = +1,
\]
the points $a$, $b$, $c$, $d$ are coplanar.

12.~If in the last exercise the intersecting plane be parallel to
any two sides of the quadrilateral, it cuts the two remaining sides
proportionally.

\textsc{Def.~x.}---If at the vertex $O$ of a trihedral angle $O$---$ABC$ we
draw normals $OA'$, $OB'$, $OC'$ to the faces $OBC$, $OCA$, $OAB$, respectively,
in such a manner that $OA'$ will be on the same side
of the plane $OBC$ as $OA$, \&c., the trihedral angle $O$---$A'B'C'$ is
called the supplementary\index{Trihedral angle!supplementary of} of the trihedral angle $O$---$ABC$.

13.~If $O$---$A'B'C'$ be the supplementary of $O$---$ABC$, prove that
$O$---$ABC$ is the supplementary of $O$---$A'B'C'$.

14.~If two trihedral angles be supplementary, each dihedral
angle of one is the supplement of the corresponding face angle of
the other.

15.~Through a given point draw a right line which will meet
two non-coplanar lines.

16.~Draw a right line parallel to a given line, which will meet
two non-coplanar lines.

17.~Being given an angle $AOB$, the locus of all the points $P$
of space, such that the sum of the projections of the line $OP$ on
$OA$ and $OB$ may be constant, is a plane.

\par\end{footnotesize}


\mybook{APPENDIX}{Prism, Pyramid, Cylinder, Sphere, and Cone}

\medskip
\textsc{i.} A \textit{polyhedron}\index{Polyhedron} is a solid figure contained by plane
figures: if it be contained by four plane figures it is
called a \textit{tetrahedron}; by six, a \textit{hexahedron}\index{Hexahedron}; by eight, an
\textit{octahedron}\index{Octahedron}; by twelve, a \textit{dodecahedron}\index{Dodecahedron}; and if by
twenty, an \textit{icosahedron}\index{Icosahedron}.

\textsc{ii.} If the plane faces of a polyhedron be equal and
similar rectilineal figures, it is called a \textit{regular polyhedron}.

\textsc{iii.} A \textit{pyramid} is a polyhedron of which all the faces
but one meet in a point. This point is called the
\textit{vertex}; and the opposite face, the \textit{base}.

\textsc{iv.} A \textit{prism}\index{Prism} is a polyhedron having a pair of parallel
faces which are equal and similar rectilineal figures, and
are called its \textit{ends}. The others, called its \textit{side faces}, are
parallelograms.

\textsc{v.} A prism whose ends are perpendicular to its sides
is called a \textit{right} prism; any other is called an \textit{oblique}
prism.

\textsc{vi.} The \textit{altitude} of a pyramid is the length of the
perpendicular drawn from its vertex to its base; and
the altitude of a prism is the perpendicular distance
between its ends.

\textsc{vii.} A \textit{parallelopiped}\index{Parallelopiped} is a prism whose bases are
parallelograms. A parallelopiped is evidently a \textit{hexahedron}.



\textsc{viii.} A \textit{cube}\index{Cube} is a rectangular parallelopiped, all whose
sides are squares.

\textsc{ix.} A \textit{cylinder}\index{Cylinder} is a solid figure formed by the revolution
of a rectangle about one of its sides, which remains
fixed, and which is called its \textit{axis}. The circles which
terminate a cylinder are called its \textit{bases} or \textit{ends}.

\textsc{x.} A \textit{cone}\index{Cone} is the solid figure described by the revolution
of a right-angled triangle about one of the legs,
which remains fixed, and which is called the \textit{axis}. The
other leg describes the \textit{base}, which is a circle.

\textsc{xi.} A \textit{sphere}\index{Sphere} is the solid described by the revolution
of a semicircle about a diameter, which remains fixed.
The \textit{centre} of the sphere is the centre of the \textit{generating}
semicircle. Any line passing through the centre of
a sphere and terminated both ways by the surface is
called a \textit{diameter}.

\addcontentsline{toc}{section}{Propositions \textsc{i.--vii.,}}
\mypropl{PROP\@.~I.---Theorem.}{Right prisms $(ABCDE$--$FGHIJ,\ A'B'C'D'E'$--$F'G'H'I'J')$ which
have ba\-ses $(ABCDE,\ A'B'C'D'E')$ that are equal and similar,
and which have equal altitudes, are equal.}


\imgflow{175}{12}{f236}

\textbf{Dem.}---Apply the bases to each other; then, since
they are equal and
similar figures, they
will coincide---that
is, the point $A$ with
$A'$, $B$ with $B'$, \&c.
And since $AF$ is
equal to $A'F'$, and
each is normal to
its respective base,
the: point $F$ will
coincide with $F'$. In
the same manner the
points $G$, $H$, $I$, $J$ will coincide respectively with the
points $G'$, $H'$, $I'$, $J'$. \textit{Hence the prisms are equal in every
respect.}



\textit{Cor}.~1.---\textit{Right prisms which have equal bases $(EF,\ E'F')$
and equal altitudes are equal in volume.}


\imgcent{213}{f237}

\textbf{Dem.}---Since the bases are equal, but not similar, we
can suppose one of them, $EF$, divided into parts $A$, $B$, $C$,
and re-arranged so as to make them coincide with the
other [I.~\textsc{xxxv.}, note]; and since the prism on $E'F'$ can
be subdivided in the same manner by planes perpendicular
to the base, the proposition is evident.

\textit{Cor}.~2.---\textit{The volumes of right prisms $(X,\ Y)$ having
equal bases are proportional to their altitudes.}

For, if the altitudes be in the ratio of $m : n$, $X$ can
be divided into $m$ prisms of equal altitudes by planes
parallel to the base; then these $m$ prisms will be all
equal. In like manner, $Y$ can be divided into $n$ equal
prisms. \textit{Hence $X : Y :: m : n$.}

\textit{Cor}.~3.---In right prisms of equal altitudes the
volumes are to one another as the areas of their bases.
This may be proved by dividing the bases into parts
so that the subdivisions will be equal, and then the
volumes proportional to the number of subdivisions in
their respective bases, that is, to their areas.

\textit{Cor}.~4.---The volume of a rectangular parallelopiped
is measured by the continued product of its three dimensions.

\myprop{PROP\@.~II\@.---Theorem.}{Parallelopipeds $(ABCD$--$EFGH,\ ABCD$--$MNOP)$, having
a common base $(ABCD)$ and equal altitudes, are equal.}


\imgcent{250}{f238}

$1^\circ$. Let the edges $MN$, $EF$ be in one right line; then
$GH$, $OP$ must be in one right line. Now $EF = MN$,

because each equal $AB$; therefore $ME=NF$; therefore
the prisms $AEM$--$DHP$, and $BFN$--$CGO$, have their
triangular bases $AEM$, $BFN$ identically equal, and
they have equal altitudes; hence they are equal; and
supposing them taken away from the entire solid, \emph{the
remaining parallelopipeds} $ABCD$--$EFGH$, $ABCD$--$MNOP$
\emph{are equal}.


\imgcent{280}{f239}

$2^{\circ}$. Let the edges $EF$, $MN$ be in different lines;
then produce $ON$, $PM$ to meet the lines $EF$ and $GH$
produced in the points $J$, $K$, $L$, $I$. Then by $1^{\circ}$ the
parallelopipeds $ABCD$--$EFGH$, $ABCD$--$MNOP$ are each
equal to the parallelopiped $ABCD$--$IJKL$. \emph{Hence they
an equal to one another}.

\emph{Cor.}---The volume of any parallelopiped is equal to
the product of its base and altitude.



\myprop{PROP\@.~III\@.---Theorem.}{A diagonal plane of a parallelopiped divides it into two
prisms of equal volume.}

$1^\circ$. When the parallelopiped is rectangular the proposition
is evident.

$2^\circ$. When it is any parallelopiped, $ABCD$--$EFGH$,
the diagonal plane bisects it.


\imgflow{118}{17}{f240}

\textbf{Dem.}---Through the vertices $A$, $E$ let planes be
drawn perpendicular to the edges
and cutting them in the points
$I$, $J$, $K$; $L$, $M$, $N$, respectively.
Then [I.~\textsc{xxxiv.}] we have $IL
= BF$, because each is equal to
$AE$. Hence $IB = LF$. In like
manner $JC = MG$. Hence the
pyramid $A$--$IJCB$ agrees in
everything but position with
$E$--$LMGF$; hence it is equal to
it in volume. To each add the
solid $ABC$--$LME$, and we have
the prism $AIJ$--$ELM$ equal to
the prism $ABC$--$EFG$. In like
manner $AJK$--$EMN=ACD$--$EGH$;
but ($1^\circ$) the prism $AIJ$--$ELM
= AJK$--$EMN$. Hence $ABC$--$EFG = ACD$--$EGH$.
\emph{Therefore the diagonal plane bisects the parallelopiped.}

\emph{Cor.}~1.---The volume of a triangular prism is equal
to the product of its base and altitude; because it is
half of a parallelopiped, which has a double base and
equal altitude.

\emph{Cor.}~2.---The volume of any prism is equal to the
product of its base and altitude; because it can be
divided into triangular prisms.



\myprop{PROP\@.~IV\@.---Theorem.}{If a pyramid $(O$--$ABCDE)$ be cut by any plane $(abcde)$
parallel to the base, the section is similar to the base.}


\imgflow{110}{14}{f241}

\textbf{Dem.}---Because the plane $AOB$ cuts the parallel
planes $ABCDE$, $abcde$, the sections
$AB$, $ab$ are parallel [XI\@.~\textsc{xvi.}]
In like manner $BC$, $bc$ are parallel.
Hence the angle $ABC = abc$
[XI\@.~\textsc{x.}]. In like manner the
remaining angles of the polygon
$ABCDE$ are equal to the corresponding
angles of $abcde$.
Again, because $ab$ is parallel to
$AB$, the triangles $ABO$, $abO$\label{wrongo} are
equiangular.\\[-\baselineskip]
\begin{flalign*}
&\text{\indent Hence }&
  AB &: BO:: ab: bO.\ \text{[VI\@.~\textsc{iv.}]}  &&\\
&\text{In like manner }&
  BO &: BC:: bO: bc;  && \\
&\text{therefore }&
  AB &: BC:: ab: bc.\ \text{[\emph{Ex {\ae}quali.}]}  &&\\
&\text{In like manner }&
  BC &: CD:: bc: cd,\ \text{\&c.}  &&
\end{flalign*}

Therefore the polygons $ABCDE$, $abcde$ are equiangular,
and have the sides about their equal angles proportional.
\emph{Hence they are similar.}

\emph{Cor.}~1.---The edges and the altitude of the pyramid
are similarly divided by the parallel plane.

\emph{Cor.}~2.---The areas of parallel sections are in the
duplicate ratio of the distances of their planes from the
vertex.

\emph{Cor.}~3.---In any two pyramids, sections parallel to
their bases, which divide their altitudes in the same
ratio, are proportional to their bases.



\myprop{PROP\@.~V.---Theorem.}{Pyramids $(P$--$ABCD,\ p$--$abc)$, having equal altitudes
$(PO,\ po)$ and bases $(ABCD,\ abc)$ of equal areas, have equal
volumes.}


\imgcent{298}{f242}

\textbf{Dem.}---If they be not equal in volume, let $abc$ be
the base of the greater; and let $ox$ be the altitude of a
prism, with an equal base, and whose volume is equal
to their difference; then let the equal altitudes $PO$, $po$
be divided into such a number of equal parts, by planes
parallel to the bases of the pyramids, that each part
shall be less than $ox$. Then [\textsc{iv.}\ \emph{Cor.}~3] the sections
made by these planes will be equal each to each. Now
let prisms be constructed on these sections as bases and
with the equal parts of the altitudes of the pyramids as
altitudes, and let the prisms in $P$--$ABCD$ be constructed
below the sections, and in $p$--$abc$, above. Then it is
evident that the sum of the prisms in $P$--$ABCD$ is
less than that pyramid, and the sum of those on the
sections of $p$--$abc$ greater than $p$--$abc$. Therefore the

difference between the pyramids is less than the difference
between the sums of the prisms, that is, less than
the lower prism in the pyramid $p$--$abc$; but the altitude
of this prism is less than $ox$ (const.). Hence the difference
between the pyramids is less than the prism
whose base is equal to one of the equal bases, and
whose altitude is equal to $ox$, and the difference is
equal to this prism (hyp.), which is impossible. \emph{Therefore
the volumes of the pyramids are equal.}


\imgflow{113}{13}{f243}

\emph{Cor.}~1.---\textit{The volume of a triangular pyramid $E$--$ABC$
is one third the volume of the prism
$ABC$--$DEF$, having the same base
and altitude.}

For, draw the plane $EAF$, then
the pyramids $E$--$AFC$, $E$--$AFD$
are equal, having equal bases $AFC$,
$AFD$, and a common altitude; and
the pyramids $E$--$ABC$, $F$--$ABC$ are
equal, having a common base and
equal altitudes. Hence the pyramid
$E$--$ABC$ is one of three
equal pyramids into which the
prism is divided. \emph{Therefore it is
one third of the prism.}

\emph{Cor.}~2.---\textit{The volume of every pyramid is one-third of
the volume of a prism having an equal base and altitude.}

Because it may be divided into triangular pyramids
by planes through the vertex and the diagonals of the
base.



\myprop{PROP\@.~VI\@.---Theorem.}{The volume of a cylinder\index{Cylinder} is equal to the product of the
area of its base by its altitude}.


\imgflow{138}{11}{f244}

\textbf{Dem}.---Let $O$ be the centre of its circular base; and
take the angle $AOB$ indefinitely
small, so that the arc
$AB$ may be regarded as a
right line. Then planes perpendicular
to the base, and
cutting it in the lines $OA$,
$OB$, will be faces of a triangular
prism, whose base
will be the triangle $AOB$,
and whose altitude will be
the altitude of the cylinder.
The volume of this prism will be equal to the area of
the triangle $AOB$ by the height of the cylinder. Hence,
dividing the circle into elementary triangles, the cylinder
will be equal to the sum of all the prisms, \textit{and
therefore its volume will be equal to the area of the base
multiplied by the altitude}.

\textit{Cor}.~1.---If $r$ be the radius, and $h$ the height of the
cylinder,
\[
\text{vol.\ of cylinder} = \pi r^2 h.
\]


\imgflow{135}{8}{f245}

\textit{Cor}.~2.---\textit{If $ABCD$ be a rectangle; $X$ a line in its plane
parallel to the side $AB$; $O$ the
middle point of the rectangle;
the volume of the solid described
by the revolution of
$ABCD$ round $X$ is equal to
the area of $ABCD$ multiplied
by the circumference of the
circle described by $O$.}

\textbf{Dem}.---Produce the lines $AD$, $BC$ to meet $X$ in the
points $E$, $F$. Then when the rectangle revolves round
$X$, the rectangles $ABFE$, $DCFE$ will describe cylinders
whose bases will be circles having $AE$, $DE$ as radii,

and whose common altitude will be $AB$. Hence the
difference between the volumes of these cylinders will
be equal to the differences between the areas of the bases
multiplied by $AB$, that is $= \pi (AE^{2} - DE^{2})\,.\,AB$.
Therefore the volume described by $ABCD$
\begin{flalign*}
&&          &= \pi\,.\,AB\,.\,(AE + DE)(AE - DE);  &&\phantom{but }\\
&\text{but }&
    AE + DE &= 2 OG,\ \text{and}\ AE-DE = AD.  &&
\end{flalign*}
Hence volume described by the rectangle $ABCD$

\hspace{.25\textwidth}$= 2\pi\,.\,OG\,.\,AB\,.\,AD$.


\hspace{.25\textwidth}$=$ rectangle $ABCD$ multiplied by the
circumference of the circle described by its middle
point $O$.

\smallskip
\begin{footnotesize}
\textsf{Observation.}---This Cor.\ is a simple case of Guldinus's celebrated
theorem. By its assistance we give in the two following
corollaries original methods of finding the volumes of the cone and
sphere, and it may be applied with equal facility to the solution
of several other problems which are usually done by the Integral
Calculus.
\par\end{footnotesize}

\smallskip
\emph{Cor.}~3.---\textit{The volume of a cone is one-third the volume
of a cylinder having the same base and altitude.}


\imgflow{160}{9}{f246}

\textbf{Dem.}---Let $ABCD$ be a rectangle whose diagonal is
$AC$. The triangle $ABC$ will describe a cone, and the
rectangle a cylinder by revolving round $AB$. Take two
points $E$, $F$ infinitely
near each other in $AC$,
and form two rectangles,
$EH$, $EK$, by drawing
lines parallel to $AD$, $AB$.
Now if $O$, $O'$ be the
middle points of these
rectangles, it is evident
that, when the whole
figure revolves round $AB$, the circumference of the
circle described by $O'$ will ultimately be twice the circumference
of the circle described by $O$; and since the
parallelogram $EK$ is equal to $EH$ [I.~\textsc{xliii}.], the solid
described by $EK$ (\emph{Cor}.~1) will be equal to twice the
solid described by $EH$. Hence, if $AC$ be divided into

an indefinite number of equal parts, and rectangles
corresponding to $EH$, $EK$ be inscribed in the triangles
$ABC$, $ADC$, the sum of the solids described by the
rectangles in the triangle $ADC$ is equal to twice the
sum of the solids described by the rectangles in the
triangle $ABC$---that is, the difference between the
cylinder and cone is equal to twice the cone. Hence
the cylinder is equal to three times the cone.

\begin{footnotesize}
\emph{Or thus:} We may regard the cone and the cylinder as limiting
cases of a pyramid and prism having the same base and altitude;
and since (v.\ \emph{Cor.}~2) the volume of a pyramid is one-third of the
volume of a prism, having the same base and altitude, the volume
of the cone is one-third of the volume of the cylinder.
\par\end{footnotesize}

\emph{Cor.}~4.---If $r$ be the radius of the base of a cone, and
$h$ its height,
\[
\text{vol.\ of cone} = \frac{ \pi r^2 h }{3}.
\]

\emph{Cor.}~5.---\textit{The volume of a sphere\index{Sphere} is two-thirds of the
volume of a circumscribed cylinder.}


\imgcent{200}{f247}

\textbf{Dem.}---Let $AB$ be the diameter of the semicircle
which describes the sphere; $ABCD$ the rectangle which

describes the cylinder. Take two points $E$, $F$ indefinitely
near each other in the semicircle. Join $EF$, which
will be a tangent, and produce it to meet the diameter
$PQ$ perpendicular to $AB$ in $N$. Let $R$ be the centre.
Join $RE$; draw $EG$, $FH$, $NL$ parallel to $AB$; and $EI$,
$FK$ parallel to $PQ$; and produce to meet $LN$ in $M$ and
$L$; and let $O$, $O'$ be the middle points of the rectangles
$EH$, $EK$.

Now the rectangle $NG\,.\,GR = PG\,.\,GQ$, because each
is equal to $GE^{2}$. Hence $NG : GP :: GQ : GR$, or
$ME : IE :: RP+RG : RG$. Now, denoting the radii of
the circles described by the points $O$, $O'$ by $\rho$, $\rho'$ respectively,
we have ultimately $\rho = GR$ and $\rho' =\frac{1}{2}(RP + RG)$.
Hence $ME : IE :: 2\rho' : \rho$; but $ME : IE ::$ rectangle
$EL:$ rectangle $EK ::$ [I.~\textsc{xliii.}] $EH : EK$;
\begin{align*}
&  \therefore EH : EK :: 2\rho' : \rho;  \\
&  \therefore 2\pi\rho \,.\,EH = 2(2\pi\rho' \,.\,EK).
\end{align*}
Hence the solid described by $EH$ equal twice the solid
described by $EK$. Therefore we infer, as in the last
Cor., that the whole volume of the sphere is equal to
twice the difference between the cylinder and sphere.
\emph{Therefore the sphere is two-thirds of the cylinder.}

\emph{Cor.}~6.---If $r$ be the radius of a sphere,
\[
\text{vol.} = \frac{4\pi r^3 }{3}.
\]

\myprop{PROP\@.~VII\@.---Theorem.}{The surface of a sphere\index{Sphere} is equal to the convex surface of
the circumscribed cylinder.}


\imgcent{190}{f248}

\textbf{Dem.}---Let $AB$ be the diameter of the semicircle
which describes the sphere. Take two points, $E$, $F$,
indefinitely near each other in the semicircle. Join
$EF$, and produce to meet the tangent $CD$ parallel to

$AB$ in $N$. Draw $EI$, $FK$ parallel to $PQ$. Produce $EI$
to meet $AB$ in $G$. Let $O$ be the centre. Join $OE$.\\[-\baselineskip]
\begin{flalign*}
&\text{Now we have }&             FE : KI :: EN :{}& IN\ \text{[VI\@.~\textsc{ii.}]};  &&\\
&\text{but }&       EN : IN ::{}& OE : EG,  &&\phantom{but }
\end{flalign*}
because the triangles $ENI$ and $OEG$ are similar.
\begin{flalign*}
&\text{Hence }&  FE : KI ::{}& OE : EG;  &&\phantom{Hence }\\
&\text{but }&          OE ={}& IG.  &&
\end{flalign*}
Hence $EF : IK :: IG : EG$; and $IG : EG ::$ circumference
of circle described by the point $I :$ circumference
of circle described by the point $E$. Hence the rectangle
contained by $EF$, and circumference of circle described
by $E$ is equal to the rectangle contained by $IK$, and circumference
of circle described by $I$---that is, the portion
of the spherical surface\index{Sphere} described by $EF$ is equal to the
portion of the cylindrical surface described by $IK$.
Hence it is evident, if planes be drawn perpendicular
to the diameter $AB$---that the portions of cylindrical and
spherical surface between any two of them are equal.
\emph{Hence the whole spherical surface is equal to the cylindrical
surface described by $CD$.}



\smallskip
\begin{footnotesize}
\emph{Or thus:} Conceive the whole surface of the sphere divided into
an indefinitely great number of equal parts, then it is evident
that each of these may be regarded as the base of a pyramid
having the centre of the sphere as a common vertex. Therefore
the volume of the sphere is equal to the whole area of the surface
multiplied by one-third of the radius. Hence if $S$ denote the
surface, we have\label{XIvi}
\begin{flalign*}
&&  S \times \frac{r}{3} &= \frac{4\pi r^3}{3}\ \text{[\textsc{vi.}, \emph{Cor.}~6]};  &&\phantom{therefore }\\
&\text{therefore }&  S &= 4\pi r^2.  &&
\end{flalign*}
That is, the area of the surface of a sphere is equal four times
the area of one of its great circles.
\end{footnotesize}

\addcontentsline{toc}{section}{Exercises,}
\exhead{Exercises.}

\begin{footnotesize}
1.~The convex surface of a cone is equal to half the rectangle
contained by the circumference of the base and the slant height.

2.~The convex surface of a right cylinder is equal to the rectangle
contained by the circumference of the base and the altitude.

3.~If $P$ be a point in the base $ABC$ of a triangular pyramid
$O$--$ABC$, and if parallels to the edges $OA$, $OB$, $OC$, through $P$,
meet the faces in the points $a$, $b$, $c$, the sum of the ratios
\[
  \frac{Pa}{OA}, \quad
  \frac{Pb}{OB}, \quad
  \frac{Pc}{OC} = 1.
\]

4.~The volume of the frustum of a cone, made by a plane
parallel to the base, is equal to the sum of the three cones whose
bases are the two ends of the frustum, and the circle whose
diameter is a mean proportional between the end diameters, and
whose common altitude is equal to one-third of the altitude of the
frustum.

5.~If a point $P$ be joined to the angular points $A$, $B$, $C$, $D$ of a
tetrahedron, and the joining lines, produced if necessary, meet the
opposite faces in $a$, $b$, $c$, $d$, the sum of the ratios
\[
  \frac{Pa}{Aa}, \quad
  \frac{Pb}{Bb}, \quad
  \frac{Pc}{Cc}, \quad
  \frac{Pd}{Dd} = 1.
\]


6.~The surface of a sphere is equal to the rectangle by its
diameter, and the circumference of a great circle.

7.~The surface of a sphere is two thirds of the whole surface
of its circumscribed cylinder.

8.~If the four diagonals of a quadrangular prism be concurrent,
it is a parallelopiped.

9.~If the slant height of a right cone be equal to the diameter
of its base, its total surface is to the surface of the inscribed sphere
as $9 : 4$.

10.~The middle points of two pairs of opposite edges of a
triangular pyramid are coplanar, and form a parallelogram.

11.~If the four perpendiculars from the vertices on the opposite
faces of a pyramid $ABCD$ be concurrent, then
\[
AB^2 + CD^2 = BC^2 + AD^2 = CA^2 + BD^2.
\]

12.~Every section of a sphere by a plane is a circle.

13.~The locus of the centres of parallel sections is a diameter of
the sphere.

14.~If any number of lines in space pass through a fixed point,
the feet of the perpendiculars on them from another fixed point
are homospheric.

15.~Extend the property of Ex.~4 to the pyramid.

16.~The volume of the ring described by a circle which revolves
round a line in its plane is equal to the area of the circle, multiplied
by the circumference of the circle described by its centre.

17.~Any plane bisecting two opposite edges of a tetrahedron
bisects its volume.

18.~Planes which bisect the dihedral angles of a tetrahedron
meet in a point.

19.~Planes which bisect perpendicularly the edges of a
tetrahedron meet in a point.

20.~The volumes of two triangular pyramids, having a common
solid angle, are proportional to the rectangles contained by the
edges terminating in that angle.

21.~A plane bisecting a dihedral angle of a tetrahedron divides
the opposite edge into portions proportional to the areas containing
that edge.


22.~The volume of a sphere: the volume of the circumscribed
cube as $\pi : 6$.

23.~If $h$ be the height, and $\rho$ the radius of a segment of a
sphere, its volume is $\dfrac{\pi h}{6} (h^2 + 3 \rho^2)$.

24.~If $h$ be the perpendicular distance between two parallel
planes, which cut a sphere in sections whose radii are $\rho_1$, $\rho_2$, the
volume of the frustum is $\dfrac{\pi h}{6} \{h^2 + 3(\rho_1^2 + \rho_2^2)\}$.

25.~If $\delta$ be the distance of a point $P$ from the centre of a sphere
whose radius is $R$, the sum of the squares of the six segments at
three rectangular chords passing through $P$ is $= 6R^2 - 2\delta^2$.

26.~The volume of a sphere $:$ the volume of an inscribed cube
as $\pi : 2$.

27.~If $O$--$ABC$ be a tetrahedron whose angles $AOB$, $BOC$,
$COA$ are right, the square of the area of the triangle $ABC$ is equal
to the sum of the squares of the three other triangular faces.

28.~In the same case, if $p$ be the perpendicular from $O$ on the
face $ABC$,
\begin{equation*}
\frac{1} {p^2} = \frac{1} {OA^2} + \frac{1} {OB^2} + \frac{1} {OC^2}.
\end{equation*}

29.~If $h$ be the height of an {\ae}ronaut, and $R$ the radius of the
earth, the extent of surface visible $= \dfrac{2\pi R^2 h} {R + h}$.

30.~If the four sides of a gauche quadrilateral touch a sphere,
the points of contact are concyclic.
\par\end{footnotesize}


\newpage
\addtocontents{toc}{\bigskip\protect\centering {\large NOTES.}\medskip}
\addcontentsline{toc}{section}{\ibksp A.---Modern theory of parallel lines,}
\begin{center}
{\LARGE NOTES.}
\medskip

\rule[1ex]{1cm}{.2pt} \quad
\medskip

{\large NOTE A.}
\medskip

MODERN THEORY OF PARALLEL LINES.
\end{center}

\begin{footnotesize}
In every plane there is one special line called the \emph{line at infinity}.
The point where any other line in the plane cuts the line at infinity
is called the \emph{point at infinity} in that line. All other points in the
line are called \emph{finite points}. Two lines in the plane which meet
the line at infinity in the same point are said to have the \emph{same
direction}, and two lines which meet it in different points to have
\emph{different directions}. Two lines which have the same direction
cannot meet in any finite point [I.\ Axiom~\textsc{x.}], and are parallel.
Two lines which have different directions must intersect in some
finite point, since, if produced, they meet the line at infinity in
different points. This is a fundamental conception in Geometry,
it is self-evident, and may be assumed as an Axiom\index{Axioms} (\emph{see} Observations
on the Axioms, Book~I.). Hence we may infer the following
general proposition:---\emph{``Any two lines in the same plane must
meet in some point in that plane; that is---$(1)$~at infinity, when the
lines have the same direction; $(2)$~in some finite point, when they
have different directions."---\emph{See} \textsc{Poncelet},\index{Poncelet} Propri{\'e}t{\'e}s Projectives},
page~52.
\end{footnotesize}

\begin{center}
\medskip
\rule[1ex]{3cm}{.2pt}
\medskip

\addcontentsline{toc}{section}{\ibksp B.---Legendre's proof of Euclid, \textsc{i., xxxii.,}}
{\large NOTE B.}
\medskip

\textsc{legendre's and hamilton's proofs of euclid, I.~xxxii.}
\end{center}

\begin{footnotesize}
The discovery of the Proposition that ``the sum of the three
angles of a triangle is equal to two right angles'' is attributed to
Pythagoras. Until modern times no proof of it, independent of
the theory of parallels, was known. We shall give here two
demonstrations, each independent of that theory. These are due
to two of the greatest mathematicians of modern times---one, the

founder of the Theory of Elliptic Functions; the other, the discoverer
of the Calculus of Quaternions.


\imgcent{268}{f260}

\textsc{Legendre's Proof.}\index{Legendre}---Let $ABC$ be a triangle, of which the
side $AC$ is the greatest. Bisect $BC$ in $D$. Join $AD$. Then $AD$ is
less than $AC$ [I.~\textsc{xix.}\ Ex.~5]. Now, construct a new triangle
$AB'C'$, having the side $AC' = 2AD$, and $AB' = AC$. Again, bisect
$B'C'$ in $D'$, and form another triangle $AB''C''$, having $AC'' = 2AD'$,
and $AB'' = AC'$, \&c. (1) The sum of the angles of the triangle $ABC
=$ the sum of the angles of $AB'C'$ [I.~\textsc{xvi.} \emph{Cor.}~1] $=$ the sum of the
angles of $AB''C'' =$ the sum of the angles of $AB'''C'''$, \&c. (2) The
angle $B'AC'$ is less than half $BAC$; the angle $B''AC''$ is less than
half $B'AC'$, and so on; hence the angle $B^{(n)} A^{(n)}$ will ultimately
become infinitely small. (3) The sum of the base angles of any
triangle of the series is equal to the angle of the preceding triangle
(\emph{see} Dem.\ I.~\textsc{xvi.}). Hence, if the annexed diagram
represent the triangle $AB^{(n+1)}C^{(n+1)}$,
the sum of the base angles $A$ and $C^{(n+1)}$ is

\imgcent{278}{f261}
equal to the angle $B^{(n)}C^{(n)}$; and when $n$ is indefinitely large, this
angle is an infinitesimal; hence the point $B^{(n+1)}$ will ultimately be
in the line $AC$, and the angle $AB^{(n+1)}C^{(n+1)}$ will become a straight
angle [I.~Def.\ \textsc{x.}], that is, it is equal to two right angles; but the
sum of the angles of $AB^{(n+1)}C^{(n+1)}$ is equal to the sum of the
angles $ABC$. \emph{Hence the sum of the three angles of $ABC$ is equal
to two right angles.}


\imgcent{215}{f249}

\addcontentsline{toc}{section}{\ibksp\;,,\protect\phantom{.---}Hamilton's\protect\phantom{\ proof },,}
\textsc{Hamilton's\index{Hamilton} Quaternion Proof.}---Let $ABC$ be the triangle.
Produce $BA$ to $D$, and make $AD$ equal to $AC$. Produce
$CB$ to $E$, and make $BE$ equal to $BD$; finally, produce $AC$ to

$F$, and make $CF$ equal to $CE$. Denote the exterior angles thus
formed by $A'$, $B'$, $C'$. Now let the leg $AC$ of the angle $A'$
be turned round the point $A$ through the angle $A'$; then the point
$C$ will coincide with $D$. Again, let the leg $BD$ of the angle $B'$
be turned round the point $B$ through the angle $B'$, until $BD$ coincides
with $BE$; then the point $D$ will coincide with $E$. Lastly,
let $CE$ be turned round $C$, through the angle $C'$, until $CE$ coincides
with $CF$, and the point $E$ with $F$. Now, it is evident that
by these rotations the point $C$ has been brought successively into
the positions $D$, $E$, $F$; hence, by a motion of mere translation
along the line $FC$, the line $CA$ can be brought into its former
position. Therefore it follows, since rotation is independent of
translation, that the amount of the three rotations is equal to one
complete revolution round the point $A$; therefore $A'+B'+C' =$ four right angles;
but\\[-\baselineskip]
\begin{flalign*}
   && A + A' + B + B' + C + C' &= \text{six right angles [I.\ \textsc{xiii.}]}; && \\
&\text{hence}
    & A + B + C &= \text{two right angles.} && \phantom{hence. }
\end{flalign*}

\textsf{Observation.}---The foregoing demonstration is the most
elementary that was ever given of this celebrated Proposition.
I have reduced it to its simplest form, and without making
any use of the language of Quaternions. The same method of
proof will establish the more general Proposition, that the sum
of the external angles of any convex rectilineal figure is equal
to four right angles.



Mr. Abbott\index{Abbott}, \textsc{f.t.c.d.}, has informed me that this demonstration
was first given by Playfair\index{Playfair} in 1826, so that Hamilton was anticipated.
It has been objected to on the ground that, applied
verbatim to a spherical triangle, it would lead to the conclusion
that the sum of the angles is two right angles, which being
wrong, proves that the method is not valid. A slight consideration
will show that the cases are different. In the proof given
in the text there are three motions of rotation, in each of which
a point describes an arc of a circle, followed by a motion of
translation, in which the same point describes \textit{a right line}, and
returns to its original position. On the surface of a sphere we
should have, corresponding to these, three motions of rotation,
in each of which the point would describe an arc of a circle,
followed by a motion of rotation about the centre of the sphere,
in which the point should describe \textit{an arc of a great circle} to
return to its original position. Hence, the proof for a plane
triangle cannot be applied to a spherical triangle.
\end{footnotesize}


\begin{center}
\medskip
\rule[1ex]{3cm}{.2pt}
\medskip

\addcontentsline{toc}{section}{\ibksp C.---To inscribe a regular polygon of seventeen sides in a circle---Ampere's solution simplified,}
{\large NOTE C.}
\medskip

\textsc{to inscribe a regular polygon of seventeen sides in a circle.}
\end{center}

\begin{footnotesize}
\textsf{Analysis.}---Let $A$ be one of the angular points, $AO$ the diameter,
$A_1$, $A_2$, $\ldots$ $A_8$ the vertices at one side of $AO$. Produce
$OA_3$ to $M$, and $OA_2$ to $P$, making $A_3M = OA_5$, and $A_2P = OA_8$.
Again, cut off $A_6N = OA_7$, and $A_1Q = OA_4$. Lastly, cut off
$OR = ON$, and $OS = OQ$. Then we have [IV\@.\ Ex.~40],\\[-\baselineskip]
\begin{flalign*}
&&  \rho_1\rho_4 = R(\rho_3 + \rho_5) &= R\,.\,OM,  \\
&&  \rho_2\rho_8 = R(\rho_6 - \rho_7) &= R\,.\,ON;  \\
&\text{but }&
  \rho_1\rho_2\rho_4\rho_8 = &R^4  & \text{[IV\@.\ Ex.~34]}; \\
&\text{therefore }&
                     OM\,.\,ON = &R^2  & \tag*{(1).}\\

&\text{In like manner, }&
  OP\,.\,OQ = &R^{2}  &  \tag*{(2).}
\end{flalign*}


\imgcent{200}{f250}

\begin{flalign*}
&\indent\text{Again,}
  & OM\,.\,ON & = (\rho_3+\rho_5)(\rho_6-\rho_7)  \\
&& & = \rho_3\rho_6 + \rho_5\rho_6 - \rho_3\rho_7 - \rho_5\rho_7  \\
&& & = R(\rho_3-\rho_8) + R(\rho_1-\rho_6) - R(\rho_2-\rho_7) - R(\rho_2-\rho_5)\ \text{[IV\@.~Ex.~40]}.  \\
&& & = R(OM - ON - OP + OQ) = R(MR - PS):  \\
&& & \qquad MR - PS = R. \\
&\indent\text{Again, }&  MR\,.\,PS &= (OM - ON)(OP - OQ)  \\
&&  &= (\rho_3+\rho_5-\rho_6+\rho_7)(\rho_2+\rho_8-\rho_1+\rho_4);
\end{flalign*}
and performing the multiplication and substituting, we get
\[
4R(OM - ON - OP + OQ) = 4R^{2}.
\]
Hence, the rectangle and the difference of the lines $MR$ and $PS$
being given, each is given; hence $MR$ is given; but $MR = OM
- ON$; therefore $OM - ON$ is given; and we have proved that
the rectangle $OM\,.\,ON = R^{2}$; therefore $OM$ and $ON$ are each
given. In like manner, $OP$ and $OQ$ are each given.

Again,
\[
  \rho_6\,.\,\rho_7 = R(\rho_1-\rho_4) = R\,.\,OQ,\ \text{and}\
  \rho_6-\rho_7 = ON.
\]
Hence, since $OQ$ and $ON$ are each given, $\rho_6$ and $\rho_7$ are each given;
therefore we can draw these chords, and we have the arc $A_{6}A_{7}$
between their extremities given; that is, the seventeenth part of
the circumference of a circle. \emph{Hence the problem is solved.}



The foregoing analysis is due to \textsc{Ampere:}\index{Ampere} see \textsc{Catalan}\index{Catalan}, \emph{Th{\'e}or{\`e}mes
et Probl{\`e}mes de G{\'e}om{\'e}trie El{\'e}mentaire}. We have abridged
and simplified \textsc{Ampere}'s solution.
\end{footnotesize}

\bigskip
\begin{center}
\rule[.5ex]{3cm}{1pt}

\addcontentsline{toc}{section}{\ibksp D.---To find two mean proportionals\index{Proportionals!two mean} between two given lines---Philo's\index{Philo, proofs by} solution,}
\medskip
{\Large NOTE D.}
\medskip

\textsc{to find two mean proportionals between two given lines.}
\end{center}

\begin{footnotesize}
The problem to find two mean proportionals is one of the most
celebrated in Geometry on account of the importance which the
ancients attached to it. It cannot be solved by the line and
circle, but is very easy by Conic Sections. The following is a
mechanical construction by the Ruler and Compass.

\textbf{Sol.}---Let the extremes $AB$, $BC$ be placed at right angles to
each other; complete the rectangle $ABCD$, and describe a circle
about it. Produce $DA$, $DC$, and let a graduated ruler be made
to revolve round the point $B$, and so adjusted that $BE$ shall be
equal to $GF$; then $AF$, $CE$ are two mean proportionals between
$AB$, $BC$.


\imgcent{178}{f251}

\textbf{Dem.}---Since $BE$ is equal to $GF$, the rectangle $BE\,.\,GE
= BF\,.\,GF$. Therefore $DE\,.\,CE$\ $= DF\,.\,AF$; hence $DE : DF
:: AF : CE$; and by similar triangles, $AB : AF :: DE : DF$, and
$CE : CB :: DE : DF$. Hence $AB : AF :: AF : CE$; and $AF : CE
:: CE : CB$. Therefore $AB$, $AF$, $CE$, $CB$ are continual proportions.
Hence [VI\@.\ Def.~\textsc{iv.}] $AF$, $CE$ are two mean proportionals
between $AB$ and $BC$.

The foregoing elegant construction is due to the ancient
Geometer \textsc{Philo} of \textsc{Byzantium}. If we join $DG$ it will be

perpendicular to $EF$. The line $EF$ is called Philo's Line; it
possesses the remarkable property of being the minimum line
through the point $B$ between the fixed lines $DE$, $DF$.

\addcontentsline{toc}{section}{\ibksp\;,,\protect\phantom{.---}Newton's solution,}
\textsc{Newton's Construction.}\index{Newton}---Let $AB$ and $L$ be the two given
lines of which $AB$ is the greater. Bisect $AB$ in $C$. With $A$ as
centre and $AC$ as radius, describe a circle, and in it place the
chord $CD$ equal to the second line $L$. Join $BD$, and draw by
trial through $A$ a line meeting $BD$, $CD$ produced in the points
$E$, $F$, so that the intercept $EF$ will be equal to the radius of the
circle. $DE$ and $FA$ are the mean proportionals\index{Proportionals!two mean} required.


\imgcent{190}{f252}

\textbf{Dem.}---Join $AD$. Since the line $BF$ cuts the sides of the
$\triangle\ ACE$, we have
\begin{flalign*}
&&  AB\,.\,CD\,.\,EF &= CB\,.\,DE\,.\,FA;\ \text{but}\ EF = CB;  &&\\
&\rlap{therefore }&
\phantom{therefore } AB\,.\,CD &= DE\,.\,FA,\ \text{or}\ \frac{CD}{DE} = \frac{FA}{AB}. \phantom{therefore } &&
\end{flalign*}
Again, since the $\triangle\ ACD$ is isosceles, we have
\begin{align*}
  ED\,.\,EC &= EA^{2} - AC^{2} = (FA + AC)^{2} - AC^{2}  \\
        &= 2FA\,.\,AC + FA^{2} = FA\,.\,AB + FA^{2}.
\end{align*}
Hence \hfill$
  ED(ED + CD) = FA(AB + FA)$,\hfill \phantom{Hence }\\[1ex]
or \hfill$\displaystyle
  DE^{2}\left(1 + \frac{CD}{DE}\right)
= FA\,.\,AB \left(1 + \frac{AF}{AB}\right) $,\hfill \phantom{or }\\[1ex]

therefore \ \hfill$DE^2 = FA\,.\,AB$, and we have $AB\,.\,CD = DE\,.\,FA$. \hfill\phantom{therefore }\\[1ex]
Hence \hfill$AB$, $DE$, $FA$, $CD$ are in continued proportion. \hfill\phantom{Hence }
\end{footnotesize}

\bigskip
\begin{center}
\rule[.5ex]{3cm}{1pt}

\bigskip
\addcontentsline{toc}{section}{\ibksp E.---M$^{c}$Cullagh's proof of the minimum property of Philo's line,}
{\large NOTE E.}
\medskip

\textsc{on philo's line.}
\end{center}

\begin{footnotesize}
I am indebted to Professor Galbraith\index{Galbraith} for the following proof
of the minimum property of Philo's Line\index{Lines!Philo's}. It is due to the late
Professor Mac\,Cullagh\index{McCullagh@M$^{c}$Cullagh}:---\emph{Let $AC$, $CB$ be two given lines, $E$ a
fixed point, $CD$ a perpendicular on $AB$; it is required to prove,
if $AE$ is equal to $DB$, that $AB$ is a minimum.}

\textbf{Dem.}---Through $E$ draw $EM$ parallel to $BC$; make $EN = EM$;
produce $AB$ until $EP = AB$. Through the points $N$, $P$ draw
$NT$, $RP$ each parallel to $AC$, and through $P$ draw $PQ$ parallel
to $BC$. It is easy to see from the figure that the parallelogram
$QR$ is equal to the parallelogram $MF$, and is therefore given.
Through $P$ draw $ST$ perpendicular to $EP$. Now, since $AE = DB$,
$BP$ is equal to $DB$; therefore $PS = CD$. Again, since $OP
= AD$, $PT$ is equal to $CD$; therefore $PS = PT$. Hence $QR$
is the maximum parallelogram in the triangle $SVT$.


\imgcent{185}{f253}

Again, if any other line $A'B$ be drawn through $E$, and produced
to $P'$, so that $EP' = AP'$, the point $P'$ must fall outside $ST$,
because the parallelogram $Q'R'$, corresponding to $QR$, will be
equal to $MF$, and therefore equal to $QR$. Hence the line $EP'$

is greater than $EP$, or $A'B'$ is greater than $AB$. Hence $AB$ is a
minimum.
\end{footnotesize}

\bigskip
\begin{center}
\rule[.5ex]{3cm}{1pt}

\bigskip
\addcontentsline{toc}{section}{\ibksp F.---On the trisection of an angle by the ruler and compass,}
{\large NOTE F.}
\medskip

\textsc{on the trisection of an angle.}\index{Angle!trisection of}
\end{center}

\begin{footnotesize}
The following mechanical method of trisecting an angle occurred
to me several years ago. Apart from the interest belonging
to the Problem, it is valuable to the student as a geometrical
exercise:---

\smallskip
\emph{To trisect a given angle $ACB$.}

\smallskip

\imgcent{204}{f254}

\textbf{Sol.}---Erect $CD$ perpendicular to $CA$; bisect the angle $BCD$
by $CG$, and make the angle $ECI$ equal half a right angle; it is
evident that $CI$ will fall between $CB$ and $CA$. Then, if we use a
jointed ruler---that is two equal rulers connected by a pivot---and
make $CB$ equal to the length of one of these rulers, and, with $C$
as centre and $CB$ as radius, describe the circle $BAM$, cutting
$CI$ in $I$: at $I$ draw the tangent $IG$, cutting $CG$ in $G$.

Then, since $ICG$ is half a right angle, and $CIG$ is right, $IGC$ is
half a right angle; therefore $IC$ is equal to $IG$; but $IC$ equal $CB$;
therefore $IG = CB$---equal length of one of the two equal rulers.
Hence, if the rulers be opened out at right angles, and placed

so that the pivot will be at $I$, and one extremity at $C$, the other
extremity at $G$; it is evident that the point $B$ will be between
the two rulers; then, while the extremity at $C$ remains fixed, let
the other be made to traverse the line $GF$, until the edge of the
second ruler passes through $B$: it is plain that the pivot moves
along the circumference of the circle. Let $CH$, $HF$, be the positions
of the rulers when this happens; draw the line $CH$; the
angle $ACH$ is one-third of $ACB$.

\textbf{Dem.}---Produce $BC$ to $M$. Join $HM$. Erect $BO$ at right
angles to $BM$. Then, because $CH=HF$, the angle $HCF = HFC$,
and the angle $DCE = ECB$ (const.). Hence the angle $HCD
= HBC$ [I.~\textsc{xxxii.}], and the right angles $ACD$, $CBO$ are equal;
therefore the angle $ACH$ is equal to $HBO$; that is [III\@.~\textsc{xxxii.}],
equal to $HMB$, or to half the angle $HCB$. Hence $ACH$ is one-third
of $ACB$.
\end{footnotesize}


\bigskip
\begin{center}
\rule[.5ex]{3cm}{1pt}

\bigskip
\addcontentsline{toc}{section}{\ibksp G.---On the quadrature of the circle,}
{\large NOTE G.}
\medskip

\textsc{on the quadrature of the circle.}\index{Quadrature of circle}
\end{center}

\begin{footnotesize}
Modern mathematicians denote the ratio of the circumference
of a circle to its diameter by the symbol $\pi$. Hence, if $r$ denote
the radius, the circumference will be $2\pi r$; and, since the area of
a circle [VI\@.~\textsc{xx.}\ Ex.~15] is equal to half the rectangle contained
by the circumference and the radius, the area will be $\pi r^2$. Hence,
if the area be known, the value of $\pi$ will be known; and, conversely,
if the value of $\pi$ be known, the area is known. On this
account the determination of the value of $\pi$ is called ``the problem
of the quadrature of the circle,'' and is one of the most celebrated
in Mathematics. It is now known that the value of $\pi$ is incommensurable;
that is, that it cannot be expressed as the ratio of
any two whole numbers, and therefore that it can be found only
approximately; but the approximation can be carried as far as we
please, just as in extracting the square root we may proceed to as
many decimal places as may be required. The simplest approximate

value of $\pi$ was found by Archimedes, namely, $22: 7$. This
value is tolerably exact, and is the one used in ordinary calculations,
except where great accuracy is required. The next to
this in ascending order, viz. $355: 113$, found by Vieta, is correct
to six places of decimals. It differs very little from the ratio
$3.1416:1$, given in our elementary books.

Several expeditious methods, depending on the higher mathematics,
are known for calculating the value of $\pi$. The following
is an outline of a very simple elementary method for determining
this important constant. It depends on a theorem which is at
once inferred from VI\@., Ex.~87, namely \textit{``If $a$, $A$ denote the reciprocals
of the areas of any two polygons of the same number of sides
inscribed and circumscribed to a circle; $a'$, $A'$ the corresponding quantities
for polygons of twice the number; $a'$ is the geometric mean
between $a $

and $A$, and $A'$ the arithmetic mean between $a'$ and $A$.''}
Hence, if $a$ and $A$ be known, we can, by the processes of finding
arithmetic and geometric means, find $a'$ and $A'$. In like manner,
from $a'$, $A'$ we can find $a''$, $A''$ related to $a'$, $A'$; as $a'$, $A'$ are to
$a$, $A$. Therefore, proceeding in this manner until we arrive at
values $a^{(n)}$, $A^{(n)}$ that will agree in as many decimal places as
there are in the degree of accuracy we wish to attain; and since
the area of a circle is intermediate between the reciprocals of
$a^{(n)}$ and $A^{(n)}$, the area of the circle can be found to any required
degree of approximation.

If for simplicity we take the radius of the circle to be unity,
and commence with the inscribed and circumscribed squares, we
have
\[
\begin{alignedat}{4}
  &a   &&=.5,              &&A   &&=.25. \\
  &a'  &&=.3535533,\qquad  &&A'  &&=.3017766. \\
  &a'' &&=.3264853,        &&A'' &&=.3141315.
\end{alignedat}
\]

These numbers are found thus: $a'$ is the geometric mean between
$a$ and $A$; that is, between $.5$ and $.25$, and $A'$ is the arithmetic
mean between $a'$ and $A$, or between $.3535533$ and $.25$.
Again, $a''$ is the geometric mean between $a'$ and $A'$; and $A''$ the
arithmetic mean between $a''$ and $A'$. Proceeding in this manner,
we find $a^{(13)}=.3183099$; $A^{(13)} =.3183099$. Hence the area of a

circle radius unity, correct to seven decimal places, is equal to the
reciprocal of $.3183099$; that is, equal to $3.1415926$; or the value
of $\pi$ correct to seven places of decimals is $3.1415926$. The
number $\pi$ is of such fundamental importance in Geometry, that
mathematicians have devoted great attention to its calculation.
\textsc{Mr.\ Shanks}, an English computer, carried the calculation to
$707$ places of decimals. The following are the first $36$ figures
of his result:---
\[
3.141,592,653,589,793,238,462,643,383,279,502,884.
\]

The result is here carried far beyond all the requirements
of Mathematics. Ten decimals are sufficient to give the circumference
of the earth to the fraction of an inch, and thirty
decimals would give the circumference of the whole visible
universe to a quantity imperceptible with the most powerful
microscope.
\end{footnotesize}


\newpage
\addcontentsline{toc}{section}{\bigskip\ibksp Conclusion,}
\begin{center}
CONCLUSION.
\end{center}

In the foregoing Treatise we have given the
Elementary Geometry of the Point, the Line, and
the Circle, and figures formed by combinations of
these. But it is important to the student to remark,
that points and lines, instead of being distinct from,
are limiting cases of, circles; and points and planes
limiting cases of spheres. Thus, a circle whose
radius diminishes to zero becomes a point. If, on
the contrary, the circle be continually enlarged, it
may have its curvature so much diminished, that
any portion of its circumference may be made to
differ in as small a degree as we please from a right
line, and become one when the radius becomes infinite.
This happens when the centre, but not the
circumference, goes to infinity.






\printindex











































































































































































































\begin{center}
THE END.
\end{center}





\newpage
\begin{center}
\textit{\textsf{THIRD EDITION, Revised and Enlarged---3/6, cloth.}}

\rule[1ex]{1cm}{1pt}
\bigskip

{\LARGE \textsf{A SEQUEL}}

\medskip
{\tiny TO THE}

\medskip
{\large FIRST SIX BOOKS OF THE ELEMENTS
OF EUCLID.}
\bigskip

{\footnotesize BY}

\bigskip
{\Large JOHN CASEY, LL\@.D., F.R.S.,}

\medskip
\textit{\textsf{Fellow of the Royal University of Ireland; Vice-President, Royal
Irish Academy; \&c.\ \&c.}}
\bigskip

\rule[1ex]{3cm}{1pt}

\textbf{Dublin: Hodges, Figgis, \& Co.\quad
London: Longmans, Green, \& Co.}

\medskip
\rule[1ex]{3cm}{1pt}
\bigskip

{\large \textsf{EXTRACTS FROM CRITICAL NOTICES.}}

\medskip
``\textsc{Nature}," \textit{April} 17, 1884.
\end{center}
\nopagebreak

\begin{footnotesize}
``We have noticed (`Nature,' vol.~xxiv., p.~52; vol.~xxvi.,
p.~219) two previous editions of this book, and are glad to find
that our favourable opinion of it has been so convincingly indorsed
by teachers and students in general. The novelty of this
edition is a Supplement of Additional Propositions and Exercises.
This contains an elegant mode of obtaining the circle tangential
to three given circles by the methods of false positions, constructions
for a quadrilateral, and a full account---for the first time in
a text-book---of the Brocard, triplicate ratio, and (what the author
proposes to call) the cosine circles. Dr.~Casey has collected together
very many properties of these circles, and, as usual with
him, has added several beautiful results of his own. He has
done excellent service in introducing the circles to the notice of
English students.\ldots We only need say we hope that this
edition may meet with as much acceptance as its predecessors,
it deserves greater acceptance."
\end{footnotesize}


\bigskip
\begin{center}
\textsc{The ``Mathematical Magazine," Erie, Pennsylvania.}
\end{center}
\nopagebreak

\begin{footnotesize}
``Dr.~Casey, an eminent Professor of the Higher Mathematics
and Mathematical Physics in the Catholic University of Ireland,
has just brought out a second edition of his unique `Sequel to
the First Six Books of Euclid,' in which he has contrived to
arrange and to pack more geometrical gems than have appeared
in any single text-book since the days of the self-taught Thomas
Simpson. `The principles of Modern Geometry contained in the
work are, in the present state of Science, indispensable in Pure
and Applied Mathematics, and in Mathematical Physics; and it
is important that the student should become early acquainted
with them.'

``Eleven of the sixteen sections into which the work is divided
exhibit most excellent specimens of geometrical reasoning and
research. These will be found to furnish very neat models for
systematic methods of study. The other five sections contain
261 choice problems for solution. Here the earnest student will
find all that he needs to bring himself abreast with the amazing
developments that are being made almost daily in the vast regions
of Pure and Applied Geometry. On pp.~152 and 153 there is an
elegant solution of the celebrated Malfatti's Problem.

``As our space is limited, we earnestly advise every lover of
the `Bright Seraphic Truth' and every friend of the `Mathematical
Magazine' to procure this invaluable book without
delay."
\end{footnotesize}

\bigskip
\begin{center}
\textsc{The ``Schoolmaster."}
\end{center}
\nopagebreak

\begin{footnotesize}
``This book contains a large number of elementary geometrical
propositions not given in Euclid, which are required by every
student of Mathematics. Here are such propositions as that \textit{the
three bisectors of the sides of a triangle are concurrent}, needed in
determining the position of the centre of gravity of a triangle;
propositions in the circle needed in Practical Geometry and Mechanics;
properties of the centres of similitudes, and the theories

of inversion and reciprocations so useful in certain electrical questions.
The proofs are always neat, and in many cases exceedingly
elegant."
\end{footnotesize}

\bigskip
\begin{center}
\textsc{The ``Educational Times.''}
\end{center}
\nopagebreak

\begin{footnotesize}
``We have certainly seen nowhere so good an introduction to
Modern Geometry, or so copious a collection of those elementary
propositions not given by Euclid, but which are absolutely indispensable
for every student who intends to proceed to the study of
the Higher Mathematics. The style and general get up of the
book are, in every way, worthy of the `Dublin University Press
Series,' to which it belongs."
\end{footnotesize}

\bigskip
\begin{center}
\textsc{The ``School Guardian.''}
\end{center}
\nopagebreak

\begin{footnotesize}
``This book is a well-devised and useful work. It consists of
propositions supplementary to those of the first six books of
Euclid, and a series of carefully arranged exercises which follow
each section. More than half the book is devoted to the Sixth
Book of Euclid, the chapters on the `Theory of Inversion' and
on the `Poles and Polars' being especially good. Its method
skilfully combines the methods of old and modern Geometry; and
a student well acquainted with its subject-matter would be fairly
equipped with the geometrical knowledge he would require for
the study of any branch of physical science."
\end{footnotesize}

\bigskip
\begin{center}
\textsc{The ``Practical Teacher.''}
\end{center}
\nopagebreak

\begin{footnotesize}
``Professor Casey's aim has been to collect within reasonable
compass all those propositions of Modern Geometry to which
reference is often made, but which are as yet embodied nowhere.\ldots
We can unreservedly give the highest praise to the matter
of the book. In most cases the proofs are extraordinarily neat.\ldots
The notes to the Sixth Book are the most satisfactory.
Feuerbach's Theorem (the nine-points circle touches inscribed
and escribed circles) is favoured with two or three proofs, all of
which are elegant. Dr.~Hart's extension of it is extremely well

proved.\ldots We shall have given sufficient commendation
to the book when we say, that the proofs of these (Malfatti's
Problem, and Miquel's Theorem), and equally complex problems,
which we used to shudder to attack, even by the powerful weapons
of analysis, are easily and triumphantly accomplished by
Pure Geometry.

``After showing what great results this book has accomplished
in the minimum of space, it is almost superfluous to say more.
Our author is almost alone in the field, and for the present need
scarcely fear rivals."
\end{footnotesize}

\bigskip
\begin{center}
\textsc{The ``Academy."}
\end{center}
\nopagebreak

\begin{footnotesize}
``Dr. Casey is an accomplished geometer, and this little book
is worthy of his reputation. It is well adapted for use in the
higher forms of our schools. It is a good introduction to the
larger works of Chasles, Salmon, and Townsend. It contains
both a text and numerous examples."
\end{footnotesize}

\bigskip
\begin{center}
\textsc{The ``Journal of Education."}
\end{center}
\nopagebreak

\begin{footnotesize}
``Dr.~Casey's `Sequel to Euclid' will be found a most valuable
work to any student who has thoroughly mastered Euclid, and
imbibed a real taste for geometrical reasoning.\ldots The
higher methods of pure geometrical demonstration, which form
by far the larger and more important portion, are admirable; the
propositions are for the most part extremely well given, and will
amply repay a careful perusal to advanced students."


\end{footnotesize}

\newpage
\begin{center}
{\LARGE PREFACE.}
\end{center}


Frequent applications having been made to \textsc{Dr. Casey}
requesting him to publish a "Key" containing the
Solutions of the Exercises in his "Elements of Euclid,"
but his professorial and other duties scarcely leaving
him any time to devote to it, I undertook, under his
direction, the task of preparing one. Every Solution
was examined and approved of by him before writing
it for publication, so that the work may be regarded
as virtually his.

The Exercises are a joint selection made by him and
the late lamented Professor Townsend, \textsc{s.f.t.c.d.}, and
form one of the finest collections ever published.

\bigskip

\begin{flushright}
JOSEPH B. CASEY. \mbox\quad
\end{flushright}
\bigskip

\noindent
\begin{tabular}{@{}c@{}}
\textsc{86, South Circular-road,}\\
\textit{December} 23, 1886.
\end{tabular}




\newpage
\noindent
\textit{\textsf{Price 4/6, post free.}}]

\bigskip
\begin{center}
THE FIRST SIX BOOKS

\medskip
{\tiny OF THE}

\bigskip
\textsf{\LARGE ELEMENTS OF EUCLID,}

\medskip
With Copious Annotations and Numerous Exercises.
\bigskip

{\footnotesize BY}

\bigskip
{\Large JOHN CASEY, LL\@.D., F.R.S.,}

\medskip
\textit{\textsf{Fellow of the Royal University of Ireland; Vice-President, Royal
Irish Academy; \&c.\ \&c.}}

\bigskip
\rule[1ex]{3cm}{1pt}

\textbf{Dublin: Hodges, Figgis, \& Co.\quad
London: Longmans, Green, \& Co.}

\medskip
\rule[1ex]{3cm}{1pt}
\bigskip

\textsf{\Large OPINIONS OF THE WORK.}
\end{center}

The following are a few of the Opinions received by
Dr. Casey on this Work:---

\begin{footnotesize}
\medskip
``Teachers no longer need be at a loss when asked which of
the numerous `Euclids' they recommend to learners. Dr.~Casey's
will, we presume, supersede all others."---\textsc{The Dublin Evening
Mail}.

\medskip
``Dr.~Casey's work is one of the best and most complete
treatises on Elementary Geometry we have seen. The annotations
on the several propositions are specially valuable to students."---\textsc{The
Northern Whig}.

\medskip
``His long and successful experience as a teacher has eminently
qualified Dr.~Casey for the task which he has undertaken.\ldots
We can unhesitatingly say that this is the best edition of Euclid
that has been yet offered to the public."---\textsc{The Freeman's
Journal}.

\medskip
\begin{center}
\textit{From the} \textsc{Rev.\ R.~Townsend, F.T.C.D.}, \&c.
\end{center}
\nopagebreak

``I have no doubt whatever of the general adoption of your
work through all the schools of Ireland immediately, and of
England also before very long."


\medskip
\begin{center}
\textit{From} \textsc{George Francis FitzGerald}, Esq., F.T.C.D.
\end{center}
\nopagebreak

"Your work on Euclid seems admirable, and is a great improvement
in most ways on its predecessors. It is a great thing
to call the attention of students to the innumerable variations in
statement and simple deductions from propositions.\ldots I should
have preferred some modification of Euclid to a reproduction, but
I suppose people cannot be got to agree to any."

\pagebreak
\begin{center}
\textit{From} \textsc{H. J. Cooke}, Esq., The Academy, Banbridge.
\end{center}
\nopagebreak

"In the clearness, neatness, and variety of demonstrations, it
is far superior to any text-book yet published, whilst the exercises
are all that could be desired."

\medskip
\begin{center}
\textit{From} \textsc{James A. Poole}, M.A., 29, Harcourt-street, Dublin.
\end{center}
\nopagebreak

"This work proves that Irish Scholars can produce Class-books
which even the Head Masters of English Schools will feel it a
duty to introduce into their establishments."

\medskip
\begin{center}
\textit{From} \textsc{Professor Leebody}, Magee College, Londonderry.
\end{center}
\nopagebreak

"So far as I have had time to examine it, it seems to me a
very valuable addition to our text-books of Elementary Geometry,
and a most suitable introduction to the `Sequel to Euclid,' which
I have found an admirable book for class teaching."

\medskip
\begin{center}
\textit{From} \textsc{Mrs. Bryant}, F.C.P., Principal of the North London Collegiate
School for Girls.
\end{center}
\nopagebreak

"I am heartily glad to welcome this work as a substitute for
the much less elegant text-books in vogue here. I have begun
to use it already with some of my classes, and find that the
arrangement of exercises after each proposition works admirably."

\medskip
\begin{center}
\textit{From the} \textsc{Rev. J. E. Reffé}, French College, Blackrock.
\end{center}
\nopagebreak

"I am sure you will soon be obliged to prepare a Second
Edition. I have ordered fifty copies more of the Euclid (this
makes 250 copies for the French College). They all like the book
here."

\medskip
\begin{center}
\textit{From the} \textsc{Nottingham Guardian}.
\end{center}
\nopagebreak

"The edition of the First Six Books of Euclid by Dr. John
Casey is a particularly useful and able work.\ldots The illustrative
exercises and problems are exceedingly numerous, and
have been selected with great care. Dr. Casey has done an
undoubted service to teachers in preparing an edition of Euclid
adapted to the development of the Geometry of the present day."


\medskip
\begin{center}
\textit{From the} \textsc{Leeds Mercury}.
\end{center}
\nopagebreak

"There is a simplicity and neatness of style in the solution of
the problems which will be of great assistance to the students in
mastering them.\ldots At the end of each proposition there is an
examination paper upon it, with deductions and other propositions,
by means of which the student is at once enabled to test
himself whether he has fully grasped the principles involved.\ldots
Dr. Casey brings at once the student face to face with the difficulties
to be encountered, and trains him, stage by stage, to solve
them."

\medskip
\begin{center}
\textit{From the} \textsc{Practical Teacher}.
\end{center}
\nopagebreak

"The preface states that this book `is intended to supply a
want much felt by Teachers at the present day--the production
of a work which, while giving the unrivalled original in all its
integrity, would also contain the modern conceptions and developments
of the portion of Geometry over which the elements
extend.'

"The book is all, and more than all, it professes to be.\ldots The
propositions suggested are such as will be found to have most
important applications, and the methods of proof are both simple
and elegant. We know no book which, within so moderate
a compass, puts the student in possession of such valuable results.

"The exercises left for solution are such as will repay patient
study, and those whose solution are given in the book itself will
suggest the methods by which the others are to be demonstrated.
We recommend everyone who wants good exercises in Geometry
to get the book, and study it for themselves."

\medskip
\begin{center}
\textit{From the} \textsc{Educational Times}.
\end{center}
\nopagebreak

"The editor has been very happy in some of the changes he
has made. The combination of the general and particular enunciations
of each proposition into one is good; and the shortening
of the proofs, by omitting the repetitions so common in Euclid, is
another improvement. The use of the contra-positive of a proved
theorem is introduced with advantage, in place of the \textit{reductio ad
absurdum}; while the alternative (or, in some cases, substituted)
proofs are numerous, many of them being not only elegant but
eminently suggestive. The notes at the end of the book are of
great interest, and much of the matter is not easily accessible.
The collection of exercises, `of which there are nearly eight
hundred,' is another feature which will commend the book to
teachers. To sum up, we think that this work ought to be read
by every teacher of Geometry; and we make bold to say that no
one can study it without gaining valuable information, and still
more valuable suggestions."


\medskip
\begin{center}
\textit{From the} \textsc{Journal of Education}, Sept. 1, 1883.
\end{center}
\nopagebreak

"In the text of the propositions, the author has adhered, in all
but a few instances, to the substance of Euclid's demonstrations,
without, however, giving way to a slavish following of his occasional
verbiage and redundance. The use of letters in brackets
in the enunciations eludes the necessity of giving a second or
particular enunciation, and can do no harm. Hints of other
proofs are often given in small type at the end of a proposition,
and, where necessary, short explanations. The definitions are
also carefully annotated. The theory of proportion, Book V., is
given in an algebraical form. This book has always appeared to
us an exquisitely subtle example of Greek mathematical logic,
but the subject can be made infinitely simpler and shorter by a
little algebra, and naturally the more difficult method has yielded
place to the less. It is not studied in schools, it is not asked for
even in the Cambridge Tripos; a few years ago, it still survived
in one of the College Examinations at St. John's, but whether
the reforming spirit which is dominant there has left it, we do
not know. The book contains a very large body of riders and
independent geometrical problems. The simpler of these are
given in immediate connexion with the propositions to which
they naturally attach; the more difficult are given in collections
at the end of each book. Some of these are solved in the book,
and these include many well-known theorems, properties of orthocentre,
of nine-point circle, \&c. In every way this edition of
Euclid is deserving of commendation. We would also express a
hope that everyone who uses this book will afterwards read the
same author's `Sequel to Euclid,' where he will find an excellent
account of more modern Geometry."
\end{footnotesize}

\begin{center}

\rule[1ex]{3cm}{1pt}

\bigskip
\textit{\textsf{\large NOW READY, Price 6s.,}}

\bigskip
{\large A KEY to the EXERCISES in the ELEMENTS of
EUCLID.}
\end{center}

\newpage
\begin{center}\textsc{Typographical Errors corrected in Project Gutenberg edition}\end{center}

p.~\pageref{def8}.~``\textsc{Def.~viii}.---When a right line
intersects \ldots'' in original, amended to ``\textsc{Def.~vii}'' in sequence.\

p.~\pageref{parallellogram}. 12 ``bisects the parallellogram'' in original,
amended to match every other occurrence as ``parallelogram''.

p.~\pageref{1cor2}. ``$\triangle ACH$ is half the
rectangle $AC\,.\,AH$ (I.\ \textit{Cor.}\ 1)'' in original. The reference is to Prop.~I.
of the current book and misnumbered, it should be (\textsc{i}.\ \textit{Cor.}\ 2).

p.~\pageref{cor3}.~``The parallelogram $CM$
is equal to $DE$ [I.~\textsc{xliii}., \textit{Cor.}~3];'' in original, amended to ``\textit{Cor.}~2''
following MS. correction: there is no \textit{Cor.}~3.

p.~\pageref{Ixlvi}.~``On $CB$ describe the square $CBEF$ I. [\textsc{xlvi}.].'' in original,
clearly meant to read [I. \textsc{xlvi}.].

p.~\pageref{remainiug}.~``The remainiug parts of the line'' in original,
obvious error amended to ``remaining''.

p.~\pageref{throuyh}.~``that which is nearest to the line throuyh the centre'' in original,
obvious error amended to ``through''.

p.~\pageref{Icor1}.~``Then this line [I., \emph{Cor.}~1]'' in original.
The reference is to Prop.~I. of the current book, so it should be [\textsc{i.}, \emph{Cor.}~1].

p.~\pageref{Idefxxii}.~``$OA$ is equal to $OC$ [I., Def.~\textsc{xxii.}]'' in original.
The reference should be [I., Def.~\textsc{xxxii.}].

p.~\pageref{concylic1}.~``the four points $A$, $C$, $D$, $B$ are concylic'' in original,
evidently intended is ``concyclic''.

p.~\pageref{Icor1again}.~as p.~\pageref{Icor1}.

p.~\pageref{thopoint}.~``Through tho point $E$'' in original,
obvious error amended to ``the''.

p.~\pageref{concylic2}.,~p.~\pageref{concylic3}.~``the points $A$, $B$, $C$, $D$ are concylic'' in original,
as p.~\pageref{concylic1}.

p.~\pageref{concylic3}.~``(Ex.~2.) \ldots or touchlng a given file and a given circle.'' in original,
obvious error amended to ``touching''.

p.~\pageref{ex21}.~``21. What is the locus of the middle points \ldots'' in original,
amended to ``31.'' in sequence.\

p.~\pageref{theifthen}.~In (21) ``the if then line $DE$ intersect the chords \ldots'' in original,
garbled phrase amended to ``then if the''.

p.~\pageref{circlesin}.~In (44) ``these circle sintersect'' in original,
misplaced space amended to ``these circles intersect''.

p.~\pageref{bisection1}.~``4.~The point of bisection (1) of the line $(OP)$'' in original,
from the diagram and following discussion this should be $(I)$.

p.~\pageref{circlesquare}.~Prop.~IX\@. ``About a given circle $(ABCD)$ to describe a circle.'' in original,
clearly this is nonsense and must mean ``About a given square''.

p.~\pageref{traingles}.~``Then the traingles $ABO$, $CBO$'' in original,
obvious error amended to ``triangles''.

p.~\pageref{enforan}.~In (52) ``and also en equilateral circumscribed polygon'' in original,
wrong letter amended to ``an''.

p.~\pageref{NoProblem}.~Heading ``\textsc{PROP\@.~XXV\@.---Problem.}'' in original,
the preamble to this book says that every Proposition in it is a \textsc{Theorem} and this
one seems to be no exception, so amended.\

p.~\pageref{VIi}.~Reference ``[I.]'' is to Proposition I. of current book, amended to ``\textsc{[i.]}''
(4 times).

p.~\pageref{VIii} sqq.~Reference ``[II\@.]'' is to Proposition II\@. of current book, amended to ``\textsc{[ii.]}''
(4 times).

p.~\pageref{bg}.~Prop V. header ``subtended bg the homologous sides'' in original,
obvious error amended to ``by''.

p.~\pageref{XVI}.~Reference ``[XVI\@.]'' corrected to ``[\textsc{xvi.}]''.

p.~\pageref{isis}.~``From the construction is is evident \ldots'' in original,
obvious error amended to ``it is''.

p.~\pageref{poiat}.~``20.~Find a poiat $O$'' in original,
obvious error amended to ``point''.

p.~\pageref{perpendiclar}.~``the lines $GH$, $GK$ each perpendiclar to $EF$'' in original,
obvious error amended to ``perpendicular''.

p.~\pageref{wrongo}.~``$O$'' when associated with a lower case letter was wrongly printed as $o$, which
is not defined. These have been corrected (3 times).

p.~\pageref{XIvi}.~Reference ``[VI\@., \emph{Cor.}~6]'' corrected to ``[\textsc{vi.}, \emph{Cor.}~6]''.

\newpage
\small
\pagenumbering{gobble}
\begin{verbatim}
*** END OF THE PROJECT GUTENBERG EBOOK ELEMENTS OF EUCLID ***

***** This file should be named 21076-t.tex or 21076-t.zip *****
***** or                    21076-pdf.pdf or 21076-pdf.pdf *****
This and all associated files of various formats will be found in:
    https://www.gutenberg.org/2/1/0/7/21076/

Updated editions will replace the previous one--the old editions will
be renamed.

Creating the works from print editions not protected by U.S. copyright
law means that no one owns a United States copyright in these works,
so the Foundation (and you!) can copy and distribute it in the
United States without permission and without paying copyright
royalties. Special rules, set forth in the General Terms of Use part
of this license, apply to copying and distributing Project
Gutenberg-tm electronic works to protect the PROJECT GUTENBERG-tm
concept and trademark. Project Gutenberg is a registered trademark,
and may not be used if you charge for an eBook, except by following
the terms of the trademark license, including paying royalties for use
of the Project Gutenberg trademark. If you do not charge anything for
copies of this eBook, complying with the trademark license is very
easy. You may use this eBook for nearly any purpose such as creation
of derivative works, reports, performances and research. Project
Gutenberg eBooks may be modified and printed and given away--you may
do practically ANYTHING in the United States with eBooks not protected
by U.S. copyright law. Redistribution is subject to the trademark
license, especially commercial redistribution.

START: FULL LICENSE

THE FULL PROJECT GUTENBERG LICENSE
PLEASE READ THIS BEFORE YOU DISTRIBUTE OR USE THIS WORK

To protect the Project Gutenberg-tm mission of promoting the free
distribution of electronic works, by using or distributing this work
(or any other work associated in any way with the phrase `Project
Gutenberg'), you agree to comply with all the terms of the Full
Project Gutenberg-tm License available with this file or online at
www.gutenberg.org/license.

Section 1. General Terms of Use and Redistributing Project
Gutenberg-tm electronic works

1.A. By reading or using any part of this Project Gutenberg-tm
electronic work, you indicate that you have read, understand, agree to
and accept all the terms of this license and intellectual property
(trademark/copyright) agreement. If you do not agree to abide by all
the terms of this agreement, you must cease using and return or
destroy all copies of Project Gutenberg-tm electronic works in your
possession. If you paid a fee for obtaining a copy of or access to a
Project Gutenberg-tm electronic work and you do not agree to be bound
by the terms of this agreement, you may obtain a refund from the
person or entity to whom you paid the fee as set forth in paragraph
1.E.8.

1.B. `Project Gutenberg' is a registered trademark. It may only be
used on or associated in any way with an electronic work by people who
agree to be bound by the terms of this agreement. There are a few
things that you can do with most Project Gutenberg-tm electronic works
even without complying with the full terms of this agreement. See
paragraph 1.C below. There are a lot of things you can do with Project
Gutenberg-tm electronic works if you follow the terms of this
agreement and help preserve free future access to Project Gutenberg-tm
electronic works. See paragraph 1.E below.

1.C. The Project Gutenberg Literary Archive Foundation (`the
Foundation' or PGLAF), owns a compilation copyright in the collection
of Project Gutenberg-tm electronic works. Nearly all the individual
works in the collection are in the public domain in the United
States. If an individual work is unprotected by copyright law in the
United States and you are located in the United States, we do not
claim a right to prevent you from copying, distributing, performing,
displaying or creating derivative works based on the work as long as
all references to Project Gutenberg are removed. Of course, we hope
that you will support the Project Gutenberg-tm mission of promoting
free access to electronic works by freely sharing Project Gutenberg-tm
works in compliance with the terms of this agreement for keeping the
Project Gutenberg-tm name associated with the work. You can easily
comply with the terms of this agreement by keeping this work in the
same format with its attached full Project Gutenberg-tm License when
you share it without charge with others.

1.D. The copyright laws of the place where you are located also govern
what you can do with this work. Copyright laws in most countries are
in a constant state of change. If you are outside the United States,
check the laws of your country in addition to the terms of this
agreement before downloading, copying, displaying, performing,
distributing or creating derivative works based on this work or any
other Project Gutenberg-tm work. The Foundation makes no
representations concerning the copyright status of any work in any
country other than the United States.

1.E. Unless you have removed all references to Project Gutenberg:

1.E.1. The following sentence, with active links to, or other
immediate access to, the full Project Gutenberg-tm License must appear
prominently whenever any copy of a Project Gutenberg-tm work (any work
on which the phrase `Project Gutenberg' appears, or with which the
phrase `Project Gutenberg' is associated) is accessed, displayed,
performed, viewed, copied or distributed:

  This eBook is for the use of anyone anywhere in the United States and
  most other parts of the world at no cost and with almost no
  restrictions whatsoever. You may copy it, give it away or re-use it
  under the terms of the Project Gutenberg License included with this
  eBook or online at www.gutenberg.org. If you are not located in the
  United States, you will have to check the laws of the country where
  you are located before using this eBook.

1.E.2. If an individual Project Gutenberg-tm electronic work is
derived from texts not protected by U.S. copyright law (does not
contain a notice indicating that it is posted with permission of the
copyright holder), the work can be copied and distributed to anyone in
the United States without paying any fees or charges. If you are
redistributing or providing access to a work with the phrase `Project
Gutenberg' associated with or appearing on the work, you must comply
either with the requirements of paragraphs 1.E.1 through 1.E.7 or
obtain permission for the use of the work and the Project Gutenberg-tm
trademark as set forth in paragraphs 1.E.8 or 1.E.9.

1.E.3. If an individual Project Gutenberg-tm electronic work is posted
with the permission of the copyright holder, your use and distribution
must comply with both paragraphs 1.E.1 through 1.E.7 and any
additional terms imposed by the copyright holder. Additional terms
will be linked to the Project Gutenberg-tm License for all works
posted with the permission of the copyright holder found at the
beginning of this work.

1.E.4. Do not unlink or detach or remove the full Project Gutenberg-tm
License terms from this work, or any files containing a part of this
work or any other work associated with Project Gutenberg-tm.

1.E.5. Do not copy, display, perform, distribute or redistribute this
electronic work, or any part of this electronic work, without
prominently displaying the sentence set forth in paragraph 1.E.1 with
active links or immediate access to the full terms of the Project
Gutenberg-tm License.

1.E.6. You may convert to and distribute this work in any binary,
compressed, marked up, nonproprietary or proprietary form, including
any word processing or hypertext form. However, if you provide access
to or distribute copies of a Project Gutenberg-tm work in a format
other than `Plain Vanilla ASCII' or other format used in the official
version posted on the official Project Gutenberg-tm website
(www.gutenberg.org), you must, at no additional cost, fee or expense
to the user, provide a copy, a means of exporting a copy, or a means
of obtaining a copy upon request, of the work in its original `Plain
Vanilla ASCII' or other form. Any alternate format must include the
full Project Gutenberg-tm License as specified in paragraph 1.E.1.

1.E.7. Do not charge a fee for access to, viewing, displaying,
performing, copying or distributing any Project Gutenberg-tm works
unless you comply with paragraph 1.E.8 or 1.E.9.

1.E.8. You may charge a reasonable fee for copies of or providing
access to or distributing Project Gutenberg-tm electronic works
provided that:

* You pay a royalty fee of 20
  the use of Project Gutenberg-tm works calculated using the method
  you already use to calculate your applicable taxes. The fee is owed
  to the owner of the Project Gutenberg-tm trademark, but he has
  agreed to donate royalties under this paragraph to the Project
  Gutenberg Literary Archive Foundation. Royalty payments must be paid
  within 60 days following each date on which you prepare (or are
  legally required to prepare) your periodic tax returns. Royalty
  payments should be clearly marked as such and sent to the Project
  Gutenberg Literary Archive Foundation at the address specified in
  Section 4, `Information about donations to the Project Gutenberg
  Literary Archive Foundation.'

* You provide a full refund of any money paid by a user who notifies
  you in writing (or by email) within 30 days of receipt that s/he
  does not agree to the terms of the full Project Gutenberg-tm
  License. You must require such a user to return or destroy all
  copies of the works possessed in a physical medium and discontinue
  all use of and all access to other copies of Project Gutenberg-tm
  works.

* You provide, in accordance with paragraph 1.F.3, a full refund of
  any money paid for a work or a replacement copy, if a defect in the
  electronic work is discovered and reported to you within 90 days of
  receipt of the work.

* You comply with all other terms of this agreement for free
  distribution of Project Gutenberg-tm works.

1.E.9. If you wish to charge a fee or distribute a Project
Gutenberg-tm electronic work or group of works on different terms than
are set forth in this agreement, you must obtain permission in writing
from the Project Gutenberg Literary Archive Foundation, the manager of
the Project Gutenberg-tm trademark. Contact the Foundation as set
forth in Section 3 below.

1.F.

1.F.1. Project Gutenberg volunteers and employees expend considerable
effort to identify, do copyright research on, transcribe and proofread
works not protected by U.S. copyright law in creating the Project
Gutenberg-tm collection. Despite these efforts, Project Gutenberg-tm
electronic works, and the medium on which they may be stored, may
contain `Defects,' such as, but not limited to, incomplete, inaccurate
or corrupt data, transcription errors, a copyright or other
intellectual property infringement, a defective or damaged disk or
other medium, a computer virus, or computer codes that damage or
cannot be read by your equipment.

1.F.2. LIMITED WARRANTY, DISCLAIMER OF DAMAGES - Except for the `Right
of Replacement or Refund' described in paragraph 1.F.3, the Project
Gutenberg Literary Archive Foundation, the owner of the Project
Gutenberg-tm trademark, and any other party distributing a Project
Gutenberg-tm electronic work under this agreement, disclaim all
liability to you for damages, costs and expenses, including legal
fees. YOU AGREE THAT YOU HAVE NO REMEDIES FOR NEGLIGENCE, STRICT
LIABILITY, BREACH OF WARRANTY OR BREACH OF CONTRACT EXCEPT THOSE
PROVIDED IN PARAGRAPH 1.F.3. YOU AGREE THAT THE FOUNDATION, THE
TRADEMARK OWNER, AND ANY DISTRIBUTOR UNDER THIS AGREEMENT WILL NOT BE
LIABLE TO YOU FOR ACTUAL, DIRECT, INDIRECT, CONSEQUENTIAL, PUNITIVE OR
INCIDENTAL DAMAGES EVEN IF YOU GIVE NOTICE OF THE POSSIBILITY OF SUCH
DAMAGE.

1.F.3. LIMITED RIGHT OF REPLACEMENT OR REFUND - If you discover a
defect in this electronic work within 90 days of receiving it, you can
receive a refund of the money (if any) you paid for it by sending a
written explanation to the person you received the work from. If you
received the work on a physical medium, you must return the medium
with your written explanation. The person or entity that provided you
with the defective work may elect to provide a replacement copy in
lieu of a refund. If you received the work electronically, the person
or entity providing it to you may choose to give you a second
opportunity to receive the work electronically in lieu of a refund. If
the second copy is also defective, you may demand a refund in writing
without further opportunities to fix the problem.

1.F.4. Except for the limited right of replacement or refund set forth
in paragraph 1.F.3, this work is provided to you--`AS-IS', WITH NO
OTHER WARRANTIES OF ANY KIND, EXPRESS OR IMPLIED, INCLUDING BUT NOT
LIMITED TO WARRANTIES OF MERCHANTABILITY OR FITNESS FOR ANY PURPOSE.

1.F.5. Some states do not allow disclaimers of certain implied
warranties or the exclusion or limitation of certain types of
damages. If any disclaimer or limitation set forth in this agreement
violates the law of the state applicable to this agreement, the
agreement shall be interpreted to make the maximum disclaimer or
limitation permitted by the applicable state law. The invalidity or
unenforceability of any provision of this agreement shall not void the
remaining provisions.

1.F.6. INDEMNITY - You agree to indemnify and hold the Foundation, the
trademark owner, any agent or employee of the Foundation, anyone
providing copies of Project Gutenberg-tm electronic works in
accordance with this agreement, and any volunteers associated with the
production, promotion and distribution of Project Gutenberg-tm
electronic works, harmless from all liability, costs and expenses,
including legal fees, that arise directly or indirectly from any of
the following which you do or cause to occur: (a) distribution of this
or any Project Gutenberg-tm work, (b) alteration, modification, or
additions or deletions to any Project Gutenberg-tm work, and (c) any
Defect you cause.

Section 2. Information about the Mission of Project Gutenberg-tm

Project Gutenberg-tm is synonymous with the free distribution of
electronic works in formats readable by the widest variety of
computers including obsolete, old, middle-aged and new computers. It
exists because of the efforts of hundreds of volunteers and donations
from people in all walks of life.

Volunteers and financial support to provide volunteers with the
assistance they need are critical to reaching Project Gutenberg-tm's
goals and ensuring that the Project Gutenberg-tm collection will
remain freely available for generations to come. In 2001, the Project
Gutenberg Literary Archive Foundation was created to provide a secure
and permanent future for Project Gutenberg-tm and future
generations. To learn more about the Project Gutenberg Literary
Archive Foundation and how your efforts and donations can help, see
Sections 3 and 4 and the Foundation information page at
www.gutenberg.org

Section 3. Information about the Project Gutenberg Literary
Archive Foundation

The Project Gutenberg Literary Archive Foundation is a non-profit
501(c)(3) educational corporation organized under the laws of the
state of Mississippi and granted tax exempt status by the Internal
Revenue Service. The Foundation's EIN or federal tax identification
number is 64-6221541. Contributions to the Project Gutenberg Literary
Archive Foundation are tax deductible to the full extent permitted by
U.S. federal laws and your state's laws.

The Foundation's business office is located at 809 North 1500 West,
Salt Lake City, UT 84116, (801) 596-1887. Email contact links and up
to date contact information can be found at the Foundation's website
and official page at www.gutenberg.org/contact

Section 4. Information about Donations to the Project Gutenberg
Literary Archive Foundation

Project Gutenberg-tm depends upon and cannot survive without
widespread public support and donations to carry out its mission of
increasing the number of public domain and licensed works that can be
freely distributed in machine-readable form accessible by the widest
array of equipment including outdated equipment. Many small donations
($1 to $5,000) are particularly important to maintaining tax exempt
status with the IRS.

The Foundation is committed to complying with the laws regulating
charities and charitable donations in all 50 states of the United
States. Compliance requirements are not uniform and it takes a
considerable effort, much paperwork and many fees to meet and keep up
with these requirements. We do not solicit donations in locations
where we have not received written confirmation of compliance. To SEND
DONATIONS or determine the status of compliance for any particular
state visit www.gutenberg.org/donate

While we cannot and do not solicit contributions from states where we
have not met the solicitation requirements, we know of no prohibition
against accepting unsolicited donations from donors in such states who
approach us with offers to donate.

International donations are gratefully accepted, but we cannot make
any statements concerning tax treatment of donations received from
outside the United States. U.S. laws alone swamp our small staff.

Please check the Project Gutenberg web pages for current donation
methods and addresses. Donations are accepted in a number of other
ways including checks, online payments and credit card donations. To
donate, please visit: www.gutenberg.org/donate

Section 5. General Information About Project Gutenberg-tm electronic works

Professor Michael S. Hart was the originator of the Project
Gutenberg-tm concept of a library of electronic works that could be
freely shared with anyone. For forty years, he produced and
distributed Project Gutenberg-tm eBooks with only a loose network of
volunteer support.

Project Gutenberg-tm eBooks are often created from several printed
editions, all of which are confirmed as not protected by copyright in
the U.S. unless a copyright notice is included. Thus, we do not
necessarily keep eBooks in compliance with any particular paper
edition.

Most people start at our website which has the main PG search
facility: www.gutenberg.org

This website includes information about Project Gutenberg-tm,
including how to make donations to the Project Gutenberg Literary
Archive Foundation, how to help produce our new eBooks, and how to
subscribe to our email newsletter to hear about new eBooks.
\end{verbatim}
\end{document}